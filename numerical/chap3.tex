
\chapter{Measuring finite size dependence of free energy in numerical simulations}
\section{Introduction to numerical simulations}

\section{Methods to measure free energy size dependence}
\subsection{The Layer adding method}
Another method is to add an extra layer of spins with Hamiltonian $\Delta H$ (see Fig. (\ref{ext}))and then write
\begin{equation}
H(\lambda) = H +\lambda \Delta H
\end{equation}
We thus have that $H(0)$ is the Hamiltonian for a system with $L-1$ planes of spins and 
$H(1)$ the Hamiltonian with $L$ planes of spins. Clearly we have that
\begin{equation}
{\partial F\over \partial \lambda} = \langle \Delta H \rangle_\lambda,
\end{equation}
the term on the right hand side. Integrating this relation then gives the change in the free energy on adding an extra layer. This is the extrapolating Hamiltonian method
%\begin{figure}
%\begin{center}
%\includegraphics[width=.95\linewidth,height=1.1\linewidth]{casfig1.pdf} 
%\end{center}
%\caption{From Vasilyev et al 2007 - scheme representing the extrapolating Hamiltonian method}\label{ext}
%\end{figure}

The whole procedure is quite time consuming and complication and it would be quite a break through to find a more efficient method. 

The results for the 3d Ising model are shown in Figs. (\ref{c++}) and (\ref{c+-}) and are taken from Vasilyev et al 2007 (note that $x<0$ corresponds to $T<T_c$).

%\begin{figure}
%\begin{center}
%\includegraphics[width=.95\linewidth,height=1.1\linewidth]{c++.pdf} 
%\end{center}
%\caption{From Vasilyev et al 2007 - $\theta_{++}$ for the 3d Ising model}\label{c++}
%\end{figure}
%
%\begin{figure}
%\begin{center}\textsl{•}
%\includegraphics[width=.95\linewidth,height=1.1\linewidth]{c+-.pdf} 
%\end{center}
%\caption{From Vasilyev et al 2007 - $\theta_{+-}$ for the 3d Ising model}\label{c+-}
%\end{figure}
\subsection{The Lopez method}
\subsection{The method of staggered fields etc applied to SOS}
