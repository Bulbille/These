\documentclass[a4paper,11pt]{book}
\usepackage[utf8]{inputenc} % accents    
\usepackage[frenchb]{babel}
\usepackage[T1]{fontenc} %accents
\usepackage{lmodern}
\usepackage{xcolor} % pour colorier
\usepackage[hidelinks]{hyperref} % pour mettre les liens cliquants

\usepackage{setspace}
\onehalfspacing

% Maths
\usepackage{amsmath}
\usepackage{geometry}
\usepackage{amssymb}
\usepackage{bm}

% Figures
\usepackage[final]{graphicx}
\usepackage{subcaption}
\usepackage{mwe}
\setkeys{Gin}{width=10cm} %% largeur par défaut
%\geometry{hmargin=2.5cm,vmargin=2cm}

\usepackage{comment}

%Pour mettre une page en paysage
\usepackage{lscape}

\usepackage{epsfig}
\usepackage{bm}
\usepackage{latexsym}
\usepackage{graphicx}
\usepackage{comment}

\usepackage{graphicx,color}
 \graphicspath{{./}{./figures/}}
\usepackage{ams math}
\usepackage{enumerate}
%\usepackage{graphicx}
\usepackage{mathbbol}
\usepackage{amsfonts}
\usepackage{natbib}
\def\figwidth{8cm}
\usepackage{bm}
\usepackage{hyperref}


%%% Commandes personnalisées  %%%
\newcommand{\nn}{\nonumber \\} % newline sans nombre dans align
\newcommand{\mH}{\mathcal{H}} %pour les hamiltoniens ronds
\newcommand{\mZ}{\mathcal{Z}} %pour les hamiltoniens ronds
\newcommand{\mC}{\mathcal{C}} %pour les fonctions de corrélation
\newcommand{\mx}{{\bf x}} %pour les vecteurs en gras
\newcommand{\mq}{{\bf q}}
\DeclareMathOperator{\sgn}{sgn}
% Les > et < se comportent normalement si c'est pour supérieur ou inférieur, sinon se comportent comme \langle
\mathchardef\less=\mathcode`<
\mathchardef\greater=\mathcode`>
\DeclareMathDelimiter{<}{\mathopen}{symbols}{"68}{largesymbols}{"0A}
\DeclareMathDelimiter{>}{\mathclose}{symbols}{"69}{largesymbols}{"0B}

%%%%%%% Début du document %%%%%
\begin{document}

\bibliographystyle{unsrt}

\tableofcontents

\chapter*{Introduction}
\addcontentsline{toc}{chapter}{Introduction} 

Chaque système statistique est décrit par un paramètre d'ordre, que ce soit la magnétisation moyenne d'un milieu aimanté ou de l'orientation moyenne des polymères par exemple. Dans ces systèmes, on appelle phase un milieu homogène selon le paramètre d'ordre. Les transitions de phase d'un système, où une phase homogène se sépare en plusieurs phases différentes, possèdent des propriétés exceptionnelles. Lors d'une transition de phase continue, la largeur de l'interface entre deux phases diverge jusqu'à atteindre une taille macroscopique. Ce confinement de l'interface mène a des effets de taille finie, notamment l'effet Casimir critique.

L'étude des propriétés statistiques des interfaces peut se faire via différentes approches, toutes complémentaires. Les modèles sur réseau, et plus préciseément le modèle d'Ising, sont particulièrement adaptés aux simulations numériques, mais plus difficiles à traiter analytiquement à cause du trop grand nombre d'interactions présentes dans le système. Une manière de diminuer les interactions est de réduire la dimensionalité du système à une dimension, ce qui nous permet d'utiliser le formalisme des matrices de transfert. À cet effet, le modèle Solid-On-Solid a été énormément étudié dans les annés 80-90 pour sa simplicité.
L'interface peut également être assimilée à un marcheur brownien qui, au lieu de bouger dans le temps, se meut dans l'espace. Ainsi, les équations de Schrödinger et les équations du mouvement de Langevin permettent de décrire les fluctuations d'une interface grâce au formalisme quantique ou stochastique. Cette méthode s'appelle la théorie des ondes capillaires et permet la résolution d'un système très analogue aux modèles SOS, en utilisant un formalisme très similaire aux matrices de transfert. 
La dernière méthode que nous aborderons dans cette thèse est celle du champ moyen, c'est-à-dire l'étude des propriétés de deux phases grâce aux équations de Landau-Ginzburg. Cette méthode a l'avantage d'offrir des calculs analytiques relativement faciles et permet d'obtenir la forme des fonctions de corrélation à plusieurs point ainsi que les longueurs de corrélation. Néanmoins, la vérification des résultats via les simulations numériques, qui nous permettrait de mettre des grandeurs mésoscopiques comme la tension superficielle ou la longueur capillaire en relation avec les grandeurs microscopiques du modèle d'Ising, est asez difficile, et nous éviterons dans la présente thèse le rapprochement.

L'étude de l'effet Casimir critique - qui est un effet de taille finie - revient au final à étudier les propriétés statistiques d'une interface, et à voir comment elles sont modifiées lorsqu'il existe des conditions aux bords. Cependant, les propriétés d'une interface varient également lorsqu'elle est mise hors-équilibre, par exemple via une force du style cisaillement, qui représente beaucoup de cas expérimentaux classiques. 
Nous nous intéressons ici particulièrement à la différence entre les états d'équilibre et hors-équilibre, pour lesquels les formalismes sont différents mais dont les simulations numériques sont similaires. La présence d'un système hors-équilibre pose également plein de questions sur la nature de l'ensemble thermodynamique que l'on se place, et nous éclaire sur les différences entre l'ensemble canonique et grand-canonique, et également entre des particules discernables et indiscernables.

Le manuscrit se décompose de la manière suivante :
\begin{itemize}
    \item Le premier chapitre introduit les différents approches historiques sur l'étude des interfaces, en s''attardant sur les principaux résultats obtenus dans la littérature pour des interfaces à l'équilibre, puis hors-équilibre
    \item Le second chapitre introduit le modèle Solid-On-Solid, qui est une approximation 1D à très basse température du modèle d'Ising en 2D. Nous y parlerons du formalisme des matrices de transfert, des principaux résultats obtenus dans ce modèle et de quelques précisions fondamentales sur les différents ensembles thermodynamique sdans lesquelles ont peut étudier nos systèmes
    \item Nous présentons dans le troisième chapitre l'alogirthme de Monter Carlo-Metropolis, un outil puissant pour explorer l'espace des phases et calculer numériquement la fonction de partition de nos systèmes
    \item Dans le quatrième chapitre, l'étude d'un système SOS - analogue à la croissance d'un cristal -  via une méthode d'intégration sur les potentitels chimiques nour permet d'obtenir l'énergie libre, et ainsi l'effet Casimir. Cette étude se termine par l'ajout du cisaillement.
        \item Dans le cinquième chapitre, nos étudions un modèle avec un champ magnétique charactéristique des expériences dans lesquelles on force une phase d'un fluide binaire dans une autre grâce à une pression de radiation exercée par un laser
    \item Un nouveau modèle découlant des considérations du second chapitre peut être créé de la même manière que le modèle SOS, en prenant en compte l'entropie. Ce nouveau modèle, baptisé Particles-Over-Paticle, fait l'objet du sixième chapitre
    \item Le septième chapitre reprend les calculs d'une publication récente de notre équipe sur un modèle de champ moyen où l'on mélange deux types de particules appartenant à des ensembles thermodynamiques différents, sous l'effet d'un cisaillement uniforme
\end{itemize}

La présente thèse a été rendue possible grâce à l'ANR FISICS, le Laboratoire Onde Matière d'Aquitaine de l'Université de Bordeaux, le Laboratoire de Physique de l'ENS Lyon et le Mésocentre de Calcul Intensif d'Aquitaine sur lesquelles ont été faites les simulations numériques. Je remercie particulièrement Josiane Parzych (LOMA) et Laurence Mauduit (ENS LYON) pour le suivi administratif, ainsi que Nguyen Ky Nguyen (MCIA) pour l'aide technique.

\chapter{Rappels théoriques sur les interfaces}
{\color{red} mettre le bruit systémaiquement à chaque équation, rappeler le bruit}
{\color{red} être sûr de définir tous les nouveaux mots. Revoir 1.64, cisaillement, quj'est-ce que v ?}
{\color{red} séparer ce qu'on fait les autres et ce que j'ai fait moi-même.}
{\color{red} faire gaffe quand ce sont des vecteurs et non des scalaires !!!}
 {\color{red} changer les exponentielles en exponentielles complètes. Ajouter de l'effet Casimir critique de Cardozo avec discussion thermodynamique.}
  
Dans ce chapitre nous analysons la dynamique des systèmes statistiques. L'analyse nous permettra de comprendre comment les transitions de phase, notament certains systèmes subissant une séparation de phase à la transition, se comportent de manière dynamique. L'exemple le plus connu est le modèle d'Ising en absence de champ magnétique, ayant la magnétisation comme paramètre d'ordre de la transition. Dans la phase haute température le système est homogène et sa magnétisation est nulle, tandis qu'en dessous de la température critique, dans le cas où le paramètre d'ordre est conservé (par exemple une dynamique de Kawasaki ou modèle B), le système va localement se séparer en deux phases de magnétisation moyenne opposée séparées par une interface minimisant l'énergie de surface entre les deux phases. 
Dans le cas où le paramètre d'ordre n'est pas conservé (par exemple une dynamique de Glauber ou modèle A), une brisure spontannée de symmétrie fera que l'une des deux phases englobe l'autre, au point de recouvrir tout le système (voir Fig \ref{clusterization}). Dans une transition de phase continue où le point critique est atteint depuis l'état désordonné vers l'état ordonné, les domaines de phase égales ont pour taille la longueur de corrélation du système. Dans les transitions de phase telles que celles du modèle d'Ising, la longueur de corrélation diverge lorsque l'on s'approche de la température critique $T_C$. En présence d'un système infini, la longueur de corrélation devient infinie, ce qui implique que le système prend un temps infini pour atteindre l'équilibre thermodynamique. Ce processus de croissance des domaines depuis la phase désordonnée s'appelle le \textit{coarsening} et la théorie de la cinétique d'ordre des phases est la théorie qui a été développée pour le comprendre.
Cette thèse s'appuie cette théorie pour déterminer les propriétés statistiques (telles que la position moyenne et sa tension superficielle) des interfaces entre deux phases coexistantes.

Tandis que l'Hamiltonien d'un système permet d'explorer toutes les configurations d'équilibre possibles, la dynamique du coarsening ne peut être étudiée qu'en construisant un modèle qui explique l'évolution de l'état du système en fonction du temps. Nous verrons tout au long de ce travail comment la conservation (ou non) du paramètre d'ordre influe sur la dynamique. Nous nous référons principalement dans ce chapitre aux références \cite{hohenberg_theory_1977,bray_theory_1994,krapivsky_kinetic_2010,halpin-healy_kinetic_1995}.

Une connaissance parfaite de la fonction de partition nécessite de connaître toutes les micro-configurations possibles du système. Les appareils de mesure possèdent tous une résolution spatiale et temporelle, c'est-à-dire qu'ils mesurent l'état moyen de toutes les particules dans un volume et dans un laps de temps donné. Plus la résolution des appareils de mesure est bonne, et plus la mesure des observables dérivées de la fonction de partition est précise. 
Concrètement, l'appareil de mesure nous donne un champ - par exemple de densité - $\phi(\mx,t)$ de notre système, qui correspond à l'intégration sur un petit volume autour de $\mx$ et une petite durée de temps autour de $t$.
{\color{red} mettre équation coarse-graining} 

\begin{figure}[h]
    \centering
    \includegraphics[width=0.9\linewidth]{intro/clusterization.pdf}
    \caption{Phénomène d'aggrégation à partir d'une trempe (\textit{quench}) dans un modèle d'Ising de $T=\infty$ à $T=T_C$ pour différents temps en étapes de Monte Carlo, pour un système $600 \times 600$ avec une dynamique non-conservée de Glauber.}
    \label{clusterization}
\end{figure}

%%%%%%%%%%%%%%%%%%%%%%%%%%%%%%%%%
\section{Équations dynamique d'un champ}
%%%%%%%%%%%%%%%%%%%%%%%%%%%%%%%%%

    \subsection{Champ avec un nombre de degrés de liberté fini}
Considérons un système dans l'ensemble canonique d'Hamiltonien  $H(\mq)$ où les $\mq_i$ ($i \in [0,N]$) représentent un nombre fini de degrés de liberté. La fonction de partition est donnée par 
\begin{align}
    Z = \int d\mq e^{-\beta H(\mq)}
\end{align}
avec la probabilité que le système se retrouve dans l'état $\mq$ égale à
\begin{align}
    P_{eq}(\mq) = \frac{e^{-\beta H(\mq)}}{Z}
    \label{eqdis}
\end{align}
À cause du trop grand nombre de degrés de libertés, la fonction de partition est rarement calculable analytiquement. Dans la limite $\beta \to \infty$ - c'est-à-dire la limite où la configuration du système minimisant le plus l'énergie est la plus probable - l'intégrale peut s'approcher par la méthode de Laplace pour l'évaluation des intégrales 
\begin{align}
    Z_{MF}= e^{-\beta H(\mq^*)}
\end{align}
Le champ $\mq^*$ est le champ qui minimise $H$ dont les degrés de liberté sont déterminés par
\begin{align}
    \frac{\partial H}{\partial q_i}|_{\mq={\bf q^*}}=0
\end{align}
Ce champ $\mq^*$ est le \textbf{champ moyen}, puisqu'il est le champ le plus probable. Dans cette approximation de champ moyen, toute observable est donnée par
\begin{align}
    < f(\mq) > = f(\mq^*)
\end{align}

On considère maintenant l'évolution temporelle du champ $\mq$ telle que son équation de Langevin donne la distribution à l'équilibre de Gibbs-Boltzmann : 
\begin{align}
    \frac{d q_i}{dt} = -L_{ij}\frac{\partial H(\mq)}{ \partial q_j} + \eta_i(t)
    \label{dynamique-langevin}
\end{align}
où $L_{ij}$ est un opérateur matriciel à définir et $\eta_i(t)$ un bruit blanc gaussien de fonction de corrélation
\begin{align}
   <  \eta_i(t)\eta_j(t') > =   \delta(t-t') \Gamma_{ij}
\end{align}
Le bruit blanc gaussien représente l'effef des fluctuations thermiques sur le système. On considère ici que le temps de corrélation de ces fluctuations est bien plus court que le temps charactéristique de l'évolution temporelle des degrés de liberté $q_i$, ce qui est de plus en plus vrai lorsque l'on se rapproche du point critique, à cause du ralentissement critique. Par symmétrie des fonctions de corrélation et de l'équation précédente nous pouvons en déduire que la matrice $\Gamma_{ij}$ doit être symmétrique et ne contenir que des valeurs propres positives.
À $T=0$, le système a tendance à minimiser son énergie, c'est-à-dire que
\begin{align}
    \frac{\partial H(\mq)}{ \partial q_j} = 0
\end{align}
Dans cette limite, l'équation \ref{dynamique-langevin} devient $\frac{d q_i}{dt}=0$, impliquant que le terme $\frac{\partial H(\mq)}{ \partial q_j}$ est le seul responsable de l'évolution du système.
Tant que la matrice $L_{ij}$ est inversible, la dynamique à $T=0$ fera tendre le système vers un minimum local de $H$, et vers son minimum global en absence de configurations métastables.
L'équation de Fokker-Planck de la densité de probabilité de la fonction des degrés de liberté $p(\mq,t)$ est 
\begin{align}
    \frac{\partial p(\mq,t)}{\partial t} = \frac{\partial}{\partial q_i} \left[\frac{1}{2}\Gamma_{ij}                 \frac{\partial p(\mq,t)}{\partial q_i} + p(\mq,t) L_{ij}\frac{\partial H(\mq)}{ \partial q_j}\right]
\end{align}
ou de manière plus concise
\begin{align}
    \frac{\partial p(\mq,t)}{\partial t} +\frac{\partial}{\partial q_i}J_i(\mq,t)=0
    \label{courant}
\end{align}
où  ${\bf J}(\mq,t)$ est le courant de probabilité. Le système respecte l'équilibre de Gibbs-Boltzmann si et seulement si $p(\mq,t)=P_{eq}(\mq)$ (donné par l'équation \ref{eqdis}) et que le courant soit nul, c'est-à-dire
\begin{align}
    \left[-\frac{\beta}{2}\Gamma_{ij} + L_{ij}\right]\frac{\partial H(\mq)}{ \partial q_j} = 0
\end{align}
Puisque cette relation est vraie quelque soit l'Hamiltonien considéré, on trouve  
\begin{align}
    \Gamma_{ij}= 2 k_B T L_{ij}
\end{align}
avec $k_B$ la constante de Boltzmann.

    \subsection{Théorie statistique des champs}

On considère maintenant un système d'Hamiltonien $H[\phi]$  dépendant d'un champ continu $\phi(\mx)$. Comme précédement, la fonction de partition est donnée par 
\begin{align}
    Z = \int d[\phi] e^{-\beta H[\phi]}
\end{align}
L'intégrale fonctionnelle sur tous les champs $\phi$ peut être prise dans la limite où $\phi$ est définie sur un réseau fini et où l'espacement entre chaque point tend vers $0$.
L'approximation du champ moyen devient maintenant
\begin{align}
Z _{MF}=  e^{-\beta H[\phi_{MF}]}
\end{align} 
où $\phi_{MF}$ est la solution de champ moyen qui minimise $H$, c'est-à-dire 
\begin{align}
    \frac{\delta H}{\delta\phi({\bf x})} = 0
\end{align}

Considérons maintenant l'Hamiltonien décrivant les modèles de type Ising
\begin{align}
    H[\phi] = \int d \mx  \frac{\kappa}{2}[\nabla \phi]^2 + V(\phi)
    \label{hamil-mean-field}
\end{align}
où le premier terme correspond à l'énergie d'interaction cherchant à diminuer les variations au sein du système, et le second terme est un potentiel symmétrique possédant deux minima globaux à basse température à $\pm \phi_C$ responsable de la séparation des phases, et possédant un minimum global à $\phi = 0$ à haute température. 

Par analogie avec le système avec un nombre fini de degré de libertés, on peut écrire l'équation de Langevin 
\begin{align}
    \frac{\partial \phi(\mx)}{\partial t}= -L \frac{\delta H}{\delta \phi(\mx)} + \eta(\mx,t).
\end{align}
avec la fonction de corrélation du bruit blanc gaussien
\begin{align}
    < \eta(\mx,t)\eta(\mx',t)> =\delta(t-t')\Gamma(\mx,\mx')
\end{align}
où  $L$ est un opérateur défini par son action sur la fonction $f$ comme
\begin{align}
    L f(\mx) = \int d\mx' L(\mx,\mx')f(\mx')
\end{align}
et de manière identique pour $\Gamma$.
De la même manière que précédement, on trouve que 
\begin{align} 
    \Gamma(\mx,\mx') =2 k_B T L(\mx,\mx')
\end{align}

Il est possible de choisir l'opérateur $L$ que l'on désire, puisque la distribution de Gibbs-Boltzmann à l'équilibre ne repose que sur la relation entre $L$ et $\Gamma$. 
Halperin et Hohenberg \cite{hohenberg_theory_1977} ont classifié les formes d'opérateurs les plus importants correspondant à des systèmes physiques.

La forme la plus simple est le \textbf{modèle A} donnée par $L(\mx,\mx')=\alpha\delta(\mx-\mx')$ 
\begin{align}
    \frac{\partial \phi(\mx)}{\partial t}= -\alpha \frac{\delta H}{\delta \phi(\mx)} + \eta(\mx,t)
    \label{MA}
\end{align}
de bruit blanc
\begin{align}
    < \eta(\mx,t)\eta(\mx',t)> =2T \alpha \delta(t-t')\delta(\mx-\mx').
\end{align}
On peut voir que la valeur moyenne $\overline \phi(t) = \frac{1}{V}\int d\mx \phi(\mx,t)$ n'est pas conservée. Le modèle A correspond alors à un système dans l'ensemble grand-canonique. Ici, le terme $\alpha$ est un coefficient cinétique décrivant le temps de relaxation du système. 
Le second modèle respectant les symmétries spatiales est le \textbf{modèle B}, et est donné par $L(\mx-\mx')= -D\nabla^2 \delta(\mx-{\bf x'})$ où le signe moins est nécessaire pour garantir la positivité de l'opérateur. On obtient l'équation d'évolution
\begin{align}
    \frac{\partial \phi(\mx)}{\partial t}= D\nabla^2 \frac{\delta H}{\delta \phi(\mx)} + \eta(\mx,t)
    \label{MB}
\end{align}
de bruit blanc
\begin{align}
    <\eta(\mx,t)\eta(\mx',t)> =-2TD   \delta(t-t')\nabla^2\delta(\mx-\mx')
\end{align}
En introduisant un bruit blanc vectoriel de composantes $\eta_i(\mx,t)$ tel que 
\begin{align}
    < \eta_i(\mx,t) \eta_i(\mx',t')> =\delta_{ij} \delta(\mx-\mx')\delta(t-t),
\end{align}
où $\delta_{ij}=1$ for $i=j$ et $0$ sinon, on peut maintenant écrire que 
\begin{align}
    \eta(\mx,t)= \nabla\cdot {\boldsymbol \eta}(\mx,t)
\end{align}
L'équation \ref{MB} devient 
\begin{align}
    \frac{\partial \phi(\mx)}{\partial t}= \nabla\cdot[ D\nabla \frac{\delta H}{\delta \phi(\mx)} + {\boldsymbol\eta}(\mx,t)]
\end{align}
Écrite sous cette forme, il est facile de voir que la valeur moyenne du paramètre d'ordre $\phi$ est conservé dans le temps. Le modèle B correspond à un système dans l'ensemble canonique, utile pour décrire les phénomènes de diffusion et d'accrétion.

À défaut de fluctuations thermiques, les équations \ref{MA} et \ref{MB} s'appellent respectivement les équations Ginzburg-Landau et de Cahn-Hillard \cite{cahn_free_nodate,langer_new_1975,kawasaki_growth_1978} et donnent l'évolution temporelle du champ moyen. 

    \subsection{Modèle $\phi^4$}
    
Le modèle standard, appelé $\phi^4$ est donné par le potentiel en double-puits de Landau-Ginzburg \cite[§ 45]{l_landau_physique_1990} 
\begin{align}
    V(\phi) = \frac{1}{2} m^2 \phi^2 + \frac{\lambda}{4!} \phi^4
    \label{phi4}
\end{align} 
où $m^2 = T-T_C$. À basse température, ce potentiel symmétrique possédant deux minima globaux à $\phi_C = \pm \sqrt{- \frac{6 m^2}{\lambda} } \pm$ décrit la séparation de phase.


Dans les expériences en laboratoire, les systèmes sont souvent couplés à des champs magnétiques ou chimiques $h(x)$ d'Hamiltonien
\begin{align}
    H_1 &= - \int d^dx h(\mx)\phi(\mx)
    \label{champ-externe}
\end{align}
qui induit un changement de stabilité entre les phases (voir Fig \ref{double-puits-temperature}). La nouvelle fonction de partition est
\begin{align}
    \mZ[h] = \int d [\phi] e^{ - \beta (\int d^d x \left( \frac{\kappa}{2} (\nabla \phi(\mx))^2 + V(\phi(\mx)) \right) + \beta \int d^d x h(\mx) \phi(\mx)}
\end{align}

\begin{figure}
    \centering
    \includegraphics[width=0.6\linewidth]{intro/double-puit-en-fonction-temp.pdf}
    \caption{Potentiel en double-puits pour $\lambda=1$ en fonction de la différence entre la température et la température critique. Dans la phase ordonnée, les mimina stables sont à $\phi_C =\pm \sqrt{- \frac{3! m^2}{\lambda} } $ et à $\phi_C = 0$ pour la phase désordonnée. En noir, l'ajout d'un champ magnétique uniforme $h(\mx) = 1$ rend la phase positive métastable. {\color{red} $m^2 = T-T_C$, ref à l'équation}}
    \label{double-puits-temperature}
\end{figure}

La valeur moyenne de $\phi$ est alors
\begin{align}
    <\phi> =  \frac{1}{\mZ[h]} \int d [\phi] \phi(\mx)e^{ - \beta (\int d^d x \left( \frac{\kappa}{2} (\nabla \phi(\mx))^2 + V(\phi(\mx)) \right) + \beta \int d^d x h(\mx) \phi(\mx)}
\end{align}
Plaçons-nous maintenant dans la phase désordonnée, où $m^2 \ge 0$ et $\lambda \simeq 0$, ce qui nous permet d'avoir une approximation gaussienne. Si on réécrit l'Hamiltonien sous forme d'opérateurs, on obtient 
\begin{align}
    H[\phi] = \frac{1}{2} \int d^d x d^d y \; \phi(\mx) \mL(\mx,\my) \phi(\my) - \int d^d x h(\mx) \phi(\mx)
\end{align}
où l'on a introduit l'opérateur $\mL(\mx,\my) = (m^2 - \kappa \nabla^2) \delta(\mx-\my)$.
La fonction de partition prend maintenant la forme gaussienne
\begin{align}
    \mZ[h] \propto e^{- \frac{\beta}{2} \int d^d x d^d y h(\mx) \mL^{-1}(\mx,\my) h(\my)}
\end{align}
où $\mL^{-1}$ est déterminé par 
\begin{align}
    (m^2 - \kappa \nabla_\my^2) \mL^{-1}(\mx,\my) = \delta(\mx-\my)
\end{align}
Par identification avec la fonction de Green $\Gamma(\mx)$ de l'opérateur $m^2 - \kappa \nabla^2$, on obtient que l'énergie libre du modèle gaussien est au final donné par 
\begin{align}
    F[h] = F_0 - \frac{1}{2} \int d^d x d^dy \;  h(\mx) \Gamma(\mx-\my) h(\my)
\end{align}

La transformée de Fourrier de l'équation de la fonction de Green $(m^2-\kappa \nabla^2) \Gamma(\mx) = \delta(\mx)$ donne
\begin{align}
    \tilde{\Gamma}(\mq) = \frac{1}{\kappa} \frac{1}{\xi^{-2} +  q^2}
\end{align}
puis
\begin{align}
    \Gamma(\mx) = \frac{1}{\kappa} \int \frac{d^dq}{(2\pi)^d} \frac{e^{i \mq \cdot \mx}}{\xi^{-2} +  q^2}
    \label{fonction-correl}
\end{align}
où l'on a introduit la longueur de corrélation du système $\xi = \sqrt{\frac{\kappa}{m^2}}$.
Par ailleurs, par différentiation fonctionnelle directe de la fonction de partition, on voit que
\begin{align}
    <\phi(\mx)> = \frac{1}{\mZ[h]} \frac{1}{\beta} \frac{\delta \mZ[H]}{\delta h(\mx)} = \frac{\delta (k_B T \ln \mZ[h])}{\delta h(\mx)} = - \frac{\delta F}{\delta h(\mx)} = \int d^d \my \Gamma(\mx-\my) h(\my)
\end{align} 
avec l'énergie libre du système  $F[h] = -k_B T \ln(\mZ[h])$. Ce résultat n'est valable que pour $h \neq 0$ ou $h \to 0$. En absence de champ extérieur, la magnétisation est nulle.
De la même manière, on peut démontrer que la fonction de corrélation est égale à 
\begin{align}
    C(\mx,{\bf y}) =  <\phi(x)\phi(y)> = \frac{1}{\beta} \frac{\delta <\phi(\mx)>}{\delta h({\bf y})} = \frac{1}{\beta} \Gamma(\mx-\my)
\end{align}
et le facteur de structure
\begin{align}
    S(k) = < \tilde{\phi}(k)\tilde{\phi}(q)> = \frac{(2\pi)^d}{\beta} \delta(k-q)  \tilde{\Gamma}(\mq)
\end{align}
{\color{red} vecteurs k et q}
Dans la phase désordonnée, où $m^2 \less 0$ et $\lambda \neq 0$, l'approximation gaussienne du modèle $\phi^4$ donne le même résultat que \ref{fonction-correl} avec le facteur $m^2$ renormalisé par l'équation auto-consistante \cite[\P 4.3]{bellac_equilibrium_2004}
\begin{align}
    m^2_0 = m^2 + \frac{1}{2} \lambda \int_\mq \frac{1}{m^2_0+\mq^2}
\end{align}
    \subsection{Tension superficielle}

La solution de champ moyen minimise l'Hamiltonien \ref{hamil-mean-field},  nous donne
\begin{align}
    \frac{\delta H}{\delta \phi(\mx)} = -\kappa \nabla^2 \phi(\mx) + V'(\phi)
    \label{interface}
\end{align}
Puisque le potentiel $V(\phi)$ est supposé symmétrique et possédant deux minima globaux en $\pm \phi_C$, en l'absence de contrainte, le système tend vers la solution homogène $\phi(\mx) = \pm \phi_C$ correspondant à l'énergie libre $F=H[\phi_C]=0$. Néanmoins, dans le cas où le paramètre d'ordre est conservé
\begin{align}
    \int d \mx \phi(\mx)=0
\end{align}
la solution $\phi(\mx)=\pm \phi_c$ est impossible. Dans ce cas, le système va se séparer en plusieurs phases homogènes $\phi(\mx)= \pm \phi_c$. 
Plaçons-nous au voisinnage de l'interface entre deux phases, c'est-à-dire que le champ $\phi$ est invariant par translation en $x$ et $y$ et que la longueur de corrélation $\xi$ est bien plus grande que la taille de l'interface, c'est-à-dire que $\phi(\mx) = \phi_K(z)$ (où $K$ désigne un \textit{kink}) et $\lim_{z\to\-\infty}=-\phi_c$ and  $\lim_{z\to\infty}=-\phi_c$, ce qui nous donne d'après \ref{interface}
\begin{align}
    \kappa \phi_K''(z) =  V'(\phi_K)
    \label{kink}
\end{align}
En multipliant de chaque côté par $\phi'_K(z)$ et en utilisant les conditions aux limites, on trouve que 
\begin{align}
    H[\phi_K]=  A\int dz\ \kappa \phi_K'^2(z)
\end{align}
où $A$ est l'aire de la surface du système dans le plan perpendiculaire à la direction $z$. On peut identifier l'intégrale à une énergie libre par unité de surface, c'est à dire la tension superficielle de l'interface $\sigma$ définie par l'équation d'Allen-Cahn
\begin{align}
    \sigma=  \int dz\ \kappa \phi_K'^2(z)
    \label{tension-superficielle}
\end{align}
Il s'ensuit que l'excès d'énergie est localisé au niveau de l'interface, et que la force principale de la croissance des domaines est la courbure du profil de l'interface, puisque l'énergie du système ne peut diminuer que par une réduction de l'aire totale de l'interface. 

Dans le cas du modèle $\phi^4$ définie à l'équation \ref{phi4}, l'équation \ref{kink} devient
\begin{align}
    \kappa \phi_K''(z) = \phi_K(z) \left( m^2 + \frac{\lambda}{3!} \phi_K(z) ^2 \right)
       \label{eq-interface-glauber}
\end{align}
dont la solution est
\begin{align}
    \phi_K(z) = \phi_C \tanh \left( \frac{z}{\sqrt{\frac{-m^2}{2 \kappa}}} \right)
       \label{profil-interface-glauber}    
\end{align}


%%%%%%%%%%%%%%%%%%%%%%%%%%%%%%%%%
    \section{Taille finie et effet Casimir critique}
    \label{sec-casimir}    
%%%%%%%%%%%%%%%%%%%%%%%%%%%%%%%%%
    
Supposons un système 2D de taille $L \times L' $ où $L \less L'$. L'énergie libre $F(\beta,L,L') = - \frac{1}{\beta} \ln ( Z(\beta,L,L'))$ est une grandeur extensive lorsque la longueur de corrélation est plus petite que la taille du système $L$. 
Cette énergie libre peut se décomposer entre l'énergie de chaque phase par unité de volume $\omega_{bulk}$, et l'énergie de tension superficielle à l'interface par unité de surface $\omega_{surf}$ \cite{cardozo_finite_2015,lopes_cardozo_critical_2014}. À noter qu'à haute température dans un système complètement homogène, ce dernier terme disparaît.

Cependant, lorsque $\xi \simeq L$, la contrainte exercée sur les fluctuations thermiques par les conditions aux bords implique une modification de l'énergie libre, créant une force sur les parois. Cet effet, premièrement prédit par Hendrik Casimir\cite{h_b_g_casimir_attraction_1948}, fut étendu aux systèmes critiques, où la divergence de la longueur de corrélation rend les expériences bien plus faciles\cite{nguyen_controlling_2013}.
{\color{red} réf originale de gennes fischer, ref 15 cardozo, review gambassi}
{\color{red} résumé notes David Casimir}

En présence d'un champ magnétique $h$ uniforme favorisant une phase par rapport à l'autre, l'énergie libre par unité d'aire d'un tel système se décompose \cite{lopes_cardozo_critical_2014,cardozo_finite_2015} en 
\begin{align}
    \frac{\Omega(\beta,L,L',h)}{L'} = L \omega_{bulk}(\beta,h) + \omega_{surf}(\beta,h) + L \omega_{ex}(\beta,L,h)
    \label{decomposition-energie}
\end{align}
où $\omega_{ex}(L,h)$ est le surplus d'énergie libre due au confinement des fluctuations, qui devient nul dans la limite $L\to \infty$.

La force de confinement par unité d'aire est définie par 
\begin{align}
    F_\perp(\beta,L,h) = - \frac{1}{L' }\frac{\partial \Omega(\beta,L,h)}{\partial L} \bigg|_{\beta,L'} = - k_B T \omega_{bulk}(\beta,h) - k_B T \frac{\partial(L \omega_{ex}(\beta,L,h))}{\partial L}\bigg|_{\beta,L'}
\end{align}
où le premier terme est la pression exercée par le système, tandis que le second terme est la force de Casimir par unité d'aire \cite{vasilyev_critical_2013} en $d$ dimensions 
\begin{align}
    f_c(\beta,L,h) = - k_B T \frac{\partial(L \omega_{ex}(\beta,L,h))}{\partial L}\bigg|_{\beta,L'} = k_B T L^{-d} \Theta(u_t,u_h)
    \label{casimir-scaling}
\end{align}
où $u_T = \frac{T-T_C}{T_C} L^\frac{1}{\nu}$ et $u_h = \frac{h}{k_B T_C} L^\frac{\beta+\gamma}{\nu}$ et où les exposants $\beta$, $\gamma$ et $\nu$ sont des exposants universels reliés aux amplitudes universelles des longueurs de corrélation du système \cite{pelissetto_critical_2002,vasilyev_critical_2013} et $\Theta(u_t,u_h)$ est une fonction universelle propre à chaque modèle. Cette fonction universelle dépend des conditions aux bords du système \cite{dantchev_casimir_2017,dantchev_exact_2016} ainsi que de l'ensemble thermodynamique dans lequel on se place \cite{gross_critical_2016,rohwer_transient_2017}.

Afin d'extraire la force de Casimir, il suffit alors de soustraire deux quantités extensives, c'est-à-dire en utilisant deux largeurs différentes $L_1$ et $L_2$
\begin{align}
    f_c(\beta,L_1,h) - f_c(\beta,L_2,h) =  \frac{1}{L' }\frac{\partial \Omega(\beta,L_2,h)}{\partial L} \bigg|_{\beta,L'} -  \frac{1}{L' }\frac{\partial \Omega(\beta,L_1,h)}{\partial L} \bigg|_{\beta,L'}
\end{align}
Puisque le surplus d'énergie dû au confinement est nul lorsque $L_2\to \infty$, on obtient que la force de Casimir est
\begin{align}
    f_c(\beta,L_1,h) \simeq \frac{1}{L' }\frac{\partial \Omega(\beta,L_2,h)}{\partial L} \bigg|_{\beta,L'} -  \frac{1}{L' }\frac{\partial \Omega(\beta,L_1,h)}{\partial L} \bigg|_{\beta,L'}
    \label{casimir-diff-omega}   
\end{align}
où en utilisant \ref{casimir-scaling}, l'approximation est valable dans le cas où $ \left( \frac{L_2}{L_1}\right)^{-d} \ll 1 $.  Cette force étant une force émergente d'origine entropique, la somme des forces exercées individuellement par chaque particule du système n'est pas égale à la force totale appliquée sur le système\cite{paladugu_nonadditivity_2016}.

La détection expérimentale de ce phénomène se fait traditionnellement dans des fluides binaires par \textit{Total Internal Reflection Microscopy} (TIRM) \cite{fukuto_critical_2005,hertlein_direct_2008,gambassi_critical_2009,edison_critical_2015-1}. La méthode consiste à mesurer le potentiel d'une sphère flottant sur un fluide binaire critique reposant sur une plaque. Cette sphère et cette plaque sont traitées chimiquement afin de favoriser l'une des deux phases à leu voisinage. Ainsi il est possible de créer des conditions aux bords $(++)$, $(+-)$ ou $(--)$ qui modifient la forme de la fonction universelle \ref{casimir-scaling}. À l'inverse, il est possible de mesurer la force de Casimir \cite{nguyen_controlling_2013} afin d'étudier les transitions de phases colloïdales.

Le modèle d'Ising appartient à une classe de modèles où il est facilitant l'obtention de résultat analytique \cite{hobrecht_critical_2017} comparables aux simulations numériques \cite{vasilyev_monte_2007,vasilyev_universal_2009,cardozo_finite_2015}.

Jusqu'à présent nous avons parlé de systèmes à l'équilibre, mais l'effet des fluctuations est également présent en dehors de l'équilibre. L'étude des interfaces nous mène à proposer des modèles de cisaillement qui influencent les propriétés statistiques des systèmes, et ainsi modifier la force de Casimir \cite{dean_out--equilibrium_2010}.

    \section{Modèles d'interface}

Dans la réalité, la phase désordonnée est extrêmement inhomogène, avec des bulles ou des digitations qui empêchent une description dynamique aisée de l'interface. Si l'on désire étudier l'interface de ces bulles ou digitations, où localement l'interface est bien définie par une fonction d'une seule variable, l'approche du champ moyen suffit. On suppose dans ce cas qu'il n'y a ni digitation ni bulles d'évaporation. Dans cette approximation, l'interface est parfaitement définie en un point et non dans un profil comme dans l'équation \ref{profil-interface-glauber}. Tous les points du champ se trouvant en bas de l'interface prennent une unique valeur strictement différente de tous les points du champ au-dessus de l'interface. 

Sans perte de généralité, nous pouvons séparer les variables spatiales par $x$ pour toutes les coordonnées parallèles à l'interface et par $z$ la coordonnée transverse. Cela se traduit par
\begin{align}
    \phi(\mx,z) = f(z-h(\mx))
    \label{capillary-wave-theory}
\end{align}
où $f(a \greater 0) = \phi_1$ et $f(a \less 0) = \phi_2$. Notre système est maintenant complètement défini par l'interface $h(x)$ d'Hamiltonien
{\color{red} $\phi$ miminise champ moyen }
\begin{align}
    H = \int d^d x \frac{\sigma}{2} (\nabla h(\mx))^2 + V(h(\mx))
    \label{hamil-cwt}
\end{align}
{\color{red} expliquer développement pour surface dans formule de Mange}

où le premier terme est l'excès d'énergie par rapport à une interface plane, et le potentiel $V$ fait référence au champ externe \ref{champ-externe}. 
Une interface se caractérise par sa hauteur moyenne $<h(t)>$ de l'interface dans l'espace et par sa fonction de corrélation parallèle à l'interface décrivant les modes de fluctuation de l'interface
\begin{align}
    C_\parallel(r,t) = <h(\mx,t)h(\mx+r,t)>_x - <h(0,t)>^2 = \sum_i A_i(\frac{r}{\xi_i}) 
\end{align}
où les $A_i$ sont des fonctions à décroissance exponentielle. Le calcul de ces fonctions sera donnée dans la section \ref{sec_laser}. 
L'épaisseur de l'interface est donnée par $\omega(t) = \sqrt{C_\parallel(0,t)} = \sqrt{<h(t)^2> - <h(t)>^2}$. 

    \subsection{Paramètre d'ordre non conservé}

Supposons une surface à laquelle viennent s'agréger des particules provenant d'un réservoir afin de créer un dépot. L'interface est alors définie par la hauteur de l'aggrégat par rapport à la surface de dépôt.

En partant de \ref{MA} et en insérant \ref{capillary-wave-theory} avec le changement de variable $u= z-h$, on a \cite{bray_interface_2001}
\begin{align}
    \frac{\partial h}{\partial t} f'(u) &= D \nabla^2 h f'(u) - V'(f) + \eta(\mx,u+h(\mx,t),t)
\end{align}
avec $\eta(x,t)$ un bruit blanc gaussien. En multipliant les deux côtés par $f'(u)$ et en intégrant de $-\infty$ à $+\infty$, puisque le terme $ \int_{-\infty}^\infty V'(f) f'(u) du = 0$, on obtient l'équation d'Edwards-Wilkinson \cite{edwards_surface_1982} 
\begin{align}
     \frac{\partial h}{\partial t} = \nu + \nabla^2 h +  \tilde{\eta}(\mx,t)
    \label{edwards-wilkinson}
\end{align}
{\color{red} ne pas mettre $\nu$, on s'emballe ??? gné ? voir notes de David. Éliminer la section, rmeplacer avec la dérivation dans les notes}
où $\nu$ désigne le flux total, et $\tilde{\eta}(x,t)$  un bruit blanc de moyenne nulle défini par
\begin{align}
    \tilde{\eta}(\mx,t) = - \frac{1}{\sigma} \int du f'(u) \eta(\mx,u+h(\mx,t),t)
\end{align}
et de corrélation 
\begin{align}
    <\tilde{\eta}(\mx,t)\tilde{\eta}(\mx',t')> = \frac{2 D}{\sigma} T\delta(\mx-\mx')\delta(t-t')
\end{align}
avec $\sigma$ la tension superficielle définie en \ref{tension-superficielle}, et $D$ le bruit thermique.
Ici $D+ \sqrt{2 D T} \tilde{\eta}(x,t)$ est le flux de particules s'aggrégeant en fonction du temps et $D \nabla^2 h$ dépend de la forme de l'interface, favorisant ou non le dépôt de particules à certains endroits.
La hauteur moyenne de l'interface varie donc comme $<h(t)> = \nu t$. En se positionant dans le référentiel de l'interface via la transformation $h \rightarrow h + \nu t$, on obtient
\begin{align}
     \frac{\partial h}{\partial t} =    \nabla^2 h +  \tilde{\eta}(\mx,t)
    \label{edwards-wilkinson-conesrved}
\end{align}

    \subsection{Paramètre d'ordre conservé}

Si l'on se place dans le référentiel du centre de l'interface, le'équation d'Edwards-Wilkinson \ref{edwards-wilkinson-conesrved} conserve le paramètre d'ordre en moyenne. Néanmoins, la transformation $h \rightarrow h + Dt$ ne prend pas en compte les fluctuations thermiques qui viennent perturber l'interface, ce qui fait que $h(t) \neq cte$. L'astuce vient ici de \cite{kawasaki_diffusion_1966,kawasaki_correlation-function_1966}, où l'on reprend l'équation de Cahn-Hilliard-Cook \ref{MB} pour les interfaces
\begin{align}
    \frac{\partial h}{\partial t} =  \nabla^2 \frac{\delta H}{\delta h} +  \tilde{\eta}(x,t)
\end{align}
{\color{red}$ H =\int d \mx \frac{\sigma (\nabla h)^2}{2} +V(h(x) $}
{\color{red} version modèle de KPZ, pas le moême modèle. Enlever $\nabla^2 (\nabla h)^2$}
qui nous donne l'équation Villain-Lai-Das Sarma\cite{villain_continuum_1991,lai_kinetic_1991}
\begin{align}
    \frac{\partial h}{\partial t} = - \nabla^4 h + \lambda \nabla^2 (\nabla h) ^2 +  \tilde{\eta}(x,t)
\end{align}
qui a fait l'objet de nombreuses études\cite{kim_conserved_1994,assis_dynamic_2015,oliveira_maximal-_2008,singha_renormalization_2016}.
L'intérêt d'un tel système est qu'il est contraint à une dynamique locale qui permet d'obtenir un système hors-équilibre. 

    \subsection{Interface hors équilibre}

Les systèmes à l'équilibre ont la particularité que les états ont une probabilité en accord avec la distribution de Gibbs-Boltzmann \ref{eqdis}, c'est-à-dire que le courant \ref{courant} est nul. La manière la plus simple de mettre le système hors-équilibre est donc d'induire un courant dans le champ $\phi(\mx,t)$. L'évolution d'un système depuis une condition initiale vers ses configurations d'équilibre est le moyen le plus simple d'étudier les conditions hors équilibre. On retrouve également dans le régime stationnaire beaucoup de systèmes hors-équilibre. Expérimentalement, les colloïdes sédimentant dans un champ gravitationel induisent un écoulement hors-équilibre. Il est également possible d'induire un flux sur des particules chargées dans un champ électrique via la pression de radiation exercée par un laser ou par un cisaillement dans un liquide visqueux\cite{girot_conical_2019}. Cette dynamique étant locale, elle ne peut exister que si le paramètre d'ordre est conservé. L'équation générale d'un système d'interface avec un cisaillement est\cite{bray_interface_2001-1,bray_interface_2001}
\begin{align}
     \frac{\partial h}{\partial t} + v \nabla h =  \mathcal{L} h +  \eta(x,t)
     \label{eq-cisaillement}
\end{align}
{\color{red} $v \nabla h$ are vectors scalar product $\nabla (vh)$ vecteur advection du à un écoulement. Mettre la formule pour un cisaillement, article Bray. Dire que champ de vitesse est constant pour faire transfo galiliéenne, écrire qu'est-ce que c'est que le cisaillement. 
Champ de cisaillement c'est à revoir}
où l'opérateur $\mathcal{L}$ est associé au modèle A ou B, et le terme $v \nabla h$ est un terme de d'advection dû au flux produit par le cisaillement. 

Dans le cas où le champ de vitesse est constant, comme la sédimentation dans un champ gravitationel, on peut parler d'écoulement. L'équation \ref{eq-cisaillement} est alors invariante par la transformation galiléenne $x \rightarrow x+vt$. 

\begin{figure}
	\begin{minipage}[t]{0.5\linewidth}
		\includegraphics[width=\linewidth]{intro/cis-ising-f-000.pdf}
	\end{minipage}%
	\begin{minipage}[t]{0.5\linewidth}
		\includegraphics[width=\linewidth]{intro/cis-ising-f-033.pdf}
	\end{minipage}
	\centering
	\begin{minipage}[t]{0.5\linewidth}
		\includegraphics[width=\linewidth]{intro/cis-ising-f-066.pdf}
	\end{minipage}
	\caption{Photos d'un système d'Ising $128 \times 50$ en fonction d'un cisaillement
	${f(y) = \omega (\frac{2 y}{L} -1)}$.  {\color{red} expliquer qu'est-ce qu'un cisaillement, comment on implémente ça dans un modèle discret}}
    \label{snap-ising-shear}	
\end{figure}  

\begin{figure}
    \centering
    \includegraphics[width=0.6\linewidth]{intro/profil-mag-ising-shear.pdf}
    \caption{Profil de magnétisation d'un système d'Ising $128 \times 50$ en fonction du cisaillement de la figure \ref{snap-ising-shear}. Plus le cisaillement est élevé, plus l'interface est confinée.}
    \label{profil-mag-ising-shear}
\end{figure}

Néanmoins, de nombreuses expériences\cite{derks_suppression_2006} et simulations numériques pour le modèle d'Ising \cite{leung_field_1986,rikvold_microstructure_2002,gonnella_nonequilibrium_2009,smith_driven_2010,smith_interfaces_2008,sadhu_non-local_2014,cohen_interface_2016,cirillo_monte_2005} ainsi qu'avec des techniques de Dynamique Moléculaire \cite{berthier_nonequilibrium_2002} montrent que le cisaillement provoque un confinement de l'interface. Dans les expériences où les effets gravitationnels sont importants, le solvant d'une suspension de colloïdes ne bouge pas, contrairement aux particules en suspension. La brisure d'invariance galiléenne qui en résulte est expliqué en détail au chapitre \ref{chap-article-dean}. \footnote{Ce chapitre a été publié dans \cite{dean_effect_2020}.}
Pour les simulations de Monte Carlo, l'invariance est brisée par la dynamique même du système, puisque les mouvements sont fait séquentiellement et non simultanément.

\section{Conclusion}

Nous avons présenté les modèles standards des transitions de phase et avons étudié l'importance de l'ensemble thermodynamique de ces systèmes.
Dans l'ensemble grand-canonique où le paramètre d'odre n'est pas conservé, les équations dynamiques du modèle A \ref{MA} appliqués au champ $\phi(\mx,t)$ permettent de calculer la fonction de corrélation du système pour le modèle $\phi^4$. Le modèle $\phi^4$ permet de basculer naturellement vers un modèle d'interface $h(\mx,t)$ qui réduit la dimensionalité du champ et rend ainsi les calculs plus simples.
	Il existe par ailleurs de nombreuses sources contribuant à l'énergie totale du système : l'énergie du volume (\textit{bulk}) du système, l'énergie de l'interface définie par sa tension superficielle \ref{tension-superficielle}, et une énergie d'excès qui donne naissance à la force de Casimir \ref{casimir-scaling}. Cette force est dûe au confinement du système par des conditions aux bords contraignant les fluctuations du champ $\phi$ selon une direction. 


Dans l'esemble canonique où le paramètre d'ordre est conservé (ou modèle B \ref{MB}), toutes les considérations antérieures se s'appliquent. Il est possible d'appliquer un flux local qui sort le système de l'équilibre. La manière la plus naturelle de le faire est de cisailler l'interface, ce qui modifie également les propriétés statistiques du système. 

Nous nous intéressons maintenant aux modèles sur réseaux, qui présentent l'avantage de la simplicité numérique et analytique et permettent de modifier facilement l'ensemble thermodynamique ainsi que le cisaillement. 
\chapter{Modèle Solid-On-Solid}
		
		\section{Hamiltonien}
	
Quelle que soit la méthode utilisée, le système se simplifie dès lors que nous désirons étudier uniquement l'interface d'un système et non son ensemble, telles que les longueurs de corrélations dans le \textit{bulk}, l'aimantation moyenne, la chaleur spécifique ou la susceptilibté magnétique. À très basse température, les interfaces sont bien délimitées et il y a très peu de gouttes d'évaporations d'une phase dans l'autre. En considérant le système très peu mélangé, il est possible de définir la présence d'une phase par rapport à la hauteur $h_i$ de l'interface. Chaque spin prend la valeur
\begin{align*}
	\sigma_{i,j} = \sgn(h_i-j)
\end{align*}
où la fonction $\sgn(x)$ est égale à $+1$ si $x>=0$ et à $-1$ sinon. Cela revient à considérer que l'énergie d'interaction perpendiculaire est prohibitif par rapport aux liaisons parallèles à l'interface $J_\perp \gg J_\parallel$. 

\begin{figure}
	\centering
	\includegraphics[scale=1]{isingtosos/sos-indiscernable.pdf}
	\caption{Une configuration possible de modèle SOS. Dans la i-ème colonne le bord horizontal de l'interface passe à la hauteur $h_i$. Toutes les particules au-dessus de l'interface sont des spins positifs et négatifs en dessous. La représentation classique du modèle SOS diffère de ce schéma par l'hypothèse que les particules sont discernables. Nous y reviendrons plus tard.}
\end{figure}


En utilisant l'identité $\min(a,b)-\max(a,b)=|a-b|$, on a
\begin{align*}
    \sum_{j=0}^L \sgn(h-j)\sgn(h'-j) = L - 2 |h-h'|
\end{align*}
Ainsi, pour un système de longueur $L_X$ et de largeur $L_Y$ l'hamiltonien du modèle d'Ising en absence de potentiel se réécrit comme 
\begin{align}
    H = 4 J L_X (2-L_Y) +4J \sum_i |h_i-h_{i+1}|
\end{align}
Le terme $|h_i-h_{i+1}|$ représente la surface de contact horizontale entre les deux phases qui dépend directement de la hauteur, tandis que le terme constant représente la surface de contact verticale.
En retirant la partie constante de l'énergie et simplifiant $4 J = J$ et en gardant $L_X = L$, nous obtenons l'hamiltonien du \textbf{modèle Solid-On-Solid}
\begin{align}
    H = J \sum_i |h_i-h_{i+1}|
    \label{hamil-sos}
\end{align}

Nous avons ici réussi à réduire la dimensionalité du système en ne prenant en compte que la hauteur $h_i$ au site $i$. L'énergie du système est alors décrite par la différence de hauteur entre deux sites voisins, et $\beta J$ prend ici alors le rôle de la tension de surface.

Puisqu'il n'y a plus de transition de phase possible dans ce modèle, la température critique $\beta_C$ du modèle d'Ising n'a plus de rôle à jouer ici. Pour rester dans le domaine de l'approximation, si nous désirons comparer les résultats avec ceux du modèle d'Ising, il faudra veiller à rester dans le domaine $T \less T_C$, l'approximation étant de plus en plus valable plus la température est basse. Par la suite, puisque nous désironts étudier le modèle SOS et non le comparer avec le modèle d'Ising, nous prendrons, sauf cas contraire explicité, $\beta = \beta_C \simeq 0.44$ et $J=1$ par soucis de simplicité. 

  \section{Matrice de Transfert}

	De manière plus générale, l'Hamiltonien d'un système avec des interactions entre les particules peut se réécrire comme $H = \sum_{\langle ij >} H(i,j)$ avec
\begin{align*}
  H(h_i,h_{i+1}) = f(h_i,h_{i+1}) + V(h_i,h_{i+1}) 
\end{align*}
où $f(h_i,h_j)$ est l'énergie d'interaction entre plus proches voisins et $V(h_i,h_j)=\frac{V(h_i)+V(h_j)}{2}$ le potentiel symmétrisé.
La fonction de partition de notre système s'écrit alors 

\begin{align*}
 Z = \sum_{h_1 h_2 ... h_L} e^{- \beta \sum_{i} H(h_i,h_{i+1})}  
   = \sum_{h_1 h_2 ... h_L} \prod_{i} e^{-\beta H(h_i,h_{i+1})} 
\end{align*}
La matrice $T(i,j) = e{-\beta H(i,j)}$ est appelée matrice de transfert. Cette matrice est périodique aux bords, c'est-à-dire $T(L,L+1) = T(L,1)$ et est symétrique, ce qui implique qu'elle est diagonalisable dans la base des vecteurs propres $|\lambda >$ de valeur propre $\lambda$. On dénote par $\lambda_0$ la plus grande valeur propre de $T$, -par $\lambda_1$ la deuxième plus grande valeur propre et ainsi de suite.
Ainsi la fonction de partition devient\cite{}
\begin{align}
  Z = \sum_{\sigma_1 \sigma_2 ... \sigma_{L}} \prod_{i} T(i,i+i) = Tr T^L  = \sum_\lambda \langle\lambda | T^L | \lambda> = \sum_\lambda \lambda^L
\end{align}

Dans la limite thermodynamique $L \to \infty$, seuls les plus grands vecteurs propres jouent un rôle. Afin de calculer les observables de notre système, il convient d'introduire la matrice des hauteurs $\tilde{M}(i,j) = \delta_{ij} i$. Nous avons donc \footnote{Pour une démonstration plus détaillée sur les matrices de transfert, se référer à \cite{matrice_transfert}}
\begin{itemize}
	\item L'énergie libre par site :  
	\begin{align}
		F =  - \frac{1}{L \beta} \ln(Z) \simeq - \frac{1}{L \beta } \ln( \lambda_0)
	\end{align}
	\item La densité de probabilité qu'un site se trouve à la hauteur $h$ : 
	\begin{align}
		p(h) = \frac{1}{Z} \sum_\lambda \lambda^L \langle\lambda | h >^2 \simeq \langle \lambda_0 | h >^2
	\end{align}
	\item La magnétisation moyenne :
	\begin{align}
		M = \langle h > = \langle \lambda_0 | \tilde{M} | \lambda_0 > 
	\end{align}
	\item La variance des hauteurs :
	\begin{align}
		\sigma = \langle (h - \langle h >)^2 > =  \langle \lambda_0 | \tilde{M}^2 | \lambda_0 >
	\end{align}
\end{itemize}

	\section{Stabilité de l'interface}

	Soit $\psi_\lambda(h)$ la projection du vecteur propre associé à la valeur propre $\lambda$ de la matrice de transfert sur la base des hauteurs dans un système infini de par et d'autre de l'interface. En absence de potentiel\cite{guyer1979}, l'équation du vecteur propre donne
\begin{align}
	\sum_{h=-\infty}^\infty T(h,h') \psi_\lambda(h) = \lambda \psi_\lambda(h')
\end{align}
En introduisant l'ersatz $\psi_\lambda(h) = \alpha^h$ qui respecte la symétrie du système, et en séparant de la somme les termes pour $h$ négatifs et positifs, on trouve aisément que 
\begin{align}
	\frac{\sinh(\beta J)}{\cosh(\beta J)-(\alpha+\frac{1}{\alpha})} \alpha^{h'} = \alpha^{h'} \lambda
\end{align}
Dans la limite thermodynamique, la probabilité de présence de l'interface à la hauteur $h$ est $p(h) = <\lambda_0|h>^2 = |\psi_0(h)|^2$. Le système ne possédant aucune brisure de symétrie particulière, la probabilité $p(h)$ se doit d'être bornée pour tout $h$. Dès lors, l'ersatz supposé $\psi_\lambda(h) = \alpha^h$ implique que $\alpha$ soit de la forme $e^{ik}$ où $k$ est la longueur d'onde associée à la valeur propre $\lambda$. On obtient que 
\begin{align}
	\psi_k(h) =& e^{ikh} \\
	\lambda_k =& \frac{\sinh(\beta J)}{\cosh(\beta J) - \cos(k)}
\end{align}


Dans le cas plus général où l'hamiltonien s'écrit de la forme $T(h,h') = f(|h-h'|)$, on trouve 
\begin{align}
	\lambda_k = \sum_{h=0}^\infty f(h)(\alpha^h+\alpha^{-h}) - f(0)
\end{align}

L'existence d'une solution de ce genre indique que l'interface n'est pas localisée dans le cas d'un système infini (ou semi-infini) en absence de tout potentiel, ce qui conduit à de nombreux problèmes numériques. 

Il est à noter qu'à $\beta=0$, c'est-à-dire pour une température infinie, la matrice de transfert est uniformément égale à $1$, menant à des vecteurs propres nuls. Dans cette limite, l'interface n'existe plus, le modèle SOS n'est donc pas valable. De même, pour une température nulle $\beta=\infty$, la matrice de transfert devient la matrice identité. Les valeurs propres deviennent toutes égales à $1$ et les vecteurs propres sont $\psi_i(h) = \delta_{h,i}$ où ici $i$ est l'indice de la i-ème valeur propre $\lambda_i = 1$. La probabilité de trouver l'interface à la hauteur $h$ devient $p(h) = \frac{1}{Z}\sum_{i} <\lambda_i | h >^2 = 1$. La température nulle a pour effet de geler l'interface sur une seule hauteur, mais toutes les hauteurs sont équiprobables. Bien que les micro-états soient extrêmement différents que pour une température finie, les propriétés macroscopiques sont identiques à cause du même poids statistique associé à chaque état.


Historiquement, une manière facile de localiser l'interface est de rajouter un potentiel $V(h) = -B \delta_{h,0}$ \cite{chui}. La présence du potentiel n'affecte pas la parité du système mais peut introduire un surplus de particules en $0$. La recherche d'un état lié nous donne un ersatz de la forme 
\begin{align}
	\psi_\lambda(h) = \begin{cases} |\alpha|^h & \text{si } h \neq 0 \\ \psi_{\lambda,0} & \text{sinon} \end{cases} 
\end{align}
L'équation du vecteur propre devient
\begin{align}
	\sum_{h=-\infty}^\infty e^{\beta |h-h'|- \beta B \delta_{h,0}} \psi_\lambda(h) = \lambda \psi_\lambda(h')
\end{align}
En notant $T(h,h') = R^{|h-h'|}$ pour $h \neq h' \neq 0$,  on obtient la même équation à un signe près dans l'exposant que l'on soit à $h'>0$ ou $h'>0$
\begin{align}
	\left( \frac{R}{\alpha} \right)^{\pm h'} \left[ \psi_{\lambda,0} + \frac{R \alpha}{1 - R \alpha} + \frac{\alpha}{R - \alpha} \right] + \left[ \frac{1}{1-R \alpha} - \frac{R}{R-\alpha} \right] = \lambda
\end{align}
Puisque cette équation est vraie pour tout $h'$, le premier terme doit être nul, ce qui nous donne
\begin{align}
	\psi_{\lambda,0} &= - \frac{\alpha}{R-\alpha}-\frac{R \alpha}{1-R \alpha} \\
	\lambda &= \frac{1}{1-R \alpha} - \frac{R}{R-\alpha}
\end{align}
L'équation du vecteur propre à $h'=0$ nous donne par ailleurs 
\begin{align}
	\psi_{\lambda,0} + 2 \frac{R \alpha}{1-R \alpha} = \lambda \psi_{\lambda,0} e^{-\beta B}
\end{align}
L'existence d'une solution cohérente $\alpha < 1$ autorise la présence d'une interface localisée grâce au pinning.

D'autres méthodes existent pour confiner l'interface. Le cisaillement d'une interface diminue sa largeur et permet de la localiser dans l'espace. On peut également proposer deux potentiels chimiques différents pour chaque phase à une hauteur de l'interface prédéfinie, comme le ferait un laser dans les expériences de cisaillement\cite{delville} dans un système semi-infini. Cet cas sera étudié plus loin. Dans un système infini, une autre possibilité est de définir un champ magnétique symétrique rendant plus difficile la présence de l'interface loin de $0$. Nous utiliserons ici un potentiel du style
\begin{align}
		  V(h) = B |h|
\end{align}

Il est facile de se convaincre que loin de $0$ le coût énergétique est si grand que la probabilité que l'interface s'y trouve soit petite, impliquant que l'interface est localisée. La position moyenne de l'interface se situe au minimum du potentiel qui est dans ce cas $0$. 


	\section{Ensemble canonique}



\begin{figure}[h]
	\centering
	\includegraphics[scale=1]{isingtosos/figure-canonique.pdf}
	\caption{Dans un modèle d'Ising à 1D, afin d'avoir une magnétisation moyenne du système à $<M>=2$, tous les états sont acceptés tant qu'il y en a d'autres afin de respecter la moyenne. Dans l'ensemble canonique, on n'a plus $<M>=2$ mais $M=2$, interdisant les micro-états rouges.}
\end{figure}

Dans l'ensemble grand-canonique, le nombre de particules dans le système varie, dépendant du potentiel chimique vis-à-vis du réservoir dans lequel il est inséré, ce qui permet à l'interface de bouger librement. Lorsque l'on se place dans un système canonique, le nombre de particules (c'est-à-dire la magnétisation totale $M$) est fixe, ce qui introduit une contrainte dans la fonction de partition
\begin{align}
	 Z(M) = \sum_{h_1 h_2 ... h_L} e^{- \beta \sum_{i} H(h_i,h_{i+1})}  \delta(\sum_i h_i = M)
\end{align}
avec la relation vis-à-vis de l'ensemble grand-canonique en l'absence de potentiel chimique
\begin{align}
	 \Xi = \sum_{M} Z(M) 
\end{align}
La position moyenne de l'interface est maintenant définie et beaucoup d'états sont interdits, ce qui change énormément les propriétés thermodynamiques de la matrice de transfert comme la distribution des hauteurs de l'interface, même si la moyenne reste la même. Malheureusement, il est impossible de réécrire la contrainte dans le langage des matrices de transfert, empêchant ainsi de calculer analytiquement les différences entre les deux ensembles. Il est possible de construire la fonction de partition \textit{ab initio}, mais le grand nombre de sites et de hauteurs permises dans un système classique empêchent le calcul dans un temps raisonnable. 


	\section{Indiscernabilité des particules : Particle-Over-Particle}
	
Dans le modèle d'Ising, les particules sont discernables puisque labellisées par leur position $(i,j)$. Cette discernabilité pose problème lorsque l'on utilise le modèle pour un système de gaz sur réseaux ou de fluides binaires par exemple. À cet égard le modèle SOS est meilleur, puisque la discernabilité ne concerne plus que les sites $i$ contenant $h_i$ particules indiscernables. 
En prenant le point de vue atomiste présent dans le modèle d'Ising, nous pouvons suivre la position de chaque particule au sein de nos sites. Cela a plusieurs implications.
La première, c'est qu'en général, seules les couches proche de l'interface sont actives, tandis que les mouvements dans le \textit{bulk} sont bien plus lents. Ainsi, en prenant une particule au hasard dans nos simulations numériques, il est possible de donner une mobilité différente aux couches du modèle SOS. Nous n'explorons pas ces systèmes dans la présente thèse. 
Deuxièmement, lors des algorithmes de Monte Carlo définit au chapitre suivant, la probabilité de choisir un site au hasard n'est pas la même que celle de choisir une particule au hasard ! 
La fonction de partition d'un système où les particules sont discernables s'écrit maintenant
\begin{align}
	Z = \sum_{h_1 h_2 ... h_L} e^{- \beta \sum_{i} H(i,i+1)} \frac{N!}{\prod_i n_i!} = N! \sum_{h_1 h_2 ... h_L} e^{- \beta \sum_{i} H(i,i+1) -\sum_i \ln(n_i!)}
\end{align}

Ce nouveau système, que l'on appellera - par analogie avec Solid-On-Solid - le modèle Particle-Over-Particle sera étudié de manière détailĺée dans les derniers chapitres.

\begin{figure}[h]
	\centering
	\includegraphics[width=0.7\linewidth]{isingtosos/figure-sos.pdf}
	\caption{Une configuration possible de modèle SOS lorsque les particules sont discernables. Cette représentation est la représentation classique dans la littérature.}
\end{figure}



\begin{comment}
		\section{SOS Hamiltonian}
		%%%%%%%%%%%%
		
		In the following, we present three Solid-On-Solid models with different magnetic fields. 
		The SOS interaction between nearest neighboors is of the form $f(i,j) = |h_i - h_j|$. 
		This kind of interaction prevents big fluctuations between two nearest neighboors and is directly related to the Ising model in the approximation where there are no overhangs between the two phases. \textcolor{blue}{see my notes for the derivation}
		
		In the absence of a magnetic field, the interface will fluctuate around its center. Shown below an typical SOS interface for an $L_X=50$ and a $L_Y=60$ after $10^5$ Monte Carlo steps.
		
		%\includegrapsics[width=10cm]{nomag.png}
		
		
		%%%%%%
		\subsection{Model $g(i) = h_i$ (model A)}
		%%%%%%
		
		This model replicates the effect of a homogeneous magnetic field. The bigger the magnetic field $B$ (which can be positive or negative), the further the interface is driven with a symetry breaking. 
		
		\begin{align}
		  H(i,j) = J |h_i-h_j| + B \frac{h_i + h_j}{2}
		\end{align}
		
		%\includegrapsics[width=10cm]{normal.png}
		
		For an infinite magnetic field, we clearly see that the interface gets flattened over on of the edge. The resulting single state avalaible in this limit is the flat interface, with all sites beeing over or under it, depending on the sign of $B$.
		
		In the limit $B \rightarrow \infty$, the interface will flatten to the bottom edge, resulting in a single state of energy 
		\begin{align}
		  F(B \rightarrow \infty) = - B L_Y
		\end{align}
		
		The free energy for at $B=0$ is thus given by align \ref{free_energy} as
		\begin{align}
		  F(0) = B L_X L_Y - \int_0^\infty m(B)dB
		  \label{energymodela}
		\end{align}
		with $m(B) = \langle\sum_i h_i> = L_Y$
		
		%%%%%%
		\subsection{Model $g(i) = |h_i|$ (model B)}
		%%%%%%
		
		This model uses a stagged magnetic field analoguous to the action of a laser on a binary mixture. The further we get from the mean position, the higher is the energy. In order to minimize the energy, the system will have a tendency to be pinned, leading to a very flat interface. 
		
		\begin{align}
		  H(i,j) = J |h_i-h_j| - B \frac{|h_i| + |h_j|}{2}
		\end{align}
		
		%\includegrapsics[width=10cm]{stagged.png}
		
		In the limit $B \rightarrow \infty$, the free energy $F$ will be equal to $0$, while the magnetisation $m(B \rightarrow \infty) = \langle\sum_i |h_i|> = 0$ also.
		
		%%%%%%
		\subsection{Model $g(i) = -|h_i|$ (model C)}
		%%%%%%
		
		This model is the same as the previous one, except with a switch of sign. In this case, the magnetic field will have a depinning effect leading to a scattering of the heights around both edges. \textcolor{red}{I don't really get yet the argument about the competition between entropy and energy.}
		
		\begin{align}
		  H(i,j) = J |h_i-h_j| - B \frac{|h_i| + |h_j|}{2}
		\end{align}
		
		%\includegrapsics[width=10cm]{negstagged.png}
		
		In the $B \rightarrow \infty$ limit, we have then a scattered system. How can we compute its free energy ? Sites will be at $h_i=\pm L_Y$ , leading to an easy $2\times 2$ transfer matrix.
		\begin{align}
		  Z = e^{\beta B L_Y L_X} Tr( (e^{-\beta J \frac{L_Y}{2} \sum_i |\sigma_i - \sigma_j| })^{L_X} )
		\end{align}
		where $\sigma_i = \pm 1$
		
		The transfer matrix is then given as
		
		\begin{align}
		T= e^{\beta B L_Y}
		  \begin{pmatrix}
		    1 & e^{-\beta 2 J L_Y} \\
		    e^{-\beta 2 J L_Y} & 1
		  \end{pmatrix}
		\end{align}
		Its eigenvalues are $\lambda_\pm = e^{\beta B L_Y}( 1 \pm e^{-\beta 2 J L_Y})$, giving a partion function 
		\begin{align}
		  Z = e^{\beta B L_Y L_X} \times ((1 - e^{-\beta 2 J L_Y})^{L_X} + (1 + e^{-\beta 2 J L_Y})^{L_X} )
		\end{align}
		
		The free energy from \ref{deffree_energy} is 
		\begin{align}
		  F(B\rightarrow \infty) = - L_Y B - \frac{1}{\beta} \ln \left( 1 + e^{-\beta 2 J L_Y} \right)
		  \label{energymodelc}
		\end{align}
		which, in the limit of $L_X \rightarrow \infty$ converges to \ref{energymodela}. This is easily explained as the energy to switch from a side to another increases so much that at some point the interface will be pinned to one of the edges, resulting in the same single state.
		
		\begin{figure}
		%  \includegrapsics[width=13cm]{comparison.pdf}
		  \caption{Computation of both terms in \ref{free_energy} in the limit $L_X \rightarrow \infty$ and $L_Y=30$ for the three different models. We see that models A and C have a very similar behaviour even for very small $B$. The linear fit does indeed give a relation $F(0) - F(B) = L_Y \times B$}
		\end{figure}
		
		%%%%%%%%%%%%
		\newpage
		\section{Numerical results}
		%%%%%%%%%%%%
		
		The diffusion of particles in a system can be mapped in an Ising model pretty easily if we assume the conservation of particles through time in our Monte Carlo dynamics. That means that $\sum h_i = K$, with $K$ a constant defined by the initial conditions. This condition can be enforced in the partition function if we only take the microstates satisfying our constraint. Sadly, this constraint is about the microstates and can not be transposed into our Hamiltonian, making the Transfer Matrix useless for such a case. 
		
		The question we want to adress is thus : how big is the difference between the constrained and the unconstrained dynamic in the computation of the free energy ? Is the limit $B^{\ast} \rightarrow \infty$ the same for the three models for both dynamics ? 
		Figure \ref{compGlau} conforts us in the conformity between the Glauber dynamic and the Transfer Matrix method. 
		
		\begin{figure}[h]
		%  \includegrapsics[width=13cm]{ModGlau.pdf}
		  \caption{Computation of the free energy (above) and the magnetisation (below) for the three models in a Glauber dynamics (unconstrained dynamics) through simulations (with error bars) and exact transfer matrix diagonalization.}
		  \label{compGlau}
		\end{figure}
		
		When comparing to the Kawasaki dynamic in Figure \ref{compKaw}, we start seeing some differences. First, in model A the magnetization of our system is always  equal to $0$, by construction of our Kawasaki dynamics, meaning that the computation of the free energy is a tricky one. Luckily, Model A is very similar to Model C with respect to the posible microstates, except some walls that do not add a significan free energy into the system. This means that the results we obtain in Model C are roughly the same as in Model A ! This allows us to drop Model A from the discussion from now on and get retrieve its general behaviour and properties from Model C.
		
		As expected, the constrained dynamics adds some variance with respect to the Transfer Matrix. 
		
		\begin{figure}
		%  \includegraphics[width=13cm]{ModKaw.pdf}
		  \caption{Computation of the free energy (above) and the magnetisation (below) for the three models in a Kawasaki dynamics (constrained dynamics) through simulations (with error bars) and exact transfer matrix diagonalization. The error in magnetization of Model A is exactly equal to 0, by construction.}
		  \label{compKaw}  
		\end{figure}

  \section{Discretization of the system with respect to continuous models}
    \subsection{Correlation length and temperature}
\end{comment}
\chapter{Méthodes numériques}
\label{chap-sim}

En 1949, Metropolis \cite{metropolis_monte_1949} découvre une méthode pour calculer via des simulations numériques de Monte Carlo, la moyenne d'observables statistiques. Si $Q$ est une quantité observable appartenant à un système statistique, comme l'énergie interne ou la densité moyenne de particules par site, alors la moyenne est calculée en pondérant la valeur de l'observable sur toutes les configurations $C$ du système par rapport au poids statistique de ces configurations. Si l'on considère le système en équilibre thermodynamique alors chaque configuration $C$ suit une distribution de Gibbs-Boltzmann, et la moyenne $<Q>$ est vaut
\begin{align}
    <Q> = \frac{\sum_{C} Q(C) \exp(-\beta E(C))}{\sum_{C} \exp(-\beta E(C))}
\end{align}
Pour un système SOS de taille $100\times100$ par exemple, petit par rapport à la limite thermodynamique comme discuté avec la figure \ref{fig-thermo-libre}, il existe $100^{100}$ configurations possibles, bien qu'une simulation numérique ne puisse explorer qu'environ $10^8$ configurations différentes en un temps CPU raisonnable.
Les modèles sur réseau se prêtent parfaitement aux simulations numériques de Monte Carlo, où le but est de calculer la valeur moyenne des observables telles que l'énergie interne ou la densité moyenne de particule par site. Toutes ces quantités peuvent être calculées directement pour le modèle SOS dans l'ensemble grand-canonique à l'aide des valeurs propres de la matrice de transfert, mais il est impossible d'utiliser une telle méthode dans l'ensemble canonique, comme expliqué dans le chapitre précédent.

Dans ce chapitre, nous commençons par expliquer le principe des simulations de Monte Carlo Metropolis, et comment choisir l'ensemble thermodyique de la simulation numérique. En plus d'étudier l'ensemble canonique, les simulations numériques offrent la possibilité d'étudier les régimes hors équilibre, dont nous justifierons la validité.
Nous finirons le chapitre par expliquer comment accélérer la vitesse de simulation grâce à la parallélisation, ainsi que d'autres astuces de programmation, en insistant sur les écueils techniques à éviter. 

Je remercie le Mésocentre de Calcul Intensif Aquitain (MCIA)\footnote{\url{https://redmine.mcia.fr/projects/mcia}} sur lequel j'ai effectué la très grande majorité de mes simulations numériques. 
L'intégralité du code produit pour cette thèse est accessible sur Github \footnote{\url{https://github.com/Bulbille/Curta}} sous la licence Creative Commons BY 3.0 \footnote{\url{https://creativecommons.org/licenses/by/3.0/fr/}}. Les simulations numériques ont été codées en C++, la parallélisation avec la librairie MPI, l'automatisation du lancement des jobs en Bash, et la visualisation des données ainsi que les diagonalisations des matrices de transfert sous Python.

{\color{red} actuellement, est-ce que j'ai le droit de diffuser librement mon code ? Le CNRS autorise la libre diffusion du code ?}

%%%%%%%%%%%%%%%%%%%%%%%%%%%%%%
    \section{Algorithme de Monte Carlo Metropolis}
%%%%%%%%%%%%%%%%%%%%%%%%%%%%%%

Les simulations de Monte Carlo explorent l'espace des configurations de manière aléatoire \cite{newman_monte_1999} avec une probabilité $p(C$ que nous définirons plus tard. En choisissant $M$ états ${C_0,...,C_M}$, l'estimateur $Q_M$ de $Q$ est donnée par
\begin{align}
    Q_M = \frac{\sum_{i=0}^M Q(C_i) p(C_i)^{-1} \exp(-\beta E(C_i))}{\sum_{i=0}^M  p(C_i)^{-1} \exp(-\beta E(C_i))}
\end{align}
Lorsque $M$ augmente, l'estimateur devient une estimation de plus en plus précise de $<Q>$, jusqu'à la limite $Q_{M\to \infty} = <Q>$. Si l'on choisit les configurations sur lesquelles on échantillone le système selon la distribution à l'équilibre de Gibbs-Boltzmann $p(\nu) = Z^{-1} e^{-\beta E(C)}$, alors l'éstimateur de $<Q>$ devient
\begin{align}
    Q_M = \frac{1}{M} \sum_{i=0}^M Q(C_i)
\end{align}
On se pose maintenant la question de savoir comment choisir les configurations afin que chacune apparaisse avec la bonne probabilité de Boltzmann. 

Une dynamique pour les systèmes avec une espace des phases discret peut être construit à partir de chaînes de Markov. On laisse la dynamique évoluer dans un discret noté $n$, et $p_n(C)$ la probabilité que le système soit dans l'état $C$ au temps $n$. Au pas de temps suivant, si le système est dans l'état $C$ il peut sauter vers un autre état $C'$ avec la probabilité de transition $\rho(C\to C')$. Le système au tempst $n+1$ dépend alors uniquement de l'état au temps $n$ : c'est un processus markovien. La probabilité $p_{n+1}(X)$ d'être dans l'état $C$ au temps $n+1$ est possible si le système était dans l'état $C$ au temps $n$ et y reste avec une probabilité $\rho(C\to C)$ , ou s'il est dans un état $C'$ et bouge vers l'état $C$ avec une probabilité $\rho(C'\to C)$. On a alors l'équation maîtresse
\begin{align}
    p_{n+1}(C) =  \rho(C\to C) p_n(C) + \sum_{C'\neq C} \rho(C'\to C) p_n(C')
\end{align}
Puisque $\rho(C' \to C)$ est une probabilité, on a la condition suivante
\begin{align}
    \sum_{C'} \rho(C' \to C) = 1
    \label{norm}
\end{align}
Maintenant, si la dynamique décrit un système physique en interaction avec un  réservoir de chaleur, la distribution à l'équilibre est donnée par
\begin{align}
    p_{eq}(C) = \frac{\exp(-\beta E(C))}{Z}
\end{align}
avec $Z$ la fonction de partition canonique. Puisque la distribution à l'équilibre n'évolue pas au cours du temps, on a
\begin{align}
    p_{eq}(C) =  \rho(C\to C) p_{eq}(C) + \sum_{C'\neq C} \rho(C'\to C)p_{eq}(C')
    \label{p-eq-mc}
\end{align}
Une autre condition que l'on impose à notre chaîne de Markov afin qu'elle génère une probabilité de distribution de Boltzmann après équilibrage, est qu'elle respecte le bilan détaillé. Afin qu'un système respecte le bilan détailĺé, il faut que le taux auquel il fait des transitions vers à partir de n'importe quel état $C$ soit égal. Mathématiquement, cela revient à dire que
\begin{align}
    \sum_{C'} p(C) \rho(C \to C') = \sum_{C'} p(C') \rho(C' \to C)
\end{align}
On peut démontrer que cette relation est équivalente à \cite{newman_monte_1999} 
\begin{align}
    \frac{\rho(C'\to C)}{\rho(C \to C')} = \frac{p(C)}{p(C')} = \frac{\exp(-\beta E(C))}{\exp(-\beta E(C'))}
\end{align} 
En adoptant le bilan détaillé, on voit facilement que la distribution à l'équilibre calculée via \ref{p-eq-mc} redonne bien la distribution de Gibbs-Boltzmann.
Durant une étape de Metropolis, la probabilité pour que la transition $C\to C'$ soit acceptée est 
\begin{align}
    p_a(C\to C')
\end{align}


Systems with a canonical heat bath can be simulated on a computer using an algorithm
obeying detailed balance. For example consider a system of $N$ Ising spins $S_i=\pm1$ interacting via a Hamiltonian $H(S_1, S_2,\cdots S_N)$. We choose $1$ of the spins randomly uniformly with a probability $p=1/N$ and calculate the new energy of the system when the spin, $S_j$ say is changed to $-S_j$. In Metropolis dynamics the probability of accepting the spin flip $p_a(S_j\to -S_j)$ is given by 
\begin{align}
p_a(S_j\to -S_j)
\end{align}
if $H(S_1, S_2,\cdots, -S_j,\cdots S_N) < H(S_1, S_2,\cdots, S_j,\cdots S_N)$ but if $H(S_1, S_2,\cdots, -S_j,\cdots S_N) > H(S_1, S_2,\cdots, S_j,\cdots S_N)$ then the flip is accepted with probability 
\begin{align}
p_a(S_j\to -S_j) = \exp\left[ -\beta\left(H(S_1, S_2,\cdots, -S_j,\cdots S_N) - H(S_1, S_2,\cdots, S_j,\cdots S_N)\right)\right] <1.
\end{align}
The total probability at a given discrete time of changing $S_j$ is thus equal to 
\begin{align}
p(S_j\to -S_j) = \frac{1}{N} p_a(S_j\to -S_j)
\end{align}
as we choose the spin $S_j$ with probability $1/N$. Therefore we have
\begin{align}
\frac{p(S_j\to -S_j)}{p(-S_j\to S_j)} = \frac{p_a(S_j\to -S_j)}{p_a(-S_j\to S_j)}.
\end{align}
In the case where the change $S_j\to -S_j$ lowers the energy we have
\begin{align}
p_a(S_j\to -S_j) =1,
\end{align}
however the reverse move $-S_j\to S_j$ costs energy so 
\begin{align}
p_a(-S_j\to S_j) =\exp\left[ -\beta\left(H(S_1, S_2,\cdots, S_j,\cdots S_N) - H(S_1, S_2,\cdots, -S_j,\cdots S_N)\right)\right],
\end{align}
which gives
\begin{eqnarray}
\frac{p(S_j\to -S_j)}{p(-S_j\to S_j)} &=& \frac{1}{\exp\left[ -\beta\left(H(S_1, S_2,\cdots, S_j,\cdots S_N) - H(S_1, S_2,\cdots, -S_j,\cdots S_N)\right)\right]} \nonumber \\
&=&\frac{ \exp\left[ -\beta H(S_1, S_2,\cdots, -S_j,\cdots S_N)\right]}{\exp\left[ -\beta H(S_1, S_2,\cdots, S_j,\cdots S_N)\right]},
\end{eqnarray}
and so in this case we see that detailed balance is respected. In the case of a move which increases the energy it is easy to see that detailed balance is again respected.  

If we consider a case where the spins $+$ represent one type of particle and the $-$ another type and insist that the total  chemical composition remains the same the above dynamics is not correct as you cannot convert a $+$ into a $-$ and vice-a-versa. However a $+$ next to a $-$ can change places. Kawasaki dynamics chooses a neighbouring pair of $+$ and $-$ and tries to switch their positions, e.g. $.+-.\to .-+.$,  the move is accepted with probability $1$ if the energy change $\Delta E<0$  and with probability $p_a=\exp\left(-\beta\Delta E\right)$ if $\Delta E>0$.

Practically in a computer program if $\Delta E >0$ one draws a uniformly distributed random
number $r\in[0,1]$ (for example {\tt rand} in Fortran and Matlab), if $r< p_a $ the move accepted but if $r>p_a$ it is refused and the system stays in its initial state.



\section{Suite}


qui définit la fonction de partition $\mZ$. Dans un algorithme de Metropolis, on met à jour le micro-état en prenant un site $i$ au hasard
\footnote{L'utilisation d'un générateur de nombre aléatoire (\textit{pRNG}) efficace est primordial. Il est déconseillé d'utiliser le générateur standard \textit{default\_random\_engine} de la librairie C++ \textit{rand} et conseillé d'opter pour des générateurs \textit{sfc64} ou \textit{xoroshiro}. Pour un pRNG booléen performant, voir \url{https://martin.ankerl.com/2018/12/08/fast-random-bool/}. Pour accélérer encore plus les calculs, ne pas oublier d'utiliser le flag d'optimisation \textit{-O3}  sur \textit{gcc} si vous codez en C/C++. Tout cela combiné accélère le code d'un facteur 20 environ. \newline
De plus, bien que la librairire OpenMP pour paralléliser le code soit simple d'utilisation, elle gère très mal - de sa nature de mémoire partagée - les pRNG. Je conseille vivement l'utilisation de la librairie MPI qui assure une étanchéité au niveau des pRNG entre chaque thread.} 
et en le changeant légèrement vers un état $\nu$. Dans un système d'Ising, nous choisissons un spin $\sigma_i$ au hasard et regardons s'il peut être renversé ou échangé avec l'un de ses plus proches voisins. Dans le modèle SOS, nous choisissons une colonne $h_i$ au hasard et regardons s'il est possible d'ajouter ou de retirer une unité à la hauteur (c'est le nombre de particules sous l'interface au site $h_i$), ou d'échanger une particule d'une colonne vers une de ses plus proches voisins.
La différence d'énergie notée $\Delta E(\mu \rightarrow \nu)$ donne la probabilité de transition entre les deux. Si l'état final $\nu$ a une énergie inférieure à l'état initial, alors il est forcément plus probable que $\mu$, et nous acceptons le changement. Dans le cas où $E_\nu \greater E_\nu$, on accepte le changement avec une probabilité satisfaisant au bilan détaillé pour une marche markovienne satisfaint à l'état d'équilibre de Botlzmann
\begin{align}
\frac{p(\mu \rightarrow \nu)}{p(\nu \rightarrow \mu)} = e^{-\Delta Ep(\nu \rightarrow \mu)}
\end{align}
ce qui nous donne la probabilité de transition $\mu \to \nu$ de Metropolis
\begin{align}
	p(\mu \rightarrow \nu) = min(1,e^{-\beta \Delta E(\mu \rightarrow \nu)})
\end{align}
Ensuite on prend un nombre aléatoire $q$ entre $0$ et $1$. Si $q < p(\mu \rightarrow \nu)$, alors la transition est validée. Une étape de Monte Carlo est achevée lorsque $L$ tentatives de transition ont été faites. Cependant, il est possible d'accélérer l'algorithme en utilisant un temps continu \cite{newman_monte_1999} ou en prenant en compte les états dont la transition a été refusée \cite{frenkel_speed-up_2004}.
L'erreur obtenue à la fin sur notre observable $<A>$ au cours d'une simulation ayant duré $t_{max}$ étapes de Monte Carlo est 
\begin{align}
	E(A) = \sqrt{\frac{2 \tau}{t_{max}} (<A^2>-<A>^2)} 
\end{align}
Cette variance dépend du temps de corrélation $\tau$ puisque si deux micro-états sont très rapprochés dans le temps , l'observable en question n'aura pas grandement évolué. En pratique, il suffit que $\frac{\tau}{t_{max}} \less 10^{-4}$ pour obtenir une erreur inférieure à $1\%$. Ce temps de corrélation $\tau$ se calcule via la fonction d'auto-corrélation 
\begin{align}
\mC(t) = <A(t')A(t+t')>-\langle A \rangle^2 = \frac{1}{T_{max}}\int_0^{T_{max}}A(t')A(t+t')-<A>^2 dt' \simeq e^{-\frac{t}{\tau}}
\end{align}
qui se comporte comme une somme d'exponentielles, mais où dans la limite thermodynamique, seul le mode de relaxation le plus long compte\cite{wansleben_monte_1991}. En supposant la limite thermodynamique, l'ordre de grandeur de $\tau$ - et donc de la variance de nos observables - est donnée par le calcul de l'intégrale\footnote{Je recommande d'intégration de Simpson.}
\begin{align}
	\tau = \int_0^{\infty} \mC(t)/\mC(0) dt
	\label{tau_cor}
\end{align}
Le calcul de la plus grande longueur de corrélation $\xi$ du système se fait de manière analogue en intégrant la fonction de corrélation spatiale définie par
\begin{align}
\mC(x) = \frac{1}{L} \sum_{x'}^L A(x')A(x+x')-<A>^2 \simeq e^{-\frac{x}{\xi}}
\end{align}
Une discussion plus rigoureuse sur la forme de la fonction de corrélation spatiale sera donnée dans la section \ref{sec_laser}.

	\subsection{Ensemble grand-canonique : algorithme de Glauber}

\begin{figure}[h]
	\centering
	\includegraphics[scale=1]{numerical/sos-glau-eq-cor.pdf}
	\caption{Courbe de l'énergie (haut) et fonction d'auto-corrélation (bas) dans avec un \textbf{paramètre d'ordre non-conservé} à partir de la condition initiale. Le temps d'équilibrage (en étapes de Monte Carlo) diminue avec la température, tandis que le temps de corrélation reste relativement constant. Le temps de corrélation étant extrêmement faible, $10^7$ étapes de Monte Carlo suffisent à avoir une erreur de moins de $0.1\%$ sur les moyennes mesurées.}
	\label{eq-glau}
\end{figure}
	
Le dépôt de particules provenant d'un réservoir permet de faire grandir un cristal à partir d'un substrat. Ce genre de systèmes est défini par le potentiel chimique $\mu$ des particules, dans le solvant et appartient à l'ensemble grand-canonique. Dans ce cas, on choisit au hasard de manière uniforme une colonne $h_i$ dans laquelle on décide de mettre ou d'enlever une particule selon le flux de particules $\nu$ vu dans l'équation d'Edwards-Wilkinson \ref{edwards-wilkinson}. Si l'on se place à l'équilibre thermodynamique, c'est-à-dire qu'autant de particules se déposent au niveau de l'interface que de particules la quittent, alors il faut que la probabilité de ces deux événements soient égales entre elles, et donc égales à $50\%$.
Dans le cas où la géométrie est infinie, les valeurs des $h_i$ ne sont pas bornées, tandis que dans une géométrie torique de hauteur $L$, on rejette toutes les configurations qui ne respectent pas aux conditions $0 \leq h_i \leq L$.
En essayant d'aller du micro-état $\mu$ vers le micro-état $\nu$ où on a fait la transformation $h_i \rightarrow h_i + \alpha$ où $\alpha=\pm 1$, on obtient que la différence d'énergie est
\begin{align}
	\Delta E &= |h_{i-1}-(h_i \pm 1)| + |h_{i+1}-(h_i \pm 1)| - |h_{i-1}-h_i| - |h_{i+1}-h_i|  \\
		&= 2 \left( (h_i \leq h_{i-1}) + (h_i \geq h_{i+1}) -1 \right )
\end{align}
où $(h_i \leq h_{i-1})$ est un booléen valant $1$ si la condition est vraie, $0$ sinon.
Le changement de magnétisation est alors $\Delta M = \alpha$, et la largeur de l'interface, définie par $\sigma = \sum_i (h_i-h_{i+1})^2$, change comme
\begin{align}
	\Delta \sigma = 2 \alpha (h_{i+1}-h_i) + 2
\end{align}
On n'a donc pas besoin, à chaque pas de temps, de recalculer ces deux grandeurs, il suffit de les actualiser dans une variable pour avoir les observables à tout instant $t$.


Afin d'accélérer le processus d'équilibrage du système, il est recommandé de commencer directement avec la valeur moyenne de magnétisation calculée à partir de la matrice de transfert. On regarde ensuite le temps d'équilibrage par la courbe $E(t)$, en attendant d'atteindre la valeur à l'équilibre. 
À l'équilibre, le taux d'évaporation des particules doit être égal au taux de dépôt sur notre système. Cependant, en l'absence d'un potentiel qui contraint l'interface, l'interface est délocalisée, l'empêchant d'atteindre l'équilibre thermodynamique. C'est la raison pour laquelle une simulation numérique dans une dynamique de Glauber se doit toujours d'avoir un potentiel permettant d'obtenir la localisation d'une interface. 

	\subsection{Ensemble canonique : algorithme de Kawasaki}

\begin{figure}
	\centering
	\includegraphics[scale=1]{numerical/sos-kaw-eq-cor.pdf}
	\caption{Courbe de l'énergie (haut) et fonction d'auto-corrélation (bas) dans avec un \textbf{paramètre d'ordre conservé} à partir de la condition initiale. Le temps d'équilibrage (en étapes de Monte Carlo) diminue avec la température, tandis que le temps de corrélation reste relativement constant. Le temps de corrélation est similaire à la dynamique de Glauber, bien que l'équilibrage soit plus long à se faire.}
	\label{eq-kaw}
\end{figure}
	
La diffusion des particules - par exemple un polymère dans un solvant - est une dynamique locale qui conserve le paramètre d'ordre du notre système, nommément la magnétisation $m$. Dans ce cas, on choisit au hasard de manière uniforme deux colonnes $h_i$ et $h_{i+1}$ dans lesquelles on va essayer d'échanger une particule entre les deux colonnes. Afin de respecter le bilan détaillé, il faut que la probabilité de choisir le mouvement $h_i \rightarrow h_{i+1}$ soit égale à $h_{i+1} \rightarrow h_i$. On peut juste définir à nouveau "l'ajout" d'une colonne vers ou à partir de l'autre via la transformation $h_i \rightarrow h_i + \alpha$ et $h_{i+1} \rightarrow h_{i+1} - \alpha$ (avec $\alpha=\pm 1$), en respectant toujours les conditions aux bords en $y$. Trois termes dans l'énergie sont modifiées\footnote{Comme précédement, il existe une version booléenne de l'équation, mais sa longueur n'offre aucun avantage en terme d'implémentation dans le code comparé au gain de temps de CPU engendré.}
\begin{align}
	\Delta E = &|h_{i-1}-(h_i \pm 1)| + |h_{i+1} \pm 1 -(h_i \pm 1)| + |h_{i+1}\pm 1-(h_{i+2} )| \\
	- &|h_{i-1}-h_i| - |h_{i+1}-h_i| - |h_{i+1}-h_{i+2}|
\end{align}

La magnétisation totale est ainsi conservée, tandis que la largeur de l'interface $\sigma$ se calcule par
\begin{align}
	\Delta \sigma = 2 \alpha  + 1
\end{align}

	\subsection{Dynamique hors-équilibre}
L'ensemble grand-canonique ne nous permet d'avoir un système qu'à l'équilibre, puisqu'il est traduit par une dynamique non-locale. Seule une dynamique locale comme la dynamique de Kawasaki peut nous donner des états hors-équilibre. L'implémentation la plus simple est d'introduire un terme de cisaillement dans notre modèle lorsque l'on décide de bouger une particule. Ce cisaillement diminue l'énergie du micro-état lorsque la particule bouge dans un sens et l'augmente si elle bouge dans l'autre sens, ce qui brise le bilan détaillé. De nombreux travaux sur les systèmes hors-équilibre dans le modèle d'Ising ont été produits \cite{smith_interfaces_2008} présentant la diminution de la largeur de l'interface lorsque le cisaillement est produit de manière parallèle. 
On peut définir deux espèces de cisaillement parallèles.
Le premier genre de cisaillement se produit aux bords d'un liquide non-visqueux, ce qui ne permet de bouger que les particules aux bords du système : il n'est donc pas adaptable à un système infini ou semi-infini. Pour un système de taille $L$ et pour un module de cisaillement de $f$, la différence d'énergie supplémentaire est 
\begin{align}
	\Delta E_{bord} = f [ (h_i == 1 || h_{i+1} == L-1) - (h_i == L-1 || h_{i+1} == 0)  ]
\end{align}
Le second genre de cisaillement se produit aux bords d'un fluide permettant un transport visqueux, ce qui entraîne un cisaillement proportionnel à la distance aux bords comme sur la figure \ref{snap-ising-shear}. En supposant que le cisaillement est nul au niveau de l'interface et que les particules vont à gauche dans la partie basse du système (et à droite dans la partice haute du système), on obtient alors
\begin{align}
	\Delta E_{prop} = f h_i
\end{align}
Cependant, pour des raisons de facilité de calcul plus tard afin de comparer les simulations numériques aux résultats analytiques, on utilise un cisaillement uniforme qui pousse les particules dans un sens. Ce type de système correspond à un flux laminaire, par exemple dû à la gravité face à une interface verticale qui tire les particules vers le bas. La différence d'énergie devient
\begin{align}
	\Delta E_{uni} = \alpha f
\end{align}
où $\alpha = 1$ si la particule va vers la droite, $-1$ sinon. 
		
	\subsection{Modèle POP}		

Dans le modèle POP, le modèle n'est plus structuré en fonction des sites $i$ mais bien des particules $\sigma_(n) = i$, la hauteur d'un site\footnote{Cette hauteur est mise à jour à chaque étape mouvement d'une particule dans un second tableau.} devenant alors
\begin{align}
	h_i = \sum_{n=0}^N \delta_{\sigma_n,i}
\end{align}

Lors d'une dynamique de Kawasaki, à chaque étape, on choisit au hasard une particule parmi les $N$ présentes dans le système pour la déplacer d'une colonne. 

Il est également possible de donner des constantes de diffusion différentes à chaque particule\footnote{Grâce à la construction d'un générateur via \textit{random::discrete\_distribution} où chaque particule a une probabilité différente d'être sélectionnée. }  afin d'émuler différents types de particules. 

La question est plus délicate lorsqu'il s'agit d'une dynamique de Glauber. Puisque chaque particule a une probabilité d'être sélectionnée pour être détruite, comment choisir la probabilité d'ajouter une particule au système ? À l'équilibre, le flux de particules entrantes est égale au flux de particules sortantes, c'est-à-dire $p_{ajout}= p_{retrait} = 0.5\%$. Dans ce cas, il suffit de choisir un booléen au hasard, puis détruire une particule et son label ou ajouter une particule à un site particulier. L'avantage de la dynamique conservée est qu'il n'est pas nécessaire de reconstruire une distrubtion pRNG à chaque étape, même si le constructeur est rapide \footnote{Le constructeur a une complexité en $\mathcal{O}(n)$ au pire. \url{http://www.cplusplus.com/reference/random/discrete_distribution/discrete_distribution/}}.

%%%%%%%%%%%%%%%%%%%%%%%%%%%%%%%%%%
\section{Conclusion}
%%%%%%%%%%%%%%%%%%%%%%%%%%%%%%%%%%

Dans ce chapitre nous avons décrit les différentes méthodes de calcul numérique qui vont de pair avec le modèle A et le modèle B, et la manière de mesurer les observables ainsi que leur barre d'erreur. Dans la pratique, les temps de corrélation sont si faibles qu'il suffit de faire environ $10^7$ étapes de Monte Carlo afin d'obtenir de bonnes statistiques, ce qui en une dimension, est extrêmement rapide. La rapidité des simulations dans le modèle SOS nous permet ainsi d'étudier une très vaste plage de paramètres, que ce soit pour différentes températures, cisaillements, hauteurs maximales ou champs externes. 
mux

\chapter{Résultats pour le modèle SOS}
    \label{chap-sos}

On considère un système de longueur $L$ et de hauteur $N$. Pour $N$ très grand, si la position de l'interface est loin de $0$ ou de $N$, on s'attend à retrouver toutes les propriétés d'un modèle infini centré en 0. Si l'interface est proche de $0$, on pourra étudier les effets au bord, comme dans un système semi-infini\footnote{Dans les simulations numériques ainsi que pour la diagonalisation des matrices nous sommes obligés de choisir une taille finie de notre système. Cependant l'équivalence avec les calculs analytiques dans les sytèmes (semi)infinis mest vraie pour de grandes tailles de systèmes}.  On fixe la hauteur moyenne $<h_i> = H$ grâce au potentiel chimique $\mu$\footnote{Calculer la magnétisation moyenne via des librairies comme \hyperlink{http://eigen.tuxfamily.org/index.php?title=Main_Page}{\underline{Eigen en C++}} (contrairement à la librairie GSL plus difficile d'accès) permet d'accélérer grandement l'étape d'équilibrage du système pour le Monte Carlo.}.

\begin{figure}[h]
	\includegraphics[width=\linewidth]{semiifgeom/hauteur-mu.pdf}
	\caption{Hauteur moyenne de l'hamiltonien $H = \sum_i |h_i-h_{i+1}| + \mu \frac{h_i+h_{i+1}}{2}$ en fonction de $\mu$ calculée grâce à la diagonalisation d'une matrice de transfert de taille $N$. Lorsque le potentiel chimique est trop faible, l'interface est délocalisée et se retrouve à la position $\frac{N}{2}$. }
	\label{hauteur-mu}
\end{figure}

	\section{Différences Glauber/Kawasaki}

figure énergie moyenne, sigma, distribution de probabilité via Airy	
	
	\section{Effet du cisaillement}

Dans les expériences, le système est soumis à un cisaillement au niveau de l'interface, ce qui modifie les observables par rapport à l'état d'équilibre. Nous optons ici pour un modèle de cisaillement uniforme dont le sens est défini par $\sgn(i-j)$, $i$ et $j$ étant des sites de hauteurs respectives $h_i$ et $h_j$. 
La différence d'énergie entre un micro-état et le suivant d'une étape de Monte Carlo sous une dynamique de diffusion (Kawasaki) est alors
\begin{align}
	\Delta E = J \sum_i \left[ |h'_i-h'_{i+1}|-|h_i-h_{i+1}| \right] \pm f 
\end{align}
où $h_i'(t) = h_i(t+1)$ - le temps étant discrétisé, $f$ le module de cisaillement et le signe du cisaillement étant pris de manière aléatoire à chaque étape, regardant ainsi si ce mouvement va dans ou contre le sens du flux. 

Pour rappel, à chaque étape on essaie de transférer une particule du site $i$ vers son voisin (gauche ou droit) $j$, ce qui se traduit par $h_i' = h_i-1$ et $h_j' = h_j+1$. 
Pour $f=0$, en remarquant que les valeurs absolues ont les propriétés suivantes
\begin{align}
	|a \pm 1| - |a| = \pm 1 \\
	|a \pm 2| - |a| = {0,\pm 2}
\end{align}
nous voyons facilement l'émergence d'une sélection des énergies possibles entre deux micro-états successifs dans ${-4,-2,0,2,4}$. Ainsi, toutes les transformations diminuant l'énergie totale du système seront toujours acceptées. En augmentant le cisaillement il devient alors possible de refuser des états réduisant l'énergie et d'accepter ceux qui l'augmente. 
Nous nous attendons alors à trois régimes différents :
\begin{itemize}
	\item $f  \less  2 J $ : à faible cisaillement, la symmétrie du système impose les observables à être paire vis-à-vis de $f$, comme le prouve le fit en carré des figures.
	\item $2 J \less f \less 4 J$ : à cisaillement moyen, certains mouvements augmentant la rugosité de l'interface sont toujours acceptés. 
	\item $f > 4 J$ : à haut cisaillement, tous les mouvements augmentant l'entropie du système sont acceptés. Une saturation du système se produit lorsque l'énergie de lien entre les sites devient négligeable face au cisaillement.
\end{itemize}


\begin{figure}[h]
	\includegraphics[width=\linewidth]{./semiifgeom/sosj1.pdf}
	\caption{Énergie $E= \langle \sum_i |h_i-h_i+1| \rangle$ (en pointillé sa dérivée), variance $\sigma = \langle (H - \langle H \rangle )^2  \rangle$ et asymétrie $\gamma = \langle (H - \langle H \rangle )^3  \rangle / \sigma^2$ pour $B^0.1$. La magnétisation est constante et égale à $\langle H \rangle = 4.51$. Le temps de corrélation du système est presque constant en fonction du cisaillement $f$, allant de  $\tau(f=0) = 5.04$ à $\tau(f=6) = 5.00$ étapes de Monte Carlo. On note une brisure à $f=2J$ et $f=4J$.
Les croix notent un fit en carré pour petit $f$, montrant la symétrie du système par inversion du signe de $f$. }
\end{figure}

\begin{figure}
	\includegraphics[width=\linewidth]{./semiifgeom/j13.pdf}
	\caption{Same as before, with $J=1.3$ We observe a net inflexion in the energy at $2 J$ for  $B=0.01$ which corresponds to $<H>=11$. Nvertheless there is not threshold at $4 J = 5.2$. My guess is that the system is too far away from the boundary in order to interact strongly with it. Interstingly enough, for $B=0.1$, $<H>=3$ and we are too close to the boundaries to see anything.}
\end{figure}

	\subsection{Différences avec le modèle d'Ising}

	Le modèle d'Ising repose sur la perméabilité entre les deux phases, avec des bulles d'une phase dans l'autre. Dans les travaux sur le rôle du cisaillement au niveau de l'interface dans un modèle d'Ising ( \cite{smith_driven_2010},\cite{smith_interfaces_2008}. ), le cisaillement s'applique à toutes les particules, ce qui a pour but d'étirer les bulles, les rendant plus fragiles au bain thermique. Cet effet participe à leur évaporation vers l'interface et à la dissipation de l'énergie injectée par le cisaillement.
	
	Au contraire, dans les modèles que nous utilisons ici, où l'interface peut fluctuer mais est imperméable, l'énergie injectée au niveau de l'interface ne peut être dissipée par évaporation, ce qui conduit à l'émergence d'une phénoménologie différente. Il est ainsi impossible de définir un cisaillement qui soit appliqué à l'intérieur des phases, puisqu'il n'existe aucune notion de particules. Plus loin nous développons un modèle qui s'abstrait de ces difficultés. 


\section{Cisaillement avec deux types de particules}

La construction naïve d'un modèle continu du cisaillement avec un seul type de particules ne donnera aucun résultat. En effet, pour que le cisaillement induise des effets hors équilibre, il faut que la dynamique des particules dans tout repère galilén soit le même. Si l'on considère une force de cisaillement uniforme qui induit la même vitesse moyenne sur toutes les particules du système, en nous plaçant dans un repère bougeant à la même vitesse que cette vitesse moyenne, nous retrouvons les mêmes propriétés à l'équilibre.
Afin de briser la symmétrie de translation, il faut soit induire un cisaillement non-uniforme, soit introduire des particules qui réagissent de manière différente vis-à-vis de cette force. Dans notre exemple sur les colloïdes, la gravité agit bien sur les polymères mais bien moins sur le solvant, ce qui brise en effet l'invariance galiléenne. 
Plusieurs études récentes portent sur le mouvement de systèmes avec plusieurs particules browniennes\cite{netz2003,dzub2002,chak2003,chak2004,lowe2009,glan2012, klym2016}, incluant le problème des électrolytes étudié par Onsager \cite{onsager} il y a longtemps.

	\subsection{Discussion about the Gaussian model}
	
The Gaussian model has a stronger interaction, been as
\begin{align}
	\Delta E = J \sum (h'_i-h'_{i+1})^2 -(h_i-h_{i+1})^2+ f (i-j)
\end{align}
In this model the bond energy between two microstates can take any integer, as 
\begin{equation*}
	(h_i-h_j+2)^2 - (h_i-h_j)^2 = 4 (h_i-h_j+1)
\end{equation*}

The gaussian interaction is very strong, so we could expect a very smooth interface. The mean difference between two sites should be about $h_i-h_{i+1} \simeq 1$. In that case, the energy difference is discretized as ${-8,-4,0,4,8}$. 
Nonetheless we cannot predict exactly the same behaviour as in the SOS model because this approximation has to be verified everytime, which is false. 1

\begin{figure}
	\includegraphics[width=\linewidth]{./semiifgeom/gauss0.pdf}
	\caption{Bond energy, thickness (variance) and skewness of the interface for two different magnetic pressures. The magnetisation is constant and is equal to $<H>=2$ for $B=0.1$. As we are very close to the boundary, we see no threshold with the drive force. Simulations take longer with this model because the interaction is stronger}
\end{figure}

	\section{Interpretation}
	
	
\chapter{Interaction gaussienne avec champ magnétique confinant}
\label{sec_laser}

Jusqu'à présent nous avons développé le modèle Solid-On-Solid et étudié ses propriétés pour des modèles d'interface en présence d'un champ magnétique uniforme, le potentiel chimique. Dans ce chapitre, nous étudions les propriétés d'un champ magnétique non-uniforme en analogie à certaines expériences effectuées au Laboratoire Onde Matière d'Aquitaine par l'équipe de Jean-Pierre Delville. Dans ces expériences, un liquide quaternaire composé de toluenne, sodium dodecyl sulfate, n-butanol et eau possède proche du point critique deux phases miscellaires séparées avec une tension de surface de $\sigma \simeq 10^{-7}N/m$ pour $T-T_C=1.5K$\cite{casner_laser-induced_2003,delville_laser_2009,girot_conical_2019}. À cela s'ajoute un laser qui par pression de radiation pousse une phase dans l'autre. 

Dans un système de taille $L'\times L$, avec $y \in [-\frac{L}{2},\frac{L}{2}]$ \footnotetext{Lorsqu'il s'agit de diagonaliser la matrice de transfert, il suffit de faire la translation $y \to y+\frac{L}{2}$}. Nous nous proposons d'étudier les propriétés statistiques d'une telle interface grâce à la présence d'un champ magnétique de la forme 
\begin{align}
	V(\sigma_{x,y}) = B \sgn(y)
\end{align}
qui se traduit dans le formalisme SOS par
\begin{align}
    V(h_i) = B |h_i|
    \label{staged}
\end{align}
similaire à l'équaiton \ref{neggstaged}, où cette fois le champ magnétique confine l'interface au voisinage de $0$.

Dans ce chapitre, nous étudions analytiquement la distribution de l'interface, le spectre d'énergie et la fonction de corrélation de ce système, qui nous permet de trouver une dépendance entre la tension superficielle avec la température et l'intensité $B$ du champ magnétique. Puisque le système expérimental possède un régime hors-équilibre, nous étudions également l'effet d'un écoulement uniforme au niveau de l'interface. 

%%%%%%%%%%%%%%%%%%%%%%%%%%%%%%%%
    \section{Interface statique}
    \subsection{Distribution de probabilité de l'interface}
%%%%%%%%%%%%%%%%%%%%%%%%%%%%%%%%


Grâce au potentiel \ref{staged}, l'interface est localisée autour de $h=0$, et possède une distribution symmétrique. La méthode décrite à la section \ref{par-stab} pour trouver la forme de la distribution nécessite au préalable un ansatz de la solution. Nous proposons ici une méthode plus puissante qui repose sur des équations continues. Dans le cas où la largeur de l'interface est grande par rapport à l'unité, on s'attend à ce que la description possède les mêmes propriétés que le système discontinu qu'est le modèle Solid-On-Solid.

Une manière de se représenter une configuration de l'interface est de le comparer à la trajectoire d'un marcheur brownien commençant au point $h_0$ et se déplaçant sur $L'$ pas de temps discontinus, jusqu'à arriver à sa position finale $h_L$, avec une trajectoire périodique de période $L'$. La fonction de partition se comprend maintenant comme la somme des trajectoires du marcheur brownien au lieu de la somme des configurations de l'interface. On associe à la trajectoire l'énergie $E = \sigma \mathcal{L}$, où $\sigma$ est la tension superficielle de notre interface et $\mathcal{L}$ la distance totale parcourue par la particule browniene. D'un point de vue continu, la distance effectuée pour de petits déplacements est $d\mathcal{L} = \sqrt{1+\frac{dh^2}{dx^2}} dx \simeq h'^2 dx$.
L'Hamiltonien discret correspondant devient gaussien
\begin{align}
    H = J \sum_i (h_i-h_{i+1})^2 + B \sum_i \frac{|h_i|+|h_{i+1}|}{2}
    \label{hamil-gauss}
\end{align}
tandis qu'une version continue du problème est 
\begin{align}
	H = \frac{\sigma}{2} \int_0^{L'} h'^2(x) dx + B \int_0^{L'} |h(x)|dx
\end{align}

{\color{red} Dans tout ce qui suit, peut-on juste mettre $\sigma \to 2 \sigma$ et enlever le facteur 2 de partout dans les équations ? Ça allège toutees les notations.}

La fonction de partition de toutes les trajectoires $h(x)$ possibles de notre particule telles que $h(0)=h(L')=h^\ast$ est alors donné par
\begin{align}
	\mZ(h,h^\ast,L') = \int_{h(0)=h} d[h] \delta(h(L')-h^\ast)e^{-\frac{\beta \sigma}{2} \int_0^{L'} h'^2(x) dx -\beta B \int_0^{L'} |h(x)|dx}
\end{align}
avec la condition initiale $\mZ(h,h',0)=\delta(h-h')$. La fonction de partition $\mZ$ obéit alors à l'équation de Schrödinger temporelle
\begin{align}
	\frac{\partial \mZ}{\partial {L'}} = \frac{1}{2 \beta \sigma} \frac{\partial^2 \mZ}{\partial h^2}  - B \beta |h| \mZ
\end{align}
avec la condition initiale $\mZ(h,h',0) = \delta(h-h')$.  En absence d'un potentiel externe on retrouve les solutions en sinus et cosinus décrivant une interface délocalisée à travers tout le système, comme dans l'équation \ref{lambda-sos}. En décomposant la fonction de partition dans la base des solutions stationnaires $\psi_E$ correspondant aux énergies propres $E$ 
\begin{align}
	Z(h,h',L') = \sum_E e^{-EL'}\psi_E(h) \psi_E(h')
	\label{schro_temp}
\end{align}
on obtient l'équation aux états propres
\begin{align}
	\epsilon \psi_\epsilon = - \frac{1}{2} \frac{\partial^2 \psi_\epsilon}{\partial h^2} + \lambda |h| \psi_\epsilon
\end{align}
où l'équation a été admiensionalisée par $\epsilon = E\beta\sigma$ et $\lambda=\beta^2 \sigma B$. À nouveau, dans la limite thermodynamique, seul l'état fondamental $E_0$ contribue à la distribution des hauteurs $p(h) = \psi_{E_0}^2(h)$.
Les solutions sont données par les fonctions d'ondes symmétriques par rapport à $h=0$
\begin{align}
	\psi_\epsilon (h) = Ai \left( (2\lambda)^\frac{1}{3}(|h|-\frac{\epsilon}{\lambda}) \right)
\end{align}
où $Ai(x)$ est la fonction de Airy. 

\begin{figure}[t]
	\begin{minipage}[t]{0.5\linewidth}
        \includegraphics[width=\linewidth]{sosequi-laser/etats-laser.pdf}
	\end{minipage}%
	\begin{minipage}[t]{0.5\linewidth}
    	\includegraphics[width=\linewidth]{sosequi-laser/energies-laser.pdf}
	\end{minipage}
    \caption{À gauche, états propres $\psi_n$ avec en noir, une référence par rapport à la distribution gaussienne. À droite, la série $\alpha_n$.}
	\label{fig-airy}    
\end{figure}

Par analogie avec l'oscillateur harmonique quantique, nous cherchons des solutions avec des conditions aux limites $\psi'_\epsilon(0) = 0$ pour les états pairs et $\psi_\epsilon(0) = 0$ pour les états impairs. Cela nous donne $\epsilon_n = 2^{-\frac{1}{3}} \lambda^\frac{2}{3}\alpha_n$ où $-\alpha_{2n} \greater 0$ est le 2n-ième zéro de la dérivée de la fonction d'Airy Ai' et $-\alpha_{2n+1} \greater 0$ est le (2n+1)-ième zéro de la fonction d'Airy Ai (voir figure \ref{fig-airy}). L'état fondamental est donné par le plus petit zéro de la fonction $\alpha_0 \simeq 1.0187$ et a une énergie de 
\begin{align}
	E_0 = \frac{\epsilon_0}{\beta \sigma} = 2^{-\frac{1}{3}} \alpha_0 \beta^\frac{1}{3}\sigma^{\frac{1}{3}}B^\frac{2}{3}
\end{align}
L'état fondamental s'écrit alors
\begin{align}
	\psi_0(h) = \frac{ Ai ( (2\lambda)^\frac{1}{3} |h|-\alpha_0 )}{ \sqrt{2 \int_0^\infty dh Ai^2 ( (2\lambda)^\frac{1}{3} |h|-\alpha_0 ) }}
\end{align}
où le dénominateur est une constante de normalisation utilisant la symétrie $p(h)=p(-h)$.  
On peut calculer les états excités suivants suivant leur parité, les états pairs s'écrivant
\begin{align}
	\psi_{2n}(h) = \frac{ Ai ( (2\lambda)^\frac{1}{3} |h|-\alpha_{2n} )}{ \sqrt{2 \int_0^\infty dh Ai^2 ( (2\lambda)^\frac{1}{3} h-\alpha_{2n} ) }}
\end{align}
et les états impairs
\begin{align}
	\psi_{2n+1}(h) = \frac{ \sgn(h) Ai ( (2\lambda)^\frac{1}{3} |h|-\alpha_{2n+1} )}{ \sqrt{2 \int_0^\infty dh Ai^2 ( (2\lambda)^\frac{1}{3} h-\alpha_{2n+1} ) }}
\end{align}
d'énergie $E_{n} = E_0 \frac{\alpha_{n}}{\alpha_0}$. 

On peut adimensionner la distribution des hauteurs par $z = (2\lambda)^\frac{1}{3}h$, et on peut définir une largeur caractéristique de l'interface 
\begin{align}
	\xi_\perp = \frac{1}{(2\beta^2 \sigma B)^\frac{1}{3}}
	\label{xi_perp}
\end{align}


Dans la limite thermodynamique, la distribution des hauteurs  devient 
\begin{align}
	p(z) = \psi_0^2(z) = \frac{ Ai^2 ( |z|-\alpha_0 )}{ 2 \int_0^\infty dz Ai^2 ( z-\alpha_0 ) }
	\label{airy}
\end{align}
	
\begin{figure}[t]
	\begin{minipage}[t]{0.5\linewidth}
		\includegraphics[width=\linewidth]{sosequi-laser/histo.pdf}
	\end{minipage}%
	\begin{minipage}[t]{0.5\linewidth}
		\includegraphics[width=\linewidth]{sosequi-laser/loghisto.pdf}
	\end{minipage}
	\caption{ Distribution de l'interface à $\beta = \beta_C$ autour d'un système centré à $L_Y=400$ en trait plein avec le fit selon la distribution de Airy \ref{airy} en échelle normale (à gauche) et en échelle log (à droite). Les écarts aux grandes fluctuations sont dues à un temps d’échantillonnage trop faible ($10^8$ MC steps) par rapport au temps de corrélation ($T_{cor} \simeq 100$), ce qui ne donne qu'environ $10^6$ états décorrélés. Par comparaison, à haut champ magnétique, le temps de corrélation est $T_{cor} \simeq 2$.}
	\label{histo_airy}
\end{figure}

Lorsque le paramètre d'ordre est conservé, la contrainte des modes de fluctuations dans la fonction de partition change radicalement les propriétés de l'interface. Dans la figure \ref{airy-eq-kae}, on constate la disparité à température et champ magnétique donné entre les ensembles statistiques, dans la limite thermodynamique. {\color{red}  Il semble que l'équivalence des ensembles thermodynamiques ne soit pas vérifiée dans ce système.} Il est possible de trouver une correspondance avec une température et un champ magnétique effectifs différents, mais les propriétés de décroissance à grande distance ne respectent cependant pas la distribution \ref{airy}. 
\begin{figure}
    \centering
	\includegraphics[width=0.5\linewidth]{sosequi-laser/comp-airy-kaw-glau.pdf}
	\caption{Comparaison de la distribution de l'interface entre une dynamique de Glauber et une dynamique de Kawasaki avec conditions aux bords périodiques en $x$ et une géométrie infinie en $y$, avec $L_X=5142$ (la longueur de corrélation parallèle à l'interface est de $\xi_\parallel = 37$ via diagonalisation de la matriice de transfert) pour $T=3$ et $B=0.01$. }
	\label{airy-eq-kae}
\end{figure}

%%%%%%%%%%%%%%%%%%%%%%%%%%
    \subsection{Fonction de corrélation}
%%%%%%%%%%%%%%%%%%%%%%%%%%

On peut également calculer la fonction de corrélation à deux-points. 
D'après l'éq \ref{schro_temp}, l'énergie de chaque état est une longueur caractéristique de chaque mode, $E_0$ étant la plus importante contribution au système. La différence d'énergie entre deux états consécutifs nous donne l'inverse de La longueur de corrélation parallèle à l'interface 
\begin{align}
	\xi_\parallel = \frac{1}{\Delta E} \simeq    2^\frac{1}{3}  \beta^{-\frac{1}{3}} \sigma^{\frac{1}{3}}B^{-\frac{2}{3}}
\end{align}


Dans la limite thermodynamique $L$ grand, la fonction de corrélation  à l'interface est
\begin{align}
	C_f(r) &= \langle f(h(0))f(h(r))\rangle - <f(h(0))><f(h(r))>
\end{align}
Puisque nous sommes dans la limite thermodynamique, la fonction de partition devient $\mZ \simeq e^{-E_0 L}$ et $<f(h(0))> = <f(h(r))> = \int_{-\infty}^\infty dh f(h) \psi_0(h)^2 $.
On obtient alors que
\begin{align}
	C_f(r) &=  \sum_{n\neq 0}\left[\int_{-\infty} ^\infty dh\  f(h)\psi_n(h)\psi_0(h)\right]^2 e^{-(E_n-E_0) r}  \nn
\end{align}
avec $E_n-E_0 = \frac{\alpha_n-\alpha_0}{\xi_\parallel}$. En particulier, si
\begin{align}
	f(h) = {\rm sign}(h-y)
\end{align}
alors $C_f(r)=C(y,r)$ est la fonction de corrélation spin-spin mesurée parallèlement à l'interface à la hauteur $y$. On peut décomposer l'intégrale en deux parties, et grâce à un changement de variable obtenir
\begin{align}
	\int_{-\infty} ^\infty dh\  f(h)\psi_n(h)\psi_0(h)=  2\int_y ^\infty dh\  \psi_n(h)\psi_0(h)
\end{align}
Puisque les $\psi_n$ sont des fonctions d'onde orthogonales répondant à l'équation de Schrödinger, l'intégrale pour $n\neq 0$ est
\begin{align}
	I(n,y)&= \int_y^\infty dh\psi_n(h)\psi_0(h)  \nn
	&= \frac{1}{2}\frac{\psi_0(x)\psi'_n(y) - \psi_n(x)\psi_0'(y)}{\epsilon_n-\epsilon_0}
\end{align} 
Comme précédement, on notant $z= \frac{y}{\xi_\perp}$, on simplifie la fonction de corrélation en
\begin{align}
	C(z ,r) = \sum_{n\neq 0} \frac{\left[ Ai(|z|-\alpha_0)Ai'(|z|-\alpha_n) -Ai(|z|-\alpha_n)Ai'(|z|-\alpha_0) \right]^2}
{ \int_0^\infty dz Ai^2 ( z-\alpha_0 )2 \int_0^\infty dz Ai^2 ( z-\alpha_n ) (\alpha_n-\alpha_0)^2}e^{-[\alpha_n-\alpha_0] \frac{r}{\xi_{||}}}
\end{align}

La constante de normalisation peut être encore simplifiée. L'intégration par partie donne
\begin{align}
	N_n = \int_0^\infty dz Ai^2 (z -\alpha_n) = [z Ai^2 (z -\alpha_n)]_0^\infty - 2\int_0^\infty dz  z Ai(z -\alpha_n)Ai'(z -\alpha_n)
\end{align}
Le terme aux limites est nul, et en utilisant l'équation d'Airy 
\begin{align}
	Ai''(z) -zAi(z)=0 \implies z Ai(z-\alpha_n) = Ai''(z-\alpha_n) + \alpha_n Ai(z-\alpha_n)
\end{align}
on obtient
\begin{align}
	N_n &= -2\int_0^\infty dz [Ai''(z -\alpha_n)+\alpha_n Ai(z -\alpha_n)] Ai'(z -\alpha_n)  \nn
	&= Ai'^2(-\alpha_n) + \alpha_n Ai^2(-\alpha_n).
\end{align}
où l'on rappelle qu'à cause de les modes pairs et impairs, on a posé $Ai(-\alpha_{2n+1})=0$ et $Ai'(-\alpha_{2n})=0$.
Cela nous donne au final
\begin{align}
	C(z,r) = \frac{1}{\alpha_0 Ai^2(-\alpha_0)} \sum_{n} \frac{\left[ Ai(|z|-\alpha_0)Ai'(|z|-\alpha_n) -Ai(|z|-\alpha_n)Ai'(|z|-\alpha_0) \right]^2}
{(Ai'^2(-\alpha_n) + \alpha_n Ai^2(-\alpha_n))  (\alpha_n-\alpha_0)^2}e^{-[\alpha_n-\alpha_0] \frac{r}{\xi_{||}}}
\end{align}

De plus, puisque $C(0,0)=1$, on démontre l'identité suivante entre les zéros de la fonction de Airy (et non de la dérivée)
\begin{align}
\sum_{n=0}^\infty \frac{1}{(\alpha_{2n+1}-\alpha_0)^2\alpha_0} = 1.
\end{align}

À grande distance, seul le terme premier état excité $n=1$ contribue à la fonction de corrélation, ce qui nous donne
\begin{align}
C(z,r) \approx \frac{1}{\alpha_0 Ai^2(-\alpha_0)}  
        \frac{\left[Ai(\frac{|z|}{\xi_{\perp}} -\alpha_{0})Ai'( \frac{|z|}{\xi_{\perp}}-\alpha_{1})-Ai(\frac{|z|}{\xi_{\perp}} -\alpha_{1})Ai'(\frac{|z|}{\xi_{\perp}} -\alpha_{0}) \right]^2}
        { Ai'^2(-\alpha_{1}) (\alpha_1-\alpha_0)^2}\exp(-[\alpha_1-\alpha_0] \frac{r}{\xi_{||}}).
\end{align}

La fonction de corrélation possède donc une décroissance de la forme
\begin{align}
    C(z,r) \approx A(\frac{z}{\xi_\perp})\exp(-[\alpha_1-\alpha_0] \frac{r}{\xi_{||}})
\end{align}
où $A(x)$ est une amplitude dépendante de la hauteur $z$ par rapport à la hauteur moyenne de l'interface, avec une longueur de corrélation  $\xi_{||}$ à grande distance indépendante de la hauteur. Qui plus est, cette longueur de corrélation peut se calculer via les plus grandes valeurs propres de la matrice de transfert $T_{ij} = J (h_i-h_j)^2 + B \frac{|h_i-\frac{L'}{2}|+|h_j-\frac{L'}{2}|}{2}$ grâce à l'équation \ref{longueur-correl-thermo}, ce qui nous donne à très grande distance la relation
\begin{align}
    \xi_{||} = -  \frac{1}{(\alpha_1-\alpha_0) \ln\left(\frac{\lambda_1}{\lambda_0} \right)} = 2^\frac{1}{3}   \beta^{-\frac{1}{3}} \sigma^{\frac{1}{3}}B^{-\frac{2}{3}}
    \label{xi_parallel}
\end{align}

%%%%%%%%%%%%%%%%%%%%%%%%%%
    \subsection{Tension superficielle effective}
%%%%%%%%%%%%%%%%%%%%%%%%%%

\begin{figure}
	\begin{minipage}[t]{0.5\linewidth}
		\includegraphics[width=\linewidth]{sosequi-laser/sigma-perp.pdf}
	\end{minipage}%
	\begin{minipage}[t]{0.5\linewidth}
		\includegraphics[width=\linewidth]{sosequi-laser/sigma-para.pdf}
	\end{minipage}
	\caption{Vérification des équations \ref{xi_perp} (gauche) et \ref{xi_parallel} (droite) en montrant la tension superficielle $\sigma$ par rapport aux longueurs charactéristiques, calculées grâce à l'Hamiltonien gaussien \ref{hamil-gauss}, pour différentes températures et champs magnétiques, pour une matrice de transfert de talle $L=400$. Chaque couleur correspond à une température, chaque symbole correspond à une intensité du champ magnétique.}
	\label{tension-airy}
\end{figure}

Dans les sections précédentes, nous avons démontré l'existence de deux longueurs de corrélation. La première, $\xi_\perp$, est définie comme la largeur de la distribution d'Airy de l'interface \ref{airy}. La deuxième, $\xi_\parallel$, est définie comme l'inverse de l'écart typique d'énergie entre deux états. En inversant les équatons \ref{xi_perp} et\ref{xi_parallel} on obtient
\begin{align}
    \sigma = \frac{1}{2 \beta^2 B \xi_\perp^3}
\end{align}
et 
\begin{align}
    \sigma = \frac{\beta B^2 (\alpha_1-\alpha_0)^3 \xi_\parallel^3}{2} =  \frac{\beta B^2 (\alpha_1-\alpha_0)^3 }{2 \ln \left(\frac{\lambda_1}{\lambda_0} \right)} 
\end{align}
où $\lambda_0$ est la plus grande valeur propre de la matrice de transfert, et $\lambda_1$ la seconde plus grande valeur propre. Ces deux expressions permettent de croiser les résultats afin de mesurer la tension superficielle effective. Dans la figure \ref{tension-airy}, calcule la valeur de $\sigma$ des deux manières. 
Afin d'obtenir une meilleure précision sur le fit de la distribution, puisque celle-ci est extrêmement piquée autour de la moyenne, il convient de faire le fit sur le logarithme de la distribution afin de donner plus de poids aux valeurs éloignées. On remarque également que pour les températures trop faibles ou des champs trop élevés, la largeur de la dsitribution devient comparable à l'unité, loin de la limite de l'approximation continue $\xi_\perp \gg 1$. Dans le cas où le champ magnétique est trop faible, l'interface est trop faiblement confinée et ne respecte pas non plus la distribution d'Airy. Le plateau que l'on voit sur la figure correspond donc au domaine de validité de l'approximation de la limite continue, où $\sigma = 2 J$. 
Cette valeur de la tension superficielle est corroborée par l'étude de $\xi_\parallel$, qui offre une plus grande robustesse numérique mais demande plus de temps de calcul si l'on désire faire des simulations de Monte Carlo. 

%%%%%%%%%%%%%%%%%%%%%%%%%%%%%%%
    \section{Interface hors-équilibre}
%%%%%%%%%%%%%%%%%%%%%%%%%%%%%%%

La présence d'une force de radiation dans les liquides binaires présentés en introduction induisent un écoulement local des liquides. Cet écoulement est analogue à la décantation de particules dans un solvant, et a pour propriété d'être uniforme. Dans le cas où les deux phases sont séparées et que l'une n'est pas impactée par l'écoulement, on peut considérer un cisaillement que nous définirons plus tard.
Plusieurs études convergent pour dire que la largeur de l'interface d'un tel système diminue avec l'intensité du cisaillement dans les modèles d'Ising \cite{smith_interfaces_2008,smith_interfaces_2008-1}.

Nous avons dans la figure \ref{comp-potentiels-chimiques} que la dynamique de Kawasaki ne peut pas être directement comparée à la dynamique de Glauber pour trouver une tension superficielle directement. Cependant, on peut extraire grâce à l'énergie le comportement global de l'interface vis-à-vis du cisaillement.


Nous nous intéressons au cas d'un écoulement uniforme d'un fluide. L'énergie associée à un écoulement vers la droite du site $i$ au site $j$ est
\begin{align}
    \Delta E_{ec} = f \sgn(i-j)
    \label{ecoulement-uniforme}
\end{align}
Ce genre de cas se retrouve lorsque le vent est en contact avec les vagues ou des nuages, créant une instabilité de Kelvin-Helmoltz augmentant la largeur moyenne de l'interface. Puisque le mouvement induit par l'écoulement peut aller de gauche à droite ou de droite à gauche sans que cela n'affecte l'énergie totale du système, on s'attend à ce que l'énergie soit une fonction paire en fonction de $f$, c'est-à-dire 
\begin{align}
    E(f) = E_{eq} + \frac{f^2}{2} E''(f) + \frac{f^4}{4!} E^{(4)}(f) + ...
\end{align}

\begin{figure}[t]
	\begin{minipage}[t]{0.5\linewidth}
		\includegraphics[width=\linewidth]{sosequi-laser/ene-kaw-airy.pdf}
	\end{minipage}%
	\begin{minipage}[t]{0.5\linewidth}
		\includegraphics[width=\linewidth]{sosequi-laser/sigma-kaw-airy.pdf}
	\end{minipage}%
	\caption{Énergie par site du modèle gaussien dans une dynamique de Kawasaki pour $T=4$ et $B=0.01$ en fonction de la force \ref{ecoulement-uniforme} -en pointillé un fit avec une puissance carrée (gauche). Largeur de l'interface par site $\frac{1}{L_X}\sum_i |h_i|^2$ en fonction de l'écoulement (droite). {\color{red} Besoin MCIA pour propre,Chip/AiryDrive}}
	\label{ene-kaw-airy}
\end{figure}

Dans la figure \ref{ene-kaw-airy}, on remarque qu'un fit en puissance carrée de dérivée seconde positive explique le comportement général en fonction du cisaillement. On remarque que l'injection d'énergie via l'écoulement vient augmenter l'énergie totale du système, et donc la largeur de l'interface. 
Contrairement au modèle d'Ising, il n'existe ici aucun mécanisme de dissipation de l'énergie. On peut immaginer que dans le modèle d'Ising, la raison pour laquelle le cisaillement diminue la largeur de l'interface \cite{smith_interfaces_2008}, c'est-à-dire diminue la température effective du système \cite{cirillo_monte_2005,winter_finite-size_2010}, est que le cisaillement accélère l'évaporation des cluster. 
Ici, l'énergie est injectée directement au niveau de l'interface et non au bulk, ce qui a tendance, comme avec les vagues, à augmenter la rugosité. 

Si l'on désire instaurer dans nos simlations un écoulement de Couette ou cisaillement, c'est-à-dire une force $\Delta E_{sh}$ qui dépend de la hauteur à laquelle s'effectue le mouvement, il faut insérer dans le modèle une notion de bulk absente des modèles SOS. Nous proposons au prochain chapitre un nouveau modèle capable de prendre en compte différents types de cisaillement.

%%%%%%%%%%%%%%%%%%%%%%%%%%
    \section{Conclusion}
%%%%%%%%%%%%%%%%%%%%%%%%%%

L'interface d'un système définit par un Hamiltonien Solid-On-Solid est une approximation directe du modèle d'Ising, mais décrit mal les systèmes macroscopiques. Afin d'étudier les systèmes continus, nous avons utilisé un Hamiltonien gaussien dont l'intégration est facilement faisable. En utilisant un champ magnétique confinant analogue à l'action d'un laser forçant une phase dans une autre dans les expériences menées au LOMA \cite{girot_conical_2019}, nous avons trouvé que les propriétés de l'interface dans l'ensemble grand-canonique étaient définies par une superposition de modes de Airy. Grâce aux deux longueurs charactéristiques du système étudié, nous avons vérifié l'équivalence entre le modèle continu et le modèle discret puisque $\sigma = 2 J$ dans une grande plage de température et de champ magnétique. Les cas où la correspondance n'est plus valable correspond aux cas où les longueurs charactéristiques sont proches de l'unité, c'est-à-dire que le système est sensible à la discrétisation du système. 
De plus,l es systèmes dans l'ensemble canonique, étudiés grâce à la dynamique de Kawasaki, présentent des propriétés très différentes de celles de l'ensemble grand-canonique. Il peut être intéressant d'explorer la raison pour laquelle l'équivalence des ensembles thermodynamiques est brisée dans ce contexte. L'ajout d'un écoulement uniforme augmente l'énergie et la largeur de l'interface, contrairement aux études sur les modèles d'Ising. Il serait intérssant de vérifier l'hypothèse selon laquelle la diminution de la largeur de l'interface dans les modèles d'Ising est due à l'évaporation accélérée des clusters du bulk.


%\subsection{Discussion about the Gaussian model}
%
%
%
%Pour rappel, à chaque étape on essaie de transférer une particule du site $i$ vers son voisin (gauche ou droit) $j$, ce qui se traduit par $h_i' = h_i-1$ et $h_j' = h_j+1$. 
%Pour $f=0$, en remarquant que les valeurs absolues ont les propriétés suivantes
%\begin{align}
%	|a \pm 1| - |a| = \pm 1 \\
%	|a \pm 2| - |a| = {0,\pm 2}
%\end{align}
%nous voyons facilement l'émergence d'une sélection des énergies possibles entre deux micro-états successifs dans ${-4,-2,0,2,4}$. Ainsi, toutes les transformations diminuant l'énergie totale du système seront toujours acceptées. En augmentant le cisaillement il devient alors possible de refuser des états réduisant l'énergie et d'accepter ceux qui l'augmente. 
%Nous nous attendons alors à trois régimes différents :
%\begin{itemize}
%	\item $f  \less  2 J $ : à faible cisaillement, la symmétrie du système impose les observables à être paire vis-à-vis de $f$, comme le prouve le fit en carré des figures.
%	\item $2 J \less f \less 4 J$ : à cisaillement moyen, certains mouvements augmentant la rugosité de l'interface sont toujours acceptés. 
%	\item $f > 4 J$ : à haut cisaillement, tous les mouvements augmentant l'entropie du système sont acceptés. Une saturation du système se produit lorsque l'énergie de lien entre les sites devient négligeable face au cisaillement.
%\end{itemize}
%
%
%La construction naïve d'un modèle continu du cisaillement avec un seul type de particules ne donnera aucun résultat. En effet, pour que le cisaillement induise des effets hors équilibre, il faut que la dynamique des particules dans tout repère galilén soit le même. Si l'on considère une force de cisaillement uniforme qui induit la même vitesse moyenne sur toutes les particules du système, en nous plaçant dans un repère bougeant à la même vitesse que cette vitesse moyenne, nous retrouvons les mêmes propriétés à l'équilibre.
%Afin de briser la symmétrie de translation, il faut soit induire un cisaillement non-uniforme, soit introduire des particules qui réagissent de manière différente vis-à-vis de cette force. Dans notre exemple sur les colloïdes, la gravité agit bien sur les polymères mais bien moins sur le solvant, ce qui brise en effet l'invariance galiléenne. 
%Plusieurs études récentes portent sur le mouvement de systèmes avec plusieurs particules browniennes, incluant le problème des électrolytes étudié par Onsager \cite{onsager} il y a longtemps.
%
%\begin{figure}[h]
%	\includegraphics[width=\linewidth]{./sosequi-laser/sosj1.pdf}
%	\caption{Énergie $E= \langle \sum_i |h_i-h_i+1| \rangle$ (en pointillé sa dérivée), variance $\sigma = \langle (H - \langle H \rangle )^2  \rangle$ et asymétrie $\gamma = \langle (H - \langle H \rangle )^3  \rangle / \sigma^2$ pour $B^0.1$. La magnétisation est constante et égale à $\langle H \rangle = 4.51$. Le temps de corrélation du système est presque constant en fonction du cisaillement $f$, allant de  $\tau(f=0) = 5.04$ à $\tau(f=6) = 5.00$ étapes de Monte Carlo. On note une brisure à $f=2J$ et $f=4J$.
%Les croix notent un fit en carré pour petit $f$, montrant la symétrie du système par inversion du signe de $f$. }
%\end{figure}
%
%\begin{figure}
%	\includegraphics[width=\linewidth]{./sosequi-laser/j13.pdf}
%	\caption{Same as before, with $J=1.3$ We observe a net inflexion in the energy at $2 J$ for  $B=0.01$ which corresponds to $<H>=11$. Nvertheless there is not threshold at $4 J = 5.2$. My guess is that the system is too far away from the boundary in order to interact strongly with it. Interstingly enough, for $B=0.1$, $<H>=3$ and we are too close to the boundaries to see anything.}
%\end{figure}
%
%Instabilité Kelvin-Helmoltz
%	
%The Gaussian model has a stronger interaction, been as
%\begin{align}
%	\Delta E = J \sum (h'_i-h'_{i+1})^2 -(h_i-h_{i+1})^2+ f (i-j)
%\end{align}
%In this model the bond energy between two microstates can take any integer, as 
%\begin{equation*}
%	(h_i-h_j+2)^2 - (h_i-h_j)^2 = 4 (h_i-h_j+1)
%\end{equation*}
%
%The gaussian interaction is very strong, so we could expect a very smooth interface. The mean difference between two sites should be about $h_i-h_{i+1} \simeq 1$. In that case, the energy difference is discretized as ${-8,-4,0,4,8}$. 
%Nonetheless we cannot predict exactly the same behaviour as in the SOS model because this approximation has to be verified everytime, which is false. 1
%
%\begin{figure}
%	\includegraphics[width=\linewidth]{./sosequi-laser/gauss0.pdf}
%	\caption{Bond energy, thickness (variance) and skewness of the interface for two different magnetic pressures. The magnetisation is constant and is equal to $<H>=2$ for $B=0.1$. As we are very close to the boundary, we see no threshold with the drive force. Simulations take longer with this model because the interaction is stronger}
%\end{figure}

\chapter{Modèle Particle-Over-Particle }



	\section{Particules indiscernables}
	Différence dans la fonction de partition en général


	\section{Avec le SOS}
	Le modèle Solid-On-Solid est l'approximation standard du modèle d'Ising car elle est étudiable analytiquement via sa matrice de transfert. 	Tout comme lorsque l'on déforme un flan seules les particules à l'interface flan-air peuvent bouger, dans le modèle SOS les particules loin de l'interface entre les deux phases ne peuvent bouger via l'agitation thermique. 
	Nous proposons alors un modèle un peu plus physique, dans lequel chaque particule a le droit de bouger. Au lieu de ne considérer que la hauteur $h_i$ au site $i$, nous considérons qu'il existe $h_i$ particules empilées les unes sur les autres. Lors d'un déplacement, nous prenons une particule au hasard pour la déplacer vers le haut de la pile, puis vers un autre site $j$. 
	Ainsi la fonction de partition devient
	
\begin{equation}
	Z = \sum_{h_1 h_2 ... h_L} e^{- \beta \sum_{i} H(i,i+1)} \frac{N!}{\prod_i n_i!} = N! \sum_{h_1 h_2 ... h_L} e^{- \beta \sum_{i} H(i,i+1) -\sum_i \ln(n_i!)}
\end{equation}

La matrice de transfert symétrisée devient donc
\begin{equation}
	T(h,h') = e^{-J |h-h'| - \frac{1}{2}(\ln(h!)-\ln(h'!)}
\end{equation}
où les termes $\ln(h)$ proviennent de l'entropie générée par la présence des particules au sein même des sites. À notre connaissance, ce modèle n'a pas été aussi étudié que le modèle SOS bien qu'il soit physiquement plus proche du modèle d'Ising initial. Le fait que la matrice de transfert ne soit pas résolvable analytiquement en est peut-être la cause. 
		\subsection{Modifications de l'algorithme Metropolis}
		au lieu de choisir un site, on choisit une particule, càd un site avec une proba pondérée.



	\section{Résultats modèle A}
	comment implémenter modèle A sur POP ? Différences avec SOS ?
	\section{Résultats modèle B}
	différences pour même hauteur moyenne, donner la distribution de hauteurs 
	on en déduit quoi ? 
	Mettre courbes de l'effet casimir, c'est pas mal
	\section{Résultats modèle A+B}
		certaines particules soumises à A, certaines à B. 
\chapter{A+B POP model : la totale}
\label{chap-article-dean}

Précédement, nous avons vu que l'écoulement injectée de l'énergie dans l'interface ne peut y être dissipée par des mécanismes d'évaporation, augmentant ainsi la largeur moyenne de l'interface. Nous avons également utilisé la présence de plusieurs types de particules afin de briser l'invariance par translation dans un repère galiléen par rapport à la vitesse moyenne induite par l'écoulement uniforme afin d'étudier les statistiques hors-équilibre.

Grâce à l'approche par la fonctionnelle de densité stochastique (SDFT) \cite{dean1996}, nous étudions dans ce chapitre la possibilité de dispersion de l'énergie injectée par l'écoulement, et nous expliquons analytiquement pourquoi l'interface devient plus rigide.
Afin de simplifier un peu les équations, nous allons supposer deux champs suivant le modèle C (selon la classification de Hohenberg et Halperin\cite{hohenberg_theory_1977}). D'un côté nous avons les colloïdes, représentés par un champs possèdant une dynamique conservée et se déplaçant à vitesse constante. De l'autre côté nous avons un couplage avec un réservoir d'un solvant. 

    \section{Le modèle C=A+B}
    
Soient deux champs scalaires $\psi$ et $\phi$ représentant les valeurs moyennes mésoscopiques de deux types de particules différentes. On suppose que le système possède deux phases stables avec une concentration moyenne du champ $\phi(\vec{x})$ données par $\psi_1$ et  $\psi_2$, la différence du paramètre d'ordre entre les deux phases étant $\Delta\psi= \psi_2 -\psi_1\greater 0$. On suppose également que $\phi(\vec{x},z\rightarrow -\infty)=\psi_2$ et $\phi(\vec{x},z\rightarrow \infty)=\psi_1$. On décompose l'hamiltonien deux parties, 
\begin{align}
    H[\psi,\phi] = H_1[\psi] +H_2[\psi,\phi]
\end{align}
Le premier Hamiltonien $H_1$, de la forme Landau-Ginzburg, correspond à l'énergie total du système
\begin{align}
    H_1[\psi]=\int d\vec{x}\left[\frac{\kappa}{2}[\nabla\psi(\vec{x})]^2 + V(\psi(\vec{x}))- gz \psi(\vec{x})\right].
\end{align}
où le premier terme représente la tension de surface et le troisième terme représente l'énergie due à la gravité. Ce même terme permet d'introduire une longueur de corrélation finie entre les deux phases. 

\begin{figure}
    \centering
    \includegraphics[width=0.5\linewidth]{pop/ab-phi.pdf}
    \caption{Schéma d'un écoulement uniforme parallèle à l'interface dépendant de la phase. Ici, on prend un modèle d'interface \ref{ab-interface} avec $\psi(\vec{x})=\psi_2$ en dessous de linterface et $\phi(\vec{x})=\psi_1$ au dessus. }
\end{figure}

Le second Hamitlonien $H_2$ est pris pour un couplage quadratique entre les deux champs
\begin{equation}
H_2 =\int d\vec{x} \frac{\lambda}{2}(1-\psi(\vec{x})-\phi(\vec{x}))^2,
\end{equation}
où l'on considère la conservation du volume totale des fractions des phases. Le champ $\phi$ peut se comprendre comme la fraction volumique locale d'un solvant d'un système de colloïdes. L'objectif de ce champ est de proposer des particules qui ne sont pas concernées par l'écoulement, peut-être parce qu'elles bougent à une échelle de temps bien plus longue. La présence de ce champ ne change cependant rien aux propriétés à l'équilibre du champ de colloïdes $\psi$. 
L'ajout de ce champ couplé permet de supprimer l'invariance galiléenne par rapport à la vitesse moyenne provoquée par l'écoulement. 

La fonction de partition s'écrit 
\begin{equation}
Z = \int d[\phi]d[\psi]\exp(-\beta H_1[\psi]- \beta H_2[\psi,\phi]) = CZ_{eff},
\end{equation}
{\color{red} expliciter l'étape d'intégration}
où $Z_{eff}$ est une fonction de partition effective ne dépendant que du champ $\psi$.
is the effective partition function for the field $\psi$, after we have integrated out the degrees of freedom corresponding to the field $\phi$,
and $C$ is a constant term resulting from this integration. The effective partition function is thus simply given by

La nouvelle fonction de partition est alors 
\begin{equation}
    Z_{eff} = \int d[\psi]\exp(-\beta H_1[\psi]),
\end{equation}
Comme expliqué plus haut, on voit que le champ $\phi$ n'a plus aucun effet sur les propriétés statistiques à l'équilibre du champ $\psi$.


On considère maintenant la dynamique de schamps. On prend la modèle de diffusivité local (modèle B selon la classification d'Hohenberg et Halpering) pour le champ $\psi$ tandis que l'on prend un modèle non conservé (modèle A) pour le solvant $\phi$.
\begin{align}
    \frac{\partial \psi(\vec{x},t)}{\partial t} +\vec{v}\cdot \vec{ \nabla}\psi(\vec{x},t)&= D\nabla^2\frac{\delta H}{\delta \psi(\vec{x})}+ \sqrt{2D T}\nabla \cdot {\bm \eta}_1(\vec{x},t) \\
    \frac{\partial \phi(\vec{x},t)}{\partial t} &= -\alpha\frac{\delta H}{\delta \phi(\vec{x})}+ \sqrt{2\alpha T}{ \eta}_2(\vec{x},t).
\end{align}
La première équation correspond à une diffusivitié locale $D$ et un terme d'advection par une champ de vitesse contstant $\vec{v}$ . La seconde équation correspond juste à l'équation du modèle A, avec un coefficient cinétique $\alpha$ correspondant au temps de relaxation du système vers l'équilibre. 


The second equation has no advection term and is simple model A dynamics. In principle we can also treat the case where the dynamics of the field $\phi$ is also diffusive and thus of model $B$ type, the analysis given here can be extended to this case but the analysis of the resulting equations is considerably more complicated. The use of model A dynamics for the solvent is justified by assuming that its dynamics is faster than that of the colloids and that the volume fraction can vary due to local conformational changes rather than  diffusive transport.

The noise terms above 
are uncorrelated and Gaussian with zero mean, their correlation functions are given by
\begin{eqnarray}
\langle \eta_{1i}(\vec{x},t) \eta_{1j}(\vec{x}',t)\rangle&=& \delta_{ij}\delta(t-t') \delta(\vec{x}-\vec{x}') \\
\langle \eta_{2}(\vec{x},t) \eta_{2}(\vec{x}',t)\rangle&=& \delta(t-t') \delta(\vec{x}-\vec{x}') ,
\end{eqnarray}
and $T$ is the temperature in units where $k_B=1$.
These dynamical equations  are thus explicitly given by
\begin{equation}
\frac{\partial \psi(\vec{x},t)}{\partial t} +\vec{v}\cdot { \nabla}\psi(\vec{x},t)= D\nabla^2[\frac{\delta H_1}{\delta \psi(\vec{x})}+\lambda(\phi(\vec{x},t) + \psi(\vec{x},t))]+ \sqrt{2D T}\nabla \cdot {\boldsymbol \eta}_1(\vec{x},t)
\end{equation}
and
\begin{equation}
\frac{\partial \phi(\vec{x},t)}{\partial t} = -\alpha\lambda[\phi(\vec{x},t) + \psi(\vec{x},t)]+ \sqrt{2\alpha T}{ \eta}_2(\vec{x},t).
\end{equation}
Taking the temporal Fourier transform, defined with the convention
\begin{equation}
\tilde F(\vec{x}, \omega) = \int_{-\infty}^\infty dt \exp(-i\omega t)F(\vec{x}, t),
\end{equation}
we can eliminate the field $\tilde \phi$ which is given by
\begin{equation}
\tilde \phi(\vec{x},\omega) = \frac{-\alpha\lambda \tilde \psi(\vec{x},\omega)+\sqrt{2\alpha T}\tilde \eta_2(\vec{x},\omega)}{i\omega +\alpha \lambda},
\end{equation}
this then gives the closed equation for $\tilde \psi$:
\begin{equation}
\left[1-\frac{\lambda D \nabla^2}{i\omega+\alpha\lambda}\right]i\omega \tilde\psi(\vec{x}, \omega) +\vec{v}\cdot\nabla\tilde\psi(\vec{x}, \omega)
= D\nabla^2 \tilde \mu(\vec{x},\omega) +  \tilde \zeta(\vec{x},\omega),
\end{equation}
where 
\begin{equation}
\mu(\vec{x},t)=\frac{\delta H_1}{\delta \psi(\vec{x},t)}
\end{equation}
is the effective chemical potential associated with the field $\psi$ and the noise term is given by
\begin{equation}
\tilde \zeta(\vec{x},\omega) = \frac{\sqrt{2\alpha T}D\lambda}{i\omega + \alpha\lambda}\nabla^2\tilde \eta_2(\vec{x},\omega) +
\sqrt{2DT}\nabla\cdot\tilde {\bm \eta}_1(\vec{x},\omega).
\end{equation}
Inverting the temporal Fourier transform then gives the effective evolution equation
\begin{equation}
\frac{\partial \psi(\vec{x},t)}{\partial t} -\lambda D\nabla^2\int_{-\infty}^t dt'
\exp(-\alpha\lambda(t-t')) \frac{\partial \psi(\vec{x},t')}{\partial t}+\vec{v}\cdot\nabla\psi(\vec{x}, t)=D\nabla^2  \mu(\vec{x},t') +  \zeta(\vec{x},t).\label{dyn1}
\end{equation}
\section{Effective interface dynamics}
We now follow the method of \cite{bray2001,bray2002} to derive the dynamical equation  for the interface between the two phases. It is assumed that the driving is in the ${\bf r}=(x,y)$ plane and that the system varies from phase $1$ to phase $2$ in the $z$ direction. The dynamical evolution for the field $\psi$ in Eq. (\ref{dyn1}) is first written as
\begin{equation}
\nabla^{-2}\left[\frac{\partial \psi(\vec{x},t)}{\partial t}+\vec{v}\cdot\nabla\psi(\vec{x}, t)\right] -\lambda D\int_{-\infty}^t dt'
\exp(-\alpha\lambda(t-t')) \frac{\partial \psi(\vec{x},t')}{\partial t'}=D  \mu(\vec{x},t') + \nabla^{-2} \zeta(\vec{x},t).\label{eqpsi}
\end{equation}
We now assume that the field $\psi$ can be written in the form
\begin{equation}
\psi(\vec{x},t) = f(z-h({\bf r},t)),
\end{equation}
and $f(z)\to \psi_2$ as $z\to -\infty$ and $f(z)\to \psi_2$ as  $z\to \infty$.
We now note the following results
\begin{eqnarray}
\frac{\partial f(z-h({\bf r},t))}{\partial t} &=& -f'(z-h({\bf r},t))\frac{h({\bf r},t)}{\partial t}\\
\nabla f(z-h({\bf r},t) )&=& [{\bf e}_z -\nabla h({\bf r},t)]f'(z-h({\bf r},t))]\\
\nabla^2 f(z-h({\bf r},t)) &=& f''(z-h({\bf r},t)[1 + [\nabla h({\bf r},t)]^2] -\nabla^2 h({\bf r},t)f'(z-h({\bf r},t)),
\end{eqnarray}
and thus we find
\begin{equation}
\mu(\vec{x},t)= -\kappa\left(f''(z-h({\bf r},t)[1 + \nabla^2 h({\bf r},t)] -\nabla^2 h({\bf r},t)f'(z-h({\bf r},t))\right) + V'(f(z-h({\bf r},t)) - gz .
\end{equation}
Multiplying both sides of the above by $f'(z-h({\bf r},t))$ yields
\begin{eqnarray}
&&f'(z-h({\bf r},t))\mu(\vec{x},t)=\nonumber \\
 &&-\kappa\left(f'(z-h({\bf r},t)f''(z-h({\bf r},t)[1 + \nabla^2 h({\bf r},t)] -\nabla^2 h({\bf r},t)f'(z-h({\bf r},t))^2\right) + V'(f(z-h({\bf r},t))f'(z-h({\bf r},t))\nonumber \\
 &&- gz f'(z-h({\bf r},t)) \nonumber
\end{eqnarray}
and then integrating over $z$ we obtain
\begin{eqnarray}
\int_{-\infty}^\infty dz f'(z-h({\bf r},t)\mu(\vec{x},t)&=& \kappa \nabla^2 h({\bf r},t)\int_{-\infty}^\infty dz\ f'(z-h({\bf r},t))^2 - \int_{-\infty}^\infty dz gz f'(z-h({\bf r},t))\nonumber \\&=&
\kappa\nabla^2 h({\bf r},t)\int_{-\infty}^\infty dz'\ f'(z')^2 - \int_{-\infty}^\infty dz' g(z' +h({\bf r},t)) f'(z')\nonumber \\
&=& \kappa\nabla^2 h({\bf r},t)\int_{-\infty}^\infty dz' \ f'(z')^2 -\Delta\psi g h({\bf r},t).
\end{eqnarray}
In the above we have assumed that $\int_{-\infty}^\infty dz' z'f'(z')=0$ by symmetry (this is also consistent with the approximation made later on in Eq. (\ref{eqdelta})). Furthermore one can show that \cite{bray2001,bray2002}
\begin{equation}
\kappa\int_{-\infty}^\infty dz' \ f'(z')^2 = \sigma,\label{mfsig}
\end{equation}
where $\sigma$ is the mean-field equilibrium Cahn-Hilliard estimate of the surface tension, obtained by  assuming that $f(z)=\psi_{MF}(z)$ is the equilibrium mean field profile of the field 
$\psi$. We thus find
\begin{equation}
\int_{-\infty}^\infty dz f'(z-h({\bf r},t)\mu(\vec{x},t) = \sigma[\nabla^2 h({\bf r},t)-m^2 h({\bf r},t)]
\end{equation}
where $m^2 = \Delta\psi g /\sigma$. We now carry out the same operation on the left hand side of Eq. (\ref{eqpsi}). First we have
\begin{eqnarray}
\nabla^{-2}\frac{\partial \psi(\vec{x},t)}{\partial t}&+&\vec{v}\cdot\nabla \psi(\vec{x},t) +\lambda D\int_{-\infty}^t dt'
\exp(-\alpha\lambda(t-t')) \frac{\partial \psi(\vec{x},t')}{\partial t'} = \nonumber \\ 
&-&\nabla^{-2}f'(z-h({\bf r},t))[\frac{\partial h({\bf r},t)}{\partial t} +\vec{v}\cdot\nabla h({\bf r},t)]  +\lambda D\int_{-\infty}^t dt'
\exp(-\alpha\lambda(t-t')) f'(z-h({\bf r},t'))\frac{\partial h({\bf r},t')}{\partial t'}\nonumber \\
&\approx& -\nabla^{-2}f'(z) [\frac{\partial h({\bf r},t)}{\partial t} +\vec{v}\cdot\nabla h({\bf r},t)] +\lambda D\int_{-\infty}^t dt'
\exp(-\alpha\lambda(t-t')) f'(z)\frac{\partial h({\bf r},t')}{\partial t'},\end{eqnarray}
where in the last line above we have neglected terms quadratic in $h$. 
Note that the neglecting of these additional terms is not strictly justified, they could potentially induce non-perturbative effects which render the surface fluctuations non-Gaussian. However we see here that the first order computation we carry out tends to reduce fluctuations with respect to equilibrium or non-driven interfaces and so if the equilibrium theory can be described by an equation which is linear in height fluctuations, it seems physically reasonable to assume that the the approximation also holds for the driven interface. 
Again, we multiply the above by $f'(z)$ and integrate over $z$. In the first term we make use of the approximation
\begin{equation}
f'(z)=\Delta\psi \delta(z)\label{eqdelta}
\end{equation}
and in the second we use the relation in Eq. (\ref{mfsig}). Putting this all together we obtain
\begin{equation}
\Delta\psi^2 \int d{\bf r} G(0,{\bf r}-{\bf r}') [\frac{\partial h({\bf r},t)}{\partial t} +\vec{v}\cdot\nabla h({\bf r},t)] +\frac{\sigma\lambda D}{\kappa}\int_{-\infty}^t dt'
\exp(-\alpha\lambda(t-t'))\frac{\partial h({\bf r},t')}{\partial t'}
= \sigma[\nabla^2 h({\bf r},t)-m^2 h({\bf r},t)] + \xi({\bf r},t),\label{em}
\end{equation}
where $G= -\nabla^{-2}$, or more explicitly
\begin{equation}
\nabla^2 G(z-z',{\bf r}-{\bf r}') = -\delta(z-z') \delta({\bf r}-{\bf r'}).
\end{equation}
The noise term $\xi$ is given by
\begin{equation}
\xi({\bf r},t) = \int_{-\infty}^{\infty} dz f'(z-h({\bf r},t)) \nabla^{-2} \zeta(\vec{x},t).
\end{equation}
Now, as the equations of motion have been derived to first order in $h$ and we wish to recover the correct equilibrium statistics for the non-driven system, we ignore the $h$ dependence in the noise and make the approximation
\begin{equation}
\xi({\bf r},t) \approx \int_{-\infty}^{\infty} dz f'(z) \nabla^{-2} \zeta(\vec{x},t).
\end{equation}
The correlation function of this noise is most easily evaluated in terms of its Fourier transform with respect to  space and time  defined by
\begin{equation}
\hat F({\bf q},\omega)=\int dt d{\bf r}\exp(-i\omega t -i{\bf q}\cdot{\bf r}) F({\bf r},t).
\end{equation}
Using the relations Eqs. (\ref{mfsig}) and (\ref{eqdelta}) one  can show that
\begin{equation}
\langle \hat \xi({\bf q},\omega)\hat \xi({\bf q}',\omega')\rangle 
=2T(2\pi)^d \delta(\omega +\omega') \delta({\bf q}+{\bf q}') \left[
\frac{\sigma}{\kappa}\frac{\alpha D^2\lambda^2}{\omega^2 +\alpha^2\lambda^2} + \frac{D\Delta\psi^2}{2q}\right].
\end{equation}
In full Fourier space the equation of motion for the field $\psi$ then reads
\begin{equation}
\left[i(\omega+{\bf q}\cdot\vec{v})\frac{\Delta\psi^2}{2q} + \frac{D\sigma\lambda}{\kappa} \frac{i\omega}{\alpha\lambda+i\omega}\right] \hat h({\bf q},\omega)= -D\sigma(q^2+m^2)\hat h({\bf q},\omega)+ \hat\xi({\bf q},\omega).\label{dyn}
\end{equation}

From this, the full Fourier transform of the correlation function of the interface height is given by
\begin{equation}
\hat C({\bf q},\omega)  = 2TD \frac{\left[ \frac{\Delta\psi^2}{2q}(\omega^2+\alpha^2 \lambda^2) + \frac{\sigma\alpha D\lambda^2}{\kappa}\right]}{\left|i[\frac{\alpha\lambda\Delta\psi^2}{2 q}(\omega + {\bf q}\cdot\vec{v}) + \frac{\lambda \sigma D}{\kappa}\omega + D\sigma(q^2+m^2)\omega]
+[\alpha\lambda D\sigma(q^2+m^2) -\frac{\Delta\psi^2}{2q}\omega(\omega+{\bf q}\cdot\vec{v})]\right|^2}.
\end{equation}
Using the above we can extract the equal time height-height correlation function in the steady states. Its spatial Fourier transform can shown to be given by
\begin{eqnarray}
\tilde C_s({\bf q}) &=& \frac{1}{2\pi} \int d\omega \hat C({\bf q}, \omega)\nonumber\\
&=&T \frac{\left(2 D\sigma q(\kappa[q^2+m^2]+\lambda)+\alpha\kappa\lambda\Delta\psi^2\right)^2 +\kappa^2 \Delta\psi^4 ({\bf q}\cdot\vec{v})^2}{\sigma[q^2+m^2]\left(2D q\sigma (\kappa[q^2+m^2]+\lambda)+\alpha \kappa\lambda \Delta\psi^2\right)^2 + \kappa\left(\kappa\sigma[q^2+m^2] + \lambda\sigma\right)\Delta\psi^4({\bf q}\cdot\vec{v})^2}.\label{eqmaind}
\end{eqnarray}
An outline of the derivation of this result is given in the Appendix to the paper.
In the absence of driving, {\em i.e.} when $\vec{v}={\bf 0}$ we recover the equilibrium correlation function
\begin{equation}
\tilde C_s({\bf q})= \tilde C_{eq}({\bf q})= \frac{T}{\sigma[q^2+m^2]},
\end{equation} 
here we see that  $1/m= \xi_{eq}$ is the so called capillary length, which is the equilibrium correlation length of the height fluctuations. We also notice that the correlation function for wave vectors perpendicular to the driving direction is simply the equilibrium one.

If we write $C_s({\bf q})= T/H_s({\bf q})$ we can interpret $H_s({\bf q})$ as an effective quadratic Hamiltonian for the height fluctuations, it is thus given by
\begin{equation}
H_s({\bf q}) = \sigma[q^2+m^2] + \frac{\kappa\lambda\sigma \Delta\psi^4 ({\bf q}\cdot\vec{v})^2}{\left(2 D\sigma q(\kappa[q^2+m^2]+\lambda)+\alpha\kappa\lambda\Delta\psi^2\right)^2 +\kappa^2 \Delta\psi^4 ({\bf q}\cdot\vec{v})^2}
\end{equation}
For small $q$ we find 
\begin{equation}
H_s({\bf q}) = \sigma m^2 + \sigma q^2(1+ \frac{v^2\cos^2(\theta)}{\alpha^2\lambda\kappa}),
\end{equation}
where $\theta$ is the angle between the wave vector ${\bf q}$ and the direction of the driving. 
This thus gives a direction dependent surface tension 
\begin{equation}
\sigma_s(\theta) = \sigma(1+ \frac{v^2\cos^2(\theta)}{v^2_0}),
\end{equation}
where we have introduced the intrinsic velocity $v_0 = \sqrt{\alpha^2\lambda\kappa}$ which depends on the microscopic {\em dynamical} quantity $\alpha$ associated with the model A dynamics of the field $\phi$, as well as the microscopic static quantities $\kappa$ (which generates the surface tension) and $\lambda$ the coupling between the field $\psi$ and $\phi$. This appearance of dynamical and static quantities that are otherwise hidden in equal time correlation functions in equilibrium is already implicit in the works of Onsager \cite{hem1996} where it is used to compute the conductivity of Brownian electrolytes and the explicit expressions were derived using stochastic density functional theory in \cite{dem2016}. We also note that the universal thermal Casimir effect between model Brownian electrolyte systems  driven by an electric field 
exhibits similar features, developing a dependency on both additional static and dynamical variables with respect to the equilibrium case \cite{dean2016}


However for this small $q$ expansion we see that the microscopic 
quantities $D$, the diffusion constant of the field $\phi$, and the order parameter jump
$\Delta\psi$ do not appear. 

From the above, we see that  in the direction of the driving the surface tension increases and the fluctuations of the surface are thus suppressed. We may also write 
\begin{equation}
H_s({\bf q}) = \sigma_s(\theta) [q^2 + m^2_e(\theta)],
\end{equation}
with 
\begin{equation}
m^2_s(\theta) =\frac{ m^2}{1+ \frac{v^2\cos^2(\theta)}{v_0^2}},
\end{equation}
this corresponds to a correlation length 
\begin{equation}
\xi_s = \xi_{eq}\sqrt{1+ \frac{v^2\cos^2(\theta)}{v_0^2}},
\end{equation}
and we see that it is increased in the direction of the driving. 

As we have just remarked  that the above results appear to be independent of the order parameter jump $\Delta \psi$ and the diffusion constant $D$, however the next order correction to $H_s$ for small $q$ is given by
\begin{equation}
H_s({\bf q}) = \sigma_s(\theta) [q^2 + m^2_e(\theta)] - \frac{4Dq \sigma^2(\lambda+\kappa m^2)( {\bf q}\cdot\vec{v})^2 }{\alpha^3 \kappa^2 \lambda^2 \Delta\psi^2},
\end{equation}
and so the small ${\bf q}$ expansion  breaks down at $\Delta\psi=0$, indeed one can see that the system has exactly the equilibrium correlation function when  $\Delta\psi=0$. 

In the limit of large $q$ we see that the effective Hamiltonian is given, to leading order, by the original equilibrium Hamiltonian and so the out of equilibrium driving has no effect on the most energetic modes of the system.

The results here predict that for unconfined surfaces the long range height fluctuations are described by an isotropic form of capillary wave theory with 
an anisotropic surface tension which is largest in the direction of driving. Numerical simulations of driven lattice gases in two dimensions \cite{leun1993} show a more drastic change upon driving and find $C_s(q)\sim  1/q^{.66}$ and thus a strong deviation from capillary wave theory.  
\section{A model of active interfaces}
We can apply the results derived in the previous section to analyse a simple model for
surfaces formed between two phases of active colloids. Activity is modelled by assuming that the colloidal field $\psi$ has a temperature different to that of  the solvent field $\phi$. This models the effect that activity leads to enhanced colloidal diffusivity over and
above the Brownian motion of particles due to thermal fluctuations \cite{gros2015}.

In the absence of any driving the dynamical equations for the field $\psi$ and $\phi$ become 
\begin{eqnarray}
\frac{\partial \psi(\vec{x},t)}{\partial t} &=& D\nabla^2\frac{\delta H}{\delta \psi(\vec{x})}+ \sqrt{2D T_1}\nabla \cdot {\bm \eta}_1(\vec{x},t) \\
\frac{\partial \phi(\vec{x},t)}{\partial t} &=& -\alpha\frac{\delta H}{\delta \phi(\vec{x})}+ \sqrt{2\alpha T_2}{ \eta}_2(\vec{x},t).
\end{eqnarray}
Following the same arguments as above we find that
\begin{equation}
\hat C({\bf q},\omega)  = 2D \frac{\left[ T_1\frac{\Delta\psi^2}{2q}(\omega^2+\alpha^2 \lambda^2) + T_2\frac{\sigma\alpha D\lambda^2}{\kappa}\right]}{\left|i\omega[\frac{\alpha\lambda\Delta\psi^2}{2 q} +  \frac{\lambda \sigma D}{\kappa} + D\sigma(q^2+m^2)]
+[\alpha\lambda D\sigma(q^2+m^2) -\frac{\Delta\psi^2}{2q}\omega^2]\right|^2}.
\end{equation}
The equal time steady state height fluctuations thus have correlation function
\begin{equation}
\tilde C_s(q) = \frac{T_1}{\sigma (q^2 + m^2)}\left[ 1 -(1-\frac{T_2}{T_1})\frac{\lambda\sigma } {\kappa }\frac{1}{\frac{\alpha\lambda \Delta \psi^2}{2Dq}+ \frac{\lambda\sigma }{\kappa} + \sigma(q^2+m^2)}\right].
\end{equation}
We see, again, that the inclusion of a non-equilibrium driving changes the statistics of height fluctuations and leads to a steady state that depends on both dynamical variables
$D$ and $\alpha$ as well as static ones $\Delta\psi,\ \lambda$ and $\kappa$ that remain hidden in the equilibrium case. This phenomenon is again seen in the behavior of the universal thermal  Casimir force between Brownian conductors held at different temperatures \cite{lu2015}.

If we assume strong activity we can take the limit $T_1\gg T_2$, in this case we find
\begin{equation}
\tilde C_s(q) = \frac{T_1}{\sigma (q^2 + m^2)}\frac{\frac{\alpha\lambda \Delta \psi^2}{2Dq}+
\sigma(q^2+m^2)}{\frac{\alpha\lambda \Delta \psi^2}{2Dq}+ \frac{\lambda\sigma }{\kappa} + \sigma(q^2+m^2)}.
\end{equation}
Interpreted in terms of an effective Hamiltonian for an equilibrium system at the temperature $T_1$ the above gives
\begin{equation}
H_s(q) = \sigma (q^2 + m^2)\left[1+\frac{\lambda\sigma }{\kappa}\frac{q}{\frac{\alpha\lambda \Delta \psi^2}{2D}+
q\sigma(q^2+m^2)}\right].
\end{equation}
In the case of an unconfined interface (where there is no gravitational effect
on the surface fluctuations) {\em i.e.} $m=0$ we see that for small $q$
\begin{equation}
H_s(q) \approx \sigma q^2 +\frac{2D\sigma^2 }{\kappa\alpha \Delta\psi^2}q^3 .
\end{equation}
We see that the effective surface tension is not modified but a reduction of fluctuations due to the presence of the term in $q^3$ arises.  As in the case of a driven system, we see that the large $q$ behavior of the effective Hamiltonian is given by the equilibrium case where $T=T_1=T_2$. 

In the case where the interface is confined, we see that for small $q$ one obtains
\begin{equation}
H_s(q) \approx \sigma m^2 \left[ 1+ \frac{2D\sigma }{\kappa\alpha \Delta\psi^2}q\right],
\end{equation}
and thus at the largest length scales of the problem there is a qualitative departure from capillary wave behavior, and the correlation length of height fluctuations at the largest length scales is given by
\begin{equation}
\xi_s = \frac{2D\sigma }{\kappa\alpha \Delta\psi^2}.
\end{equation}
The above result should be compared with that obtained in \cite{zia1991} for 
systems with anisotropic thermal white noise, which breaks detailed balance and mimics random driving of the system parallel to the interface; for free interfaces it was found that $C_s(q)\sim 1/q$.
\section{Conclusions}
We have presented a model to analyse the effect of uniform driving on the dynamics of the interface in a two phase system. In order to generate a non-equilibrium state a second {\em hidden} order parameter was introduced. This models the behaviour of a local or solvent degree of freedom which is not influenced by the driving field. In this way, we obtain out of equilibrium interface fluctuations which are described by Gaussian statistics as found in the experimental study of \cite{derks2006}. The agreement with this experimental study also extends to qualitative agreement with the increase of the effective surface tension in the direction of driving and also an increase in the correlation length of the height fluctuations with respect to a non-driven equilibrium interface. However, we  note that numerical simulations of a sheared Ising interface \cite{smith2008,smith2010} also reveal a reduction of interface fluctuations but the lateral correlation length is found to be reduced.

The basic idea underlying this study would be interesting to apply to a number of possible variants of this model, for instance both the dynamics
of the main field $\phi$ and the solvent field $\phi$ could be varied. To make a direct link with driven colloidal interfaces one should study model H type dynamics for the main field $\phi$ and other variants for the dynamics of the 
solvent field $\phi$ could also be considered. 

As mentioned above, in lattice based models driving induces non-equilibrium states even in the simple Ising lattice gas. A model analogous to that studied here can be formulated in a lattice based systems using the Hamiltonian 
\begin{equation}
H = -J\sum_{(ij)}S_i S_j (1+ \sigma_{(ij)}),
\end{equation}
where $S_i=\pm1$ are Ising spins at the lattice sites $i$, and $\sigma_{(ij)}=\pm 1$ are Ising like dynamical solvent variables associated with the lattice links $(ij)$. The static partition function is given by
\begin{equation}
Z = {\rm Tr}_{\sigma_{ij},S_i} \exp\left[\beta J\sum_{(ij)}S_i S_j (1+ \sigma_{(ij)})\right],
\end{equation}
and the trace over the solvent variables can be trivially carried out to give
\begin{equation}
Z = {\rm Tr}_{S_i}\left( \exp\left[\beta J\sum_{(ij)}S_i S_j \right]\prod_{(ij)}2\cosh(\beta JS_iS_j)\right )= [2\cosh(\beta J)]^L{\rm Tr}_{S_i}\exp(\beta J\sum_{(ij)}S_i S_j ),
\end{equation}
where $L$ is the number of links on the lattice of the model. We thus see that the underlying effective static model is precisely the zero field Ising model. 

This model can then be driven in a number of ways, for instance using conserved Kawasaki dynamics for the Ising spins to model diffusive dynamics in the presence of a uniform driving field parallel to the surface between the two phases at a temperature below the ferromagnetic ordering temperature $T_c$. The dynamics of the Ising spins on the lattice links can  be given by non-conservative single spin flip, for instance Glauber, dynamics to keep the analogy with the continuum model discussed in the paper but diffusive dynamics or indeed a mixture of diffusive and non-conserved dynamics 
could be implemented. It would be interesting to see to what extent this modification of the driven lattice gas model affects the non-equilibrium driven states that arise. 

It is also clear that this lattice model can be used to simulate the effect of activity where the Ising spins $S_1$ corresponding to the colloid field undergo  Kawasaki dynamics at the temperature $T_1$ where as the link variables $\sigma_{(ij)}$ undergo single spin flip non-conserved dynamics
at the temperature $T_2$.

\section{Acknowledgements}
The authors acknowledge support from the ANR (France) Grant FISICS \\
\appendix
\section{Evaluating Fourier integrals}
Here we outline how the Fourier integration leading to Eq. (\ref{eqmaind}) is carried out. Defining
\begin{equation}
I(f(\omega)) = \int \frac{d\omega}{2\pi} \frac{f(\omega)}{\left|i(A\omega + B) + (C-D\omega-E \omega^2)\right|}
\end{equation}
we see that the integral we need to evaluate can be written in the form
\begin{equation}
I = a I(\omega^2) + b I(1).
\end{equation}
The calculation leading to Eq. (\ref{dyn}) can be carried out in the presence of a forcing term on the height profile in order to compute the response function for the surface which has a denominator of the form
\begin{equation}
{\rm Den} = i(A\omega + B) + (C-D\omega-E \omega^2),
\end{equation}
and due to causality the above only has poles in the upper complex plane (due to the convention of Fourier transforms used here). Consequently we find that
\begin{equation}
\int \frac{d\omega}{2\pi} \frac{1}{i(A\omega + B) + (C-D\omega-E \omega^2)} = 0,\label{key}
\end{equation}
as one may close the integration contour in the lower half of the complex plane. Taking the real and imaginary part of Eq. (\ref{key}) leads to
\begin{eqnarray}
C I(1) -D I(\omega) - E I(\omega^2) = 0 \\
AI(\omega) + B I(1) = 0.
\end{eqnarray}
Using this we can express $I(\omega^2)$ as a function of $I(1)$, and explicitly we have 
\begin{equation}
I(\omega^2) = \frac{I(1)}{E}[C+ \frac{DB}{A}].
\end{equation}

To evaluate $I(1)$ we now use
\begin{equation}
I(1) = -{\rm Im} \int \frac{d\omega}{2\pi}\frac{1}{A\omega +B} \frac{1}{i(A\omega + B) + (C-D\omega-E \omega^2)}.
\end{equation}
The integrand above has no poles in the lower half of the complex plane but has a {\em half pole} at $\omega=-B/A$ on the real axis, thus using standard complex analysis we find
\begin{equation}
I(1) = \frac{1}{2(CA + BD - \frac{EB^2}{A})}.
\end{equation}
Then after some laborious, but straightforward algebra, the results Eq. (\ref{eqmaind}) is obtained.




The dynamics of discrete particle systems is however affected by uniform driving of identical particles. The study of driven lattice gases has revealed a wide range of intriguing physical phenomena and indeed shown how driving can even lead to phase separation \cite{katz1984,zia1991,leun1993,schm1995,schm1998}. The discrete nature of the dynamics of these systems, both in space and time, means that no Galilean transformation to an equilibrium state exists. Analytical studies of these systems require a phase ordering kinetics description in terms of a continuum order parameter. In order to break Galilean invariance the local mobility of the particles can be taken to be dependent on the local order parameter, this is then sufficient to induce non-trivial steady states under driving \cite{katz1984,leun1993,schm1995,schm1998}. Interfaces between the separated phases in uniformly driven systems have non capillary behaviors which are, even today, not fully understood \cite{leun1993}.
Taking random driving in a given direction also leads to non-equilibrium steady states, if the noise is Gaussian and white, the fluctuation dissipation theorem is violated and novel interface fluctuations are induced which, again, are  not of  the capillary type \cite{zia1991}. 



\input{intro/conclusion}


\bibliography{biblio.bib}
\end{document}

\begin{comment}
\chapter{Phase ordering kinetics}
\section{Introduction}
In this chapter we will analyse the dynamics of statistical systems. The analysis will allow us to understand how phase transitions occur dynamically. For instance we know that certain systems undergo what is known phase separation, for instance if we take the Ising model with zero magnetic field, in the high temperature phase the system is homogeneous and the average magnetisation, which is the order parameter for the transition is zero. Below the critical temperature if the overall magnetisation is conserved (for instance for Kawasaki dynamics), which would be the case if spins corresponded to different types of particles, the system will separate into two  phases of opposite average magnetisation, separated by an interface which will be roughly flat in order to minimise the surface energy between the two phases. For nonconserved systems, where the overall magnetisation in not conserved (for example Glauber dynamics), eventually one of the two phases will make up the system (spontaneous symmetry breaking). In a continuous phase transition as the critical point is reached from the disordered to ordered, domains of phases of positive and negative magnetisation form and the size of these domains is given by the correlation length of the system. For continuous phase transitions such as that in the Ising model the correlation length diverges as as the critical point is approached, for instance as $T\to T_c$ if the temperature is varied. The size of the domains thus have to become infinite if the system is infinite, this means that for an infinite system it will take an infinite time to relax to the equilibrium state. The process of domain growth is known as coarsening and phase ordering kinetics is the theory that has been developed to understand the phenomenon of coarsening. Furthermore, for systems with a conserved order parameter which separate into two phases, the two phases will be separated by an interface. This interface will be characterised by a surface tension, its average position will be fixed but it will exhibit fluctuations. Later we will see how model of phase ordering kinetics and be used to determine the static and dynamical properties of interfaces between two coexisiting phases. 

While the phase diagram of a system can be determined via 
its Hamiltonian and equilibrium statistical mechanics, the dynamics of coarsening depends on details of the systems dynamics that do not show up in single time thermodynamic observables. Therefore one needs to construct dynamical models that capture the underlying evolution of the state of the system, in particular there is a big difference between systems where the order parameter is conserved and those where it is not conserved.

\section{Statics of systems with a finite number of degrees of freedom}

Thermodynamic systems are naturally described in terms of fields, for example densities. This means that one is naturally lead to consider statistical field theories where the system is described in terms of a local field $\phi({\bf x})$. Statistical field theories can be applied to both statics, to understand phase diagrams, and dynamics to understand phase ordering. However to start with we will examine the case of systems with a finite number of degrees of freedom. 

Consider a system in the canonical ensemble with a Hamiltonian $H({\bf q})$ where $q_i$ for 
$1\leq i\leq N$ represent a finite number of continuous degrees of freedom. The partition function for the system is given by
\begin{equation}
Z = \int d{\bf q} \exp\left(-\beta H({\bf q})\right),
\end{equation}
and in equilibrium the probability density function $P_{eq}({\bf q})$ of the degrees of freedom is given by 
\begin{equation}
P_{eq}({\bf q}) = \frac{\exp\left(-\beta H({\bf q}\right)}{Z}.\label{eqdis}
\end{equation}
In general the integral which gives the  partition function cannot be computed analytically.
The simplest approximation to compute $Z$ is the mean field approximation where the integral 
is approximated by the integrand at its largest value - in mathematics this is the Laplace method for approximating an integral. The mean field approximation is thus
\begin{equation}
Z_{MF}= \exp\left(-\beta H({\bf q}^*)\right),
\end{equation}
where ${\bf q}^*$ is the value of ${\bf q}$ which minimises $H$ (note that the approximation becomes exact in the zero temperature limit - $\beta \to \infty$   - as the system will minimise its energy). The values $q_i^*$ are determined from
\begin{equation}
\frac{\partial H}{\partial q_i}|_{{\bf q}={\bf q^*}}=0.
\end{equation}
Within this approximation any thermodynamic observable is given by
\begin{equation}
\langle f({\bf q}) \rangle = f({\bf q}^*).
\end{equation}

We now consider how one can model dynamics of such systems. We will look for a Langevin equation which is chosen to give the correct equilibrium Gibbs-Boltzmann distribution. We write
\begin{equation}
\frac{d q_i}{dt} = -L_{ij}\frac{\partial H({\bf q})}{ \partial q_j} + \eta_i(t),
\end{equation}
where $L_{ij}$ is a matrix which discuss later and $\eta_i(t)$ is zero mean Gaussian white noise  with correlation function 
\begin{equation}
\langle \eta_i(t)\eta_j(t')\rangle =  \Gamma_{ij} \delta(t-t'),\label{cfn}
\end{equation}
The Gaussian white noise represents the effects of thermal fluctuations on the system we assume that the correlation time of these fluctuations is extremely short with respect to the dynamics of the degrees of freedom $q_i$ (in fact in critical systems the dynamics becomes very slow, critical slowing down, and this approximation becomes better and better as one approaches the critical point).  As Eq. (\ref{cfn}) is for a correlation function the matrix $\Gamma_{ij}$ must be symmetric and cannot have any negative eigenvalues.

In the absence of noise or thermal fluctuations, so at zero temperature, the system will simply minimise its energy. Therefore if 
\begin{equation}
\frac{\partial H({\bf q})}{ \partial q_j} =0, 
\end{equation}
with no noise we have $\frac{d q_i}{dt}=0$, that is to say it is the term $\frac{\partial H({\bf q})}{ \partial q_j}$ that drives the dynamics if there is no noise. As long as the matrix $L_{ij}^{-1}$ exists the zero temperature dynamics will take the system to the local minimum of $H$ and to the absolute minimum if there are no metastable configurations. 

Under these assumptions, the Fokker-Planck equation for the probability density function of the degrees of freedom is 
\begin{align}
\frac{\partial p({\bf q},t)}{\partial t} = \frac{\partial}{\partial q_i} \left[\frac{1}{2}\Gamma_{ij} \frac{\partial p({\bf q},t)}{\partial q_i} + p({\bf q},t) L_{ij}\frac{\partial H({\bf q})}{ \partial q_j}\right].
\end{align}
This can be written as 
\begin{align}
\frac{\partial p({\bf q},t)}{\partial t} +\frac{\partial}{\partial q_i}J_i({\bf q},t)=0,
\end{align}
where the ${\bf J}({\bf q},t)$ is the probability current. We now insist that the system is in equilibrium with zero current when $p({\bf q},t)= P_{eq}({\bf q})$ as given by Eq. (\ref{eqdis}), this gives
\begin{align}
\left[-\frac{\beta}{2}\Gamma_{ij} + L_{ij}\right]\frac{\partial H({\bf q})}{ \partial q_j},
\end{align}
and this holds for any choice of $H$ is we chose.
\begin{align}
\Gamma_{ij}= 2T L_{ij}
\end{align}
where we have taken units where Boltzmann's constant $k_B=1$. 
\section{Statistical field theory}
We now consider a system with Hamiltonian $H[\phi]$ which depends on a continuous field 
$\phi({\bf x})$. The partition function is given by a functional integral
\begin{align}
Z = \int d[\phi] \exp(-\beta H[\phi]),
\end{align}
the functional integral over all possible fields $\phi$ can be taken as a limit where $\phi$ is defined at a finite number of points on a lattice and then the lattice spacing is taken to zero. 

The mean field approximation to partition function is then given by
\begin{align}
Z _{MF}=  \exp(-\beta H[\phi_{MF}]),
\end{align} 
where $\phi_{MF}$ is the mean field solution which minimises $H$. The definition of a functional derivative of a functional 
\begin{align}
F[\phi+\delta\phi]-F[\phi]= \int d{\bf x} \frac{\delta F}{\delta\phi({\bf x})} \delta\phi({\bf x}).
\end{align}
Therefore if a field $\phi$ maximises $H$ we must have 
\begin{align}
\frac{\delta H}{\delta\phi({\bf x})}=0.
\end{align}

We now consider the standard Landau-Ginzburg Hamiltonian describing Ising like systems where
\begin{align}
H[\phi] = \int d{\bf x} \ \frac{\kappa}{2}[\nabla \phi]^2 + V(\phi) .
\end{align}
The first term represents an energetic cost of varying the field $\phi$ while the second potential term has two minima at $\phi=\pm \phi_c$, and without loss of generality we can chose  $V(\phi_c)=V(-\phi_c)$, in the low temperature or phase separated phase and a single minimum at $\phi=0$ in the high temperature phase. The standard, so called, $\phi^4$ form is
\begin{align}
V(\phi) = \frac{1}{2} m^2 \phi^2 + \frac{\lambda}{4!} \phi^4,\label{p4}
\end{align} 
where 
\begin{align}
m^2 = T-T_c.
\end{align}
It is easy to see that 
\begin{align}
\frac{\delta H}{\delta \phi({\bf x})} = -\kappa \nabla^2 \phi({\bf x}) + V'(\phi).\label{cm}
\end{align}
If there is non constraint on the system if can simply chose $\phi({\bf x}) =\phi_c$ or $\phi({\bf x}) =-\phi_c$ everywhere which corresponds to a  free energy $F=H[\phi_c]=0$. However in a system with a conserved order parameter
\begin{align}
\int d{\bf x} \  \phi({\bf x})=0, 
\end{align}
then the solutions $\phi=\pm \phi_c$ cannot hold. In this case the system will separate into a two phases where $\phi({\bf x})= \pm \phi_c$. We therefore choose an interface at $z=0$ where 
and take $\phi({\bf x}) = \phi_K(z)$ ($K$ standing for kink as it is known as the kink solution in the literature) where $\lim_{z\to\-\infty}=-\phi_c$ and  $\lim_{z\to\infty}=-\phi_c$. 
We therefore find from Eq. (\ref{cm}) that
\begin{align}
-\kappa \frac{d^2 }{dz^2}\phi_K(z)  + V'(\phi_K) = 0 \label{kk0}
\end{align}
This align can be solved for the potential in Eq. (\ref{p4}) ({\em you should do it and fill in the details}) but even without knowing the explicit solution we can write
\begin{align}
H[\phi_K]=  A\int dz \ \frac{\kappa}{2}\left(\frac{d\phi_K(z)}{dz}\right)^2 + V(\phi_K(z)),\label{kk1}
\end{align}
where $A$ is the surface area of the system in the plane perpendicular to the direction $z$. 
However if we multiply Eq. (\ref{kk0}) by $d\phi/dz$ and integrate we find
\begin{align}
-\frac{\kappa}{2} (\frac{d\phi_K}{dz})^2 + V(\phi) = C,
\end{align}
where $C$ is a constant. However as $\phi_K(z)\to \pm \phi_c$ as $z\to \pm \infty$ and $V(\pm\phi_c) =0$ we find that $C=0$. Using this we obtain 
\begin{align}
H[\phi_K]=  A\int dz\  {\kappa}\left(\frac{d\phi_K(z)}{dz}\right)^2 .
\end{align}
If the interface has a free energy per unit area of $\sigma$ then we have the Cahn-Hillard estimate of the surface tension 
\begin{align}
\sigma=  \int dz\  {\kappa}\left(\frac{d\phi_K(z)}{dz}\right)^2 .
\end{align}

Now we return to dynamics. If we compare with systems with a discrete number of variables we
should have a Langevin align of the form
\begin{align}
\frac{\partial \phi({\bf x})}{\partial t}= -L \frac{\delta H}{\delta \phi({\bf x})} + \eta({\bf x},t).
\end{align}
The white noise correlator should have the form
\begin{align}
\langle \eta({\bf x},t)\eta({\bf x}',t)\rangle =\delta(t-t')\Gamma({\bf x},{\bf x'}),
\end{align}
where here  $\Gamma({\bf x},{\bf x'})$ is an operator (before it was a matrix) defined by its action on functions $f$ as
\begin{align}
\Gamma f({\bf x}) = \int d{\bf x}' \Gamma({\bf x},{\bf x}')f({\bf x}'),
\end{align}
and $L$ is also an operator with 
\begin{align}
L f({\bf x}) = \int d{\bf x}' L({\bf x},{\bf x}')f({\bf x}'),
\end{align}
Following the same arguments for systems with a finite number of degrees of freedom we thus have the relation
\begin{align} 
\Gamma({\bf x},{\bf x}') =2T L({\bf x},{\bf x}').
\end{align}
The simplest form of dynamics is given by $L({\bf x},{\bf x}')=\alpha\delta({\bf x}-{\bf x}')$ which gives the model A dynamics
\begin{align}
\frac{\partial \phi({\bf x})}{\partial t}= -\alpha \frac{\delta H}{\delta \phi({\bf x})} + \eta({\bf x},t),
\end{align}
with the noise correlator
\begin{align}
\langle \eta({\bf x},t)\eta({\bf x}',t)\rangle =2T \alpha \delta(t-t')\delta({\bf x}-{\bf x'}).
\end{align}
The average value of $\phi$ 
\begin{align}
\overline \phi(t) = \frac{1}{V}\int d{\bf x} \phi({\bf x},t),
\end{align}
is clearly not generally conserved by this dynamics.

Model $B$ dynamics amounts to choosing
\begin{align}
L({\bf x}-{\bf x}')= -D\nabla^2 \delta({\bf x}-{\bf x'}),
\end{align}
here the fact that $L$ is a positive semi-definite operator can be seen by taking its Fourier transform. The evolution align here is
\begin{align}
\frac{\partial \phi({\bf x})}{\partial t}= D\nabla^2 \frac{\delta H}{\delta \phi({\bf x})} + \eta({\bf x},t),
\label{MB}
\end{align}
and where
\begin{align}
\langle \eta({\bf x},t)\eta({\bf x}',t)\rangle =-2TD   \delta(t-t')\nabla^2\delta({\bf x}-{\bf x'}).
\end{align}
We notice that if we introduce the vectorial white noise with components $\eta_i({\bf x},t)$ such that
\begin{align}
\langle \eta_i({\bf x},t) \eta_i({\bf x}',t')\rangle =\delta_{ij} \delta({\bf x}-{\bf x'})\delta(t-t),
\end{align}
where $\delta_{ij}=1$ for $i=j$ and is zero otherwise,  we can write
\begin{align}
\eta({\bf x},t)= \nabla\cdot {\boldsymbol \eta}({\bf x},t),
\end{align}
as one can verify the two noises have the same correlation function. In this way Eq. (\ref{MB}) becomes 
\begin{align}
\frac{\partial \phi({\bf x})}{\partial t}= \nabla\cdot[ D\nabla \frac{\delta H}{\delta \phi({\bf x})} + {\boldsymbol\eta}({\bf x},t)].
\end{align}
From this it is easy to see that the order parameter is conserved - thus model  B describes conserved phase ordering dynamics.

The above argument is rather phenomenological and a microscopic derivation of conserved order parameter dynamics gives the so called Dean-Kawasaki align. Model B is a simplified form of this exact align - you could say more.

\end{comment}