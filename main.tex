\documentclass[11pt]{book}

\input{Preambule}		% Liste des packages et de leurs options
\input{CommandesPerso}	% Commandes et environnements perso
\input{PageDeGarde}


%%% Commandes personnalisées  %%%
\newcommand{\nn}{\nonumber \\} % newline sans nombre dans align
\newcommand{\mC}{\mathcal{C}} %pour les fonctions de corrélation
\newcommand{\bx}{{\bf x}} %pour les vecteurs en gras
\newcommand{\by}{{\bf y}} %pour les vecteurs en gras
\newcommand{\bq}{{\bf q}}
\newcommand{\bk}{{\bf k}}
\newcommand{\br}{{\bf r}}
\newcommand{\bv}{{\bf v}}
\DeclareMathOperator{\sgn}{sgn}
% Les > et < se comportent normalement si c'est pour supérieur ou inférieur, sinon se comportent comme \langle
\mathchardef\less=\mathcode`<
\mathchardef\greater=\mathcode`>
\DeclareMathDelimiter{<}{\mathopen}{symbols}{"68}{largesymbols}{"0A}
\DeclareMathDelimiter{>}{\mathclose}{symbols}{"69}{largesymbols}{"0B}

%% Ne pas casser les équations inline $ $ 
\relpenalty=5000
\binoppenalty=5000


% Infos de la page de garde
\author{Paul Gersberg}
\title{Confinement and driving effects on continuous and discrete model interfaces}
\specialite{Laser, Matière, Nanosciences}
\directeur{David S. Dean}
\encadrant{Peter C.W. Holdsworth}
\date{16 juillet 2020}
\jurya{Steve Bramwell, University College London}{Professeur}{Rapporteur}
\juryb{Jorge Kurchan, ENS Paris}{Directeur de recherche}{Rapporteur}
\juryc{Pierre Pujol, LPT Toulouse }{Professeur}{Examinateur}
\juryd{Jérôme Cayssol, LOMA}{Professeur}{Examinateur}
\ecole{Université de Bordeaux}

% Méta-données du PDF
\hypersetup{
    pdfauthor={Paul Gersberg},
    pdfsubject={Manuscrit de thèse de doctorat},
    pdftitle={Confinement and driving effects on continuous and discrete model interfaces},
}

%%%%%%% Début du document %%%%%
\begin{document}
%\bibliographystyle{unrst}

% Préambule
	\pagenumbering{roman}
	\pagedegarde

	\cleardoublepage
{\bf Effets d'écoulements et de confinement dans les modèles discrets et continus d'interfaces} \\

Lorsqu'un système statistique est confiné ou soumis à un écoulement, on observe une modification de l'énergie libre et des propriétés sous-jacentes du système, dont une force de confinement connue dans les systèmes critiques sous le nom de force de Casimir critique. Dans cette thèse, nous nous intéressons aux systèmes à une interface dans la phase ordonnée. 

À partir de la théorie statistique des champs, nous développons les équations de la dynamique d'une interface continue dans les modèles A et B. En utilisant les intégrales de chemin sur des interfaces possédant un hamiltonien gaussien, nous trouvons la distribution de probabilité de l'interface et la force de confinement pour une interface libre ou soumise à une pression constante. Nous mettons également au point un système d'équations couplées propre au modèle C pour montrer qu'un écoulement uniforme parallèle à l'interface diminue la fluctuation des hauteurs et augmente la longueur de corrélation du système.

Pour les modèles discrets, nous introduisons le modèle Solid-On-Solid issu du modèle d'Ising à basse température, et utilisons le formalisme des matrices de transfert afin de calculer la force de confinement. Nous généralisons également une méthode de mesure de l'énergie libre dans les simulations numériques, et en profitons pour observer les différences d'énergie libre entre les ensembles thermodynamiques. Finalement, nous proposons un nouveau modèle discret, qui contrairement au modèle SOS, n'est pas un modèle discret d'interface mais est une meilleure approximation du modèle d'Ising à basse température.
\\ \\
{\bf Mots-clés :} théorie statistique des champs, modèle A/B, Ising, Solid-On-Solid, force de Casimir, écoulements \\

{\bf Confinement and driving effects on continuous and discrete model interfaces}\\

When a statistical system is confined or driven, the free energy is modified, leading to effects such as the critical Casimir force. In this thesis, we are interested in interface systems in the ordered phase.

From statistical field theory, we develop the dynamical equations of continuous interfaces both in model A and B. Using path integrals on interfaces having gaussian hamiltonians, we find the probability density distribution and the confinement force for a free or under constant pressure interface. We also propose a coupled system as in model C, and show that uniform driving along the interface suppreses height fluctuations and increases the correlation length of fluctuations. 

For discrete models, we introduce the Solid-On-Solid model which is a low-temperature approximation of the Ising model, and use the transfer matrix method to compute the confinement force. We also generalize a method to numerically compute the free energy and use it to describe the free energy difference between statistical ensembles. Finally, we propose a new model, which contrary to SOS, is not an interface model but is a better approximation of the Ising model at low temperature.
\\ \\
{\bf Keywords : }  statistical field theory, model A/B, Ising, Solid-On-Solid, Casimir force, driving

\newpage
{\bf \huge Résumé en français}

\newpage
{\bf \huge Abstract}


	\cleardoublepage
		% Table des matières
			\setcounter{tocdepth}{1}	% Pas besoin de trop détailler le sommaire ici (chapitres/sections)
			\dominitoc						% Génération des mini-toc	\pagenumbering{arabic}
			\setcounter{tocdepth}{4}
			\setcounter{secnumdepth}{4}
			\tableofcontents
		% Liste des figures
			\renewcommand*\listfigurename{Liste des figures}
			%\listoffigures
		% Liste des tableaux
		%\listoftables

\chapter*{Introduction}
\addcontentsline{toc}{chapter}{Introduction} 

Chaque système statistique est décrit par un paramètre d'ordre, que ce soit la magnétisation moyenne d'un milieu aimanté ou de l'orientation moyenne des polymères par exemple. Dans ces systèmes, on appelle phase un milieu homogène selon le paramètre d'ordre. Les transitions de phase d'un système, où une phase homogène se sépare en plusieurs phases différentes, possèdent des propriétés exceptionnelles. Lors d'une transition de phase continue, la largeur de l'interface entre deux phases diverge jusqu'à atteindre une taille macroscopique. Ce confinement de l'interface mène a des effets de taille finie, notamment l'effet Casimir critique.

L'étude des propriétés statistiques des interfaces peut se faire via différentes approches, toutes complémentaires. Les modèles sur réseau, et plus préciseément le modèle d'Ising, sont particulièrement adaptés aux simulations numériques, mais plus difficiles à traiter analytiquement à cause du trop grand nombre d'interactions présentes dans le système. Une manière de diminuer les interactions est de réduire la dimensionalité du système à une dimension, ce qui nous permet d'utiliser le formalisme des matrices de transfert. À cet effet, le modèle Solid-On-Solid a été énormément étudié dans les annés 80-90 pour sa simplicité.
L'interface peut également être assimilée à un marcheur brownien qui, au lieu de bouger dans le temps, se meut dans l'espace. Ainsi, les équations de Schrödinger et les équations du mouvement de Langevin permettent de décrire les fluctuations d'une interface grâce au formalisme quantique ou stochastique. Cette méthode s'appelle la théorie des ondes capillaires et permet la résolution d'un système très analogue aux modèles SOS, en utilisant un formalisme très similaire aux matrices de transfert. 
La dernière méthode que nous aborderons dans cette thèse est celle du champ moyen, c'est-à-dire l'étude des propriétés de deux phases grâce aux équations de Landau-Ginzburg. Cette méthode a l'avantage d'offrir des calculs analytiques relativement faciles et permet d'obtenir la forme des fonctions de corrélation à plusieurs point ainsi que les longueurs de corrélation. Néanmoins, la vérification des résultats via les simulations numériques, qui nous permettrait de mettre des grandeurs mésoscopiques comme la tension superficielle ou la longueur capillaire en relation avec les grandeurs microscopiques du modèle d'Ising, est asez difficile, et nous éviterons dans la présente thèse le rapprochement.

L'étude de l'effet Casimir critique - qui est un effet de taille finie - revient au final à étudier les propriétés statistiques d'une interface, et à voir comment elles sont modifiées lorsqu'il existe des conditions aux bords. Cependant, les propriétés d'une interface varient également lorsqu'elle est mise hors-équilibre, par exemple via une force du style cisaillement, qui représente beaucoup de cas expérimentaux classiques. 
Nous nous intéressons ici particulièrement à la différence entre les états d'équilibre et hors-équilibre, pour lesquels les formalismes sont différents mais dont les simulations numériques sont similaires. La présence d'un système hors-équilibre pose également plein de questions sur la nature de l'ensemble thermodynamique que l'on se place, et nous éclaire sur les différences entre l'ensemble canonique et grand-canonique, et également entre des particules discernables et indiscernables.

Le manuscrit se décompose de la manière suivante :
\begin{itemize}
    \item Le premier chapitre introduit les différents approches historiques sur l'étude des interfaces, en s''attardant sur les principaux résultats obtenus dans la littérature pour des interfaces à l'équilibre, puis hors-équilibre
    \item Le second chapitre introduit le modèle Solid-On-Solid, qui est une approximation 1D à très basse température du modèle d'Ising en 2D. Nous y parlerons du formalisme des matrices de transfert, des principaux résultats obtenus dans ce modèle et de quelques précisions fondamentales sur les différents ensembles thermodynamique sdans lesquelles ont peut étudier nos systèmes
    \item Nous présentons dans le troisième chapitre l'alogirthme de Monter Carlo-Metropolis, un outil puissant pour explorer l'espace des phases et calculer numériquement la fonction de partition de nos systèmes
    \item Dans le quatrième chapitre, l'étude d'un système SOS - analogue à la croissance d'un cristal -  via une méthode d'intégration sur les potentitels chimiques nour permet d'obtenir l'énergie libre, et ainsi l'effet Casimir. Cette étude se termine par l'ajout du cisaillement.
        \item Dans le cinquième chapitre, nos étudions un modèle avec un champ magnétique charactéristique des expériences dans lesquelles on force une phase d'un fluide binaire dans une autre grâce à une pression de radiation exercée par un laser
    \item Un nouveau modèle découlant des considérations du second chapitre peut être créé de la même manière que le modèle SOS, en prenant en compte l'entropie. Ce nouveau modèle, baptisé Particles-Over-Paticle, fait l'objet du sixième chapitre
    \item Le septième chapitre reprend les calculs d'une publication récente de notre équipe sur un modèle de champ moyen où l'on mélange deux types de particules appartenant à des ensembles thermodynamiques différents, sous l'effet d'un cisaillement uniforme
\end{itemize}

La présente thèse a été rendue possible grâce à l'ANR FISICS, le Laboratoire Onde Matière d'Aquitaine de l'Université de Bordeaux, le Laboratoire de Physique de l'ENS Lyon et le Mésocentre de Calcul Intensif d'Aquitaine sur lesquelles ont été faites les simulations numériques. Je remercie particulièrement Josiane Parzych (LOMA) et Laurence Mauduit (ENS LYON) pour le suivi administratif, ainsi que Nguyen Ky Nguyen (MCIA) pour l'aide technique.


\mainmatter
\chapter{Equilibrium interface dynamics}
\section{Introduction}

Dans ce chapitre nous analysons la dynamique des systèmes statistiques. L'analyse nous permettra de comprendre comment les transitions de phase, notament certains systèmes subissant une séparation de phase à la transition, se comportent de manière dynamique. L'exemple le plus connu est le modèle d'Ising en absence de champ magnétique, le paramètre d'ordre de la transition étant la magnétisation totale du système. Dans la phase haute température, le système est homogène et sa magnétisation est nulle. En dessous de la température critique, dans le cas où le paramètre d'ordre est conservé (par exemple une dynamique de Kawasaki ou modèle B), le système va localement se séparer en deux phases de magnétisation moyenne opposée séparées par une interface minimisant l'énergie de surface entre les deux phases. 

Dans le cas où le paramètre d'ordre n'est pas conservé (par exemple une dynamique de Glauber ou modèle A), une brisure spontannée de symmétrie fera que l'une des deux phases englobe l'autre, au point de recouvrir tout le système (voir Fig \ref{clusterization}). Dans une transition de phase continue où le point critique est atteint depuis l'état désordonné vers l'état ordonné, les domaines de phase égales sont de taille égale à la longueur de corrélation du système. Dans les transitions de phase telles que celles du modèle d'Ising, cette longueur de corrélation diverge lorsque l'on s'approche de la température critique $T_C$. Dans un système thermodnamique, elle devient infinie, impliquant que le système prend un temps infini pour atteindre l'équilibre thermodynamique : c'est le ralentissement critique. Ce processus de croissance des domaines depuis la phase désordonnée s'appelle le \textit{coarsening} et la théorie de la cinétique d'ordre des phases est la théorie développée pour le comprendre.
Cette thèse s'appuie sur cette théorie afin de déterminer les propriétés statistiques (telles que la position moyenne et la tension superficielle) des interfaces entre deux phases coexistantes.
\begin{figure}[t]
    \centering
    \includegraphics[width=0.9\linewidth]{intro/clusterization.pdf}
    \caption{Phénomène d'aggrégation à partir d'une trempe (\textit{quench}) dans un modèle d'Ising de $T=\infty$ à $T=T_{2D,C}$ \cite{onsager_crystal_1944} pour différents temps en étapes de Monte Carlo, pour un système $600 \times 600$ avec une dynamique non-conservée de Glauber.}
    \label{clusterization}
\end{figure}


In this chapter we will analyse the dynamics of statistical systems. The analysis will allow us to understand how phase transitions occur dynamically. For instance we know that certain systems undergo what is known phase separation, for instance if we take the Ising model with zero magnetic field, in the high temperature phase the system is homogeneous and the average magnetisation, which is the order parameter for the transition is zero. Below the critical temperature if the overall magnetisation is conserved (for instance for Kawasaki dynamics), which would be the case if spins corresponded to different types of particles, the system will separate into two  phases of opposite average magnetisation, separated by an interface which will be roughly flat in order to minimise the surface energy between the two phases. For nonconserved systems, where the overall magnetisation in not conserved (for example Glauber dynamics), eventually one of the two phases will make up the system (spontaneous symmetry breaking). In a continuous phase transition as the critical point is reached from the disordered to ordered, domains of phases of positive and negative magnetisation form and the size of these domains is given by the correlation length of the system. For continuous phase transitions such as that in the Ising model the correlation length diverges as as the critical point is approached, for instance as $T\to T_c$ if the temperature is varied. The size of the domains thus have to become infinite if the system is infinite, this means that for an infinite system it will take an infinite time to relax to the equilibrium state. The process of domain growth is known as coarsening and phase ordering kinetics is the theory that has been developed to understand the phenomenon of coarsening. Furthermore, for systems with a conserved order parameter which separate into two phases, the two phases will be separated by an interface. This interface will be characterised by a surface tension, its average position will be fixed but it will exhibit fluctuations. Later we will see how model of phase ordering kinetics and be used to determine the static and dynamical properties of interfaces between two coexisiting phases. 

While the phase diagram of a system can be determined via 
its Hamiltonian and equilibrium statistical mechanics, the dynamics of coarsening depends on details of the systems dynamics that do not show up in single time thermodynamic observables. Therefore one needs to construct dynamical models that capture the underlying evolution of the state of the system, in particular there is a big difference between systems where the order parameter is conserved and those where it is not conserved.

\section{Statics of systems with a finite number of degrees of freedom}

Thermodynamic systems are naturally described in terms of fields, for example densities. This means that one is naturally lead to consider statistical field theories where the system is described in terms of a local field $\phi({\bf x})$. Statistical field theories can be applied to both statics, to understand phase diagrams, and dynamics to understand phase ordering. However to start with we will examine the case of systems with a finite number of degrees of freedom. 

Consider a system in the canonical ensemble with a Hamiltonian $H({\bf q})$ where $q_i$ for 
$1\leq i\leq N$ represent a finite number of continuous spatial degrees of freedom and where in a classical system we have already integrated over the corresponding momenta. The partition function for the system is given by
\begin{equation}
Z = \int d{\bf q} \exp\left(-\beta H({\bf q})\right),
\end{equation}
and in equilibrium the probability density function $P_{eq}({\bf q})$ of the degrees of freedom is given by 
\begin{equation}
P_{eq}({\bf q}) = \frac{\exp\left(-\beta H({\bf q})\right)}{Z}.\label{eqdis}
\end{equation}
In general the integral which gives the  partition function cannot be computed analytically.
The simplest approximation to compute $Z$ is the mean field approximation where the integral 
is approximated by the integrand at its largest value - in mathematics this is the Laplace method for approximating an integral and in this context it is just an expansion about the minimum energy configuration of the system. The mean field approximation is thus
\begin{equation}
Z_{MF}= \exp\left(-\beta H({\bf q}^*)\right),
\end{equation}
where ${\bf q}^*$ is the value of ${\bf q}$ which minimises $H$ (note that the approximation becomes exact in the zero temperature limit - $\beta \to \infty$   - as the system will minimise its energy). The values $q_i^*$ are determined from
\begin{equation}
\frac{\partial H}{\partial q_i}|_{{\bf q}={\bf q^*}}=0.
\end{equation}
Within this approximation any thermodynamic observable is given by
\begin{equation}
\langle f({\bf q}) \rangle = f({\bf q}^*).
\end{equation}

We now consider how one can model dynamics of such systems. We will look for a Langevin equation which is chosen to give the correct equilibrium Gibbs-Boltzmann distribution. We write
\begin{equation}
\frac{d q_i}{dt} = -L_{ij}\frac{\partial H({\bf q})}{ \partial q_j} + \eta_i(t),
\end{equation}
where $L_{ij}$ is a matrix which discuss later and $\eta_i(t)$ is zero mean Gaussian white noise  with correlation function 
\begin{equation}
\langle \eta_i(t)\eta_j(t')\rangle =  \Gamma_{ij} \delta(t-t')\label{cfn}
\end{equation}
The Gaussian white noise represents the effects of thermal fluctuations on the system we assume that the correlation time of these fluctuations is extremely short with respect to the dynamics of the degrees of freedom $q_i$ (in fact in critical systems the dynamics becomes very slow, critical slowing down, and this approximation becomes better and better as one approaches the critical point).  There is no momentum term in this Langevin equation and for this reason it is often called the over damped Langevin equation. Overdamped Langevin equations can also be derived staring from Newton's laws in the presence of friction, due to a solvent, and again white noise (again due to molecular collisions with the solvent) and by taking the limit where the frictional forces are greater than the acceleration term in Newton's equations (equivalent to setting the particle masses to zero).


As Eq. (\ref{cfn}) is for a correlation function the matrix $\Gamma_{ij}$ must be symmetric and cannot have any negative eigenvalues.

In the absence of noise or thermal fluctuations, so at zero temperature, the system will simply minimise its energy. Therefore if 
\begin{equation}
\frac{\partial H({\bf q})}{ \partial q_j} =0, 
\end{equation}
with no noise we have $\frac{d q_i}{dt}=0$, that is to say it is the term $\frac{\partial H({\bf q})}{ \partial q_j}$ that drives the dynamics if there is no noise. As long as the matrix $L_{ij}^{-1}$ exists the zero temperature dynamics will take the system to the local minimum of $H$ and to the absolute minimum if there are no metastable configurations. 

Under these assumptions, the Fokker-Planck equation for the probability density function of the degrees of freedom is 
\begin{equation}
\frac{\partial p({\bf q},t)}{\partial t} = \frac{\partial}{\partial q_i} \left[\frac{1}{2}\Gamma_{ij} \frac{\partial p({\bf q},t)}{\partial q_i} + p({\bf q},t) L_{ij}\frac{\partial H({\bf q})}{ \partial q_j}\right].
\end{equation}
This can be written as 
\begin{equation}
\frac{\partial p({\bf q},t)}{\partial t} +\frac{\partial}{\partial q_i}J_i({\bf q},t)=0,
\end{equation}
where the ${\bf J}({\bf q},t)$ is the probability current. We now insist that the system is in equilibrium with zero current when $p({\bf q},t)= P_{eq}({\bf q})$ as given by Eq. (\ref{eqdis}), this gives
\begin{equation}
\left[-\frac{\beta}{2}\Gamma_{ij} + L_{ij}\right]\frac{\partial H({\bf q})}{ \partial q_j},
\end{equation}
and this holds for any choice of $H$ is we chose.
\begin{equation}
\Gamma_{ij}= 2T L_{ij}
\end{equation}
where we have taken units where Boltzmann's constant $k_B=1$. 
\section{Statistical field theory}
We now consider a system with Hamiltonian $H[\phi]$ which depends on a continuous field 
$\phi({\bf x})$. The partition function is given by a functional integral
\begin{equation}
Z = \int d[\phi] \exp(-\beta H[\phi]),
\end{equation}
the functional integral over all possible fields $\phi$ can be taken as a limit where $\phi$ is defined at a finite number of points on a lattice and then the lattice spacing is taken to zero. 
In many cases  the system has been coarse grained and $\phi$ represents a spatially varying order parameter, for instance the local density averaged over some small volume. In this case the Hamiltonian $H$ is strictly speaking a free energy  and contains terms that depend on the temperature.

The mean field approximation to partition function is then given by
\begin{equation}
Z _{MF}=  \exp(-\beta H[\phi_{MF}]),
\end{equation} 
where $\phi_{MF}$ is the mean field solution which minimises $H$. The definition of a functional derivative of a functional is
\begin{equation}
F[\phi+\delta\phi]-F[\phi]= \int d{\bf x} \frac{\delta F}{\delta\phi({\bf x})} \delta\phi({\bf x}).
\end{equation}
Therefore if a field $\phi$ maximises $H$ we must have 
\begin{equation}
\frac{\delta H}{\delta\phi({\bf x})}=0.
\end{equation}

We now consider the standard Landau-Ginzburg Hamiltonian describing Ising like systems where
\begin{equation}
H[\phi] = \int d{\bf x} \ \frac{\kappa}{2}[\nabla \phi]^2 + V(\phi) .
\end{equation}
The first term represents an energetic cost of varying the field $\phi$ while the second potential term has two minima at $\phi=\pm \phi_c$, and without loss of generality we can chose  $V(\phi_c)=V(-\phi_c)$, in the low temperature or phase separated phase and a single minimum at $\phi=0$ in the high temperature phase. The standard, so called, $\phi^4$ form is
\begin{equation}
V(\phi) = \frac{1}{2} m^2 \phi^2 + \frac{\lambda}{4!} \phi^4,\label{p4}
\end{equation} 
where 
\begin{equation}
m^2 = T-T_c.
\end{equation}
It is easy to see that 
\begin{equation}
\frac{\delta H}{\delta \phi({\bf x})} = -\kappa \nabla^2 \phi({\bf x}) + V'(\phi).\label{cm}
\end{equation}
If there is non constraint on the system if can simply chose $\phi({\bf x}) =\phi_c$ or $\phi({\bf x}) =-\phi_c$ everywhere which corresponds to a  free energy $F=H[\phi_c]=0$. However in a system with a conserved order parameter
\begin{equation}
\int d{\bf x} \  \phi({\bf x})=0, 
\end{equation}
then the solutions $\phi=\pm \phi_c$ cannot hold. In this case the system will separate into a two phases where $\phi({\bf x})= \pm \phi_c$. We therefore choose an interface at $z=0$ where 
and take $\phi({\bf x}) = \phi_K(z)$ ($K$ standing for kink as it is known as the kink solution in the literature) where $\lim_{z\to\-\infty}=-\phi_c$ and  $\lim_{z\to\infty}=-\phi_c$. 
We therefore find from Eq. (\ref{cm}) that
\begin{equation}
-\kappa \frac{d^2 }{dz^2}\phi_K(z)  + V'(\phi_K) = 0 \label{kk0}
\end{equation}
This equation can be solved for the potential in \eqref{p4} ({\em you should do it and fill in the details}) but even without knowing the explicit solution we can write
\begin{equation}
H[\phi_K]=  A\int dz \ \frac{\kappa}{2}\left(\frac{d\phi_K(z)}{dz}\right)^2 + V(\phi_K(z)),\label{kk1}
\end{equation}
where $A$ is the surface area of the system in the plane perpendicular to the direction $z$. 
However if we multiply Eq. \eqref{kk0} by $d\phi/dz$ and integrate we find
\begin{equation}
-\frac{\kappa}{2} (\frac{d\phi_K}{dz})^2 + V(\phi_K) = C,
\end{equation}
where $C$ is a constant. However as $\phi_K(z)\to \pm \phi_c$ as $z\to \pm \infty$ and $V(\pm\phi_c) =0$ we find that $C=0$. Using this we obtain 
\begin{equation}
H[\phi_K]=  A\int dz\  {\kappa}\left(\frac{d\phi_K(z)}{dz}\right)^2 .
\end{equation}
If the interface has a free energy per unit area of $\sigma$ then we have the Cahn-Hillard estimate of the surface tension 
\begin{equation}
\sigma=  \int dz\  {\kappa}\left(\frac{d\phi_K(z)}{dz}\right)^2 .\label{CHST}
\end{equation}

Now we return to dynamics. If we compare with systems with a discrete number of variables we
should have a Langevin equation of the form
\begin{equation}
\frac{\partial \phi({\bf x})}{\partial t}= -L \frac{\delta H}{\delta \phi({\bf x})} + \eta({\bf x},t).
\end{equation}
The white noise correlator should have the form
\begin{equation}
\langle \eta({\bf x},t)\eta({\bf x}',t)\rangle =\delta(t-t')\Gamma({\bf x},{\bf x'}),
\end{equation}
where here  $\Gamma({\bf x},{\bf x'})$ is an operator (before it was a matrix) defined by its action on functions $f$ as
\begin{equation}
\Gamma f({\bf x}) = \int d{\bf x}' \Gamma({\bf x},{\bf x}')f({\bf x}'),
\end{equation}
and $L$ is also an operator with 
\begin{equation}
L f({\bf x}) = \int d{\bf x}' L({\bf x},{\bf x}')f({\bf x}'),
\end{equation}
Following the same arguments for systems with a finite number of degrees of freedom we thus have the relation (which is sometimes called the fluctuation dissipation theorem as it essentially is equivalent)
\begin{equation} 
\Gamma({\bf x},{\bf x}') =2T L({\bf x},{\bf x}').\label{gnoise}
\end{equation}
The simplest form of dynamics is given by $L({\bf x},{\bf x}')=\alpha\delta({\bf x}-{\bf x}')$ which gives the model A dynamics
\begin{equation}
\frac{\partial \phi({\bf x})}{\partial t}= -\alpha \frac{\delta H}{\delta \phi({\bf x})} + \eta({\bf x},t),\label{MA}
\end{equation}
with the noise correlator
\begin{equation}
\langle \eta({\bf x},t)\eta({\bf x}',t)\rangle =2T \alpha \delta(t-t')\delta({\bf x}-{\bf x'}).
\end{equation}
The average value of $\phi$ 
\begin{equation}
\overline \phi(t) = \frac{1}{V}\int d{\bf x}\  \phi({\bf x},t),
\end{equation}
is clearly not generally conserved by this dynamics.

Model $B$ dynamics amounts to choosing
\begin{equation}
L({\bf x}-{\bf x}')= -D\nabla^2 \delta({\bf x}-{\bf x'}),
\end{equation}
here the fact that $L$ is a positive semi-definite operator can be seen by taking its Fourier transform. The evolution equation here is
\begin{equation}
\frac{\partial \phi({\bf x})}{\partial t}= D\nabla^2 \frac{\delta H}{\delta \phi({\bf x})} + \eta({\bf x},t),
\label{MB}
\end{equation}
and where
\begin{equation}
\langle \eta({\bf x},t)\eta({\bf x}',t)\rangle =-2TD   \delta(t-t')\nabla^2\delta({\bf x}-{\bf x'}).
\end{equation}
We notice that if we introduce the vectorial white noise with components $\eta_i({\bf x},t)$ such that
\begin{equation}
\langle \eta_i({\bf x},t) \eta_i({\bf x}',t')\rangle =\delta_{ij} \delta({\bf x}-{\bf x'})\delta(t-t),
\end{equation}
where $\delta_{ij}=1$ for $i=j$ and is zero otherwise,  we can write
\begin{equation}
\eta({\bf x},t)= \nabla\cdot {\boldsymbol \eta}({\bf x},t),
\end{equation}
as one can verify the two noises have the same correlation function. In this way Eq. (\ref{MB}) becomes 
\begin{equation}
\frac{\partial \phi({\bf x})}{\partial t}= \nabla\cdot[ D\nabla \frac{\delta H}{\delta \phi({\bf x})} + {\boldsymbol\eta}({\bf x},t)].
\end{equation}
From this it is easy to see that the order parameter is conserved - thus model  B describes conserved phase ordering dynamics.

\section{Models for equilibrium interfaces}
Here we discuss effective models of interfaces. The simplest model is to assume that the 
interface is parameterised by a height profile $h({\bf r})$, however one also has to assume that 
$h({\bf r})$ is a single valued function of ${\bf r}$. Given this one can write
\begin{equation}
H[h] = \sigma A[h]
\end{equation}
where $A_h$ is the area of the interface. However the interface area is given by
\begin{equation}
A[h] = \int_A d{\bf r}\sqrt{1+[\nabla h]^2},
\end{equation}
where the integral is over the plane perpendicular to the $z$ axis which is taken to be of area $A$. When the fluctuations of the interface are small, we can expand the above to quadratic order in $h$ to obtain
\begin{equation}
H[h]= A\sigma +\frac{\sigma}{2} \int_A d{\bf r} \ [\nabla h]^2.
\end{equation}
The first term is independent of the height so we can write the effective Hamiltonian for the surface as
\begin{equation}
H_{eff} [h]= \frac{\sigma}{2} \int_A d{\bf r}\  [\nabla h]^2.\label{heff}
\end{equation}
\section{Effective dynamics of interface heights}\label{heightd}
We will now try and derive an approximation for the dynamics of the  height of the interface from the original phase ordering kinetics. Here we use the method of Bray and Cavagnha - put in the reference, which was used to study the dynamics of sheared interfaces, in the absence of shear to determine the dynamical properties of interfaces in phase separated systems for both model A and model B dynamics.

We imagine that the system is phase separated in the direction  $z$, on average the interface is taken to be at $z=0$, and we write
\begin{equation}
\phi(z,{\bf r},t) = f(z-h({\bf r},t))\label{hans}
\end{equation}
where $f(z)=\phi_K(z)$ is the kink solution from mean field theory.
\\

\noindent{\bf Model A dynamics}

For model A dynamics, we substitute Eq. \eqref{hans} into Eq. \eqref{MA} and make use of the following results
\begin{eqnarray}
\frac{\partial f(z-h({\bf r},t))}{\partial t}&=& -f'(z-h({\bf r},t))\frac{\partial h({\bf r},t)}{\partial t}\\
\nabla f(z-h({\bf r},t))&=& [{\bf e}_z-\nabla h({\bf r},t)]f'(z-h({\bf r},t))  \\
\nabla^2 f(z-h({\bf r},t))&=& f''(z-h({\bf r},t))- \nabla^2 h({\bf r},t)f'(z-h({\bf r},t))+ [\nabla h({\bf r},t)]^2 
f''(z-h({\bf r},t))
\end{eqnarray}
and thus find
\begin{eqnarray}
&&-f'(z-h({\bf r},t))\frac{\partial h({\bf r},t)}{\partial t}= \alpha\kappa\times\\ &&\left[f''(z-h({\bf r},t))- \nabla^2 h({\bf r},t)f'(z-h({\bf r},t))+ [\nabla h({\bf r},t)]^2 
f''(z-h({\bf r},t))\right] - \alpha V'(f'(z-h({\bf r},t))) + \eta({\bf r},z,t).
\end{eqnarray}
We now multiply both sides of this equation by $f'(z-h({\bf r},t))$ and defining $\zeta=z-h({\bf r},t)$ we integrate  $\zeta$ over $[-\infty,\infty]$ and use the following identities
\begin{eqnarray}
\int_{-\infty}^\infty d\zeta f'(\zeta)f''(\zeta) &=& [\frac{1}{2}f'^2(\zeta)]_{-\infty}^\infty =0\\
\int_{-\infty}^\infty d\zeta f'(\zeta) V'(f) &=& \int_{-\infty}^\infty d\zeta\frac{d V(f)}{d\zeta}= [V(f(\zeta))]_{-\infty}^\infty=0,
\end{eqnarray} 
note that the first relation above holds as $f(\zeta)=\pm \phi_c$ as $\zeta\to\pm \infty$ and the second as
$V(\phi_c)=V(-\phi_c)=0$.
The terms that are left then give
\begin{equation}
-\int_{-\infty}^\infty f'^2(\zeta)d\zeta\ \frac{\partial h({\bf r},t)}{\partial t}
= -\alpha\int_{-\infty}^\infty f'^2(\zeta)d\zeta \ \kappa \nabla^2 h({\bf r},t) + \int_{-\infty}^\infty d\zeta \eta({\bf r},\zeta+ h({\bf r},t))f'(\zeta)
\end{equation}
Now using the Cahn-Hillard estimate of the surface tension Eq. \eqref{CHST} this becomes
\begin{equation}
\frac{\sigma}{\kappa} \frac{\partial h({\bf r},t)}{\partial t}
= \alpha\sigma \nabla^2 h({\bf r},t) +\xi({\bf r},t),
\end{equation}
where the noise term is given by
\begin{equation}
\xi({\bf r},t)= \int_{-\infty}^\infty d\zeta \eta({\bf r},\zeta+ h({\bf r},t))f'(\zeta)
\end{equation}
The noise term has zero mean and correlation function
\begin{eqnarray}
\langle \xi({\bf r},t)\xi({\bf r}',t')\rangle &=&2\alpha T\delta(t-t')\delta({\bf r}-{\bf r}')\int_{-\infty}^\infty d\zeta d\zeta' \delta(\zeta-\zeta')f'(\zeta)f'(\zeta')\\
&=& 2\alpha T\delta(t-t')\delta({\bf r}-{\bf r}')\int_{-\infty}^\infty d\zeta f'^2(\zeta)= \frac{2\alpha T\sigma}{\kappa}\delta(t-t')\delta({\bf r}-{\bf r}').
\end{eqnarray}
This now gives
\begin{equation}
\frac{\partial h({\bf r},t)}{\partial t}= \kappa\alpha \nabla^2 h({\bf r},t) + \eta({\bf r},t)
\end{equation}
where 
\begin{equation}
\langle \eta({\bf r},t)\eta({\bf r}',t')\rangle = \frac{2\alpha T\kappa}{\sigma}\delta(t-t')\delta({\bf r}-{\bf r}').
\end{equation}
Now defining $\alpha' = \frac{\kappa\alpha}{\sigma}$ we can write
\begin{equation}
\frac{\partial h({\bf r},t)}{\partial t}= \alpha' \sigma\nabla^2 h({\bf r},t) + \eta({\bf r},t).
\end{equation}
This has the form of model $A$ dynamics (as in Eq. \eqref{MA})  for the height profile with Hamiltonian
$H_{eff}$ as given in \eqref{heff}, that is to say we can write
\begin{equation}
\frac{\partial h({\bf r},t)}{\partial t}= -\alpha' \frac{\delta H_{eff}[h]}{\delta h({\bf r})} + \eta({\bf r},t),
\label{ew}
\end{equation}
and where 
\begin{equation}
\langle \eta({\bf r},t)\eta({\bf r}',t')\rangle= 2T\alpha'\delta(t-t').
\end{equation}
This dynamical calculation is thus consistent with the idea of describing the surface in terms of a height variable with an energy given by the surface tension. The equation \eqref{ew} is known as the Edwards-Wilkinson equation. We can use this equation to determine how the domains of a coarsening systems grow at low temperatures. To do this we ignore the noise term and assume that at $t=0$ the correlations of the height are short range so
\begin{equation}
C({\bf r}-{\bf r}',0)= \langle h({\bf r},0)h({\bf r}',0)\rangle =C_0 \delta({\bf r}-{\bf r}').
\end{equation}
In Fourier space the noiseless Edwards-Wilkinson equation becomes
\begin{equation}
\frac{\partial\tilde h({\bf k},t)}{\partial t} = -\alpha'\sigma \tilde h({\bf k},t) ,
\end{equation}
and so we find
\begin{equation}
\tilde h({\bf k},t) = h({\bf k},0)\exp(-\alpha'\sigma{\bf k}^2 t).
\end{equation}
We thus find 
\begin{equation}
\langle \tilde h({\bf k},t)\tilde h({\bf k}',t')\rangle = \langle h({\bf k},0)h({\bf k}',0)\rangle \exp(-\alpha'\sigma[k^2+k'^2] t).
\end{equation}
Now recall that if 
\begin{equation}
\langle  h({\bf r},t) h({\bf r}',t')\rangle =C({\bf r}-{\bf r}',t)
\end{equation}
then
\begin{equation}
\langle \tilde h({\bf k},t)\tilde h({\bf k}',t')\rangle= (2\pi)^d \delta({\bf k}+{\bf k}') \tilde C({\bf k},t),
\end{equation}
where 
\begin{equation}
\tilde C({\bf k},t)= \int d{\bf r} \exp(-i{\bf k}\cdot {\bf r})C({\bf r},t),
\end{equation}
is the Fourier transform of the correlation function which is a function of a single position due to invariance by translation in space, and $d$ is the dimension of space (so here $d=2$ for a surface in 3d space and $d=1$ for a surface in a 2d space). Putting all this together gives
\begin{equation}
\tilde C({\bf k},t)= C_0 \exp(-2\alpha'\sigma k^2 t).
\end{equation}
Inverting the Fourier transform gives
\begin{equation}
C({\bf r},t)= \frac{C_0}{(8\pi \alpha'\sigma t)^{\frac{d}{2}}} \exp(-\frac{{\bf r}^2}{16\pi \alpha'\sigma t}).
\end{equation}
From this we see that if $C({\bf r},t)\sim g(\frac{{\bf r}}{\ell(t)})r(t)$ then the length scale $\ell(t)\sim t^{\frac{1}{2}}$, this agrees with is found in the Ising model under Glauber dynamics, where the growth exponent is also given by $z=\frac{1}{2}$.
\\

\noindent{\bf Model B dynamics}

For model B dynamics, we take the same ansatz as in Eq. \eqref{hans} but we rewrite the model B dynamics as
\begin{equation}
-\nabla^{-2} \frac{\partial\phi({\bf x},t)}{\partial t} = -D \frac{\delta H}{\delta \phi({\bf x})}+ 
\theta({\bf x},t),
\end{equation}
here $-\nabla^{-2}$ represents the Green's function $G$ which obeys 
\begin{equation}
\nabla^2 G({\bf x}-{\bf x'}) = -\delta({\bf x}-{\bf x}'),
\end{equation}
and 
\begin{equation}
\theta({\bf x},t)=-\nabla^{-2}\eta({\bf x},t)= \int d{\bf x}'G({\bf x}-{\bf x'})\eta({\bf x},t)
\end{equation}
The correlation function of $\theta({\bf x},t)$ is given by
\begin{eqnarray}
\langle \theta({\bf x},t)\theta({\bf y},t')\rangle &=&-2DT\delta(t-t') \int d{\bf x}'G({\bf x}-{\bf x'})d{\bf y}'G({\bf y}-{\bf y'})\nabla^2\delta({\bf x}'-{\bf y}') \\
&=& -2DT\delta(t-t') \int d{\bf x}'G({\bf x}-{\bf x'})d{\bf y}'\nabla^2G({\bf y}-{\bf y'})\delta({\bf x}'-{\bf y}') \\
&=& 2DT\delta(t-t') G({\bf x}-{\bf y}).
\end{eqnarray}
where we have integrated by parts in the second line and used 
\begin{equation}
-\nabla^2G({\bf y}-{\bf y'})= \delta({\bf y}-{\bf y}'),
\end{equation}
in the third.

Now mutliplying by $f'(z-h({\bf r},t))$ and integrating $z$ over $[-\infty,\infty]$, we find
\begin{eqnarray}
&&-\int dz f'(z-h({\bf r},t))\int dz'd{\bf r}'\  G(z-z',{\bf r}-{\bf r}') f'(z'-h({\bf r}',t))\frac{\partial h({\bf r}',t)}{\partial t} = \\
&&-D\sigma \nabla^2 h({\bf r},t) + \chi({\bf r},t),
\end{eqnarray}
with the noise
\begin{equation}
\chi({\bf r},t)= \int dz f'(z-h({\bf r},t)) \theta({\bf r},z,t).
\end{equation}
As we assume that the height fluctuations are small we keep only the lowest order terms in $h$ in the deterministic terms and the noise, we will see later that this is compatible thermodynamically. We thus have
\begin{eqnarray}
&&-\int dz \ f'(z)\int dz'd{\bf r}'\  G(z-z',{\bf r}-{\bf r}') f'(z')\frac{\partial h({\bf r}',t)}{\partial t} = \\
&&-D\sigma \nabla^2 h({\bf r},t) + \chi({\bf r},t),
\end{eqnarray}
and now  the noise is given by
\begin{equation}
\chi({\bf r},t)= \int dz\  f'(z) \theta({\bf r},z',t).
\end{equation}
This equation which is linear in $h$ can now be Fourier transformed in the plane ${\bf r}$ and in terms of the Fourier transform of $h$ we find
\begin{equation}
-\int dz \ f'(z)\int dz'd{\bf r}'\ \tilde G(z-z',{\bf k}) f'(z')\frac{\partial\tilde h({\bf k},t)}{\partial t} = 
Dk^2\sigma \tilde h({\bf k},t) + \tilde \chi({\bf k},t).\label{bstep}
\end{equation}
The Fourier transform of $G$ in the ${\bf r}$ plane obeys
\begin{equation}
\frac{d^2 \tilde G(z-z',{\bf k} )}{dz^2}-k^2 \tilde G(z-z',{\bf k} )=-\delta(z-z')
\end{equation}
and the solution to this equation (with the boundary condition that $\tilde G(z-z',{\bf k} )\to 0$ as $|z-z'|\to\infty$)  is
\begin{equation}
\tilde G(z-z',{\bf k}) = \frac{\exp(-k|z-z'|)}{2k},
\end{equation}
and note that $k=|{\bf k}|$. 
Next we make the sharp interface approximation where we write
\begin{equation}
f(z) = 2\phi_c \delta(z),\label{sharp}
\end{equation}
that is to say we have replaced the smooth kink solution with a step like solution
$f(z) = \phi_c\  {\rm sgn}(z)$. This then gives
\begin{equation}
-4\phi_c^2 \tilde G(0,k) \frac{\partial h({\bf k},t)}{\partial t} = 
Dk^2\sigma  \tilde h({\bf k},t) + \tilde \chi({\bf k},t),
\end{equation}
which we rewrite as
\begin{equation}
\frac{\partial \tilde h({\bf k},t)}{\partial t} = -\frac{Dk^3\sigma}{2\phi_c^2} \tilde h({\bf k},t) + \tilde \xi({\bf k},t),
\end{equation}
with 
\begin{equation}
\tilde \xi({\bf k},t)= - \frac{k}{2\phi_c^2}\tilde \chi({\bf k},t),
\end{equation}
where 
\begin{equation}
\tilde \chi({\bf k},t)= \int dz\  f'(z) \tilde\theta({\bf k},z,t),
\end{equation}
The correlation function of $\tilde \theta({\bf k},t)$ is 
\begin{equation}
\langle \theta({\bf k},t)\theta({\bf k}',t')\rangle = 2DT(2\pi)^d \delta(t-t') \delta({\bf k}+{\bf k'}) \tilde G(z-z',k)
\end{equation}
and from this we find 
\begin{equation}
\langle \chi({\bf k},t)\chi({\bf k}',t')\rangle  = 2DT(2\pi)^d \delta(t-t') \delta({\bf k}+{\bf k'}) \int dz dz'
f(z) f(z') \tilde G(z-z',k),\label{bstep2}
\end{equation}
now using the sharp interface approximation Eq. (\ref{sharp}) we obtain
\begin{equation}
\langle \chi({\bf k},t)\chi({\bf k}',t')\rangle  = 2DT(2\pi)^d \delta(t-t') \delta({\bf k}+{\bf k'}) \frac{2\phi_c^2}{k},
\end{equation}
and consequently
\begin{equation}
\langle \xi({\bf k},t)\xi({\bf k}',t')\rangle = 2DT(2\pi)^d \delta(t-t') \delta({\bf k}+{\bf k'}) \frac{k}{2\phi_c^2}.
\end{equation}
Finally in we find the interface dynamics for model B in Fourier space is
\begin{equation}
\frac{\partial h({\bf k},t)}{\partial t} = -\frac{Dk^3\sigma}{2\phi_c^2} \tilde h({\bf k},t) + \tilde \xi({\bf k},t),\label{modBFT}
\end{equation}
In real space this has the form
\begin{equation}
\frac{\partial h({\bf r})}{\partial t}= -L \frac{\delta H_{eff}}{\delta h ({\bf r})} + \xi({\bf r},t).
\end{equation}
where the operator $L$ is defined via its Fourier transform
\begin{equation}
\tilde L({\bf k}) = \frac{Dk}{2\phi_c^2}.
\end{equation}
Now if we look at Eq. \eqref{modBFT} we see that solving the equation without noise will give a function of $k^3t$, which in real space corresponds to $x^3/t$. From this we see that the coarsening length scale grows as $\ell(t) \sim t^{\frac{1}{3}}$ and consequently the coarsening exponent is $z=\frac{1}{3}$.  Coarsening for conserved model B or diffusive dynamics is slower than that of model A. One of the reasons for this slowing down with respect to nonconserved dynamics is that material must be physically transported by diffusion (by exchanging spins in the language of lattice spin models), where as for  model A dynamics the composition can change at any given point by {\em spin flipping}. As a cautionary note, if we had taken the
Hamiltonian in Eq. \eqref{heff} and applied model B conserved dynamics, as in Eq. \eqref{MB},  for the height field we would not have obtained this equation. 

We can actually do better than the above sharp interface approximation as Eq. \eqref{bstep} can be written as
\begin{equation}
Q(k)\frac{\partial\tilde h({\bf k},t)}{\partial t} = 
-Dk^2\sigma \tilde h({\bf k},t) - \tilde \chi({\bf k},t).\label{bstep2}
\end{equation}
where 
\begin{equation}
Q(k)= \int dzdz' \ f'(z)\ \tilde G(z-z',{\bf k}) f'(z').
\end{equation}
Notice that from Eq. \eqref{bstep2} that
\begin{equation}
\langle \chi({\bf k},t)\chi({\bf k}',t')\rangle  = 2DT(2\pi)^d \delta(t-t') \delta({\bf k}+{\bf k'}) Q(k).\end{equation}
and so 
\begin{equation}
\frac{\partial\tilde h({\bf k},t)}{\partial t}= -\tilde L(k) \tilde \mu({\bf k}) + \eta({\bf k}),
\end{equation}
where $\mu({\bf x})=\delta H_{eff}/\delta h({\bf x}) $ and $\tilde L(k) = D/Q(k)$ and 
\begin{equation}
\langle \eta({\bf k},t)\eta({\bf k}',t')\rangle  = 2T(2\pi)^d \delta(t-t') \delta({\bf k}+{\bf k'}) \tilde L(k).\end{equation}
\section{Systems driven by imposed hydrodynamic flows}
{\bf Here you need to discuss the experiments get some nice photos etc}

Here we consider what happens when a system is driven out of equilibrium, by driven we mean that energy is injected into the system by a laser for instance as discussed in the introduction or by inducing a hydrodynamics flow, for instance a shear flow induced in a Couette cell. In principle we should analyse this system with model H dynamics which couples diffusive model B dynamics to hydrodynamics in the low Reynolds number Stokes flow regime. In this dynamics the order parameter field will itself induce a hydrodynamics flow which will modify
the imposed one. However this full situation is very difficult to analyse and to a first approximation we can assume that the back reaction of the order parameter field on the hydrodynamic flow is small with respect to the imposed hydrodynamic flow and so we can simply write
\begin{equation}
\frac{\partial \phi ({\bf x},t)}{\partial t} + \nabla\cdot({\bf  v}({\bf x})\phi({\bf x},t)) =- L\frac{\delta H}{\delta \phi({\bf x})} + \eta({\bf x},t),\label{drive}
\end{equation}
where $L$ is given by the underlying model A or B dynamical operatorand the noise has the correlation function as given by Eq. \eqref{gnoise}, and ${\bf v}({\bf x})$ is the imposed (time independent) hydrodynamic flow or can equally well be an external drive imposed on the colloidal particles, due to the gravitational or electric field for example. 

The simplest case one can consider is where the driving field ${\bf v}({\bf x}) ={\bf v}_0$ is uniform . Unfortunately this simple driving does not lead to a new steady state. Basically all of the colloidal particles acquire an an average velocity ${\bf v}$ and so move along at the same speed relative to each other. Mathematically this can be seen by making the Galilean transformation
\begin{equation}
\phi ({\bf x},t)= \phi({\bf x}-{\bf v}_0t,t) = \phi({\bf y},t).
\end{equation}
this transform eliminates the driving from the evolution equation \eqref{drive} and so we find an equilibrium system. 

The most studies example is where the driving is a shear flow, this corresponds to the experiments of Derks et al and the numerical simulations of Smith et al and the analytical work
of Bray and Cavagnah. In Bray and Cavagna, the effective dynamics of the surface term 
in the presence of a shear flow, parallel to the interface,
\begin{equation}
{\bf v}({\bf x}) = \gamma z {\bf e}_x
\end{equation}
was studied using the method explained in section (\ref{heightd}). The addition of a shear flow leads to the appearance of a nonlinear term in $h$ and the interface statistics thus become non-Gaussian.


\chapter{Méthodes numériques}
\label{chap-sim}

En 1949, Metropolis \cite{metropolis_monte_1949} découvre une méthode pour calculer via des simulations numériques de Monte Carlo, la moyenne d'observables statistiques. Si $Q$ est une quantité observable appartenant à un système statistique, comme l'énergie interne ou la densité moyenne de particules par site, alors la moyenne est calculée en pondérant la valeur de l'observable sur toutes les configurations $C$ du système par rapport au poids statistique de ces configurations. Si l'on considère le système en équilibre thermodynamique alors chaque configuration $C$ suit une distribution de Gibbs-Boltzmann, et la moyenne $<Q>$ est vaut
\begin{align}
    <Q> = \frac{\sum_{C} Q(C) \exp(-\beta E(C))}{\sum_{C} \exp(-\beta E(C))}
\end{align}
Pour un système SOS de taille $100\times100$ par exemple, petit par rapport à la limite thermodynamique comme discuté avec la figure \ref{fig-thermo-libre}, il existe $100^{100}$ configurations possibles, bien qu'une simulation numérique ne puisse explorer qu'environ $10^8$ configurations différentes en un temps CPU raisonnable.
Les modèles sur réseau se prêtent parfaitement aux simulations numériques de Monte Carlo, où le but est de calculer la valeur moyenne des observables telles que l'énergie interne ou la densité moyenne de particule par site. Toutes ces quantités peuvent être calculées directement pour le modèle SOS dans l'ensemble grand-canonique à l'aide des valeurs propres de la matrice de transfert, mais il est impossible d'utiliser une telle méthode dans l'ensemble canonique, comme expliqué dans le chapitre précédent.

Dans ce chapitre, nous commençons par expliquer le principe des simulations de Monte Carlo Metropolis, et comment choisir l'ensemble thermodyique de la simulation numérique. En plus d'étudier l'ensemble canonique, les simulations numériques offrent la possibilité d'étudier les régimes hors équilibre, dont nous justifierons la validité.
Nous finirons le chapitre par expliquer comment accélérer la vitesse de simulation grâce à la parallélisation, ainsi que d'autres astuces de programmation, en insistant sur les écueils techniques à éviter. 

Je remercie le Mésocentre de Calcul Intensif Aquitain (MCIA)\footnote{\url{https://redmine.mcia.fr/projects/mcia}} sur lequel j'ai effectué la très grande majorité de mes simulations numériques. 
L'intégralité du code produit pour cette thèse est accessible sur Github \footnote{\url{https://github.com/Bulbille/Curta}} sous la licence Creative Commons BY 3.0 \footnote{\url{https://creativecommons.org/licenses/by/3.0/fr/}}. Les simulations numériques ont été codées en C++, la parallélisation avec la librairie MPI, l'automatisation du lancement des jobs en Bash, et la visualisation des données ainsi que les diagonalisations des matrices de transfert sous Python.

{\color{red} actuellement, est-ce que j'ai le droit de diffuser librement mon code ? Le CNRS autorise la libre diffusion du code ?}

%%%%%%%%%%%%%%%%%%%%%%%%%%%%%%
    \section{Algorithme de Monte Carlo Metropolis}
%%%%%%%%%%%%%%%%%%%%%%%%%%%%%%

Les simulations de Monte Carlo explorent l'espace des configurations de manière aléatoire \cite{newman_monte_1999} avec une probabilité $p(C$ que nous définirons plus tard. En choisissant $M$ états ${C_0,...,C_M}$, l'estimateur $Q_M$ de $Q$ est donnée par
\begin{align}
    Q_M = \frac{\sum_{i=0}^M Q(C_i) p(C_i)^{-1} \exp(-\beta E(C_i))}{\sum_{i=0}^M  p(C_i)^{-1} \exp(-\beta E(C_i))}
\end{align}
Lorsque $M$ augmente, l'estimateur devient une estimation de plus en plus précise de $<Q>$, jusqu'à la limite $Q_{M\to \infty} = <Q>$. Si l'on choisit les configurations sur lesquelles on échantillone le système selon la distribution à l'équilibre de Gibbs-Boltzmann $p(\nu) = Z^{-1} e^{-\beta E(C)}$, alors l'éstimateur de $<Q>$ devient
\begin{align}
    Q_M = \frac{1}{M} \sum_{i=0}^M Q(C_i)
\end{align}
On se pose maintenant la question de savoir comment choisir les configurations afin que chacune apparaisse avec la bonne probabilité de Boltzmann. 

Une dynamique pour les systèmes avec une espace des phases discret peut être construit à partir de chaînes de Markov. On laisse la dynamique évoluer dans un discret noté $n$, et $p_n(C)$ la probabilité que le système soit dans l'état $C$ au temps $n$. Au pas de temps suivant, si le système est dans l'état $C$ il peut sauter vers un autre état $C'$ avec la probabilité de transition $\rho(C\to C')$. Le système au tempst $n+1$ dépend alors uniquement de l'état au temps $n$ : c'est un processus markovien. La probabilité $p_{n+1}(X)$ d'être dans l'état $C$ au temps $n+1$ est possible si le système était dans l'état $C$ au temps $n$ et y reste avec une probabilité $\rho(C\to C)$ , ou s'il est dans un état $C'$ et bouge vers l'état $C$ avec une probabilité $\rho(C'\to C)$. On a alors l'équation maîtresse
\begin{align}
    p_{n+1}(C) =  \rho(C\to C) p_n(C) + \sum_{C'\neq C} \rho(C'\to C) p_n(C')
\end{align}
Puisque $\rho(C' \to C)$ est une probabilité, on a la condition suivante
\begin{align}
    \sum_{C'} \rho(C' \to C) = 1
    \label{norm}
\end{align}
Maintenant, si la dynamique décrit un système physique en interaction avec un  réservoir de chaleur, la distribution à l'équilibre est donnée par
\begin{align}
    p_{eq}(C) = \frac{\exp(-\beta E(C))}{Z}
\end{align}
avec $Z$ la fonction de partition canonique. Puisque la distribution à l'équilibre n'évolue pas au cours du temps, on a
\begin{align}
    p_{eq}(C) =  \rho(C\to C) p_{eq}(C) + \sum_{C'\neq C} \rho(C'\to C)p_{eq}(C')
    \label{p-eq-mc}
\end{align}
Une autre condition que l'on impose à notre chaîne de Markov afin qu'elle génère une probabilité de distribution de Boltzmann après équilibrage, est qu'elle respecte le bilan détaillé. Afin qu'un système respecte le bilan détailĺé, il faut que le taux auquel il fait des transitions vers à partir de n'importe quel état $C$ soit égal. Mathématiquement, cela revient à dire que
\begin{align}
    \sum_{C'} p(C) \rho(C \to C') = \sum_{C'} p(C') \rho(C' \to C)
\end{align}
On peut démontrer que cette relation est équivalente à \cite{newman_monte_1999} 
\begin{align}
    \frac{\rho(C'\to C)}{\rho(C \to C')} = \frac{p(C)}{p(C')} = \frac{\exp(-\beta E(C))}{\exp(-\beta E(C'))}
\end{align} 
En adoptant le bilan détaillé, on voit facilement que la distribution à l'équilibre calculée via \ref{p-eq-mc} redonne bien la distribution de Gibbs-Boltzmann.
Durant une étape de Metropolis, la probabilité pour que la transition $C\to C'$ soit acceptée est 
\begin{align}
    p_a(C\to C')
\end{align}


Systems with a canonical heat bath can be simulated on a computer using an algorithm
obeying detailed balance. For example consider a system of $N$ Ising spins $S_i=\pm1$ interacting via a Hamiltonian $H(S_1, S_2,\cdots S_N)$. We choose $1$ of the spins randomly uniformly with a probability $p=1/N$ and calculate the new energy of the system when the spin, $S_j$ say is changed to $-S_j$. In Metropolis dynamics the probability of accepting the spin flip $p_a(S_j\to -S_j)$ is given by 
\begin{align}
p_a(S_j\to -S_j)
\end{align}
if $H(S_1, S_2,\cdots, -S_j,\cdots S_N) < H(S_1, S_2,\cdots, S_j,\cdots S_N)$ but if $H(S_1, S_2,\cdots, -S_j,\cdots S_N) > H(S_1, S_2,\cdots, S_j,\cdots S_N)$ then the flip is accepted with probability 
\begin{align}
p_a(S_j\to -S_j) = \exp\left[ -\beta\left(H(S_1, S_2,\cdots, -S_j,\cdots S_N) - H(S_1, S_2,\cdots, S_j,\cdots S_N)\right)\right] <1.
\end{align}
The total probability at a given discrete time of changing $S_j$ is thus equal to 
\begin{align}
p(S_j\to -S_j) = \frac{1}{N} p_a(S_j\to -S_j)
\end{align}
as we choose the spin $S_j$ with probability $1/N$. Therefore we have
\begin{align}
\frac{p(S_j\to -S_j)}{p(-S_j\to S_j)} = \frac{p_a(S_j\to -S_j)}{p_a(-S_j\to S_j)}.
\end{align}
In the case where the change $S_j\to -S_j$ lowers the energy we have
\begin{align}
p_a(S_j\to -S_j) =1,
\end{align}
however the reverse move $-S_j\to S_j$ costs energy so 
\begin{align}
p_a(-S_j\to S_j) =\exp\left[ -\beta\left(H(S_1, S_2,\cdots, S_j,\cdots S_N) - H(S_1, S_2,\cdots, -S_j,\cdots S_N)\right)\right],
\end{align}
which gives
\begin{eqnarray}
\frac{p(S_j\to -S_j)}{p(-S_j\to S_j)} &=& \frac{1}{\exp\left[ -\beta\left(H(S_1, S_2,\cdots, S_j,\cdots S_N) - H(S_1, S_2,\cdots, -S_j,\cdots S_N)\right)\right]} \nonumber \\
&=&\frac{ \exp\left[ -\beta H(S_1, S_2,\cdots, -S_j,\cdots S_N)\right]}{\exp\left[ -\beta H(S_1, S_2,\cdots, S_j,\cdots S_N)\right]},
\end{eqnarray}
and so in this case we see that detailed balance is respected. In the case of a move which increases the energy it is easy to see that detailed balance is again respected.  

If we consider a case where the spins $+$ represent one type of particle and the $-$ another type and insist that the total  chemical composition remains the same the above dynamics is not correct as you cannot convert a $+$ into a $-$ and vice-a-versa. However a $+$ next to a $-$ can change places. Kawasaki dynamics chooses a neighbouring pair of $+$ and $-$ and tries to switch their positions, e.g. $.+-.\to .-+.$,  the move is accepted with probability $1$ if the energy change $\Delta E<0$  and with probability $p_a=\exp\left(-\beta\Delta E\right)$ if $\Delta E>0$.

Practically in a computer program if $\Delta E >0$ one draws a uniformly distributed random
number $r\in[0,1]$ (for example {\tt rand} in Fortran and Matlab), if $r< p_a $ the move accepted but if $r>p_a$ it is refused and the system stays in its initial state.



\section{Suite}


qui définit la fonction de partition $\mZ$. Dans un algorithme de Metropolis, on met à jour le micro-état en prenant un site $i$ au hasard
\footnote{L'utilisation d'un générateur de nombre aléatoire (\textit{pRNG}) efficace est primordial. Il est déconseillé d'utiliser le générateur standard \textit{default\_random\_engine} de la librairie C++ \textit{rand} et conseillé d'opter pour des générateurs \textit{sfc64} ou \textit{xoroshiro}. Pour un pRNG booléen performant, voir \url{https://martin.ankerl.com/2018/12/08/fast-random-bool/}. Pour accélérer encore plus les calculs, ne pas oublier d'utiliser le flag d'optimisation \textit{-O3}  sur \textit{gcc} si vous codez en C/C++. Tout cela combiné accélère le code d'un facteur 20 environ. \newline
De plus, bien que la librairire OpenMP pour paralléliser le code soit simple d'utilisation, elle gère très mal - de sa nature de mémoire partagée - les pRNG. Je conseille vivement l'utilisation de la librairie MPI qui assure une étanchéité au niveau des pRNG entre chaque thread.} 
et en le changeant légèrement vers un état $\nu$. Dans un système d'Ising, nous choisissons un spin $\sigma_i$ au hasard et regardons s'il peut être renversé ou échangé avec l'un de ses plus proches voisins. Dans le modèle SOS, nous choisissons une colonne $h_i$ au hasard et regardons s'il est possible d'ajouter ou de retirer une unité à la hauteur (c'est le nombre de particules sous l'interface au site $h_i$), ou d'échanger une particule d'une colonne vers une de ses plus proches voisins.
La différence d'énergie notée $\Delta E(\mu \rightarrow \nu)$ donne la probabilité de transition entre les deux. Si l'état final $\nu$ a une énergie inférieure à l'état initial, alors il est forcément plus probable que $\mu$, et nous acceptons le changement. Dans le cas où $E_\nu \greater E_\nu$, on accepte le changement avec une probabilité satisfaisant au bilan détaillé pour une marche markovienne satisfaint à l'état d'équilibre de Botlzmann
\begin{align}
\frac{p(\mu \rightarrow \nu)}{p(\nu \rightarrow \mu)} = e^{-\Delta Ep(\nu \rightarrow \mu)}
\end{align}
ce qui nous donne la probabilité de transition $\mu \to \nu$ de Metropolis
\begin{align}
	p(\mu \rightarrow \nu) = min(1,e^{-\beta \Delta E(\mu \rightarrow \nu)})
\end{align}
Ensuite on prend un nombre aléatoire $q$ entre $0$ et $1$. Si $q < p(\mu \rightarrow \nu)$, alors la transition est validée. Une étape de Monte Carlo est achevée lorsque $L$ tentatives de transition ont été faites. Cependant, il est possible d'accélérer l'algorithme en utilisant un temps continu \cite{newman_monte_1999} ou en prenant en compte les états dont la transition a été refusée \cite{frenkel_speed-up_2004}.
L'erreur obtenue à la fin sur notre observable $<A>$ au cours d'une simulation ayant duré $t_{max}$ étapes de Monte Carlo est 
\begin{align}
	E(A) = \sqrt{\frac{2 \tau}{t_{max}} (<A^2>-<A>^2)} 
\end{align}
Cette variance dépend du temps de corrélation $\tau$ puisque si deux micro-états sont très rapprochés dans le temps , l'observable en question n'aura pas grandement évolué. En pratique, il suffit que $\frac{\tau}{t_{max}} \less 10^{-4}$ pour obtenir une erreur inférieure à $1\%$. Ce temps de corrélation $\tau$ se calcule via la fonction d'auto-corrélation 
\begin{align}
\mC(t) = <A(t')A(t+t')>-\langle A \rangle^2 = \frac{1}{T_{max}}\int_0^{T_{max}}A(t')A(t+t')-<A>^2 dt' \simeq e^{-\frac{t}{\tau}}
\end{align}
qui se comporte comme une somme d'exponentielles, mais où dans la limite thermodynamique, seul le mode de relaxation le plus long compte\cite{wansleben_monte_1991}. En supposant la limite thermodynamique, l'ordre de grandeur de $\tau$ - et donc de la variance de nos observables - est donnée par le calcul de l'intégrale\footnote{Je recommande d'intégration de Simpson.}
\begin{align}
	\tau = \int_0^{\infty} \mC(t)/\mC(0) dt
	\label{tau_cor}
\end{align}
Le calcul de la plus grande longueur de corrélation $\xi$ du système se fait de manière analogue en intégrant la fonction de corrélation spatiale définie par
\begin{align}
\mC(x) = \frac{1}{L} \sum_{x'}^L A(x')A(x+x')-<A>^2 \simeq e^{-\frac{x}{\xi}}
\end{align}
Une discussion plus rigoureuse sur la forme de la fonction de corrélation spatiale sera donnée dans la section \ref{sec_laser}.

	\subsection{Ensemble grand-canonique : algorithme de Glauber}

\begin{figure}[h]
	\centering
	\includegraphics[scale=1]{numerical/sos-glau-eq-cor.pdf}
	\caption{Courbe de l'énergie (haut) et fonction d'auto-corrélation (bas) dans avec un \textbf{paramètre d'ordre non-conservé} à partir de la condition initiale. Le temps d'équilibrage (en étapes de Monte Carlo) diminue avec la température, tandis que le temps de corrélation reste relativement constant. Le temps de corrélation étant extrêmement faible, $10^7$ étapes de Monte Carlo suffisent à avoir une erreur de moins de $0.1\%$ sur les moyennes mesurées.}
	\label{eq-glau}
\end{figure}
	
Le dépôt de particules provenant d'un réservoir permet de faire grandir un cristal à partir d'un substrat. Ce genre de systèmes est défini par le potentiel chimique $\mu$ des particules, dans le solvant et appartient à l'ensemble grand-canonique. Dans ce cas, on choisit au hasard de manière uniforme une colonne $h_i$ dans laquelle on décide de mettre ou d'enlever une particule selon le flux de particules $\nu$ vu dans l'équation d'Edwards-Wilkinson \ref{edwards-wilkinson}. Si l'on se place à l'équilibre thermodynamique, c'est-à-dire qu'autant de particules se déposent au niveau de l'interface que de particules la quittent, alors il faut que la probabilité de ces deux événements soient égales entre elles, et donc égales à $50\%$.
Dans le cas où la géométrie est infinie, les valeurs des $h_i$ ne sont pas bornées, tandis que dans une géométrie torique de hauteur $L$, on rejette toutes les configurations qui ne respectent pas aux conditions $0 \leq h_i \leq L$.
En essayant d'aller du micro-état $\mu$ vers le micro-état $\nu$ où on a fait la transformation $h_i \rightarrow h_i + \alpha$ où $\alpha=\pm 1$, on obtient que la différence d'énergie est
\begin{align}
	\Delta E &= |h_{i-1}-(h_i \pm 1)| + |h_{i+1}-(h_i \pm 1)| - |h_{i-1}-h_i| - |h_{i+1}-h_i|  \\
		&= 2 \left( (h_i \leq h_{i-1}) + (h_i \geq h_{i+1}) -1 \right )
\end{align}
où $(h_i \leq h_{i-1})$ est un booléen valant $1$ si la condition est vraie, $0$ sinon.
Le changement de magnétisation est alors $\Delta M = \alpha$, et la largeur de l'interface, définie par $\sigma = \sum_i (h_i-h_{i+1})^2$, change comme
\begin{align}
	\Delta \sigma = 2 \alpha (h_{i+1}-h_i) + 2
\end{align}
On n'a donc pas besoin, à chaque pas de temps, de recalculer ces deux grandeurs, il suffit de les actualiser dans une variable pour avoir les observables à tout instant $t$.


Afin d'accélérer le processus d'équilibrage du système, il est recommandé de commencer directement avec la valeur moyenne de magnétisation calculée à partir de la matrice de transfert. On regarde ensuite le temps d'équilibrage par la courbe $E(t)$, en attendant d'atteindre la valeur à l'équilibre. 
À l'équilibre, le taux d'évaporation des particules doit être égal au taux de dépôt sur notre système. Cependant, en l'absence d'un potentiel qui contraint l'interface, l'interface est délocalisée, l'empêchant d'atteindre l'équilibre thermodynamique. C'est la raison pour laquelle une simulation numérique dans une dynamique de Glauber se doit toujours d'avoir un potentiel permettant d'obtenir la localisation d'une interface. 

	\subsection{Ensemble canonique : algorithme de Kawasaki}

\begin{figure}
	\centering
	\includegraphics[scale=1]{numerical/sos-kaw-eq-cor.pdf}
	\caption{Courbe de l'énergie (haut) et fonction d'auto-corrélation (bas) dans avec un \textbf{paramètre d'ordre conservé} à partir de la condition initiale. Le temps d'équilibrage (en étapes de Monte Carlo) diminue avec la température, tandis que le temps de corrélation reste relativement constant. Le temps de corrélation est similaire à la dynamique de Glauber, bien que l'équilibrage soit plus long à se faire.}
	\label{eq-kaw}
\end{figure}
	
La diffusion des particules - par exemple un polymère dans un solvant - est une dynamique locale qui conserve le paramètre d'ordre du notre système, nommément la magnétisation $m$. Dans ce cas, on choisit au hasard de manière uniforme deux colonnes $h_i$ et $h_{i+1}$ dans lesquelles on va essayer d'échanger une particule entre les deux colonnes. Afin de respecter le bilan détaillé, il faut que la probabilité de choisir le mouvement $h_i \rightarrow h_{i+1}$ soit égale à $h_{i+1} \rightarrow h_i$. On peut juste définir à nouveau "l'ajout" d'une colonne vers ou à partir de l'autre via la transformation $h_i \rightarrow h_i + \alpha$ et $h_{i+1} \rightarrow h_{i+1} - \alpha$ (avec $\alpha=\pm 1$), en respectant toujours les conditions aux bords en $y$. Trois termes dans l'énergie sont modifiées\footnote{Comme précédement, il existe une version booléenne de l'équation, mais sa longueur n'offre aucun avantage en terme d'implémentation dans le code comparé au gain de temps de CPU engendré.}
\begin{align}
	\Delta E = &|h_{i-1}-(h_i \pm 1)| + |h_{i+1} \pm 1 -(h_i \pm 1)| + |h_{i+1}\pm 1-(h_{i+2} )| \\
	- &|h_{i-1}-h_i| - |h_{i+1}-h_i| - |h_{i+1}-h_{i+2}|
\end{align}

La magnétisation totale est ainsi conservée, tandis que la largeur de l'interface $\sigma$ se calcule par
\begin{align}
	\Delta \sigma = 2 \alpha  + 1
\end{align}

	\subsection{Dynamique hors-équilibre}
L'ensemble grand-canonique ne nous permet d'avoir un système qu'à l'équilibre, puisqu'il est traduit par une dynamique non-locale. Seule une dynamique locale comme la dynamique de Kawasaki peut nous donner des états hors-équilibre. L'implémentation la plus simple est d'introduire un terme de cisaillement dans notre modèle lorsque l'on décide de bouger une particule. Ce cisaillement diminue l'énergie du micro-état lorsque la particule bouge dans un sens et l'augmente si elle bouge dans l'autre sens, ce qui brise le bilan détaillé. De nombreux travaux sur les systèmes hors-équilibre dans le modèle d'Ising ont été produits \cite{smith_interfaces_2008} présentant la diminution de la largeur de l'interface lorsque le cisaillement est produit de manière parallèle. 
On peut définir deux espèces de cisaillement parallèles.
Le premier genre de cisaillement se produit aux bords d'un liquide non-visqueux, ce qui ne permet de bouger que les particules aux bords du système : il n'est donc pas adaptable à un système infini ou semi-infini. Pour un système de taille $L$ et pour un module de cisaillement de $f$, la différence d'énergie supplémentaire est 
\begin{align}
	\Delta E_{bord} = f [ (h_i == 1 || h_{i+1} == L-1) - (h_i == L-1 || h_{i+1} == 0)  ]
\end{align}
Le second genre de cisaillement se produit aux bords d'un fluide permettant un transport visqueux, ce qui entraîne un cisaillement proportionnel à la distance aux bords comme sur la figure \ref{snap-ising-shear}. En supposant que le cisaillement est nul au niveau de l'interface et que les particules vont à gauche dans la partie basse du système (et à droite dans la partice haute du système), on obtient alors
\begin{align}
	\Delta E_{prop} = f h_i
\end{align}
Cependant, pour des raisons de facilité de calcul plus tard afin de comparer les simulations numériques aux résultats analytiques, on utilise un cisaillement uniforme qui pousse les particules dans un sens. Ce type de système correspond à un flux laminaire, par exemple dû à la gravité face à une interface verticale qui tire les particules vers le bas. La différence d'énergie devient
\begin{align}
	\Delta E_{uni} = \alpha f
\end{align}
où $\alpha = 1$ si la particule va vers la droite, $-1$ sinon. 
		
	\subsection{Modèle POP}		

Dans le modèle POP, le modèle n'est plus structuré en fonction des sites $i$ mais bien des particules $\sigma_(n) = i$, la hauteur d'un site\footnote{Cette hauteur est mise à jour à chaque étape mouvement d'une particule dans un second tableau.} devenant alors
\begin{align}
	h_i = \sum_{n=0}^N \delta_{\sigma_n,i}
\end{align}

Lors d'une dynamique de Kawasaki, à chaque étape, on choisit au hasard une particule parmi les $N$ présentes dans le système pour la déplacer d'une colonne. 

Il est également possible de donner des constantes de diffusion différentes à chaque particule\footnote{Grâce à la construction d'un générateur via \textit{random::discrete\_distribution} où chaque particule a une probabilité différente d'être sélectionnée. }  afin d'émuler différents types de particules. 

La question est plus délicate lorsqu'il s'agit d'une dynamique de Glauber. Puisque chaque particule a une probabilité d'être sélectionnée pour être détruite, comment choisir la probabilité d'ajouter une particule au système ? À l'équilibre, le flux de particules entrantes est égale au flux de particules sortantes, c'est-à-dire $p_{ajout}= p_{retrait} = 0.5\%$. Dans ce cas, il suffit de choisir un booléen au hasard, puis détruire une particule et son label ou ajouter une particule à un site particulier. L'avantage de la dynamique conservée est qu'il n'est pas nécessaire de reconstruire une distrubtion pRNG à chaque étape, même si le constructeur est rapide \footnote{Le constructeur a une complexité en $\mathcal{O}(n)$ au pire. \url{http://www.cplusplus.com/reference/random/discrete_distribution/discrete_distribution/}}.

%%%%%%%%%%%%%%%%%%%%%%%%%%%%%%%%%%
\section{Conclusion}
%%%%%%%%%%%%%%%%%%%%%%%%%%%%%%%%%%

Dans ce chapitre nous avons décrit les différentes méthodes de calcul numérique qui vont de pair avec le modèle A et le modèle B, et la manière de mesurer les observables ainsi que leur barre d'erreur. Dans la pratique, les temps de corrélation sont si faibles qu'il suffit de faire environ $10^7$ étapes de Monte Carlo afin d'obtenir de bonnes statistiques, ce qui en une dimension, est extrêmement rapide. La rapidité des simulations dans le modèle SOS nous permet ainsi d'étudier une très vaste plage de paramètres, que ce soit pour différentes températures, cisaillements, hauteurs maximales ou champs externes. 
mux

%%%%%%%%%%%%%%%%%%%%
\chapter{Equilibrium Interface models and their finite size effects}
%%%%%%%%%%%%%%%%%%%%

Models for interfaces arise naturally in phase separated systems, as explained in \ref{chap-int-dyn}. Finite size corrections are manifested when the correlation length becomes of the order of magnitude of the system's size. When undergoing a continuous phase separation, the system exhibits finite size corrections which are manifested by a long range critical Casimir interaction, which we describe in the first section. In the second section we examine finite size effects in continuous interface models in one dimension, and show that while they have similar long-range interactions, the forces induced by interface confinement are quite different, and will be compared to the GSOS model. In the last section, we compute the size-dependent eigenvalues of the transfer matrix for the free SOS model, and compare the results with previous works.

%%%%%%%%%%%%%%%%%%%%
\section{The Casimir effect}
\label{sec-casimir}
%%%%%%%%%%%%%%%%%%%%

Here we explain the critical Casimir effect. For completeness we start by explaining the quantum Casimir effect as it was in the quantum context that the effect was first observed \cite{casimir_attraction_1948}.
We also describe the basis of the Lifshitz theory that generalises Casimir's contribution to 
general dielectric materials beyond the perfectly conducting plate paradigm.

%%%%%%%%%%%%%%%%%%%% 
\subsection{Quantum Casimir effect}
%%%%%%%%%%%%%%%%%%%%
In an ideal conductor, the free charges can move arbitrarily quickly to cancel out any electric in the plane \cite{feynman_feynman_1963}. Thus, a perfectly conducting plate in the $(x,y)$ plane imposes boundary conditions on the electromagnetic field
\begin{equation}
{\bf E}\times {\bf n} = {\bf 0};\ {\bf B}\cdot {\bf n} = 0
\end{equation}
The quantum Hamiltonian for the electromagnetic field is given by 
\begin{equation}
H = \sum_{\bk,\lambda}\hbar \omega(\bk,\lambda)\left[a^\dagger(\bk,\lambda)a(\bk,\lambda) + {1\over 2}\right]
\end{equation}
Here $\lambda$ denotes the polarisation (there are two polarisation states) and $\bk$ the wave vector. The dispersion 
relation for photons is
\begin{equation}
\omega(\bk,\lambda) = |\bk|c.
\end{equation}
The ground state energy of the electromagnetic field \cite{casimir_attraction_1948} is given by
\begin{equation}
E_0 = < 0|H|0> = H = \sum_{\bk,\lambda}{1\over 2}\hbar \omega(\bk,\lambda)= \sum_\bk\hbar |\bk|c
\end{equation}

The presence of conduction plates at $z=0$ and $z=L$ means that the wave vectors $k_z$
must be quantised according to $k_z= n\pi/L$ where $n \in \{0,\ 1, \ 2, \ \cdots\}$ while
if the $(x,y)$ plane has a large area $A$ we can write
\begin{equation}
\sum_{k_x,k_y}\ \cdot= {A\over (2\pi)^2}\int d^2\bk \ \cdot
\end{equation}
This then gives 
\begin{eqnarray}
E_0(L) &=& {\hbar c A\over (2\pi)^2}\sum_{n=0}^\infty \int d^2\bk \left(\bk^2 + {n^2\pi^2\over L^2}\right)^{1\over 2} \\
&=& {\hbar c A\over (2\pi)}\sum_{n=0}^\infty \int_0^\infty kdk \left(\bk^2 + {n^2\pi^2\over L^2}\right)^{1\over 2}
\end{eqnarray}
The problem with the above expression is that it is clearly divergent. However it can be rendered finite by cutting off the high momentum degrees of freedom by writing
\begin{equation}
E_0(L) = {\hbar c A\over (2\pi)}\sum_{n=0}^\infty \int_0^\infty kdk \left(\bk^2 + {n^2\pi^2\over L^2}\right)^{1\over 2} f\left((\bk^2 + {n^2\pi^2\over L^2})^{1\over 2}\right)
\end{equation}
where $f$ is a smooth function such that $f(p)=1$ for $p\ll\Lambda$ and $f(p)=0$ for $p\gg\Lambda$. Here, $\Lambda$ is an ultraviolet cut-off and $f$ thus only counts the contribution of photons with a momentum less than $\hbar\Lambda$. For this sort of calculation to make physical sense the physical result we get at the end should be independent of both the choice of $f$ and $\Lambda$. 




In the limit $L\to\infty$ we can replace the sum over discrete modes by an integral, as usual in statistical physics, 
\begin{equation}
E_0(L) = {\hbar c A\over (2\pi)}\int_{0}^\infty {L\over \pi}d\nu \int_0^\infty kdk \left(\bk^2 + \nu^2\right)^{1\over 2}f\left((\bk^2 + \nu^2)^{1\over 2}\right)
\end{equation}
where we have used $d\nu = \pi/L$. We thus see that for large $L$ we have
\begin{equation}
E_0(L) = AL \epsilon_{bulk}
\end{equation}
where $\epsilon_{bulk}$ is a bulk energy density per unit of volume, that is to say the total energy is extensive.  The computation above only calculates the energy of the EM field between the plates. If the physical system extends up to $L'\gg L$, then the total energy of both the interior and the exterior of the plates is
given by
\begin{equation}
E_{total}(L) = E_0(L) + A(L'-L)\epsilon_{bulk}
\end{equation}
We see that the part of the energy that depends on $L$ is given by
\begin{equation}
U(L) = E_0(L)- AL\epsilon_{bulk}
\end{equation}
It is the derivative of $U$ which gives the physical interaction between the two plates, the pressure associated with this interaction in the colloid science literature is called the disjoining pressure \cite{stubenrauch_disjoining_2003} where to compute the effective interaction the bulk pressure has to be subtracted.
Now we simply write 
\begin{equation}
AL\epsilon_{bulk} = {\hbar c A\over (2\pi)}\int_{0}^\infty dn \int_0^\infty kdk \left(\bk^2 + {n^2\pi^2\over L^2}\right)^{1\over 2} f\left((\bk^2 + {n^2\pi^2\over L^2})^{1\over 2}\right)
\end{equation}
where we have put the $L$ dependence in the integral. This then gives
\begin{equation}
U(L) = {\hbar c A\over (2\pi)}\left[\sum_{n=0}^\infty g(n) -\int_0^\infty dn \ g(n)\right]
\end{equation}

where
\begin{equation}
g(n) = \int_0^\infty kdk \left(\bk^2 + {n^2\pi^2\over L^2}\right)^{1\over 2} f\left((\bk^2 + {n^2\pi^2\over L^2})^{1\over 2}\right) = {1\over 2}\int_{n^2\pi^2\over L^2}^\infty du u^{1\over 2} f(u^{1\over 2})
\end{equation}
We now use the Euler-Mauclarin formula
\begin{equation}
\sum_{n=0}^\infty g(n) -\int_0^\infty dn \ g(n) = -B_1g(0) -{1\over 2} B_2 g'(0) - {1\over 24} B_4 g'''(0) - \cdots
\end{equation}
where $B_n$ are the Bernoulli numbers\footnote{The first Bernoulli numbers are explicitly given by $B_1 =1\ , B_2 = {1\over 2} ,\ B_4 = -{1\over 30}$}.
We find that
\begin{equation}
g'(n) = -{\pi^3\over L^3} n^2 f({n\pi\over L})
\end{equation}
and noticing than in the region around $n=0$, $f=1$ is a constant, we show that
\begin{eqnarray}
g'(0) &=& 0 \\
g''(0) &=& 0 \\
g'''(0) &=& -{2\pi^3\over L^3}
\end{eqnarray}
Higher order derivatives are zero so the full result is given by the first three terms of the Euler-Maclaurin formula. We thus find
\begin{equation}
U(L) = {\hbar c A\over (2\pi)}\left[ -g(0) -{\pi^3\over 360 L^3}\right]
\end{equation}
The first term independent of $L$ can be interpreted as a surface energy. The effective $L$ dependent interaction is given by
\begin{equation}
U_{int}(L) = -{\hbar \pi^2 c A\over 720 L^3}
\end{equation}
We see that the effective interaction is attractive. Interestingly Casimir thought that his calculation could explain the stability of the electron \cite{casimir_attraction_1948,carazza_casimir_1986} The model of the electron is one of a perfectly conducting shell carrying an electric charge $e$. If the radius of the shell is $a$ then the electrostatic energy of due to the charge is given by
\begin{equation}
E_{Charge} = {e^2\over 8\pi a\epsilon_0}
\end{equation}
There is thus a repulsive force on the shell which should make it expand. Casimir thought that the Casimir force on a spherical geometry, 
if is an attractive force as is the case for the parallel plate geometry, could stabilise the electron. Clearly by dimensional analysis
\begin{equation}
E_{Cas} = -{Z\hbar c\over a}
\end{equation}
The balance of the Casimir and electric forces would then require
\begin{equation}
Z = {e^2\over 8\pi \hbar c}.
\end{equation}
However, in the case of conducting spherical shell, the constant $Z \simeq -0.046175$ is negative \cite{boyer_quantum_1968,milton_casimir_1978,bowers_casimir_1998}, while the same calculation for a cylindrical geometry predicts an attractive force \cite{milton_casimir_1978}. 

%%%%%%%%%%%%%%%%%%%%
\subsection{Lifshitz Theory}
%%%%%%%%%%%%%%%%%%%%
The Casimir calculation is based on the boundary conditions imposed on the EM field 
due to a conductor. However, this is an ideal mathematical limit, conductors being conductors because free charges can move to cancel out the electric field in the conducting surface.
The Casimir force can also be seen as due to correlations induced in the charge fluctuations in each plate, which allows for an alternative method based on sources which recovers the Casimir force \cite{schwinger_casimir_1978,schwinger_casimir_1992}. In a sense therefore the effect can be interpreted without reference to the zero point energy of the vacuum and the Casimir calculation works due to the fact that the mathematical limit in going to a perfect conductor works. 
Using a stochastic formulation of electrodynamics by Rytov \cite{rytov_principles_1989}, the Casimir calculation was generalized by Lifshitz for interactions between arbitrary electrical bodies, characterized by their local electric and magnetic response\cite{lifshitz_theory_1955}. 
Even though this theory is very general, the microscopic justification is not completely rigorous, source terms (random currents and dipole fluctuations) are introduced to Maxwell's equations to give a Langevin formulation of Maxwell's equations in the presence of dielectric bodies. The correlation functions of the white noise terms depend on the temperature of the system and are determined via the quantum fluctuation dissipation theorem. The Lifshitz theory is computationally difficult to work with and it was reformulated in a way more useful for practical calculations and that can be applied to experimental setups \cite{van_kampen_macroscopic_1968,ninham_van_1970}.
Rytov's formulation has the advantage that it can be used to treat out of equilibrium situations where different bodies are held at different temperatures. This allows both the computation of out of equilibrium forces and radiative heat transfer. 

The theory in the presence of electromagnetic media is written in terms of the electric and magnetic fields ${\bf E}$ and ${\bf B}$ and the displacement and magnetizing fields ${\bf D}$ and ${\bf H}$ which are assumed to obey local relations in real space and Fourier space
\begin{equation}
\tilde {\bf D}(\omega)=\epsilon(\omega)\tilde{\bf E}(\omega); \ \tilde {\bf B}(\omega)= \mu(\omega) \tilde{\bf H}(\omega)
\end{equation}
where $\tilde\epsilon( \omega)$ and $\tilde\mu( \omega)$ are the frequency dependent
permittivity and permeability. The boundary conditions at the interface between two materials $1$ and $2$ are given by
\begin{eqnarray}
B_{1n}= B_{2n} && \ D_{1n}=D_{2n} \\
E_{1t} = E_{2t} && \ H_{1t}=H_{2t}
\end{eqnarray}
where $n$ denotes the normal component and $t$ the tangential component to the interface.

Forces can be computed using the vacuum (assuming that the surface where the force is computed is next to the vacuum) Maxwell stress tensor.
\begin{equation}
T_{ij}= \epsilon\left(E_iE_j -{1\over 2}\delta_{ij}E^2\right) + {1\over \mu}\left( B_i B_j -{1\over 2}\delta_{ij}B^2\right)
\end{equation}
Notice that the stress tensor is quadratic in the fields ${\bf E}$ and ${\bf B}$, this means that even if the fields are on average zero, both thermal and quantum fluctuations give rise to forces.

In media Maxwells equations are
\begin{eqnarray}
\nabla \times{\bf E} &=& -{\partial {\bf B}\over \partial t} \\
\nabla \times{\bf H} &=& {\bf J}-{\partial {\bf D}\over \partial t} \\
\nabla \cdot {\bf D} &=& \rho\\
\nabla \cdot {\bf B} &=& 0 
\end{eqnarray}

In a dielectric medium or conductor where there are no applied external fields there is no free charge or current . As such, the average values of ${\bf E}$ and ${\bf B}$ are zero. Rytov's idea was to add a random current to induce both thermal and quantum fluctuations
into the problem. If we assume that the only contribution to the current comes from a fluctuating polarization density ${\bf P}$ we can write
\begin{equation}
{\partial \rho\over \partial t} +\nabla \cdot {\bf J}= 0\implies\nabla\cdot \left[-{\partial {\bf P}\over \partial t} +{\bf J}\right] = 0
\end{equation}
where we have used
\begin{equation}
\rho = -\nabla \cdot {\bf P}
\end{equation}
This means that the current is given by
\begin{equation}
{\bf J}={\partial {\bf P}\over \partial t} 
\end{equation}
or in Fourier space
\begin{equation}
\tilde {\bf J}(\omega)= i\omega \tilde P(\omega)
\end{equation}
Now if we assume that the fluctuations in the polarization density are uncorrelated in space, the fluctuation dissipation theorem tells us that the correlation function of the polarization density in Fourier space is given by
\begin{equation}
< P_\alpha(\omega; \bx)P^{\dagger}_\beta(\omega; \bx')>_{sym}=
{\hbar \epsilon''(\omega)\over 2}\coth\left({\hbar\omega\over 2k_B T}\right)\delta(\omega-\omega')\delta(\bx-\bx')\delta_{\alpha\beta}
\end{equation}

\begin{equation}
\epsilon(\omega) = \epsilon'(\omega)+i\epsilon''(\omega)
\end{equation}
The Lifshitz calculation for slab geometries gives a force per unit area between two slabs of media separated by a distance $L$
\begin{eqnarray}
{F\over A} &=& -{k_BT \over \pi c^3}\sum_{n=0}^\infty \omega_n^3 \int_1^\infty dp p^2 \ 
\left[ 1-{(s_1+p)(s_2+p)\over (s_1-p)(s_2-p)}\exp(-{2p\omega_n L\over c})\right]\nonumber \\
&+&\left[ 1-{(s_1+p\varepsilon_1)(s_2+p\varepsilon_2)\over (s_1-p\varepsilon_1)(s_2-p\varepsilon_2)}\exp(-{2p\omega_n L\over c})\right]
\end{eqnarray}
where $\epsilon = \epsilon_0\varepsilon$, $s_i = \sqrt{\epsilon_i -1 +p^2}$,
$\omega = {2\pi nk_BT \over \hbar}$ are the Mastubara frequencies \cite{matsubara_new_1955} and ${\varepsilon_i
=\varepsilon_i(i\omega_n)}$. Note that the integral over real frequencies has become a sum over discrete Matsubara frequencies, they come from the poles in the hyperbolic cotangent.

One needs to know the dielectric response at imaginary frequency, this is done using the Kramers-Kronig relation
\begin{equation}
\varepsilon(i\omega)= 1 +{2\over \pi}\int_0^\infty d\zeta {\zeta \varepsilon''(\zeta)\over \omega^2 + \zeta^2}
\end{equation}


%%%%%%%%%%%%%%%%%%%%
\subsection{Critical Casimir effect}
%%%%%%%%%%%%%%%%%%%%


In systems having a continuous phase transition the correlation length diverges as the critical point is approached. This means that the correlation length has a size comparable to that of the system size, this  leads to strong finite-size effects in the free energy. Following the arguments of Fisher and de Gennes \cite{gambassi_casimir_2009}, we describe how a version of  the Casimir effect is manifested in  critical systems. 

%%%%%%%%%%%%%%%%%%%%
\subsubsection{Bulk scaling for near critical systems}
%%%%%%%%%%%%%%%%%%%%
The free energy for a system consisting of $N$ spins has a singular part at a critical temperature $T_c$ which can be written as
\begin{equation}
F(t,h) = N f(t,h)
\end{equation}
where $t = (T-T_c)/ T_c$ measures the distance from the critical point and $h$ is the external applied magnetic field. We assume that we are in a system where the only relevant parameters are $T$ and $h$ (equivalently the concentration or chemical potential of a binary mixture), which is true for $d\less 4$ \cite{amit_field_2005}.
Now if we carry out a renormalisation group transformation
blocking spins in blocks of linear size $b$ into new effective spins, the RG transformation
gives
\begin{equation}
N f(t,h) = N'f(t',h')
\end{equation}
Clearly the number of spins in the blocked system is given by $b^d N'=N$ and the RG transformation for $t$ and $h$ are given by $t' = b^{y_1} t$ and $h' = b^{y_2}h$, where $y_1$ and $y_2$ are positive and are the RG exponents for the fields $t$ and $h$ (from which all critical exponents can be deduced). This then means that
\begin{equation}
f(t,h) = {1\over b^d}f(b^{y_1} t, b^{y_2} h) 
\label{rg}
\end{equation}
We begin by working with $t\greater 0$ but the arguments here are trivially generalisable to the case $t\less 0$. In Eq. (\ref{rg}) is we choose $b$ such that $b^{y_1}t=1$, then, at the critical field $h=0$, we find
\begin{equation}
f(t,0) = t^{d\over y_1} f(1,0)
\end{equation}
The singularity in the specific heat is defined via
\begin{equation}
c\sim {\partial^2 \over \partial t^2}f(t,0)
\end{equation}
and so we find
\begin{equation}
c\sim t^{{d\over y_1}-2} \sim t^{-\alpha}
\end{equation}
where $\alpha$ is the exponent associated with the divergence of the specific heat. This means that
\begin{equation}
\alpha = 2 -{d\over y_1}
\end{equation}
The RG transformation for the correlation function has the form
\begin{equation}
C(r,t,h) = \lambda^2(b) C(r/b, b^{y_1}t, b^{y_2} h)
\end{equation}
Clearly length scales transform as $r' = r/b$. Again setting $h=0$ and choosing $b^{y_1}t=1$ gives
\begin{equation}
C(r,t,h) = \lambda^2(t^{-{1\over y_1}}) C(r/t^{-{1\over y_1}}, 1, 0).
\end{equation}
The correlation function, by definition is given by
\begin{equation}
C(r,t) \sim f(r/\xi),
\end{equation}
where $\xi$ is the correlation length. This immediately tells us that $\xi = t^{-{1\over y_1}}$ and from the usual definition 
\begin{equation}
\xi \sim t^{-\nu}
\end{equation}
we have $\nu = 1/y_1$. These two formula for $y_1$ then give the hyper scaling relation
\begin{equation}
\alpha = 2-d\nu.
\end{equation}
The exponents $\alpha$ and $\nu$ are the ones that are important in the critical Casimir effect.

%%%%%%%%%%%%%%%%%%%%
\subsubsection{Finite size scaling}
%%%%%%%%%%%%%%%%%%%%
Consider a system which is finite in one direction with either periodic boundaries or 
two surfaces. While the critical system has $h=0$ there can be local surface fields at each surface $a$ and $b$. This represents a preference of the surfaces for one phase or the other. 
The finite scaling hypothesis for a slab system of large area $A$ but with finite width $L$ can be stated as
\begin{equation}
f(t,h_a,h_b,L^{-1}) = {1\over b^{d}}f(b^{y_1}t, b^{y_a} h_a, b^{y_b} h_b, bL^{-1}) \label{ffs}
\end{equation}
We thus see that the field $L^{-1}$ is a relevant field with RG exponent $y_L=1$. 
The surface fields are not necessarily relevant so we can have $y_a$ and $y_b$ positive or negative. The important point about finite size scaling is that when $L$ is finite the singularity due to the thermodynamic phase transition is smoothed out by the system's finite size (note that we assume that the system has no two-dimensional phase transition in the region we are looking at). First we see that when $L$ is large there should be a bulk contribution to the free energy plus a surface term (so we are considering the limit $L\to\infty$ before $\xi\to\infty$)
\begin{align}
f(t,h_a,h_b,L^{-1}) =& f(t,h_a,h_b,0) + L^{-1}{\partial f(t,h_a,h_b,0)\over \partial x_4} \nn
=& {1\over b^d} f(b^{y_1}t, b^{y_a} h_a, b^{y_b} h_b,0)+{1\over b^{d-1}}L^{-1}{\partial f(b^{y_1}t, b^{y_a} h_a, b^{y_b} h_b, 0)\over \partial x_4}
\end{align}
where we have carried out the Taylor expansion for $L^{-1}$ small using both versions of Eq. (\ref{ffs}) and ${\partial \over \partial x_n}$ indicates the partial derivative with respect to the $n^{th}$ argument. 
The second term gives a total contribution to the singular part of the free energy of the order $AL \times L^{-1}$ and is thus a surface tension $\gamma$ and so we have
\begin{equation}
\gamma= {1\over b^{d-1}}{\partial f(b^{y_1}t, b^{y_a} h_a, b^{y_b} h_b, 0)\over \partial x_4}
\end{equation}


Setting $bt^{y_1}=1$ then gives close to the critical point
\begin{equation}
\gamma \sim t^{d-1\over y_1}{\partial f(1,\lim_{t\to 0} t^{-\frac{y_a}{y_1}}h_a, \lim_{t\to 0} t^{-\frac{y_a}{y_1}}h_b,0)\over \partial x_4} = t^{(d-1)\nu}C' = \xi^{-(d-1)}C'
\end{equation}
The formula relating the surface tension and the correlation length, in the above $C'$ is a constant depending on the universality class. 

Now if we keep $L$ finite and set $b^{y_1}t=1$ in Eq. (\ref{ffs}) we find
\begin{equation}
f(t,h_a,h_b,L^{-1}) = t^{d\over y_1}f(1, t^{-{y_a\over y_1}} h_a, t^{-{y_b\over y_1}} h_b, t^{-{1\over y_1}}L^{-1})
\label{ffs}
\end{equation}
which can be written as
\begin{equation}
f(t,h_a,h_b,L^{-1}) = {1\over \xi^d}f(1,\xi^{y_a} h_a, \xi^{y_b} h_b, \xi/L)
\end{equation}
This can then be written as
\begin{equation}
f(t,h_a,h_b,L^{-1}) = {1\over L^d}\theta({L\over \xi}, \xi^{y_a} h_a, \xi^{y_b} h_b)
\end{equation}
Now crucially as $\xi \to \infty$ the function $\theta$ is analytic so we can take the limit $\xi\to\infty$ without any problems to find
\begin{equation}
f(0,h_a,h_b,L^{-1}) = {1\over L^d}\theta(0, \lim_{\xi\to \infty} \xi^{y_a} h_a,\lim_{\xi\to \infty} \xi^{y_b} h_b)
\end{equation}
Clearly for each surface we have 3 possibilities: $\lim_{\xi\to \infty} \xi^{y_a} h_a = \pm \infty$, if the surface fields are relevant, as well as $\lim_{\xi\to \infty} \xi^{y_a} h_a = 0$ if the surface fields are irrelevant. There is clearly also a similar argument when the system has periodic boundary conditions and there are no surface fields. Near the critical point depending on the boundary conditions there should be scaling functions when the surface fields attract the same phase $\theta_{++}(x)$, where they attract different phases and
$\theta_{+-}(x)$, and $\theta_{pbc}(x)$ when the boundary conditions are periodic. There should also be a zero surface field case $\theta_{00}$ when the surfaces fields are irrelevant or zero (this is however unlikely). Fisher and de Gennes argued, without proof, that the force for $(++)$ boundary conditions should be attractive where as the $(+-$) case should produce repulsive forces \cite{gambassi_casimir_2009,gambassi_critical_2009} . 

The total singular part of the free energy is thus given by
\begin{equation}
F = ALf(t,h_a,h_b,L^{-1}) = {A\over L^{d-1}}\theta({L\over \xi}, \xi^{y_a} h_a, \xi^{y_b} h_b).
\end{equation}

The scale of the energy is set by the energy of thermal fluctuations $k_BT$ we thus find
that 
\begin{equation}
F(t=0) = {k_BT A C\over L^{d-1}}
\label{casfreeenergy}
\end{equation}
where $C$ is a constant depending on the surface universality class.


%%%%%%%%%%%%%%%%%%%%
\section{Finite size scaling in one dimensional interface models}
\label{sec-continuous-interface}
%%%%%%%%%%%%%%%%%%%%

We have seen that confinement generated Casimir forces appear when the correlation length of the fluctuations in a system becomes of the order of the minimum size of the system,  be it for perfect conductor plates in vacuum or in confined critical systems. We have also seen that interfaces between two phases can described by surface models. If we consider just a surface model, it is physically obvious that finite size effects will also arise in these systems. In what follows we will study the size dependence of a number of continuous and discrete interface models (corresponding to the interfaces of two dimensional systems). While long range forces are generated by confinement we find that these models have quantitatively different behaviours to the critical Casimir effect. However if one assumes a phenomenological proposal by Privman \cite{privman_finite-size_1988-1} to introduce a finite size correction to the surface tension, the critical Casimir effect can be quantitatively recovered.

%%%%%%%%%%%%%%%%%%%%
\subsection{Continuous models in one dimension}
%%%%%%%%%%%%%%%%%%%%

In one dimension the partition function for a surface model of the type discussed in Sec \ref{sec-continuous} can be written as a path integral 
\begin{equation}
Z(t)  = \int d[h]\exp\left(-\frac{\beta\sigma}{2}\int_0^t h'^2(x) dx -\beta\int_0^t  V(h(x)) dx\right)
\end{equation}
where $t$ is the length of the system and the notation is so chosen as the variable $x$ con be thought of as a time in path integral language.
It is convenient to fix both the starting point $h(0)=x$ and the end point $h(t)=x$ and define what is known as the propagator \cite{matsubara_new_1955}
\begin{equation}
K(h,h',t)=\int_{h(0)=h} d[h] \delta(h' -h(t)) \exp\left(-\frac{\beta}{2}\int_0^t \sigma h'^2(x) dx -\beta \int_0^t  V(h(x)) dx\right)\label{prog}
\end{equation}
The propagator is an example of a path integral and is the sum over all paths going between 
$h$ and $h'$ in what can be taken to be the time $t$.  It can be shown \cite{kleinert_path_2009} that the path integral obeys an imaginary time Schr\"odinger equation
\begin{equation}
\frac{\partial  K(h,h',t)}{\partial t} = -\hat H K(h,h',t)
\end{equation}
where $\hat H$ is the Hamiltonian operator
\begin{equation}
\hat H = -\frac{1}{2\sigma\beta}\frac{\partial^2 }{\partial h^2} + \beta V(h)
\end{equation}
and, with a suitable normalisation, the initial condition
\begin{equation}
K(h,h',t)=\delta(h-h')
\end{equation}
If the Hamiltonian operator $\hat H$ has eigenfunctions $\psi_n$, normalised so that
\begin{equation}
\int dh \ \psi^2_n(h) = 1
\end{equation}
 and with eigenvalues $\epsilon_n$, it is easy to see that the propagator can be written as
\begin{equation}
K(h,h',t)= \sum_n \exp(-t\epsilon_n)\psi_n(h)\psi_n(h')
\end{equation}
If we take a system with periodic boundary conditions but otherwise leave the initial value $h(0)$ of the height to be free, then using the normalisation of the eigenfunctions, we find
\begin{equation}
Z(t) = \int dh K(h,h,t) = \sum_n \exp(-t\epsilon_n)
\end{equation}
Now in the thermodynamic limit $t\to\infty$ if there is a gap between the ground state energy
$\epsilon_0$ and the first excited state, $g=\epsilon_1-\epsilon_0$ which is non-zero, we can apply ground state dominance 
\begin{equation}
Z(t) =\exp(-t\epsilon_0)
\end{equation}
which gives the free energy per unit length as
\begin{equation}
f=\frac{1}{\beta}\epsilon_0
\end{equation}
As well as the free energy we are interested in the probability distribution of the height at a single point (which is independent of the point chooses due to invariance by translation of the system). For instance the probability distribution of $h(0)$ is given by
\begin{align}
p_1(h)=& \frac{\int d[h]\delta(h(0)-h)\exp\left(-\frac{\beta}{2}\int_0^t h'^2(x) dx -\beta\int_0^t  V(h(x)) dx\right)}{Z(t)}\nn
=& \frac{K(h,h,t)}{Z(t)} \nn
=& \frac{\sum_n \exp(-t\epsilon_n)\psi_n^2(h)}{\sum_n \exp(-t\epsilon_n)}
\end{align}
and so as $t\to\infty$, ground state dominance gives
\begin{equation}
p_1(h)= \psi_0^2(h)
\end{equation}
We see that the normalisation of the probability density function for $h$ follows from the 
normalisation of the wave functions.

The joint probability density function for two heights separated by a time or distance $x$ is given by
\begin{align}
p_2(h,h',x)=& \frac{\int d[h]\delta(h(0)-h)\delta(h(x)-h')\exp\left(-\frac{\beta}{2}\int_0^t h'^2(x) dx -\beta\int_0^t  V(h(x)) dx\right)}{Z(t)}\nn
=& \frac{K(h,h',x)K(h',h,t-x)}{Z(t)}  \nn
=& \frac{\sum_{nm} \exp(-x\epsilon_n)\psi_n(h)\psi_n(h')\exp(-[L-x]\epsilon_m)\psi_m(h')\psi_m(h)}{\sum_n \exp(-t\epsilon_n)}
\end{align}
Due to ground state dominance only the term with $m=0$ survives in the sum above (as
as $x$ is taken such that $x\ll t$) as we find
\begin{eqnarray}
p_2(h,h',x) &=& \sum_{n} \psi_0(h')\psi_0(h)\psi_n(h')\psi_n(h)\exp(-x[\epsilon_n-\epsilon_0]))\\
&=& p_1(h)p_1(h') + \sum_{n>0} \psi_0(h')\psi_0(h)\psi_n(h')\psi_n(h)\exp(-x[\epsilon_n-\epsilon_0]))
\label{eqp2}
\end{eqnarray}
From this we see that when $x[\epsilon_n-\epsilon_0] \gg1 $ for all $n$ and so in particular $x[\epsilon_1-\epsilon_0] \gg1$ we have 
\begin{equation}
p_2(h,h',x) \sim p_1(h)p_2(h')
\end{equation}
so that the height at large distances are uncorrelated or equivalently are independent random variables. This gives a correlation length
\begin{equation}
\xi = \frac{1}{\epsilon_1-\epsilon_0}.\label{clq}
\end{equation}

%%%%%%%%%%%%%%%%
    \subsection{The confined elastic line}
%%%%%%%%%%%%%%%%
Here we consider the case where $V(h)= 0$ for $0 \less h \less L$ and $V(h)=0$ otherwise. This corresponds to a one dimensional elastic line confined between two impenetrable walls separated by a distance $L$. The Hamiltonian $\hat H$ is that for a quantum well of width $L$ and with eigenfunctions \cite{cohen-tannoudji_mecanique_2018} 
\begin{equation}
\psi(h) = \sqrt{\frac{2}{L}}\sin(\frac{\pi(n+1)h}{L})
\end{equation}
where $n\geq 0$ are integers. From this we see that the ground state energy is 
\begin{equation}
\epsilon_0 = \frac{1}{2\sigma\beta}\frac{\pi^2}{L^2}
\end{equation}
and so, in the thermodynamic limit, the free energy per unit length is
\begin{equation}
f = \frac{1}{2\sigma\beta^2}\frac{\pi^2}{L^2}= \frac{T^2\pi^2}{2\sigma L^2}
\end{equation}
Here the pressure (in this case pressure in a force per unit length) is given by
\begin{equation}
P = -\frac{\partial f}{\partial L} = \frac{\pi^2 T^2}{\sigma L^3},\label{pfree}
\end{equation}
and we see that it is repulsive. Physically, the fluctuations of the surface repel the walls. 
The pressure has the Casimir like characteristic that it behaves as a long range power law type interaction, however a two dimensional critical Casimir system (see \eqref{casfreeenergy}) would have a free energy per unit length $f=CT/L$. We also see that the free energy scales  a $T^2$ rather than $T$ (as is the case for the Casimir interaction).

Having said this, a critical system has zero surface tension and so using a model with a finite surface tension for the critical interface is clearly not appropriate. However it has been conjectured by Privman \cite{privman_finite-size_1988-1}  that the term $\sigma$, which Privman refers to the stiffness, should be modified by finite size effects (although he considers a case where the height $L$ of the system and the length $t$ are of the same order).
 The simplest conjecture proposed by Privman\cite{privman_finite-size_1988-1} is that
\begin{equation}
\sigma(L)\approx \sigma_b + \frac{T a}{L}
\end{equation}
so that $a$ has no dimensions.
This amounts to  assuming that the corrections are analytic in the variable $1/L$. It seems difficult to justify this from a more microscopic view, however if we use this we find that
\begin{equation}
f = \frac{1}{2\sigma(L)\beta^2}\frac{\pi^2}{L^2}= \frac{T^2\pi^2}{2[ \sigma_b + \frac{T a}{L}]L^2}
\end{equation}
Now if we associate the critical point where $\sigma_b=0$ we find
\begin{equation}
f_c= \frac{T\pi^2}{2 a L}
\end{equation}
and this does have exactly the form predicted for a critical system in Eq. \eqref{casfreeenergy}.

Using Eq. (\ref{clq}) we find that  the correlation length is given by
\begin{equation}
\xi = \frac{2}{3}\frac{\sigma L^2}{T\pi^2}\label{corel}
\end{equation}
thus it increases as the surface tension is increased or the temperature is lower. This makes physical sense as the surface should become {\em flatter} under these conditions. Also, as the system becomes more confined, the correlation length increases, again as  confinement  
kills fluctuations. The correlation length tells us that if we wanted to simulate this system then we need to take
\begin{equation}
t\gg \xi 
\end{equation}
in order to be in the thermodynamics limit and so $t \gg \frac{\sigma L^2}{T\pi^2}$, thus for 
$L$ large, in general we would need to simulate rather large systems.
The probability distribution function of the height at a single point is given by
\begin{equation}
p_1(h) =\frac{2}{L}\sin^2(\frac{\pi h}{L})
\end{equation}
and from this we find 
\begin{equation}
\langle h\rangle = \frac{L}{2}  
\end{equation}
which is rather obvious. The width of the interface is given by 
\begin{equation}
w=\sqrt{\langle h^2\rangle - \langle h\rangle^2}
\end{equation}
and here it is given by
\begin{equation}
w= L\sqrt{\frac{1}{12}-\frac{1}{2\pi^2}}= 0.180756\  L
\end{equation}

During this thesis we have considered models of surfaces where the overall surface integral is fixed. In magnetic systems this corresponds to systems with conserved magnetisation. It is surprisingly difficult to deal with this constraint in a hard way both for continuous surfaces, treated via the Schr\"odinger equation, and for discrete systems with the transfer matrix. In principle one can always introduce a magnetic field to fix the average total magnetisation to zero. Howeve,r there will always be fluctuations around the average value. Normally in thermodynamics one can fix the magnetisation per site to zero by applying an external magnetic field (which is zero in systems having an up down symmetry). If there are $N$ sites, the fluctuations of the magnetisation by site scale as $1/\sqrt{N}$.  However, the total magnetisation has fluctuations of the order of $\sqrt{N}$ and so the condition of fixed  total magnetisation is only imposed approximately by an applied magnetic field.

In the confined Edwards Wilkinson model we consider the magnetisation $M$ defined
by 
\begin{equation}
M = \int_0^t dx\  h(x)
\end{equation}
We know, from symmetry arguments, that without an external applied field we have
\begin{equation}
\langle M\rangle = \frac{tL}{2}
\end{equation}
which  can be shown explicitly from the formula
\begin{equation}
\langle M\rangle = t\int_0^L dh  h \psi_0^2(h)
\end{equation}
Interestingly, if we write things in terms of the traditional bra and ket notation of quantum mechanics, we see that
\begin{equation}
\langle M\rangle = t \langle 0| h|0\rangle
\end{equation}
and we note that 
\begin{equation}
\langle 0| h|0\rangle = \Delta \epsilon_{0,1}(h)
\end{equation}
where $\Delta \epsilon_{0,1}(h)$ is the shift in the ground state energy to first order in perturbation theory induced by a perturbation of the potential $\Delta V(h) = h$. However this is
just the obvious thermodynamic expression
\begin{equation}
\langle M\rangle = -t\frac{\partial}{\partial \lambda}f (\lambda)|_{\lambda=0},
\end{equation}
for a potential $U(h) = V(h) + \lambda  h$.
The variance of $M$ can be computed by using Eq. \eqref{eqp2}, which gives
\begin{align}
\langle M^2\rangle_c =& \int_0^t dx dx'[ h(x)h(x')-\langle h^2\rangle] \nn
=& \sum_{n>0}^\infty
\left(\int_0^L dh\ h \psi_0(h)\psi_n(h)\right)^2\int _0^t dxdx' \exp(-[\epsilon_n-\epsilon_0]|x-x'|)
\end{align}
and for large $t$ carrying out the integration over $x$ and $x'$ gives
\begin{equation}
\langle M^2\rangle_c = 2t \sum_{n \greater 0} \frac{1}{\epsilon_n-\epsilon_0}\left(\int_0^L dh\ h \psi_0(h)\psi_n(h)\right)^2
\end{equation}
From second order perturbation theory we see that this is just equivalent to the thermodynamic
identity
\begin{equation}
\langle M^2\rangle_c = -Tt\frac{\partial^2}{\partial \lambda^2}f (\lambda)|_{\lambda=0}
\end{equation}
Using the explicit form of the eigenfunctions we find
\begin{equation}
\langle M^2\rangle_c=\frac{16L^4 t\sigma\beta }{\pi^2}\sum_{n=1}^\infty \frac{1}{(n+1)^2-1}\left[ \int_0^1 du \ u 
\sin(\pi u) \sin(\pi (n+1)u)\right]^2
\end{equation}
It can then be shown that
\begin{equation}
\int_0^1 du \ u 
\sin(\pi u) \sin(\pi (n+1)u)= -\frac{2(n+1) (1+ \cos((n+1)\pi)}{\pi^2[(n+1)^2-1]^2}
\end{equation}
From this we see that only the modes where $n$ is odd contribute to the fluctuations of the magnetisation. Consequently we find
\begin{eqnarray}
\langle M^2\rangle_c&=&\frac{256 L^4 t\sigma\beta }{\pi^6}\sum_{n,\ {\rm odd}\ ,=1}^\infty \frac{(n+1)^2}{[(n+1)^2-1]^5}\\
&=& \frac{1024L^4 t\sigma\beta }{\pi^6}\sum_{k=0}^\infty \frac{(k+1)^2}{[(2k+2)^2-1]^5}\\
&=& \frac{L^4 t\sigma\beta(15-\pi^2) }{\pi^4}.
\end{eqnarray}
This formula deserves some comment. A first trivial comment is that that $\langle M^2\rangle_c\sim t$ in accordance with the thermodynamic arguments given above. Secondly the variance diverges as $\sigma\beta\to\infty$, which is normal as the low energy configuration zero mode  - a straight line - is unaffected by the confining walls and so in principal this line can lie anywhere on $[0,L ]$, explaining the scaling with $L^4$.


\subsection{The Airy line}
In the  above well known example we confine the surface and then compute the pressure. This is an example of the constant volume ensemble. Physically we could also consider the case of a system which is confined softly by an externally imposed pressure $P_0$ (which can also be treated as a chemical potential depending on the context) in the constant pressure ensemble. In this case the potential is given by
\begin{align}
V(h) = \begin{cases} P_0 h  &\ { \rm for }\ h \greater 0 \nn \infty &\ {\rm for}\ h\leq 0 \end{cases}
\end{align}
The time independent Schr\"odinger equation for the eigenfunctions here is
\begin{equation}
-\frac{1}{2\sigma\beta}\frac{d^2 \psi_n(h)}{dh^2} + P_0\beta h \psi_n(h) = \epsilon_n\psi_n(h)
\end{equation}
The corresponding eigenfunctions have boundary conditions $\psi_n(0)=0$ due to the hard wall potential at $h=0$ and they must also decay to zero as $h\to \infty$ so as to be normalisable.

The key to finding the eigenfunctions is to transform  the Schr\"odinger into the Airy equation which is
\begin{equation}
\frac{d^2y(x)}{dx^2}- x y(x)=0.
\end{equation}
This equation has solutions ${\rm Ai}(x)$ which decay as 
\begin{equation}
{\rm Ai}(x) \sim \frac{\exp(-\frac{2}{3} x^{\frac{3}{2}}) \Gamma(\frac{5}{6})\Gamma(\frac{1}{6})}{4\pi^{\frac{3}{2}} x^{\frac{1}{4}}},
\end{equation}
as $x\to\infty$ and so are normalizable as eigenfunctions. For $x<0$ the Airy function  oscillates and has an 
infinite number of negative zeros $-\alpha_n$ such that ${\rm Ai}(-\alpha_n)=0$. 

We make the change of variable $h=\ell z'$ to find
\begin{equation}
\frac{1}{2\sigma\beta\ell^2}\frac{d^2 \psi_n(z')}{dz'^2}- P_0\beta \ell(z'-\varepsilon_n)\psi_n(z')=0,
\end{equation}
where $\varepsilon_n= \epsilon_n/(P_0\beta\ell)$. Now  we chose $\ell$ so that
\begin{equation}
2\sigma\beta^2P_0 \ell^3=1,
\end{equation}
and we see that  
\begin{equation}
\ell = \left(\frac{1}{2\beta^2\sigma P_0}\right)^{\frac{1}{3}},
\end{equation}
is an intrinsic length scale.
\begin{equation}
\frac{d^2 \psi_n(z')}{dz'^2}- (z'-\varepsilon_n)\psi_n(z')=0.
\end{equation}
Finally if we use $z=z'-\varepsilon_n$ we obtain Airy's equation \cite{albright_integrals_1977}
\begin{equation}
\frac{d^2 \psi_n(z)}{dz^2}- z\psi_n(z)=0
\end{equation}
and so
\begin{equation}
\psi_n(z)= c_n {\rm Ai}(z),
\end{equation}
where $c_n$ is a normalisation constant. This means that in terms of the original height variable $h$, 
\begin{equation}
\psi_n(h)= c_n {\rm Ai}(\frac{h}{\ell}-\varepsilon_n).
\end{equation}
The boundary condition $\psi_n(h)$ then show that we must choose $\varepsilon_n=\alpha_{n+1}$. This means that the ground state energy is
\begin{equation}
\epsilon_0 = \alpha_1P_0\beta\ell= \frac{\alpha_1P_0\beta}{(2\sigma\beta^2P_0)^\frac{1}{3}}= \frac{\alpha_1 P_0^\frac{2}{3}\beta^\frac{1}{3}}{2^\frac{1}{3} \sigma^\frac{1}{3}},
\end{equation}
and where we note that $\alpha_1 = 2.33811$.
\begin{equation}
f= \frac{\alpha_1 P_0^\frac{2}{3}}{2^\frac{1}{3} \sigma^\frac{1}{3}\beta^\frac{2}{3}}.
\end{equation}
From the original partition function we see that $h$ is conjugate to $P_0$ and so we find the average height is given by
\begin{equation}
\bar h= \langle h\rangle = \frac{\partial f}{\partial P_0} = \frac{2}{3}\frac{\alpha_1 }{2^\frac{1}{3} \sigma^\frac{1}{3}\beta^\frac{2}{3}P_0^\frac{1}{3}} = \frac{2}{3}\alpha_1\ell,\label{h1}
\end{equation}
and solving for $P_0$ in terms of $\overline h$ gives
\begin{equation}
P_0 = \frac{4}{27}\frac{\alpha_1^3 T^2}{\sigma \overline h^3},
\end{equation}
we see that $P_0$ behaves exactly in the same way as the pressure of a confined elastic line
in term of the temperature and surface tension. Only the overall numerical prefactor is different.

The correlation length is given by
\begin{equation}
\xi = \frac{2^\frac{1}{3}(\sigma T)^\frac{1}{3}}{(\alpha_2-\alpha_1)P_0^\frac{2}{3}}.
\end{equation}
When written in terms of $\overline h$ the above correlation length behaves in the same way
as for the free elastic line, however when $P_0$ is fixed we see that the behavior as a function 
of $T$ and $\sigma$ is quite different. The correlation length still increases with $\sigma$ but now decreases as the temperature is decreases.

The probability density function for the height of the interface at a single point is given
by
\begin{equation}
p_1(h) = \frac{{\rm Ai}^2(\frac{h}{\ell}-\alpha_1)}{\int_0^\infty dh' {\rm Ai}^2(\frac{h'}{\ell}-\alpha_1)}
\end{equation}
However if we write the height variable in terms of the length scale $\ell$, $h(x)= \ell z(x)$ we find that $z$ has the single point probability density function
\begin{equation}
p(z)= \frac{{\rm Ai}^2(z-\alpha_1)}{\int_0^\infty dz' {\rm Ai}^2(z'-\alpha_1)}
=\frac{{\rm Ai}^2(z-\alpha_1)}{{\rm Ai}{^{'2}}(-\alpha_1)}
\label{airyprob}.
\end{equation}


 \begin{figure}[t]
\begin{center}
\includegraphics[angle=0,width=15cm ]{finite-size/etats-laser.pdf}
\caption{The scaled probability density function $p(z)$ for the distribution of the height at
a single point for the Airy line given in Eq. (\ref{airyprob})}
\label{1dpot}
\end{center}
\end{figure}
Using this we find the average height is given by
\begin{equation}
\langle h\rangle = \overline h= \ell z_0,
\end{equation}
where 
\begin{equation}
z_0 = \frac{\int_0^\infty dz z {\rm Ai}^2(z-\alpha_1)}{\int_0^\infty dz {\rm Ai}^2(z-\alpha_1)}.
\end{equation}
Interestingly comparison with the thermodynamic calculation giving Eq. \eqref{h1} shows that the identity
\begin{equation}
z_0 = \frac{2}{3}\alpha_1,
\end{equation}
must hold- this surprising identity can be verified numerically. Here we find that the average height given by
\begin{equation}
\langle h\rangle= 0.697089 \ \ell.
\end{equation}
The variance of the magnetisation is then given by
\begin{eqnarray}
\langle M^2\rangle_c &=& -TL\frac{\partial^2}{\partial \lambda^2}f (\lambda)|_{\lambda=0}=-TL\frac{\partial^2}{\partial P^2}f (P)|_{P=P_0}\\
&=& L\frac{\alpha_1 }{2^\frac{1}{3} \sigma^\frac{1}{3}\beta^\frac{5}{3}}\frac{2}{9}P_0^{-\frac{4}{3}}.
\end{eqnarray}
In terms of the average height this then gives
\begin{equation}
\langle M^2\rangle_c = \frac{9}{4\alpha_1^3}L\sigma\beta \overline h^4
\end{equation}


%%%%%%%%%%%%%%%%%%%%
\section{The generalized Lopes-Jacquin-Holdsworth Method}
\label{gen-lopes}
%%%%%%%%%%%%%%%%%%%%

In Sec \ref{sec-lopes}, we have shown a way to numerically compute the free energy of a system at a chemical potential $\mu$ in absence of another potential. Here we generalise the method for any kind of external potential. We will explain the method for the Ising model, but the derivation for the SOS model is straightforward.

For any external field which can be written as $B V(\sigma)$, where $V(\sigma)$ is a function of the internal microscopic variables $\sigma_i$,  the Hamiltonian of the Ising model is
\begin{align}
H = - J \sum_{} \sigma_i \sigma_j - B \sum_i V(\sigma_i)
\end{align}
The mean value of the external potential is
\begin{align}
    <  \sum_i V(\sigma_i) > =&  \sum_{\bf h} \sum_i V(\sigma_i) \exp(-\beta H) \nn
    =&  - \frac{\partial F(\mu)}{\partial B}
\end{align}
where $F$ is the free energy of the system. We see that for any potential of the form \eqref{lopes-gen}, we can integrate the previous equation to find
\begin{align}
   F(B_1) - F(B_2) = - \int_{B_1}^{B_2} d B'  < \sum_i V(\sigma_i) >_{B'} 
   \label{lopes-gen}
\end{align}

In the case where we know the analytical form of the free energy in the limits $B_2 \to \infty$ or $B_1 \to 0$, this method provides a way to directly measure it for any temperature or size by integrating over the chemical potential.
From the total free energy, we recover the Casimir form through Eq \eqref{cas-lopes}.
The limit $B_1 \to 0$ is the free system limit, and the free energy can not be computed analytically. However, when $B_2 \to \infty$, for  the majority of external fields $B V(\sigma)$ in which we are interested, there is often a configuration limit whose free energy can be computed analytically. For example, if $V(\sigma)=\sigma$, the configuration limit is the one where all spins point towards the same direction, leading to a free energy of $0$. Thus, we have
\begin{align}
   F(B_1) - F_{analytic}(\infty) = - \int_{B_1}^{\infty} d B'  < \sum_i V(\sigma_i) >_{B'} 
   \label{lopes-gen}
\end{align}
In numerical simulations, it is not possible to range over infinity, and a criterion has to be defined to know the error made between the analytic case $\mu_2 = \infty$ and the maximal $\mu_2$ achieved in simulations. As in Eq \eqref{function-d}, we define the function
\begin{align}
    D(B,L_1,L_2) =  < M^\ast(L_1)-M^\ast(L_1-1) - (M^\ast(L_2)-M^\ast(L_2-1) >
\end{align}
with the generalized magnetization $M^\ast = \sum_i V(\sigma_i)$. A suitable upper limit of integration if we want to get the Casimir force is when the function $D$ reaches $0$ within the precision of the simulation.


For the SOS Hamiltonian
\begin{align}
    H =  J \sum_i |h_i -h_{i+1}|  + B \sum_i V(h_i)
\end{align}
we define the generalised mean height as
\begin{align}
    h^\ast = < \sum_i V(h_i) >
\end{align}
Eqquation \eqref{lopes-gen} writes as
\begin{align}
   F(B_1) - F(B_2) = -  \int_{B_1}^{B_2} d\mu' h^\ast(B')
   \label{diff-gene}
\end{align}
which can be directly be verified with the transfer matrix. 
In the limit $B \to \infty$, the generalised height is zero, while the free energy $F(\infty)$ can often be computed analytically. 
To minimize the error between the analytical limit and the numerical simulations, a suitable choice of the upper integration's limit $B_2$ is given by
\begin{align}
    \int_{B_2}^\infty  d B' h^\ast(B')) \ll \int_{B_1}^{B_2}  dB' h^\ast(B')
\end{align}
An heuristic argument to find a suitable upper limit for integration is when $h^\ast(B_2) \ll h^\ast(B_1)$.
In Fig \ref{integration-free-ene}, we see the free energy computed from the matrix transfer, compared to the integration procedure \eqref{lopes-gen} for the SOS model for the chemical potential $V(h_i)=h_i$ in Monte Carlo simulations, where we see the agreement for $B_2$ large enough.

\begin{figure}
    \centering
    \includegraphics[width=0.7\linewidth]{finite-size/integration-free-ene.pdf}
    \caption{Difference in free energy directly computed from transfer matrix, compared to numerical integration over the generalized height, for different upper limit $B_2$. The parameters are $L' = 256$, $L=200$ and $\beta=1$ for $5e7$ Monte Carlo steps.}
    \label{integration-free-ene}
\end{figure}

Since the order parameter is conserved in model B, the generalized Lopes-Jacquin-Holdswroth method can be used to compute the free energy for Kawasaki dynamics for potentials different from the chemical potential. 
As a proof of concept, we take a potential of the form
\begin{align}
    V(h_i) = - |h_i-\frac{L}{2}|
    \label{neggstaged}
\end{align}
Such potential will press the interface along $h=0$ and $h=L$ compared to the classical chemical potential which presses the interface at $h=0$, as seen in Fig \ref{fig-negstagged}. Far away from $\frac{L}{2}$, both potentials are equivalent in symmetric fashion, and shall behave similarly for large $B$, because the free energy only depends on the interface fluctuations and not the mean height. 


\begin{figure}
    \centering
	\includegraphics[width=0.7\linewidth]{int-dyn/comp-potentiels-chimiques.pdf}
	\caption{Snapshots of systems for the potential \eqref{neggstaged} and the chemical potential for $\beta=1$ and $B=2$ with $L=40$ and $L'=256$}
    \label{fig-negstagged}	
    \centering
   	\includegraphics[width=0.7\linewidth]{finite-size/integration-free-ene-neg.pdf}
    \caption{Difference in free energy directly computed from transfer matrix with the potential \eqref{neggstaged}, compared to numerical integration over the generalized height. The parameters are $L' = 256$, $L=20$ and $\beta=1$, $B_2 = 1$for $10^7$ Monte Carlo steps. } 
   	\label{fig-int-negstagged}    
\end{figure}  

In the $B \rightarrow \infty$ limit, the system has two equilibrium positions $h=0$ et $h=L$, which gives the transfer matrix
\begin{align}
T= e^{\beta B \frac{L}{2}}
  \begin{pmatrix}
    1 & e^{-\beta  J L} \\
    e^{-\beta  J L} & 1
  \end{pmatrix}
\end{align}
The eigenvalues are $\lambda_\pm = e^{ \beta B \frac{L}{2}}( 1 \pm e^{-\beta J L})$, which gives us the free energy 
\begin{align}
  F(B \rightarrow \infty) = - B \frac{L}{2}
\end{align}
We plotted in Fig \ref{fig-int-negstagged} the difference of free energy computed from the transfer matrix between $B_1$ finite and $B_2=1$, and the integration procedure \eqref{diff-gene} with the generalized height with the matrix transfer and Monte Carlo simulations, both for Glauber and Kawasaki dynamics. The disagreement between the expected value and the simulation results are from a $B_2$ too small, which we also see in the integration of the generalized height from the transfer matrix. We can convince ourselves by doing the integration from the transfer matrix for a larger $B_2$. Also, it is worth noting that for this system, there is no significant difference in results for the two different dynamics.

This method opens a new way to compute the free energy for any kind of external potential of the form $B V(h)$, or $B V(\sigma)$ in the case of the Ising or SOS models for conserved and non-conserved dynamics, such as non-uniform external fields \cite{bissacot_phase_2010}. 



%%%%%%%%%%%%%%%%%%%%
\section{The confined solid on solid model}
%%%%%%%%%%%%%%%%%%%%


From exact diagonalization of the SOS transfer matrix in the infinite case \cite{guyer_sine-gordon_1979}, finite-size effects were studied both for the SOS and RSOS model \cite{svrakic_finite-size_1988,privman_finite-size_1988}. Nevertheless the derivation of eigenvectors and eigenvalues were not explicit in the latter case. Those eigenvalues are a multiple of an integer, and the study of the eigenvalues issued from an odd integer where also not discussed. We also add an analysis to the correlation length and the limits of high and low temperatures for the free energy. 

We consider here the free interface confined between $0$ and $L$, with no external field. The SOS transfer matrix is thus given by 
\begin{align}
T(h_i,h_j) = \exp(-\beta J |h_i-h_j|)
\end{align}
Since positions are comprised from $0$ to $L$, we can write the transfer matrix as
\begin{equation}
T_{ij} = \exp(-\beta J|i-j|).
\end{equation}
We introduce 
\begin{equation}
r=\exp(-\beta\ J)
\end{equation}
To find the eigenvectors of $T$, we consider the vector denoted by $[a]$ which has components
\begin{equation}
[a]_i = a^i,
\end{equation}
where $i$ is an index ranging from $0$ to $L$. 
The action of the SOS transfer matrix on this vector is given by
\begin{equation}
\left[T\ [a]\right]_i = \sum_{j=0}^L r^{ |i-j|} a^j
\end{equation}
and we find
\begin{align}
[T\ [a]]_i
=& r^i \sum_{j=0}^i r^{-j} a^j + r^{-i} \sum_{j= i+1}^L r^j a^j \nn
=& r^i \sum_{j=0}^i r^{-j} a^j + r^{-i} \sum_{k= 0}^{L-i-1} r^{i+1+k} a^{i+1+k} \nn
=& r^i \frac{1- r^{-(i+1)} a^{i+1}}{1- r^{-1} )a} + r a^{i+1}\frac{1- r^{L-i} a^{L-i}}{1- r a} \nn
=&\left[\frac{ ra}{1-ra}- \frac{ \frac{a}{r}}{1-\frac{a}{r}}\right]a^i +\frac{r^i}{1-\frac{a}{r}}-\frac{r^{L+1-i}a^{L+1}}{1-ra}
\end{align}
We now define
\begin{equation}
\lambda(a)= \frac{ ra}{1-ra}- \frac{ \frac{a}{r}}{1-\frac{a}{r}} = \frac{\frac{1}{r}-r}{\frac{1}{r}+r
- a-\frac{1}{a}}
\label{elam}
\end{equation}
and notice that
\begin{equation}
\lambda(a) = \lambda(a^{-1})
\end{equation}
We can thus write
\begin{equation}
[T\ [a]]_i= \lambda(a)a^i +\frac{r^i}{1-\frac{a}{r}}-\frac{r^{L+1-i}a^{L+1}}{1-ra}
\end{equation}
Now, considering the action of the transfer matrix on the vector $[a^{-1}]$, we find
\begin{equation}
\left[T\ [a^{-1}]\right]_i = \lambda(a) a^{-i} + \frac{r^i}{1-\frac{1}{ra}}-\frac{r^{L+1-i}a^{-(L+1)}}{1-\frac{r}{a}}
\end{equation}
We now look for an eigenvector of the form
\begin{equation}
{\bf v} = [a]+ c[a^{-1}]
\end{equation}
The action of $T$ on ${\bf v}$ is 
\begin{equation}
\left[T\ ([a]+c [a^{-1}]\right]_i = \lambda(a)[a^i + c a^{-i}]
+ r^i\left(\frac{1}{1-\frac{a}{r}}+ \frac{c}{1-\frac{1}{ra}}\right)
- r^{L+1-i} \left(\frac{a^{L+1}}{1-ra} + c\frac{a^{-(L+1)}}{1-\frac{r}{a}}\right)
\end{equation}
and we see that ${\bf v}$ is an eigenvector, with eigenvalue $\lambda(a)$, if
\begin{eqnarray}
\frac{1}{1-\frac{a}{r}}+ \frac{c}{1-\frac{1}{ra}}&=&0 \\
\frac{a^{L+1}}{1-ra} + c\frac{a^{-(L+1)}}{1-\frac{r}{a}} &=& 0
\end{eqnarray}
The above equations imply that 
\begin{equation}
c= -\frac{ra-1}{a(r-a)}
\end{equation}
and
\begin{equation}
c^2 = a^{2L}
\end{equation}
Therefore we find
\begin{equation}
v_i = a^i \pm a^{L-i}
\end{equation}

\subsubsection*{Ground state eigenvector}
We expect the ground state eigenvector (corresponding to the largest eigenvalue) to be symmetric with respect to the middle of the system and so 
\begin{equation}
v_i= v_{L-i}
\end{equation}
which implied that we should have $c= a^L$. 

This then gives the equation determining the values of a for the largest eigenvalue, and in general for the eigenvalues which are symmetric ( $c=1$),
\begin{equation}
a^{L+1} = \frac{1-ra}{r-a}.
\label{aeq}
\end{equation}

As a check on the above derivation we can consider the case $L=1$ so we have two sites. Here we see that the transfer matrix is given explicitly by
\begin{equation}
T =\begin{pmatrix} & 1 & r\\& r &1\end{pmatrix}
\end{equation}
and the largest eigenvector is easily seen to be given by
\begin{equation}
\lambda_0 = 1+r 
\end{equation}
In this case, we seethat Eq. (\ref{aeq}) gives
\begin{equation}
a^{2} = \frac{1-ra}{r-a}
\end{equation}
which has three solutions
\begin{eqnarray}
a_1&=& -1\\
a_2&=& \frac{1}{2} \left(-\sqrt{r^2+2
r-3}+r+1\right) \\
a_3&=& \frac{1}{2} \left(\sqrt{r^2+2
r-3}+r+1\right)\end{eqnarray}
We see that $a_2=1/a_3$, and $|a_2|=|a_3|=1$, and that
\begin{equation}
\lambda(-1)= \frac{1-r}{1+r}
\end{equation}
while 
\begin{equation}
\lambda(a_2)=\lambda(a_3)= 1+r
\end{equation}
corresponds to the maximal eigenvalue. Note that $\lambda(-1)$ is not the other eigenvalue of the transfer matrix, this has to be found by considering solutions with $c=-1$, as we will see later.

The equation (\ref{aeq}) determining $a$ can also be written as
\begin{equation}
a^{L} = - \frac{r-\frac{1}{a}}{r-a}
\end{equation}
From this we see that if $a$ is a solution then $1/a$, and that $a=-1$ is always a solution.


We now introduce $\theta$ and 
\begin{align}
a=\exp(i\theta)
\label{def-theta}
\end{align}
Then the parameter of the eigenvector is
\begin{equation}
\exp(i L \theta) = - \frac{r-\exp(-i\theta)}{r-\exp(i\theta)}
\label{thetamas}
\end{equation}
From Eq \eqref{elam}, we have
\begin{equation}
\lambda(\theta) = \frac{\sinh(\beta J)}{\cosh(\beta J) - \cos(\theta)}.
\end{equation}
Notice that in order to construct a real eigenvector corresponding to $\lambda_0$ we can use the fact that both $v_i(a) = a^i + a^{L-i}$ and $v_i(a^{-1}) = a^{-i} + a^{-L+i}$ are both eigenvectors with the same eigenvalue. This means that $u_i(a) = v_i(a) + v_i(-a)$ is also an eigenvector and all its components are real. 

Clearly the largest eigenvalue corresponds to the value of $\theta$ closest to $0$,  so we look for an eigenvalue such that $L \theta \sim 1$. We write
\begin{equation}
\phi = L \theta
\end{equation}
For $L$ large, this gives
\begin{equation}
\exp(i\phi) \approx -\frac{r-1+ i\frac{\phi}{L}}{ r-1- i\frac{\phi}{L}}\approx -1 +2 i\frac{\phi}{L(1-r)}
\end{equation}
and so we find to leading order in $1/L$
\begin{equation}
\theta = \frac{(2n+1)\pi}{L}
\end{equation}
However we notice that this approximation is only valid if $L(1-r) \gg1$. For large $\beta$ this approximation is simply equivalent to $L\gg1$, however when $\beta$ is small it requires
that $H \beta \gg 1$.

The closest eigenvector to the real axis has $n=0$ and so we have
\begin{equation}
\lambda_0 \approx \frac{\sinh(\beta J)}{\cosh(\beta J) - \cos(\frac{\pi}{L})} \approx \frac{\sinh(\beta J)}{\cosh(\beta J) - 1+ \frac{\pi^2}{2 L^2}} \approx \coth(\frac{\beta J}{2})(1 - \frac{\pi^2}{4\sinh^2(\frac{\beta J}{2}) L^2})
\label{ground-sos}
\end{equation}
In the limit $L\to \infty$, the ground-state eigenvalue is the same as the Sine-Gordon chain of length $L' \to \infty$ fixed at $h(0) = h(L') = 0$ and using a SOS interaction between nearest neighboors \cite{guyer_sine-gordon_1979}, which is normal since boundary conditions on the x-axis are negligible in the thermodynamic limit.

\subsubsection*{First excited state eigenvector}
In order to compute the second eigenvalue $\lambda_1$ we look for an odd or antisymmetric solution with $c=-1$. We thus find
\begin{equation}
\exp(i L\theta) = \frac{r-\exp(-i\theta)}{r-\exp(i\theta)}
\label{theta}
\end{equation}

For large $L$ we look for a solution of the form $\theta=\phi/L$ and this gives
\begin{equation}
\exp(i\phi) \approx 1
\end{equation}
and so we chose solutions $\phi = 2n\pi$ for integer $n$. However the solution $n=0$ which corresponds to $a=1$ has $v(i) = a^i-a^{L-i} =0$ and so does not correspond to an eigenvector. We thus take the next solution $\phi = 2\pi$ which gives
\begin{equation}
\lambda_1 \approx \frac{\sinh(\beta J)}{\cosh(\beta J) - \cos(\frac{2\pi}{L})} \approx \frac{\sinh(\beta J)}{\cosh(\beta J) - 1+ \frac{2\pi^2}{L^2}}\approx \coth(\frac{\beta J}{2})(1 - \frac{\pi^2}{\sinh^2(\frac{\beta J}{2}) L^2})
\label{excited-sos}
\end{equation}
In Fig \ref{large-l-limit}, we show the agreement between the computation of the first two eigenvalues !!!! computed by numerical diagonalisation and compared with the analytical approximations Eq. \eqref{ground-sos} and Eq. \eqref{excited-sos} which are valid for the large L limit
\begin{figure}
\centering
\includegraphics[width=0.7\linewidth]{finite-size/null_deanJ.pdf}
\caption{$\lambda_0$ and $\lambda_1$ as a function of $L$ computed by numerical diagonalisation of the transfer matrix, compared to the analytical approximations for large $L$: Eq. \eqref{ground-sos} and Eq. \eqref{excited-sos}. Here we have chosen  for $J=1$ and $\beta=1$.}
\label{large-l-limit} 
\end{figure}

The correlation length is then given by
\begin{equation}
\xi =\frac{1}{\ln(\frac{\lambda_0}{\lambda_1})} = \frac{1}{\ln(\frac{\cosh(\beta J) - \cos(\frac{\pi}{L})}{\cosh(\beta J) - \cos(\frac{2\pi}{L})})}\approx \frac{4}{3}\frac{\sinh^2(\frac{\beta J}{2})L^2}{\pi^2}
\end{equation}
and we see that this has the same form as that for the free elastic line in Eq. (\ref{corel}).
Furthermore, the free energy per site is given in the thermodynamic limit and for large $L$ by
\begin{equation}
f=-\frac{1}{\beta}\ln(\lambda_0) \approx -\frac{1}{\beta}\left[ \ln(\coth(\frac{\beta J}{2}))- \frac{\pi^2}{4\sinh^2(\frac{\beta J}{2}) L^2}\right]
\end{equation}
and this gives a pressure
\begin{equation}
P= -\frac{\partial f}{\partial L}= \frac{T\pi^2}{2 \sinh^2(\frac{\beta}{2}) L^3}
\end{equation}


This has the same form as the pressure for the elastic line in Eq. \eqref{pfree} if we make the identification of the effective surface tension to be used in the elastic line model
\begin{equation}
\sigma_{eff} = \frac{2}{\beta}\sinh^2(\frac{\beta J}{2})
\end{equation}
We should note that this is also consistent with the equality deduced by comparing the correlation length of the two models.

We see that in the limit of large $L$ and for appropriately low temperatures, the finite size SOS model reproduced the phenomenology of the elastic line (confined Edwards-Wilkinson surface). 
This is not surprising as a low temperatures jumps of more that two lattice spacings in the height are suppressed by a factor or $\exp(-\beta J)$ with respect to staying at the same height moving up or down by one site. The low temperature SOS model thus becomes effectively equivalent to the RSOS model and thus is equivalent to a local random walk model. 

\subsubsection*{High temperature limit}

To explore the high temperature limit we can note that if we write
\begin{equation}
z= r-\exp(-i\theta)
\end{equation}
we can write Eq. \eqref{thetamas} as
\begin{equation}
\exp(i L\theta) = -\frac{z}{\overline z} = \exp(2i\psi + i\pi)
\end{equation}
where
\begin{equation}
\tan(\psi) = \frac{\sin(\theta)}{r-\cos(\theta)}
\end{equation}
This then gives 
\begin{equation}
L \theta = 2\psi + \pi
\end{equation}
and so
\begin{equation}
\tan(\psi) = \frac{\sin(\theta)}{r-\cos(\theta)}= \tan(\frac{L\theta}{2} +\frac{\pi}{2})= - \cot(\frac{L\theta}{2})
\end{equation}
which finally gives
\begin{equation}
\tan(\frac{L\theta}{2}) = \frac{\cos(\theta)-r}{\sin(\theta)}
\label{mde}
\end{equation}
In this form we see that our calculations agree with those of Svravick et al \cite{svrakic_finite-size_1988}. Futhermore when $\beta\to 0$ we know that the elements of the transfer matrix all tend to one and that the largest eigenvalue has all components equal. This means that in the infinite temperature limit, $\theta=0$. Therefore in Eq. \eqref{mde} we look for solutions where $\theta$ is small. Taylor expanding gives to leading order
\begin{equation}
\frac{L\theta^2}{2} \approx1-r-\frac{\theta^2}{2}
\end{equation}
which gives
\begin{equation}
\theta \approx \sqrt{\frac{2(1-r)}{L+1}}
\end{equation}
However the above expansion assumes that $\theta L\ll1$ and so
\begin{equation}
\sqrt{2L(1-r)} \ll 1
\end{equation}
This means that the height can fluctuate by of order $L$ from site to site. The high temperature approximation is thus equivalent to
\begin{equation}
\theta \approx \sqrt{\frac{2\beta J}{L+1}}.
\end{equation}
Therefore at high temperature this means that $L \beta J\ll1$. 
This gives a maximal eigenvalue
\begin{equation}
\lambda_0 = L+1
\end{equation}
and the free energy
\begin{equation}
f=-\frac{1}{\beta}\ln(L+1)
\end{equation}
which is the obvious result coming from the infinite temperature entropy. 
This result suggests that the solution for $\theta$ at small $\beta$ can be written as a perturbation series of the form
\begin{equation}
\theta = \sqrt{\beta J}\sum_{n=0}^\infty b_n (\beta J)^n
\end{equation}
The first two terms give
\begin{equation}
\theta = \sqrt{\beta J}\left[\sqrt{\frac{2\beta J}{L+1}} -\beta J \frac{2 + 2L +L^2}{6\sqrt{2}(1+L)^{\frac{3}{2}}}\right]
\end{equation}
and from this we find
\begin{equation}
f=-\frac{1}{\beta}\ln(L+1-\beta J\frac{L^2+2L}{3})
\label{high-temperature} 
\end{equation}
 and where we show in Fig \ref{fig-high-temp} the agreement of the high-temperature approximation \eqref{high-temperature} with respect to the direct diagonalization of the transfer matrix.
As pointed out above this result gives the high temperature entropy but it also exhibits the correct average energy $\epsilon$ per unit length at high temperature. To see this we note that all values of $h$ are equiprobable at infinite temperature and so
\begin{equation}
\epsilon = \frac{1}{(L+1)^2} J \sum_{i,j=0}^L |i-j| = J\frac{L^2+2L}{3}
\end{equation}

\begin{figure}
\centering
\includegraphics[width=0.7\linewidth]{finite-size/high_temperature.pdf}
\caption{Free energy with respect to $\beta$ for $L=100$ and $J=1$ in the high-temperature limit, by direct diagonalization of the transfer matrix and by Eq \eqref{high-temperature}.}
\label{fig-high-temp} 
\end{figure}


%%%%%%%%%%%%%%%%
\section{Conclusion}
%%%%%%%%%%%%%%%%

Finite-size effects corrections in the free energy are important when the correlation length becomes of the order of magnitude of the system's size. The derivative of the free energy with respect to the system size yields a confinement pressure, which can be seen for electromagnetic fields \cite{casimir_attraction_1948,rytov_principles_1989,lifshitz_theory_1955} and for critical systems \cite{gambassi_casimir_2009}, which has long-range interaction. 

For continuous 1D interface systems, we use the path integral method \cite{matsubara_new_1955} to compute the energy of all states, which gives us in the thermodynamic limit $f = \frac{1}{\beta} \epsilon_0$ and the correlation length $\xi = \frac{1}{\epsilon_1-\epsilon_0}$. The computation of the free energy of a continuous 1D interface is thus mapped to a 1D quantum problem.  This method can be used for all potentials $V(h)$, and so we apply it to two speficic cases. In the confined elastic line, which gives us the free energy per unit length $f = \frac{T^2 \pi^2}{2 \sigma L^2}$, which gives a different power-law than the critical Casimir force. That is to be expected, since critical systems have no surface tension, but using a conjecture \cite{privman_finite-size_1988-1} about the finite-size corrections on the surface tension, we show the correspondence between both models in the case $\sigma=0$. In the semi-infinite geometry though, we compute the average height when the  interface is under pressure. \\


For discrete  Solid-On-Solid models, the path integral cannot be directly applied, and methods adapted to discrete systems must be used. 

The first method is to compute the free energy through numerical simulations. We generalize the LJH method \cite{lopes_cardozo_critical_2014} to any external potential both for Ising and SOS models, and compare it to the transfer matrix in the case of SOS for Kawsasaki dynamics in the SOS model - which was the special case where neither the Layer method nor the LJH method were pertinent.

For the confined SOS interface, we followed Švrakić \cite{svrakic_finite-size_1988} to obtain the exact eigenvalues and eigenvectors of the transfer matrix in the case $V(h)=0$. This gives us a free energy which has a the same dependence to the system's size than the confined elastic line.

\chapter{Beyond Solid-On-Solid : the Particles-Over-Particles model}
\label{chap-pop}

In Sec \ref{sec-sos} we introduced the Solid-On-Solid model with Hamiltonian
\begin{align}
H_{SOS} = \sum_i |h_i-h_{i+1}| + \frac{V(h_i)+V(h_{i+1})}{2}
\end{align}
We discuss in this chapter a variant the model which is formed distinct particle types, and explain the numerical algorithm it obeys. For a single particle type, numerical simulations show that it is a better approximation to the Ising model than the SOS model for temperatures lower than $T_C$. This model allows for model C dynamics where particle types belong to different thermodynamic ensembles and have different kinetic of diffusive coefficients.
In the multiple particles systems, we show that driving one of the particles types leads to an increase in surface fluctuations. We theoretically explain these results in terms of stochastic functional theory.

%%%%%%%%%%%%%%%%%
\section{The Single-type Particles-Over-Particles model}
%%%%%%%%%%%%%%%%%

\begin{figure}[!htb]
\centering
\includegraphics{pop/hauteur-tm-pop.pdf}
\caption{Mean height of the POP model with respect to chemical potential $\mu$ through transfer matrix with different maximal heights in the thermodynamic limit $L'\to \infty$, compared to the Striling's approximation Eq \eqref{stirling-pop},at $\beta=1$. {\color{red} add MC sim}}
\label{haut-tm-pop} 
\includegraphics{pop/comparaison-modeles.pdf}
\caption{Comparison of the internal mean total energy of the Ising, SOS, RSOS and POP models with respect to temperature in absence of external field and $\mu=0$, from Monte Carlo simulations at $L'=126$ and $L=30$ with Glauber dynamics. We used Eq \ref{energie-sos-ising} to compare energies between the interface models and the Ising one.
{\color{red} redo sims for Ising, need MCIA, takes too long otherwise}}
\label{comp-models}
\end{figure}

The SOS Hamiltonian is directly derived from the Ising model thanks to the overhangs' absence approximation. Even though, while doing numerical simulations in both systems, it is striking to see that simulations do not behave identically. In the Ising model, if there are $n_{+,i}$ positive spins and $n_{-,i}$ negative spins at column $i$, in Monte Carlo simulations for each spin can be randomly chosen with a uniform probability for a spin-flip attempt. On the contrary, in the SOS model as we described in Chapter \ref{chap-sim}, we randomly choose a column $i$ with a uniform probability. We see that even though the Hamiltonian is the same, the dynamic is different.

If the height profiles represent particle numbers, fixing the total number of particles to be $N$ and taking them to be identical, the partition function is given by
\begin{equation}
Z_{POP}(N) = \frac{1}{N!}\sum_{h_1,h_2\cdots h_{L'}} \delta_{\sum_{i=1}^L h_i, N}\frac{N!}{\prod_{i=1}^{L'} h_i!} \exp\left(-\beta J \sum_{i=1}^{L'} |h_{i+1}-h_i| -\beta\sum_{i=1}^{L'} V(h_i)\right)
\end{equation}
Here the combinatorial term $\frac{N!}{\prod_{i=1}^L h_i!}$ represents the number of ways that the $h_i$ particles on each site can be chosen from the $N$ particles available. In the same fashion as the Solid-On-Molid, we call this model the \textbf{Particles-Over-Particles model}, since we stack particles in columns of height $h_i$ .The constraint on the particle number makes the computation of the partition function at fixed $N$ complicated both analytically and numerically. However, if we change to the grand canonical ensemble using
the formula
\begin{equation}
\Xi = \sum_{N} \exp(\beta\mu N) Z_N
\end{equation}
where $\Xi$ is the grand partition function and $\mu$ the chemical potential, we find
\begin{equation}
\Xi_{POP} = \sum_{h_1,h_2\cdots h_{L'}} \frac{1}{\prod_{i=1}^{L'} h_i!} \exp\left(-\beta J \sum_{i=1}^{L'} |h_{i+1}-h_i| -\beta\sum_{i=1}^{L'}[ V(h_i)-\mu h_i]\right)
\end{equation}
The model differs from the usual solid on solid model in that a number of particle configurations give rise to the same height configurations. The grand partition function can then be written as 
\begin{equation}
\Xi = \sum_{h_1,h_2\cdots h_{L'}} \exp\left(-\beta H_{eff}(h_1,h_2\cdots h_{L'})\right)
\end{equation}
where 
\begin{equation}
H_{eff}= J \sum_{i=1}^{L'} |h_{i+1}-h_i| +\sum_{i=1}^{L'} [V(h_i)-\mu h_i +\frac{1}{\beta}\ln(h_i !)].
\end{equation}

The transfer matrix is
\begin{align}
T_{POP}(h,h') = T_{SOS}(h,h') \exp \left( \frac{\ln(h)+\ln(h')}{2} \right)
\end{align}

The Monte Carlo implementation is as follows. Each particle is labeled with the site $i$ in which it is, and $h_i$ is the number of particles numbered at that site. 
In the Glauber dynamics, particles can be exchanged with a reservoir. With a probability $1/2$ one attempts to add a particle and with probability $1/2$ one attempts to takeaway a particle. In the first case, a site $i$ is chosen with a uniform distribution, and we attempt to create a new particle labeled at site $i$ with probability $min(1,\exp(-\beta \Delta E))$. In the latter, a particle is selected uniformly between all existing particles, and an attempt to remove it is done
\footnote{In C++, we can use a $std::vector$ in which we add or remove particles. After each success attempt, we rebuild the distribution $std::uniform\_int\_distribution(0,N-1)$, where N is the number of particles. This operation is lightweight and should not cause any slowing down.}.
Kawasaki dynamics is implemented by randomly choosing a particle $n$ with probability $\frac{1}{N}$ at each Monte Carlo step, then trying to move the particle to the left or right using Metropolis acceptance rate. 

In Fig \ref{comp-models}, we plot the internal energy of the Ising, SOS, RSOS and POP models with respect to temperature in absence of external fields and $\mu=0$ wtih non-conserved dynamics. {\color{red} discuss the graph once plots are better}

Contrary to the SOS model where there needs to be a confining external field in order to localize the interface \cite{burkhardt_localisation-delocalisation_1981,chui_pinning_1981}, the entropic term gives a stable position for the interface. In absence of external field, the effective potential is given by
\begin{align} 
    V_{eff}(h) = - \mu h + \frac{1}{\beta}\ln(h!)
\end{align}
If the chemical potential is large enough, the number of particles $N$ is large enough, so we can use Striling's formula and approximate a continuous derivative with the finite-difference in $h$, so we have
\begin{align} 
    V_{eff}(h)' = - \mu +\frac{1}{\beta} \ln(h)
\end{align}which gives the mean height 
\begin{align} 
    <h> = \exp(\beta \mu) 
\label{stirling-pop}
    \end{align}
    In Fig \ref{haut-tm-pop}, we show the mean height \eqref{stirling-pop} compared to the transfer matrix diagonalisation with different matrix size and the Monte Carlo simulations at$\beta=1$. We see that when $ \gg 1$, the Stirling's formula becomes valid and Eq \eqref{stirling-pop} becomes accurate. Since $<h>$ cannot exceed the maximum size of the system, saturation occurs at large $\mu$. 


%%%%%%%%%%%%%%%%%%%%
    \section{$M$-particles POP system}
%%%%%%%%%%%%%%%%%%%%


\begin{figure}
    \centering
    \includegraphics[width=0.7\linewidth]{pop/figure-pop.pdf}
    \caption{Possible POP configuration with two types of particles $p_1$ and $p_2$. The red line shows the origin $z=0$. In the $i$-th column the interface is at height $h_i$, with $n_{p_0,i}$ particles of type $p_0$ at site $i$, and same for particles $p_2$. Over the interface, there are no particles. }
    \label{fig-pop}
\end{figure}

We consider a model of a surface delimiting a bulk phase of $L'$  sites which contains $M$ different particle types $p_1 ... p_M$, and $n_{m,i}$ denote the number of particles of type $m$ at site $i$, and the interface height is $h_i = \sum_m n_{m,i}$. Taking into account the entropic contribution, the effective Hamiltonian for the model is
\begin{align}
    H[M] = J \sum_i |h_i-h_{i+1}| + \sum_i V(h_i) - \sum_m \mu_m \sum_i n_{m,i} + \frac{1}{\beta} \sum_m \sum_i \ln(n_{m,i})
    \label{ham-pop-c}
\end{align}
We assume that the particles in each column are demixed, i.e. the permitted particle configurations are taken to be stacked vertically such that the stack of $p_{m+1}$ particles lies on top  of the $p_m$ particles, as seen in Fig \ref{fig-pop} for $M=2$. The first term in the Hamiltonian corresponds to the surface tension with a gas phase above the stacks of particles. As disccused with the SOS model, we can have a restricted or gaussian version of Eq \ref{ham-pop-c}. 

The grand partition function is given by
\begin{align}
    \Xi = \sum_{p_1...p_M} \exp(-\beta H[M])
\end{align}

The motivation for a $M$ particle theory is that when one develops a continuous, off latice theory, based on Brownian dynamics, any single particle theory is insensitive to constant driving due to Galilean invariance, the presence of two or more different particle types breaks this invariance and yields non-equilibrum effects.

%%%%%%%%%%%%%%%%
\chapter{Driven interfaces}
\label{chap-driven}
%%%%%%%%%%%%%%%%

One of the most natural ways of creating a non-equilibrium steady state is by applying
external driving forces, as shown in Fig \ref{fig-driven}. Driving arises naturally in sedimenting systems due to gravity, in systems with free charges under the action of an electric field, and also due to the radiation pressure exerted by a laser. Experiments where a phase separated colloidal system is sheared parallel to the interface show that driving due to shear tends to suppress surface fluctuations \cite{derks_suppression_2006}, and similar results are found where Ising models are numerically sheared \cite{smith_interfaces_2008,smith_lateral_2010}. These results are somewhat surprising, for instance they
are contrary to the observation that wind generates waves on the ocean. One may think that the precise nature of the driving plays a role, for instance uniformly driving a system may be intrinsically diferent to applying a shear field which is manifestly nonuniform. 

In this capter we investigate driving using three different methos. 
In the first section - which is almost a verbatim of a paper we have published \cite{dean_effect_2020} -  we develop model C interfaces (model B interface is invariant under galilian transformation under driving) and find that the height fluctuations are suppressed and the correlation length of the fluctuations is increased. 
In the second section, we discuss driving in the SOS model, where the galilean invariance is broken by the discrete-time nature of numerical simulations, and find an increase of the interface width with respect to driving.

%%%%%%%%%%%%%%%%
    \section{The effect of driving on model C interfaces}
%%%%%%%%%%%%%%%%

Constructing a continuum model which is analytically tractable and is also affected by uniform driving is straightforward but contains some subtleties. In a continuum system it is clear that uniform driving can only move a system away from equilibrium when the driving acts differently on different particle types. For instance, consider  a system of identical interacting Brownian particles driven by a uniform  force. The force will induce the same average velocity on all the particles, consequently, in the frame moving with this average velocity, we will recover the unmodified equilibrium state. However, when multiple particle types are present, the mean velocity induced on different species are different and no Galilean transformation is possible. Perhaps the first such study of this phenomenon was due to Onsager \cite{lars_onsager_collected_1996}, who studied the conductivity of electrolytes and in doing so showed how the correlation functions in the steady state were modified by the electric field. 
Recently there have been many studies of driven multi-particle Brownian systems \cite{dzubiella_lane_2002,chakrabarti_dynamical_2003,chakrabarti_reentrance_2004,sutterlin_dynamics_2009,glanz_nature_2012, klymko_microscopic_2016}, including the electrolyte problem,  and rich new physics has been found, even in the case of purely Gaussian theories \cite{demery_conductivity_2016,poncet_universal_2017} based on stochastic density functional theory \cite{dean_renormalization_1996}. 

The dynamics of discrete particle systems is however affected by uniform driving of identical particles. The study of driven lattice gases has revealed a wide range of intriguing physical phenomena and indeed shown how driving can even lead to phase separation \cite{katz_nonequilibrium_1984,zia_interfacial_1991,leung_anomalous_1993,schmittmann_driven_1998}. The discrete nature of the dynamics of these systems, both in space and time, means that no Galilean transformation to an equilibrium state exists. Analytical studies of these systems require a phase ordering kinetics description in terms of a continuum order parameter. In order to break Galilean invariance the local mobility of the particles can be taken to be dependent on the local order parameter, this is then sufficient to induce non-trivial steady states under driving \cite{katz_nonequilibrium_1984,leung_anomalous_1993,smith_interfaces_2008,smith_lateral_2010}. Interfaces between the separated phases in uniformly driven systems have non capillary behaviors which are, even today, not fully understood \cite{leung_anomalous_1993}.
Taking random driving in a given direction also leads to non-equilibrium steady states, if the noise is Gaussian and white, the fluctuation dissipation theorem is violated and novel interface fluctuations are induced which, again, are  not of  the capillary type \cite{zia_interfacial_1991}. 

Driving can also be deterministic but space dependent, for instance if one considers applied shear flows, the spatial dependence of the flow means no Galilean transformation to an equilibrium steady state is possible and this therefore leads to non-equilibrium steady states. The effect of shear on interfaces in these type of systems yields interface equations of the stochastic Burgers type and the statistics are no thus longer Gaussian due to the presence of nonlinearities \cite{bray_interface_2001,bray_interface_2001-1,smith_interfaces_2008,smith_lateral_2010,thiebaud_nonequilibrium_2010,thiebaud_nonlinear_2014}

In this chapter we analyse what is known, in the classification of Hohenberg and Halperin \cite{hohenberg_theory_1977}, as model C type dynamics for two fields, one with conserved model B type dynamics, which is in addition convected at a uniform velocity to mimic driving. We refer to this first field as the colloid field.
This colloid field is coupled to an additional field which undergoes model A non-conserved dynamics and which is not subjected to the driving. The model A field can be thought of a passive solvent and its coupling to the model B field is chosen in such a way that it has no influence on the non-driven equilibrium steady state. We then derive the effective dynamics between two separated low temperature phases by using a
method introduced in \cite{bray_interface_2001,bray_interface_2001-1} for the study of interfaces under shear flow. This method yields a Gaussian theory for the interface statistics and driving introduces interesting new physics, notably we find that the effective surface tension of the system is increased but also the correlation length of interface fluctuations (due to an effective gravitational term) are increased. These observations are in qualitative agreement with experimental results on sheared low tension interfaces in phase separated colloidal systems \cite{derks_suppression_2006}. In this experimental system the interface fluctuations were also found to be well described by Gaussian statistics and this is our principal motivation for studying theories which remain Gaussian but are  modified by driving. While the long wavelength theory we find is of a capillary type, we also find new, higher derivative terms, which  are generated in the spectrum of the height fluctuations. 

As an aside, we also show how the model introduced here can be used to analyse the effect of activity on the dynamics of the surface between two phases of active colloids. The activity is implemented by taking a different temperature for the colloid and solvent fields, this difference in temperatures leads to significantly modified surface statistics which again develop dependencies on static and dynamical variables of the model which otherwise remain hidden for the equilibrium version of the problem.

%%%%%%%%%%%%%%%%
    \subsection{The underling two field  model}
%%%%%%%%%%%%%%%%

We consider a coarse grained model for two scalar fields $\psi$ and $\phi$ with Hamiltonian
\begin{equation}
    H[\psi,\phi] = H_1[\psi] +H_2[\psi,\phi]
\end{equation}
The Hamiltonian $H_1$ is of the classic Landau-Ginzburg form
\begin{equation}
    H_1[\psi]=\int d\bx\left[\frac{\kappa}{2}[\nabla\psi(\bx)]^2 + V(\psi(\bx))
- gz \psi(\bx)\right]
\end{equation}
The last term represents the energy due to a gravitational field and will introduce a finite correlation length in the fluctuations between the two phases. We assume that the above Hamiltonian has two stable phases with average concentrations of the field $\phi(\bx)$ given by the constant values $\psi_1$ and $\psi_2$, the difference between the order parameter in  the two different phases is denoted by 
by $\Delta\psi= \psi_2 -\psi_1 \greater 0$. This means that we find the phase $1$ as $z\to\infty$ and the phase $2$ as $z\to-\infty$. The term $H_2$ is taken to be a simple quadratic coupling between the fields
\begin{equation}
    H_2 =\int d\bx \frac{\lambda}{2}(1-\psi(\bx)-\phi(\bx))^2
\end{equation}
which is an approximative conservation law of total volume fraction of the phases. The field $\phi$ can be though of as the local volume fraction of the solvent in a colloidal system. However the presence of this solvent field does not change the effective equilibrium statistical mechanics of the colloid field $\psi$ as the partition function can be written as 
\begin{equation}
    Z = \int d[\phi]d[\psi]\exp(-\beta H_1[\psi]- \beta H_2[\psi,\phi]) = CZ_{eff}
\end{equation}
where $Z_{eff}$ is the effective partition function for the field $\psi$, after we have integrated out the degrees of freedom corresponding to the field $\phi$,
and $C$ is a constant term resulting from this integration. The effective partition function is thus simply given by
\begin{equation}
    Z_{eff} = \int d[\psi]\exp(-\beta H_1[\psi])
\end{equation}
and, as stated above, we see that the field $\phi$ thus has no effect on the equilibrium statistical mechanics of the field $\psi$.

\begin{figure}
    \centering
    \includegraphics[width=0.7\linewidth]{drivenC/driven.pdf}
    \caption{Schematics of the advection term ${\bf v}\cdot { \nabla}\psi(\bx,t)$ for a field under the interface approximation \eqref{dynamical-interface}. The red and blue phase respectively correspond to $\phi_2$ and $\psi_1$, with $\phi_2 \greater \psi_1$, and the solid line the interface between phases.}
    \label{fig-driven}    
\end{figure}

We now consider the dynamics of the fields. We take local diffusive model B dynamics for the field $\psi$ and non-conserved model A dynamics for the field $\phi$
\begin{eqnarray}
\frac{\partial \psi(\bx,t)}{\partial t} +{\bf v}\cdot { \nabla}\psi(\bx,t)&=& D\nabla^2\frac{\delta H}{\delta \psi(\bx)}+ \sqrt{2D T}\nabla \cdot {\bm \eta}_1(\bx,t) \\
\frac{\partial \phi(\bx,t)}{\partial t} &=& -\alpha\frac{\delta H}{\delta \phi(\bx)}+ \sqrt{2\alpha T}{ \eta}_2(\bx,t).
\end{eqnarray}
The first equation corresponds to standard model B dynamics but with an advection term by a constant velocity field $\bf v$. 
In Fig \ref{fig-driven}, we show the effect of that advection term in the presence of an interface. Since phase 1 is more diluted than phase 2, there are less particles which are driven at constant velocity $\bv$.
The second equation has no advection term and is simple model A dynamics. In principle we can also treat the case where the dynamics of the field $\phi$ is also diffusive and thus of model $B$ type, the analysis given here can be extended to this case but the analysis of the resulting equations is considerably more complicated. The use of model A dynamics for the solvent is justified by assuming that its dynamics is faster than that of the colloids and that the volume fraction can vary due to local conformational changes rather than  diffusive transport.

The noise terms above 
are uncorrelated and Gaussian with zero mean, their correlation functions are given by
\begin{eqnarray}
< \eta_{1i}(\bx,t) \eta_{1j}(\bx',t)>&=& \delta_{ij}\delta(t-t') \delta(\bx-\bx') \\
< \eta_{2}(\bx,t) \eta_{2}(\bx',t)>&=& \delta(t-t') \delta(\bx-\bx') ,
\end{eqnarray}
and $T$ is the temperature in units where $k_B=1$.
These dynamical equations  are thus explicitly given by
\begin{equation}
    \frac{\partial \psi(\bx,t)}{\partial t} +{\bf v}\cdot { \nabla}\psi(\bx,t)= D\nabla^2[\frac{\delta H_1}{\delta \psi(\bx)}+\lambda(\phi(\bx,t) + \psi(\bx,t))]+ \sqrt{2D T}\nabla \cdot {\boldsymbol \eta}_1(\bx,t)
\end{equation}
and
\begin{equation}
    \frac{\partial \phi(\bx,t)}{\partial t} = -\alpha\lambda[\phi(\bx,t) + \psi(\bx,t)]+ \sqrt{2\alpha T}{ \eta}_2(\bx,t).
\end{equation}
Taking the temporal Fourier transform, defined with the convention
\begin{equation}
    \tilde F(\bx, \omega) = \int_{-\infty}^\infty dt \exp(-i\omega t)F(\bx, t),
\end{equation}
we can eliminate the field $\tilde \phi$ which is given by
\begin{equation}
    \tilde \phi(\bx,\omega) = \frac{-\alpha\lambda \tilde \psi(\bx,\omega)+\sqrt{2\alpha T}\tilde \eta_2(\bx,\omega)}{i\omega +\alpha \lambda},
\end{equation}
this then gives the closed equation for $\tilde \psi$:
\begin{equation}
    \left[1-\frac{\lambda D \nabla^2}{i\omega+\alpha\lambda}\right]i\omega \tilde\psi(\bx, \omega) +{\bf v}\cdot\nabla\tilde\psi(\bx, \omega)
= D\nabla^2 \tilde \mu(\bx,\omega) +  \tilde \zeta(\bx,\omega),
\end{equation}
where 
\begin{equation}
    \mu(\bx,t)=\frac{\delta H_1}{\delta \psi(\bx,t)}
\end{equation}
is the effective chemical potential associated with the field $\psi$ and the noise term is given by
\begin{equation}
    \tilde \zeta(\bx,\omega) = \frac{\sqrt{2\alpha T}D\lambda}{i\omega + \alpha\lambda}\nabla^2\tilde \eta_2(\bx,\omega) +
\sqrt{2DT}\nabla\cdot\tilde {\bm \eta}_1(\bx,\omega).
\end{equation}
Inverting the temporal Fourier transform then gives the effective evolution equation
\begin{equation}
    \frac{\partial \psi(\bx,t)}{\partial t} -\lambda D\nabla^2\int_{-\infty}^t dt'
\exp(-\alpha\lambda(t-t')) \frac{\partial \psi(\bx,t')}{\partial t}+{\bf v}\cdot\nabla\psi(\bx, t)=D\nabla^2  \mu(\bx,t') +  \zeta(\bx,t).\label{dyn1}
\end{equation}

%%%%%%%%%%%%
\subsection{Effective interface dynamics}
%%%%%%%%%%%%

Following Sec \ref{sec-heightd}, we derive the dynamical equation for the interface $h(\br)$, where
\begin{equation}
    \psi(\bx,t) = f(z-h({\bf r},t))
    \label{dynamical-interface}
\end{equation}
and $f(z)\to \psi_2$ as $z\to -\infty$ and $f(z)\to \psi_2$ as  $z\to \infty$, and we use the sharp interface aproximation
\begin{equation}
    f'(z)=\Delta\psi \delta(z)
    \label{eqdelta}
\end{equation}
We also assume that the driving is in the ${\bf r}=(x,y)$. The dynamical evolution for the field $\psi$ in Eq. (\ref{dyn1}) is first written as
\begin{equation}
\nabla^{-2}\left[\frac{\partial \psi(\bx,t)}{\partial t}+{\bf v}\cdot\nabla\psi(\bx, t)\right] -\lambda D\int_{-\infty}^t dt'
\exp(-\alpha\lambda(t-t')) \frac{\partial \psi(\bx,t')}{\partial t'}=D  \mu(\bx,t') + \nabla^{-2} \zeta(\bx,t).\label{eqpsi}
\end{equation}
Using the relations \eqref{triple-interface} and \eqref{fdhdt}, where $V(\psi(\bx)) = V(\psi(\bx))-gz \psi(\bx)$, multiplying both sides by  $f'(z-h(\br,t))$ and integrating over $z$ as in Eq \eqref{int-potentiel-phi}, we obtain
\begin{eqnarray}
\int_{-\infty}^\infty dz f'(z-h({\bf r},t)\mu(\bx,t)&=& \kappa \nabla^2 h({\bf r},t)\int_{-\infty}^\infty dz\ f'(z-h({\bf r},t))^2 - \int_{-\infty}^\infty dz gz f'(z-h({\bf r},t))\nonumber \\&=&
\kappa\nabla^2 h({\bf r},t)\int_{-\infty}^\infty dz'\ f'(z')^2 - \int_{-\infty}^\infty dz' g(z' +h({\bf r},t)) f'(z')\nonumber \\
&=& \kappa\nabla^2 h({\bf r},t)\int_{-\infty}^\infty dz' \ f'(z')^2 -\Delta\psi g h({\bf r},t).
\end{eqnarray}
By using the Cahn-Hilliard estimate of the surface tension \eqref{CHST}, we thus find
\begin{equation}
    \int_{-\infty}^\infty dz f'(z-h({\bf r},t)\mu(\bx,t) = \sigma[\nabla^2 h({\bf r},t)-m^2 h({\bf r},t)]
\end{equation}
where $m^2 = \Delta\psi g /\sigma$. 


We now carry out the same operation on the left hand side of Eq. (\ref{eqpsi}). First we have
\begin{align}
\nabla^{-2}\frac{\partial \psi(\bx,t)}{\partial t}&+{\bf v}\cdot\nabla \psi(\bx,t) +\lambda D\int_{-\infty}^t dt'
\exp(-\alpha\lambda(t-t')) \frac{\partial \psi(\bx,t')}{\partial t'} = \nn
&-\nabla^{-2}f'(z-h({\bf r},t))[\frac{\partial h({\bf r},t)}{\partial t} +{\bf v}\cdot\nabla h({\bf r},t)]  +\lambda D\int_{-\infty}^t dt'
\exp(-\alpha\lambda(t-t')) f'(z-h({\bf r},t'))\frac{\partial h({\bf r},t')}{\partial t'}\nn
&\approx -\nabla^{-2}f'(z) [\frac{\partial h({\bf r},t)}{\partial t} +{\bf v}\cdot\nabla h({\bf r},t)] +\lambda D\int_{-\infty}^t dt'
\exp(-\alpha\lambda(t-t')) f'(z)\frac{\partial h({\bf r},t')}{\partial t'}
\end{align}
where in the last line above we have neglected terms quadratic in $h$. 
Note that the neglecting of these additional terms is not strictly justified, they could potentially induce non-perturbative effects which render the surface fluctuations non-Gaussian. However we see here that the first order computation we carry out tends to reduce fluctuations with respect to equilibrium or non-driven interfaces and so if the equilibrium theory can be described by an equation which is linear in height fluctuations, it seems physically reasonable to assume that the the approximation also holds for the driven interface. 
Again, we multiply the above by $f'(z)$ and integrate over $z$. 

Putting this all together we obtain
\begin{align}
    \Delta\psi^2 \int d{\bf r} G(0,{\bf r}-{\bf r}') [\frac{\partial h({\bf r},t)}{\partial t} +{\bf v}\cdot\nabla h({\bf r},t)] +\frac{\sigma\lambda D}{\kappa}\int_{-\infty}^t dt'
\exp(-\alpha\lambda(t-t'))\frac{\partial h({\bf r},t')}{\partial t'}
= \nn
 \sigma[\nabla^2 h({\bf r},t)-m^2 h({\bf r},t)] + \xi({\bf r},t)
    \label{em}
\end{align}
where $G= -\nabla^{-2}$, or more explicitly
\begin{equation}
    \nabla^2 G(z-z',{\bf r}-{\bf r}') = -\delta(z-z') \delta({\bf r}-{\bf r'})
\end{equation}
The noise term $\xi$ is given by
\begin{equation}
    \xi({\bf r},t) = \int_{-\infty}^{\infty} dz f'(z-h({\bf r},t)) \nabla^{-2} \zeta(\bx,t).
\end{equation}
Now, as the equations of motion have been derived to first order in $h$ and we wish to recover the correct equilibrium statistics for the non-driven system, we ignore the $h$ dependence in the noise and make the approximation
\begin{equation}
    \xi({\bf r},t) \approx \int_{-\infty}^{\infty} dz f'(z) \nabla^{-2} \zeta(\bx,t).
\end{equation}
The correlation function of this noise is most easily evaluated in terms of its Fourier transform with respect to  space and time  defined by
\begin{equation}
    \hat F({\bf q},\omega)=\int dt d{\bf r}\exp(-i\omega t -i{\bf q}\cdot{\bf r}) F({\bf r},t).
\end{equation}
Using the relations Eqs. \eqref{CHST} and \eqref{eqdelta} one  can show that
\begin{equation}
    < \hat \xi({\bf q},\omega)\hat \xi({\bf q}',\omega')> =2T(2\pi)^d \delta(\omega +\omega') \delta({\bf q}+{\bf q}') \left[ \frac{\sigma}{\kappa}\frac{\alpha D^2\lambda^2}{\omega^2 +\alpha^2\lambda^2} + \frac{D\Delta\psi^2}{2q}\right]
\end{equation}
In full Fourier space the equation of motion for the field $\psi$ then reads
\begin{equation}
    \left[i(\omega+{\bf q}\cdot{\bf v})\frac{\Delta\psi^2}{2q} + \frac{D\sigma\lambda}{\kappa} \frac{i\omega}{\alpha\lambda+i\omega}\right] \hat h({\bf q},\omega)= -D\sigma(q^2+m^2)\hat h({\bf q},\omega)+ \hat\xi({\bf q},\omega)
    \label{dyn}
\end{equation}

From this, the full Fourier transform of the correlation function of the interface height is given by
\begin{equation}
    \hat C({\bf q},\omega)  = 2TD \frac{\left[ \frac{\Delta\psi^2}{2q}(\omega^2+\alpha^2 \lambda^2) + \frac{\sigma\alpha D\lambda^2}{\kappa}\right]}{\left|i[\frac{\alpha\lambda\Delta\psi^2}{2 q}(\omega + {\bf q}\cdot{\bf v}) + \frac{\lambda \sigma D}{\kappa}\omega + D\sigma(q^2+m^2)\omega]
+[\alpha\lambda D\sigma(q^2+m^2) -\frac{\Delta\psi^2}{2q}\omega(\omega+{\bf q}\cdot{\bf v})]\right|^2}
\end{equation}
Using the above we can extract the equal time height-height correlation function in the steady states. Its spatial Fourier transform can shown to be given by
\begin{align}
    \tilde C_s({\bf q}) = \frac{1}{2\pi} \int d\omega \hat C({\bf q}, \omega)
    \label{fourier-steady-state}
\end{align}

This integral has the same form as 
\begin{equation}
I(f(\omega)) = \int \frac{d\omega}{2\pi} \frac{f(\omega)}{\left|i(A\omega + B) + (C-D\omega-E \omega^2)\right|}
\end{equation}
so we see that the integral we need to evaluate can be written in the form
\begin{equation}
I = a I(\omega^2) + b I(1)
\end{equation}
The calculation of Eq. \eqref{fourier-steady-state} can be carried out in the presence of a forcing term on the height profile in order to compute the response function for the surface which has a denominator of the form
\begin{equation}
{\rm Den} = i(A\omega + B) + (C-D\omega-E \omega^2)
\end{equation}
and due to causality the above only has poles in the upper complex plane (due to the convention of Fourier transforms used here). Consequently we find that
\begin{equation}
    \int \frac{d\omega}{2\pi} \frac{1}{i(A\omega + B) + (C-D\omega-E \omega^2)} = 0
    \label{key}
\end{equation}
as one may close the integration contour in the lower half of the complex plane. Taking the real and imaginary part of Eq. (\ref{key}) leads to
\begin{eqnarray}
C I(1) -D I(\omega) - E I(\omega^2) = 0 \\
AI(\omega) + B I(1) = 0
\end{eqnarray}
Using this we can express $I(\omega^2)$ as a function of $I(1)$, and explicitly we have 
\begin{equation}
I(\omega^2) = \frac{I(1)}{E}[C+ \frac{DB}{A}]
\end{equation}

To evaluate $I(1)$ we now use
\begin{equation}
I(1) = -{\rm Im} \int \frac{d\omega}{2\pi}\frac{1}{A\omega +B} \frac{1}{i(A\omega + B) + (C-D\omega-E \omega^2)}
\end{equation}
The integrand above has no poles in the lower half of the complex plane but has a {\em half pole} at $\omega=-B/A$ on the real axis, thus using standard complex analysis we find
\begin{equation}
I(1) = \frac{1}{2(CA + BD - \frac{EB^2}{A})}
\end{equation}

Then after some laborious, but straightforward algebra, we obtain that 
\begin{align}
\tilde C_s({\bf q} = T \frac{\left(2 D\sigma q(\kappa[q^2+m^2]+\lambda)+\alpha\kappa\lambda\Delta\psi^2\right)^2 +\kappa^2 \Delta\psi^4 ({\bf q}\cdot{\bf v})^2}{\sigma[q^2+m^2]\left(2D q\sigma (\kappa[q^2+m^2]+\lambda)+\alpha \kappa\lambda \Delta\psi^2\right)^2 + \kappa\left(\kappa\sigma[q^2+m^2] + \lambda\sigma\right)\Delta\psi^4({\bf q}\cdot{\bf v})^2}\label{eqmaind}
\end{align}

In the absence of driving, {\em i.e.} when ${\bf v}={\bf 0}$ we recover the equilibrium correlation function
\begin{equation}
    \tilde C_s({\bf q})= \tilde C_{eq}({\bf q})= \frac{T}{\sigma[q^2+m^2]},
\end{equation} 
here we see that  $1/m= \xi_{eq}$ is the so called capillary length, which is the equilibrium correlation length of the height fluctuations. We also notice that the correlation function for wave vectors perpendicular to the driving direction is simply the equilibrium one.

If we write $C_s({\bf q})= T/H_s({\bf q})$ we can interpret $H_s({\bf q})$ as an effective quadratic Hamiltonian for the height fluctuations, it is thus given by
\begin{equation}
    H_s({\bf q}) = \sigma[q^2+m^2] + \frac{\kappa\lambda\sigma \Delta\psi^4 ({\bf q}\cdot{\bf v})^2}{\left(2 D\sigma q(\kappa[q^2+m^2]+\lambda)+\alpha\kappa\lambda\Delta\psi^2\right)^2 +\kappa^2 \Delta\psi^4 ({\bf q}\cdot{\bf v})^2}
\end{equation}
For small $q$ we find 
\begin{equation}
    H_s({\bf q}) = \sigma m^2 + \sigma q^2(1+ \frac{v^2\cos^2(\theta)}{\alpha^2\lambda\kappa}),
\end{equation}
where $\theta$ is the angle between the wave vector ${\bf q}$ and the direction of the driving. 
This thus gives a direction dependent surface tension 
\begin{equation}
    \sigma_s(\theta) = \sigma(1+ \frac{v^2\cos^2(\theta)}{v^2_0})
    \label{direction-surface-tension}
\end{equation}
where we have introduced the intrinsic velocity $v_0 = \sqrt{\alpha^2\lambda\kappa}$ which depends on the microscopic {\em dynamical} quantity $\alpha$ associated with the model A dynamics of the field $\phi$, as well as the microscopic static quantities $\kappa$ (which generates the surface tension) and $\lambda$ the coupling between the field $\psi$ and $\phi$. This appearance of dynamical and static quantities that are otherwise hidden in equal time correlation functions in equilibrium is already implicit in the works of Onsager \cite{lars_onsager_collected_1996} where it is used to compute the conductivity of Brownian electrolytes and the explicit expressions were derived using stochastic density functional theory in \cite{demery_conductivity_2016}. We also note that the universal thermal Casimir effect between model Brownian electrolyte systems  driven by an electric field 
exhibits similar features, developing a dependency on both additional static and dynamical variables with respect to the equilibrium case \cite{dean_nonequilibrium_2016}


However for this small $q$ expansion we see that the microscopic 
quantities $D$, the diffusion constant of the field $\phi$, and the order parameter jump
$\Delta\psi$ do not appear. 

From the above, we see that  in the direction of the driving the surface tension increases and the fluctuations of the surface are thus suppressed. We may also write 
\begin{equation}
    H_s({\bf q}) = \sigma_s(\theta) [q^2 + m^2_e(\theta)],
\end{equation}
with 
\begin{equation}
    m^2_s(\theta) =\frac{ m^2}{1+ \frac{v^2\cos^2(\theta)}{v_0^2}},
\end{equation}
this corresponds to a correlation length 
\begin{equation}
    \xi_s = \xi_{eq}\sqrt{1+ \frac{v^2\cos^2(\theta)}{v_0^2}},
\end{equation}
and we see that it is increased in the direction of the driving. 

As we have just remarked  that the above results appear to be independent of the order parameter jump $\Delta \psi$ and the diffusion constant $D$, however the next order correction to $H_s$ for small $q$ is given by
\begin{equation}
    H_s({\bf q}) = \sigma_s(\theta) [q^2 + m^2_e(\theta)] - \frac{4Dq \sigma^2(\lambda+\kappa m^2)( {\bf q}\cdot{\bf v})^2 }{\alpha^3 \kappa^2 \lambda^2 \Delta\psi^2},
\end{equation}
and so the small ${\bf q}$ expansion  breaks down at $\Delta\psi=0$, indeed one can see that the system has exactly the equilibrium correlation function when  $\Delta\psi=0$. 

In the limit of large $q$ we see that the effective Hamiltonian is given, to leading order, by the original equilibrium Hamiltonian and so the out of equilibrium driving has no effect on the most energetic modes of the system.

The results here predict that for unconfined surfaces the long range height fluctuations are described by an isotropic form of capillary wave theory with 
an anisotropic surface tension which is largest in the direction of driving. Numerical simulations of driven lattice gases in two dimensions \cite{leung_anomalous_1993} show a more drastic change upon driving and find $C_s(q)\sim  1/q^{.66}$ and thus a strong deviation from capillary wave theory.  

%%%%%%%%%%%%%%%%
    \subsection{A model of active interfaces}
%%%%%%%%%%%%%%%%

We can apply the results derived in the previous section to analyse a simple model for
surfaces formed between two phases of active colloids. Activity is modelled by assuming that the colloidal field $\psi$ has a temperature different to that of  the solvent field $\phi$. This models the effect that activity leads to enhanced colloidal diffusivity over and
above the Brownian motion of particles due to thermal fluctuations \cite{grosberg_nonequilibrium_2015}.

In the absence of any driving the dynamical equations for the field $\psi$ and $\phi$ become 
\begin{eqnarray}
\frac{\partial \psi(\bx,t)}{\partial t} &=& D\nabla^2\frac{\delta H}{\delta \psi(\bx)}+ \sqrt{2D T_1}\nabla \cdot {\bm \eta}_1(\bx,t) \\
\frac{\partial \phi(\bx,t)}{\partial t} &=& -\alpha\frac{\delta H}{\delta \phi(\bx)}+ \sqrt{2\alpha T_2}{ \eta}_2(\bx,t).
\end{eqnarray}
Following the same arguments as above we find that
\begin{equation}
    \hat C({\bf q},\omega)  = 2D \frac{\left[ T_1\frac{\Delta\psi^2}{2q}(\omega^2+\alpha^2 \lambda^2) + T_2\frac{\sigma\alpha D\lambda^2}{\kappa}\right]}{\left|i\omega[\frac{\alpha\lambda\Delta\psi^2}{2 q} +  \frac{\lambda \sigma D}{\kappa} + D\sigma(q^2+m^2)]
+[\alpha\lambda D\sigma(q^2+m^2) -\frac{\Delta\psi^2}{2q}\omega^2]\right|^2}.
\end{equation}
The equal time steady state height fluctuations thus have correlation function
\begin{equation}
    \tilde C_s(q) = \frac{T_1}{\sigma (q^2 + m^2)}\left[ 1 -(1-\frac{T_2}{T_1})\frac{\lambda\sigma } {\kappa }\frac{1}{\frac{\alpha\lambda \Delta \psi^2}{2Dq}+ \frac{\lambda\sigma }{\kappa} + \sigma(q^2+m^2)}\right].
\end{equation}
We see, again, that the inclusion of a non-equilibrium driving changes the statistics of height fluctuations and leads to a steady state that depends on both dynamical variables
$D$ and $\alpha$ as well as static ones $\Delta\psi,\ \lambda$ and $\kappa$ that remain hidden in the equilibrium case. This phenomenon is again seen in the behavior of the universal thermal  Casimir force between Brownian conductors held at different temperatures \cite{lu_out--equilibrium_2015}.

If we assume strong activity we can take the limit $T_1\gg T_2$, in this case we find
\begin{equation}
    \tilde C_s(q) = \frac{T_1}{\sigma (q^2 + m^2)}\frac{\frac{\alpha\lambda \Delta \psi^2}{2Dq}+
\sigma(q^2+m^2)}{\frac{\alpha\lambda \Delta \psi^2}{2Dq}+ \frac{\lambda\sigma }{\kappa} + \sigma(q^2+m^2)}.
\end{equation}
Interpreted in terms of an effective Hamiltonian for an equilibrium system at the temperature $T_1$ the above gives
\begin{equation}
    H_s(q) = \sigma (q^2 + m^2)\left[1+\frac{\lambda\sigma }{\kappa}\frac{q}{\frac{\alpha\lambda \Delta \psi^2}{2D}+
q\sigma(q^2+m^2)}\right].
\end{equation}psi
In the case of an unconfined interface (where there is no gravitational effect
on the surface fluctuations) {\em i.e.} $m=0$ we see that for small $q$
\begin{equation}
    H_s(q) \approx \sigma q^2 +\frac{2D\sigma^2 }{\kappa\alpha \Delta\psi^2}q^3 .
\end{equation}
We see that the effective surface tension is not modified but a reduction of fluctuations due to the presence of the term in $q^3$ arises.  As in the case of a driven system, we see that the large $q$ behavior of the effective Hamiltonian is given by the equilibrium case where $T=T_1=T_2$. 

In the case where the interface is confined, we see that for small $q$ one obtains
\begin{equation}
    H_s(q) \approx \sigma m^2 \left[ 1+ \frac{2D\sigma }{\kappa\alpha \Delta\psi^2}q\right],
\end{equation}
and thus at the largest length scales of the problem there is a qualitative departure from capillary wave behavior induced by activity, and the correlation length of height fluctuations at the largest length scales is given by
\begin{equation}
    \xi_a = \frac{2D\sigma }{\kappa\alpha \Delta\psi^2}.
\end{equation}
The above result should be compared with that obtained in \cite{zia_interfacial_1991} for 
systems with anisotropic thermal white noise, which breaks detailed balance and mimics random driving of the system parallel to the interface; for free interfaces it was found that $C_s(q)\sim 1/q$.

%%%%%%%%%%%%%%%%%%%%
    \subsection{Conclusion}
%%%%%%%%%%%%%%%%%%%%

We have presented a model to analyse the effect of uniform driving on the dynamics of the interface in a two phase system. In order to generate a non-equilibrium state a second {\em hidden} order parameter was introduced. This models the behaviour of a local or solvent degree of freedom which is not influenced by the driving field. In this way, we obtain out of equilibrium interface fluctuations which are described by Gaussian statistics as found in the experimental study of \cite{derks_suppression_2006}. The agreement with this experimental study also extends to qualitative agreement with the increase of the effective surface tension in the direction of driving and also an increase in the correlation length of the height fluctuations with respect to a non-driven equilibrium interface. However, we  note that numerical simulations of a sheared Ising interface \cite{smith_interfaces_2008-1,smith_lateral_2010} also reveal a reduction of interface fluctuations but the lateral correlation length is found to be reduced.

The basic idea underlying this study would be interesting to apply to a number of possible variants of this model, for instance both the dynamics
of the main field $\phi$ and the solvent field $\phi$ could be varied. To make a direct link with driven colloidal interfaces one should study model H type dynamics for the main field $\phi$ and other variants for the dynamics of the 
solvent field $\phi$ could also be considered. 

As mentioned above, in lattice based models driving induces non-equilibrium states even in the simple Ising lattice gas. A model analogous to that studied here can be formulated in a lattice based systems using the Hamiltonian 
\begin{equation}
    H = -J\sum_{(ij)}S_i S_j (1+ \sigma_{(ij)}),
\end{equation}
where $S_i=\pm1$ are Ising spins at the lattice sites $i$, and $\sigma_{(ij)}=\pm 1$ are Ising like dynamical solvent variables associated with the lattice links $(ij)$. The static partition function is given by
\begin{equation}
    Z = {\rm Tr}_{\sigma_{ij},S_i} \exp\left[\beta J\sum_{(ij)}S_i S_j (1+ \sigma_{(ij)})\right],
\end{equation}
and the trace over the solvent variables can be trivially carried out to give
\begin{equation}
    Z = {\rm Tr}_{S_i}\left( \exp\left[\beta J\sum_{(ij)}S_i S_j \right]\prod_{(ij)}2\cosh(\beta JS_iS_j)\right )= [2\cosh(\beta J)]^L{\rm Tr}_{S_i}\exp(\beta J\sum_{(ij)}S_i S_j ),
\end{equation}
where $L$ is the number of links on the lattice of the model. We thus see that the underlying effective static model is precisely the zero field Ising model. 

This model can then be driven in a number of ways, for instance using conserved Kawasaki dynamics for the Ising spins to model diffusive dynamics in the presence of a uniform driving field parallel to the surface between the two phases at a temperature below the ferromagnetic ordering temperature $T_c$. The dynamics of the Ising spins on the lattice links can  be given by non-conservative single spin flip, for instance Glauber, dynamics to keep the analogy with the continuum model discussed in the paper but diffusive dynamics or indeed a mixture of diffusive and non-conserved dynamics 
could be implemented. It would be interesting to see to what extent this modification of the driven lattice gas model affects the non-equilibrium driven states that arise. 

It is also clear that this lattice model can be used to simulate the effect of activity where the Ising spins $S_1$ corresponding to the colloid field undergo  Kawasaki dynamics at the temperature $T_1$ where as the link variables $\sigma_{(ij)}$ undergo single spin flip non-conserved dynamics
at the temperature $T_2$.   

%%%%%%%%%%%%%%%%%%
    \section{Driven SOS model}
%%%%%%%%%%%%%%%%%%

In the previous section, coupling of a model B field with a model A one was done in order to break the Galilean transformation, leading to out-of-equilibrium steady state. In lattice based numerical simulations, the invariance is broken because of the discrete-time nature of the algorithm.
In a SOS model under Kawasaki dynamics, the implementation of a constant driving flow is as follows. From the configuration $C$ we choose a configuration $C'$ as explained is Sec \ref{algo-kawasaki}, meaning we chose a random site $i$ and a nearest neighboor $i\pm1 $ to which give one of its particles. Under a flow, the difference of energy between both states is
\begin{align}
\Delta E_d = \Delta E_{eq} \pm v
\end{align}
where $v$ is the intensity of the drive, and the sign depends on the direction of the flow. For example, if the flow goes to the right, then every configuration which moves a particle to the right will have an additional energy $+v$, while if the particle goes against the flow, it will have an additional energy $-v$.

Implementing a shear $v |L/2-y|$ as in Fig \ref{snap-ising-shear} is tricky, because it requires to know the height of the particle and thus have access to bulk information. In SOS models, we consider that only the particle at the interface moves and changes height accordingly to the interface's height of the neighboring site. The vertical movement of the particle, couple to the horizontal one, is what makes the SOS model different to the Ising one, and physical arguments forbids the use $h_i$, $h_{i+1}$ or even the average $\frac{h_i+h_{i+1}}{2}$ as the shear contribution might be zero depending of the configuration, even though it should always be present. 

Under periodic boundary conditions, the direction of the flow should not alter the steady state. The average total energy has thus to be an even function with respect to the drive $v$, ie
\begin{align}
    <E(v)> = <E(-v)>
\end{align}
and same goes for the surface tension $\sigma(v)$.
From Eq \eqref{CHST} and \eqref{profil-interface-glauber}, we have the surface tension
\begin{align}
    \sigma(v) = \int_{-\infty}^{\infty} dz J\  \left( \frac{d \tanh \left(\frac{z}{\xi_\perp(v)} \right)}{dz} \right)^2 = \int_{-\infty}^{\infty} dz  J \frac{1}{\cosh^4(\frac{x}{\xi_\perp(v)}) \xi_\perp^2}
    \label{int-sigma}
\end{align}
where $\xi_\perp(v)$ is the interfacial correlation length. This interface width has the same qualitative behaviour as the interface width $<w> = \sqrt{<h^2>-<h>^2}$. 

In Fig \ref{fig-correl-drive} we show the interface width and the surface tension with respect to the drive computed through Kawasaki dynamics. The first thing we notice is that the interface width increases with the imposed driving in an almost linear way, as does the mean height of wind generated waves with respect to the wind surface velocity \cite{maat_roughness_1991}.
The second thing we notice is that there are a change of regime at $v=2J$ and $v=4J$, with a net change in the derivative. The change of energy $\Delta E$ has values in $[-4,-2,0,2,4]$, so we see that there are three regimes : the weak driving regime for $v \less 2J$ , the middle regime for $2J \less v \less 4J$ where some moves with adding energy to the system are always accepted, and the strong driving regime for $v \greater 4J$ where all moves adding energy are accepted, ending with a saturation when the bond energy becomes negligible with respect to the driving. 
 
While this result is in direct contradiction with Eq. \eqref{1dpot} in the model C and Ising simulations \cite{smith_interfaces_2008-1,smith_lateral_2010} where the surface tension increases with $v$, we provide no formal physical theory explaining why there is a diference between those two interface models. 

\begin{figure}
    \centering
    \includegraphics[width=0.7\linewidth]{drivenC/tension-drive.pdf}
    \caption{Interface width $w=\sqrt{<h^2>-<h>^2}$ and surface tension $\sigma$ computed from the integral \eqref{int-sigma} with respect to the drive $v$,  with $L'=256$, $L=200$, $<h> = 4.5$, and $\beta = J = 1$ for $5 \cdot 10^7$ MC steps.}
    \label{fig-correl-drive}    
\end{figure}

%%%%%%%%%%%%%%%%%%%
%    \section{Driven POP model}
%%%%%%%%%%%%%%%%%%%
%
%    \subsection{Conserved diffusive dynamics}
%%%%%%%%%%%%%%%%
%We now study a system of two particle types $A$ and $B$, where one is a a solvent and the other one a particle in suspension, for example.
%Assuming Brownian dynamics for the particles, the stochastic density functional equations for the continuum fields $n_A$ and $n_B$ are
%given by
%\begin{equation}
%\frac{\partial n_A(x)}{\partial t} = \frac{\partial}{\partial x} \beta D_A n_A(x)\frac{\partial}{\partial x} \frac{\delta H}{\delta n_A(x)} + \frac{\partial}{\partial x} \sqrt{2D_A n_A(x)} \eta_A(x,t)
%\end{equation}
%and 
%\begin{equation}
%\frac{\partial n_B(x)}{\partial t} = \frac{\partial}{\partial x} \beta D_B n_B(x)\frac{\partial}{\partial x} \frac{\delta H}{\delta n_B(x)} + \frac{\partial}{\partial x} \sqrt{2D_Bn_B(x)} \eta_B(x,t)
%\end{equation}
%where $D_A$ and $D_B$ are the diffusion constants of the particles. The noise terms are independent , zero mean, spatiotemporal Gaussian white noise with
%\begin{eqnarray}
%\langle \eta_A(x,t) \eta_A(x',t')\rangle = \langle \eta_B(x,t) \eta_B(x',t')\rangle =\delta(x-x')\delta(t-t').
%\end{eqnarray}
%As we are interested in what happens when one of the species is driven, we add a term
%\begin{equation}
%H_D = -\int dx xf n_A(x)
%\end{equation}
%to the Hamiltonian, this corresponds to a force which pushes the particles of type $A$ to the right. We also assume periodic boundary conditions, and thus a current will exist in the resulting steady state.
%This introduces term in the equation for $n_A(x)$ which becomes
%\begin{equation}
%\frac{\partial n_A(x)}{\partial t}+ \frac{\partial}{\partial x}D_A\beta f n_A(x) = \frac{\partial}{\partial x} \beta D_A n_A(x)\frac{\partial}{\partial x} \frac{\delta H}{\delta n_A(x)} + \frac{\partial}{\partial x} \sqrt{2D_A n_A(x)} \eta_A(x,t).
%\end{equation}
%It is important to note that in the absence of the particles of type $B$, the resulting equation for the field $n_A(x,t)$ can be rendered independent of
%the force $f$ via the Galilean transformation
%\begin{equation}
%n(x,t) = n(x-vt,t)
%\end{equation}
%where $v= D_A\beta f$ is the induced drift on the particles of type $A$.
%Note that the Galilean invariance can also be broken if the force $f$ acts on both particle types but the diffusion coefficients $D_A$ and $D_B$ are different.
%
%We now expand the deterministic part of the two equations to first order in the density fluctuations about its mean value and the noise terms to zeroth order. This approximation respects detailed balance for the effective quadratic Hamiltonian and has been used with accuracy in a wide variety of contexts. The resulting dynamics of of a model B and Fourier transforming in space gives
%\begin{equation}
%\frac{\partial \tilde{\boldmath \Phi}(k,t)}{\partial t}= -\beta \tilde A(k){\boldmath \Phi}(k,t) + {\boldmath \tilde\eta}(k,t),
%\end{equation}
%where
%\begin{equation}
%{\boldmath \Phi}(k,t) =\begin{pmatrix} &\tilde \phi_A(k,t) \\ &\tilde \phi_B(k,t)\end{pmatrix}
%\end{equation}
%The noise correlation function is given by
%\begin{equation}
%\langle {\tilde\boldmath\eta}^{T}(k,t) {\tilde\boldmath{\eta}}(k',t')\rangle = 4\pi\tilde R(k) \delta(t-t')\delta(k+k')
%\end{equation}
%where 
%\begin{equation}\tilde R(k)= 2\begin{pmatrix} & D_A\overline n_A k^2 & 0\\
%& 0 & D_B\overline n_B k^2\end{pmatrix},
%\end{equation}
%and 
%\begin{equation}
%\tilde A(k) =\sigma\begin{pmatrix}& D_A\overline n_A k^2(k^2 + m_A^2) -i\frac{D_A k f}{\sigma} & 
%D_A\overline n_A k^4 \\ & D_B\overline n_B k^4 & D_B \overline n_B k^2(k^2 + m_B^2)\end{pmatrix}
%\end{equation}
%The Fourier transform steady state correlation function matrix defined by
%\begin{equation}
%\langle {\boldmath\tilde\Phi}^T(k){\boldmath\tilde\Phi}(k')\rangle =
%2\pi \delta(k+k') \tilde C(k)
%\end{equation}
%is then given by the solution to the Lyapounov equation
%\begin{equation}
%\tilde A(k)\tilde C(k) + \tilde C(k)\tilde A^T(-k) = 2T \tilde R(k)
%\end{equation}
%Solving this we find that
%\begin{equation}
%\tilde C_{hh}(k)= \frac{T}{\sigma}\frac{k^2(m_A^2+m_B^2)(D_A\overline n_A[k^2+m_A^2] +D_B\overline n_B[k^2+m_B^2] )^2 +\frac{f^2 }{\sigma^2}D_A^2(2k^2+m_A^2 + m_B^2)}{k^2(D_A\overline n_A[k^2+m_A^2] +D_B\overline n_B[k^2+m_B^2] )^2(m_A^2 m_B^2 + k^2(m_A^2 +m_B^2))
%+\frac{f^2 }{\sigma^2}D_A^2(k^2+m_A^2)(k^2+m_B^2)}.
%\end{equation}
%In the equilibrium or non-driven system where $f=0$ the above formula
%yields the static result Eq. (\ref{stat}). Of particular interest is the strong driving limit where we find that as $f\to\infty$ the result
%\begin{equation}
%\tilde C_{hh}(k)=\frac{T}{\sigma}\left[\frac{1}{k^2+m_A^2}
%+\frac{1}{k^2+m_B^2}\right]
%\end{equation}
%The effect of strong  driving is to decouple the fluctuations of $n_A$ and $n_B$ and we see that the total height fluctuation is that of the sum two independent interfaces. The height variance is then given by
%\begin{equation}
%\langle h^2\rangle_s = \frac{T}{2\sigma m_d}
%\end{equation}
%where 
%\begin{equation}
%m_d = \frac{m_A m_B}{m_A+m_B}
%\end{equation}
%From this we find that
%\begin{equation}
%\frac{\langle h^2\rangle_s }{\langle h^2\rangle_{eq}}=\frac{m_A+m_B}{\sqrt{m_A^2 + m_B^2}}.
%\end{equation}
%We also see that, as $m_A$ and $m_B$ are positive, the fluctuations in limit of infinite driving are always larger than in the equilibrium state. 
%
%
%Imagine now that the $B$ particles constitute the bottom layer and that $D_B\ll D_A$ this mimics the solid on solid dynamics where only the top layer of the particles participate in the dynamics. This gives
%\begin{equation}
%\tilde C_{hh}(k)= \frac{T}{\sigma}\frac{k^2(m_A^2+m_B^2)(\overline n_A[k^2+m_A^2] )^2 +\frac{f^2 }{\sigma^2}(2k^2+m_A^2 + m_B^2)}{k^2(\overline n_A[k^2+m_A^2]  )^2(m_A^2 m_B^2 + k^2(m_A^2 +m_B^2))
%+\frac{f^2 }{\sigma^2}(k^2+m_A^2)(k^2+m_B^2)}.
%\end{equation}
%Assuming the $A$ layer is thinner that the $B$ layer so $m_B\ll m_A$ gives
%\begin{equation}
%\tilde C_{hh}(k)= \frac{T}{\sigma}\frac{k^2 m_A^2(\overline n_A[k^2+m_A^2] )^2 +\frac{f^2 }{\sigma^2}(2k^2+m_A^2 )}{k^2(\overline n_A[k^2+m_A^2]  )^2(m_A^2 m_B^2 + k^2 m_A^2)
%+\frac{f^2 }{\sigma^2}(k^2+m_A^2)(k^2+m_B^2)}.
%\end{equation}
%which simplifies to give
%\begin{eqnarray}
%\tilde C_{hh}(k)&=& \frac{T}{\sigma}\frac{k^2 m_A^2(\overline n_A[k^2+m_A^2] )^2 +\frac{f^2 }{\sigma^2}(2k^2+m_A^2 )}{(k^2+m_A^2)(k^2 + m_B^2)[(n_A^2m_A^2 k^2(k^2+m_A^2)
%+\frac{f^2 }{\sigma^2}]} \\
%\tilde C_{hh}(k)&=& \frac{T}{\sigma}\frac{1}{k^2 + m_B^2}
%+\frac{T}{\sigma}\frac{\frac{f^2 }{\sigma^2}k^2}{(k^2+m_A^2)(k^2 + m_B^2)[(n_A^2m_A^2 k^2(k^2+m_A^2)
%+\frac{f^2 }{\sigma^2}]} 
%\\
%&=&\frac{T}{\sigma}\left[\frac{1}{(k^2 + m_B^2)}
%+\frac{ f^2 }{\sigma^2\overline n_A^2 m_A^2}\frac{k^2}{(k^2+m_A^2)(k^2 + m_B^2)(k^4+m_A^2 k^2 +\frac{f^2 }{\sigma^2n_A^2 m_A^2})}\right]
%\end{eqnarray}
%The height fluctuations are then given by
%\begin{equation}
%\langle h^2\rangle = \frac{T}{2\sigma m_B} +  \frac{T}{2\sigma}\frac{ f^2 }{\sigma^2\overline n_A^2 m_A^2}\frac{m_A + m_B+ m_++ m_-}{(m_A+m_B) (m_++m_B)(m_-+m_B)(m_A+m_+)(m_A+m_+)(m_++m_-)},
%\end{equation}
%where 
%\begin{equation}
%m_\pm = \sqrt{\frac{m_A^2\pm\sqrt{m_A^4 -\frac{4f^2 }{\sigma^2n_A^2 m_A^2}}}{2}}
%\end{equation}
%
%%%%%%%%%%%%%%%%
%    \subsection{Non conserved dynamics}
%%%%%%%%%%%%%%%%
%We now consider the case where the particles of type $B$ are in contact with 
%a reservoir of the same particles in a vapour phase. To model this we modify the dynamics of the B phase by introducing a component of non-conserved dynamics for these particles
%\begin{equation}
%\frac{\partial n_B(x)}{\partial t} = \frac{\partial}{\partial x} \beta D_B n_B(x)\frac{\partial}{\partial x} \frac{\delta H}{\delta n_B(x)} + \frac{\partial}{\partial x} \sqrt{2D_Bn_B(x)} \eta_B(x,t)
%-K_B\beta \frac{\delta H}{\delta n_B(x)} + \sqrt{2K_B}\eta'_B(x,t),
%\end{equation}
%here if $\eta'_B(x,t)$ is a new spatio-temporal white noise independent of the 
%others, the undriven system obeys detailed balance.  Now the average value of the $n_B$ is determined by taking the average in the steady state. As the system is invariant under translation, the average of the first diffusive term on the right-hand-side is zero and so we find
%\begin{equation}
%\langle \frac{\delta H}{\delta n_B(x)}\rangle=0,
%\end{equation}
%where the averaging is over the system in the steady state. Again invariance by translation in space can be applied to write
%\begin{equation}
%\langle \frac{\partial V(n_A,n_B)}{\partial n_A}\rangle  =0
%\end{equation}
%Here we have $V(n_A,n_B)= U(n_A) + U(n_B)$ where $U(x) = Tx(\ln(x)-1)-\mu x$ and so expanding about $\overline n_A$ we find to second order that 
%\begin{equation}
%\langle U'(\overline n_A) + U''(\overline n_A)\phi_A +\frac{1}{2}U'''(\overline n_A)\phi_A ^2 \rangle =0,
%\end{equation}
%the equation to first order gives 
%\begin{equation}
%U'(\overline n_A)=0
%\end{equation}
%which gives $\overline n_A= n_m$ where $n_m$ is the value for which $U$ attains its minimum. However if we keep the next order term we find 
%\begin{equation}
%U'(\overline n_A)+\frac{1}{2} U'''(\overline n_A)\langle\phi_A ^2 \rangle=0.
%\end{equation}
%If the renormalization of the average value of $\overline n_A$ is assumed to be small we can write
%\begin{equation}
%\overline n_A= n_m +\delta,
%\end{equation}
%which gives
%\begin{equation}
%\delta = -\frac{1}{2} \frac{U'''(n_m)}{U''(n_m)}\langle\phi_A ^2 \rangle.
%\end{equation}
%For the (entropic) potential in question here we have
%\begin{equation}
%\delta = \frac{1}{2n_m} \langle\phi_A ^2 \rangle,
%\end{equation}
%we thus see that the average height of the interface is increased due to fluctuations. The non-equilibrium fluctuations are stronger an thus the height increases under driving. 


%%%%%%%%%%%%%%%%
    \section{Conclusions}
%%%%%%%%%%%%%%%%

We have present two models to analyse the effect of uniform driving on the dynamics of the interface in a two phase system. Using dynamics derived from model B\cite{bray_interface_2001,bray_interface_2001-1}, we added a coupling field from model A in order to remove galilean invariance in the equations, and found that that the driving does increase the effective surface tension in the direction of driving and increases the correlation length along the interface with respect to a non-driven equilibrium interface. This work resulted in a published article \cite{dean_effect_2020}.
We also studied the effect of driving in SOS models, and found that contrary to Ising systems \cite{smith_interfaces_2008-1} and model C systems, the driving does reduce the surface tension. While no physical argument is provided as to why there is such a qualitative difference between models, it is in agreement with wave generated waves phenomenon \cite{maat_roughness_1991}.
%\input{intro/conclusion}


%\usepackage{biblatex}
\bibliographystyle{unsrt}
\bibliography{biblio}
\end{document}
