\chapter*{Introduction}
\addcontentsline{toc}{chapter}{Introduction} 


Chaque système statistique est décrit par un paramètre d'ordre, que ce soit la magnétisation moyenne d'un milieu aimanté ou de l'orientation moyenne des polymères par exemple. Dans ces systèmes, on appelle phase un milieu homogène selon le paramètre d'ordre. Les transitions de phase d'un système, où une phase homogène se sépare en plusieurs phases différentes, possèdent des propriétés exceptionnelles. Lors d'une transition de phase continue, la largeur de l'interface entre deux phases diverge jusqu'à atteindre une taille macroscopique, atteignant ainsi les bords de la cellule expérimentale. Ce confinement de l'interface mène a des effets de taille finie, notamment l'effet Casimir critique.

L'étude des propriétés statistiques des interfaces peut se faire via différentes approches, toutes complémentaires. Les modèles sur réseau, et plus préciseément le modèle d'Ising, sont particulièrement adaptés aux simulations numériques, mais plus difficiles à traiter analytiquement à cause du trop grand nombre de degrés de liberté présents dans le système. Une manière de diminuer la quantité de degrés de liberté est de réduire la dimensionalité du système à une dimension, ce qui nous permet d'utiliser le formalisme des matrices de transfert. À cet effet, le modèle Solid-On-Solid a été énormément étudié dans les annés 80-90 pour sa simplicité.
L'interface peut également être assimilée à un marcheur brownien qui, au lieu de bouger dans le temps, se meut dans l'espace. Ainsi, les équations de Schrödinger et les équations du mouvement de Langevin permettent de décrire les fluctuations d'une interface grâce au formalisme quantique ou stochastique. Cette méthode s'appelle la théorie des ondes capillaires et permet la résolution d'un système très analogue aux modèles SOS, en utilisant un formalisme très similaire aux matrices de transfert. 
La dernière méthode que nous aborderons dans cette thèse est celle du champ moyen, c'est-à-dire l'étude des propriétés de deux phases grâce aux équations de Landau-Ginzburg. Cette méthode a l'avantage d'offrir des calculs analytiques relativement faciles et permet d'obtenir la forme des fonctions de corrélation à plusieurs point ainsi que les longueurs de corrélation. Néanmoins, la vérification des résultats via les simulations numériques, qui nous permettrait de mettre des grandeurs mésoscopiques comme la tension superficielle ou la longueur capillaire en relation avec les grandeurs microscopiques du modèle d'Ising, est asez difficile, et nous éviterons dans la présente thèse le rapprochement.

L'étude de l'effet Casimir critique - qui est un effet de taille finie - revient au final à étudier les propriétés statistiques d'une interface, et à voir comment elles sont modifiées lorsqu'il existe des conditions aux bords. Cependant, les propriétés d'une interface varient également lorsqu'elle est mise hors-équilibre, par exemple via une force du style cisaillement, qui représente beaucoup de cas expérimentaux classiques. 
Nous nous intéressons ici particulièrement à la différence entre les états d'équilibre et hors-équilibre, pour lesquels les formalismes sont différents mais dont les simulations numériques sont similaires. La présence d'un système hors-équilibre pose également plein de questions sur la nature de l'ensemble thermodynamique que l'on se place, et nous éclaire sur les différences entre l'ensemble canonique et grand-canonique, et également entre des particules discernables et indiscernables.

Le manuscrit se décompose de la manière suivante :
\begin{itemize}
    \item Le premier chapitre introduit les différents approches historiques sur l'étude des interfaces, en s''attardant sur les principaux résultats obtenus dans la littérature pour des interfaces à l'équilibre, puis hors-équilibre
    \item Le second chapitre introduit le modèle Solid-On-Solid, qui est une approximation 1D à très basse température du modèle d'Ising en 2D. Nous y parlerons du formalisme des matrices de transfert, des principaux résultats obtenus dans ce modèle et de quelques précisions fondamentales sur les différents ensembles thermodynamique sdans lesquelles ont peut étudier nos systèmes
    \item Nous présentons dans le troisième chapitre l'alogirthme de Monter Carlo-Metropolis, un outil puissant pour explorer l'espace des phases et calculer numériquement la fonction de partition de nos systèmes
    \item Dans le quatrième chapitre, l'étude d'un système SOS - analogue à la croissance d'un cristal -  via une méthode d'intégration sur les potentitels chimiques nour permet d'obtenir l'énergie libre, et ainsi l'effet Casimir. Cette étude se termine par l'ajout du cisaillement.
        \item Dans le cinquième chapitre, nos étudions un modèle avec un champ magnétique charactéristique des expériences dans lesquelles on force une phase d'un fluide binaire dans une autre grâce à une pression de radiation exercée par un laser
    \item Un nouveau modèle découlant des considérations du second chapitre peut être créé de la même manière que le modèle SOS, en prenant en compte l'entropie. Ce nouveau modèle, baptisé Particles-Over-Paticle, fait l'objet du sixième chapitre
    \item Le septième chapitre reprend les calculs d'une publication récente de notre équipe sur un modèle de champ moyen où l'on mélange deux types de particules appartenant à des ensembles thermodynamiques différents, sous l'effet d'un cisaillement uniforme
\end{itemize}

La présente thèse a été rendue possible grâce à l'ANR PHYSICS, le Laboratoire Onde Matière d'Aquitaine de l'Université de Bordeaux, le Laboratoire de Physique de l'ENS Lyon et le Mésocentre de Calcul Intensif d'Aquitaine sur lesquelles ont été faites les simulations numériques. Je remercie particulièrement Josiane Parzych (LOMA) et Laurence Mauduit (ENS LYON) pour le suivi administratif, ainsi que Nguyen Ky Nguyen (MCIA) pour l'aide technique.
