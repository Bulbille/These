\chapter*{Introduction}
\addcontentsline{toc}{chapter}{Introduction} 

Most statistical mechanical  models can be  described using an order parameter, such as the mean magnetization in a magnetic system, the density of a liquid or gas or the average orientation of liquid crystals [R]. During a continuous phase transition, the correlation length diverges [R]. When the correlation length becomes of the same order of magnitude as the experimental or numerical system size, finite size effect[R] become stronger, leading to interesting new physics, for example the  critical Casimir effect [R].

In a system where the order parameter is conserved, or when it is fixed at values corresponding to two different phases, an interface is formed between the two phases.
In systems with a rectangular geometry this interface is at a first order of approximation flat as the system tends to minimise the area between the two phases to minimize the interfacial energy generated by the surface tension. However thermal fluctuations induce fluctuations of the interface. From a theoretical point of view, one can study the statistical properties, both static and dynamic, of interfaces between two phases through different though complementary methods. Historically, the first method was through lattice models, and more precisely the Ising model. Those models are well-suited for numerical analysis due to their discrete nature, while posing analytical challenges due to the large number of degrees of freedom. Analytic studies of the interface are limited to two dimensional systems [R-cite a paper by Douglas Abrahams on the two d Ising mode]. Numerical simulations of phase separated Ising models are straightforward, however the identification of the interface and even an unambiguous definition of the position of the interface are not obvious. For this reason coarse grained models of the interface have been proposed.

For instance, the Solid-On-Solid model [R] is an approximation of the Ising model in $d$ dimensions which describes the position of an interface under certain simplifying circumstances. For two dimensional Ising systems this means that the transfer matrix method can be applied either analytically or numerically and the results are directly comparable with numerical simulations.
Coarse graining the Ising model leads the Landau-Ginzburg Hamiltonian description in terms of a continuous field [R]. Using this description on can derives approximate descriptions for both the dynamics and statics of the interface. In particular within this formalism the effective partition function of the interface describes a random walker in a potential, then using the Feynman-Kac formula [R] two dimensional systems can be analysed using a quantum mechanical treatment.

When modeling experiments using statistical mechanics it is important to identify the correct thermodynamic ensemble to describe the system. For instance in an Ising spin which describes interacting spins with $s=\pm 1$ the total magnetisation is not necessarily conserved as a single spin can change its sign. However if the spins are regarded as fixed but can exchange with each other by swapping sites (to describe a lattice gas for example), the overall all magnetization is fixed. In the latter case the magnetization is thus fixed and so the two ensembles are clearly different. It is thus interesting to study how ensemble differences in the underlying lattice model affect the statics and dynamics of interfaces in these models.


The thesis' outline is as following :
\begin{itemize}
    \item In the first chapter we discuss interface. We explain how there statics can be modeled in terms of various discrete and continuous interface models. In particular we review phase ordering kinetics for the evolution of the order parameter in coarse grained systems. From this one can deduce using a method introduced by Bray and coworkers [R] to derive the equilibrium dynamics of the interface. We then discuss discrete models and discuss the links between Ising and lattice gas models and the appropriate thermodynamic ensembles. The approximate description of these systems in terms of Solid-On-Solid type models is then addressed as well as their solution via the transfer matrix method. Finally we discuss the case of interfaces which are driven and whose steady states are non-equilibrium ones, giving some examples of numerical simulations and experimental results which motivate our later study of driven interface models for model C type systems.
    \item In the second chapter we discuss the basis of numerical simulations, in particular how to compute the free energy for equilibrium systems. We also discuss some numerical tips we have found particularly useful
    \item The third chapter is devoted to finite size effects for various models. A particularly interesting manifestation of finite size effects is the so called Casimir interaction [R cite a book on the Casimir effect]. For completeness we describe the original quantum calculation of Casimir for two perfectly conducting plates at zero temperature [R- cite Casimir]. Then we discuss the generalization to arbitrary  dielectric materials,the so called Lifshitz theory [R]. We then explain the critical Casimir effect which was first predicted by Fisher and de Gennes, this due to thermal, rather than quantum, fluctuations in critical or near critical systems. A natural question to ask is whether interface models can capture the same finite size scaling as predicted by Fisher and de Gennes. To this end we analyse finite size effects for an number of interface models for two dimensional systems. First we analyse continuous elastic line models where the underlying interface is described by a Brownian motion in an external potential. These models have been studied extensively in the literature but we find a number of new results. First we consider an elastic surface (corresponding to the Edwwards-Wilkinson model) confined between two hard walls. The resulting free energy is well known and does not correspond to what is expected from the critical Casimir effect. However we show that by including a phenomenological finite size correction to the surface tension (or line stiffness) proposed by Privman [Cite the Privman paper I mention in the chapter] one can recover the quantitative form predicted by the critical Casimir effect. 
We also derive the statistics of the equilibrium interface, in particular the fluctuations of the integrated height (corresponding to the average magnetisation in a spin model). 

Then we analyse the so called Airy line, corresponding to an interface above a hard wall but with a linear potential (corresponding to an applied pressure in the constant pressure ensemble) pushing the interface toward the wall.

    Finally we examine finite size effects in the Solid-on-Solid model. These were first analysed in [R-the first Privman paper], however we give an alternative derivation which allows one to compute the correlation length and also show that for large systems the physics is essentially the same as for continuum Brownian models and determine the effective surface tension.
    
   
    \item The fourth chapter is essentially a published paper \cite{dean_effect_2020}. 
    It considers an interface model which can be driven out of equilibrium by a uniform driving field. In order for a uniform field to have an effect on the system we introduce two fields, a colloidal field with model B conserved dynamics and a solvent field with model A dynamics. This combination of dynamics is called model C dynamics [Ref to a phase ordering paper]. The system is driven out of equilibrium by assuming that the driving acts only on the colloidal field. Using the method of Bray et al [R] we derive the interface dynamics and compute its correlation function in the resulting non-equilibrium steady state. The properties are considerably different to the standard capillary wave properties of equilibrium surfaces and some of the results can be interpreted in terms of an effective surface tension. The calculations exhibit some of the phenomenology seen in experiments on sheared interface [R Derks]. Finally we discuss the same model without driving but with different temperatures for the two fields to simulate active colloid systems, again a rich phenomenology of non-equilibrium states emerges.
    \item In the last chapter we introduce a new lattice model which is a better approximation to the Ising model than the Solid-On-Solid model. This new model, the Particles-Over-Particles model, takes into account the entropy, in comparison to SOS.
\end{itemize}

This thesis has been possible thanks to the ANR's grant FISICS, the Laboratoire Onde Matière d'Aquitaine from Université de Bordeaux, and the Laboratoire de Physique  from ENS Lyon. The numerical simulations benefited from the numerical resources of the Mésocentre de Calcul Intensif Aquitain \cite{noauthor_mesocentre_nodate}, with the help of Nguyen Ky Nguyen. I also wish to thank Josiane Parzych (LOMA) and Laurence Mauduit (ENS LYON) for all the administration procedures.
