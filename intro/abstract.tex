{\bf Effets d'écoulements et de confinement dans les modèles discrets et continus d'interfaces} \\

Cette thèse examine les propriétés de l'interface entre deux phases dans un système avec des phases séparées. Nous regardons comment les effets de taille finis modifient les propriétés statistiques de ces interfaces, en particulier comment la dépendance de l'énergie libre par rapport à la taille du système donne lieu à des interactions de Casimir critique à longue portée proche du point critique.
Souvent, les interfaces sont décrites par des modèles simplifiés ou coarse-grained dont les seuls degrés de liberté sont les hauteurs de l'interface. Nous rappelons comment les propriétés statiques et dynamiques de ces interfaces sont retrouvées à partir de modèles microscopiques de spins et de théorie statistique des champs.
Nous étudions ensuite les effets de taille finie pour les interfaces continues comme le modèle Edwards-Wilkinson ou discrètes comme le modèle Solid-On-Solid, et discutons leur pertinence dans le cadre de l'effet Casimir critique.

Dans la seconde partie de la thèse, nous examinons des modèles d'interfaces sous écoulement qui possède des états stationnaires hors-équilibres. Nous développons 
ces équations dans le cadre du modèle C d'une interface, qui a un état stationnaire hors-équilibre lorsque soumis à un écoulement uniforme. L'état stationnaire hors-équilibre résultant exhibe des propriétés retrouvées dans les expériences sur des colloïdes sous cisaillement, notamment la suppression des fluctuations de la hauteur de l'interface et une augmentation de la longueur de corrélation des fluctuations. 
Finalement, nous proposons un nouveau modèle pour des interfaces uni-dimensionnelles qui est une modification du modèle Solid-on-Solid qui contient un terme supplémentaire d'entropie, ce qui le rend plus approprié pour l'étude de la diffusion et du cisaillement.
\\ \\
{\bf Keywords : } Modèles d'interface, Théorie d'ordonnancement des phases, Modèle d'Ising, Modèle Solid-On-Solid, Force de Casimir, Écoulements stationnaires hors-équilibre



{\bf Confinement and driving effects on continuous and discrete model interfaces}\\
This thesis examines the properties  of the interface between two phases in  phase separated systems. We are interested 
in how finite size effects modify the statistical properties of these interfaces, in particular  the dependence of the free energy on the system size 
gives rise to long range critical Casimir forces close to the critical point. Often the interfaces in phase separated systems are
described by simplified or coarse grained models whose only degrees of freedom are the interface height. We review how the statics and dynamics of
these  interface models can be derived from microscopic spin models and statistical field theories. We then examine finite size effects for 
continuous interface models such as the Edwards Wilkinson model and discrete models such as the Solid-On-Solid model  and discuss their 
relevance to the critical Casimir effect. In the second part of the thesis we examine models of driven interfaces which have nonequilibrium steady states. 
We develop a model C type model of an interface which shows a nonequlibrium steady
state even with constant driving. The resulting nonequlibrium steady state shows properties seen in experiments on sheared colloidal systems,
notably the suppression of height fluctuations but   a increase in the  correlation length of the fluctuations. Finally we propose a new model for
one dimensional interfaces which is a modification of  the solid-on-solid model and which contains extra entropic terms which make it more
appropriate to study diffusive dynamics and driving.
\\ \\
{\bf Keywords : }  Interface models, phase ordering dynamics, Ising model, Solid-On-Solid model, Casimir force, driven nonequilibrium steady states

\cleardoublepage
{\bf \huge Résumé en français} \\

La majorité des systèmes statistiques peuvent être décrits par un paramètre d'ordre, comme la magnétisation moyenne dans les systèmes magnétiques, la densité dans un fluide ou l'orientation moyenne des polymères dans les cristaux liquides. Les propriétés statiques et dynamiques de tels systèmes est bien décrite par la théorie statistique des champs. Dans cette théorie, le champ $\phi$ est soumis à un hamiltonien $H(\phi)$, et ses propriétés peuvent être dérivée de la fonction de partition $Z$.
Lorsque ces systèmes possèdent une transition de phase au point critique, il y a une discontinuité de l'énergie libre dûe à la modification des micro-configurations possibles. 
Dans les systèmes magnétiques on passe ainsi d'une magnétisation nulle à une magnétisation finie, les liquides qui étaient auparavant mélangés se séparent, et les polymères adoptent une direction moyenne commune. Dans la phase ordonnée on retrouve alors des composantes connexes où la valeur de $\phi$ est quasi-constante. Entre ces composantes connexes se situe l'interface entre les phases. 
Dans cette thèse, nous nous intéressons particulièrement aux propriétés statiques et dynamiques des interfaces dans des modèles continus et discrets avec des contraintes telles que le confinement ou le cisaillement. \\


{\bf \large Description d'une interface} \\

La première partie de cette thèse est consacrée à la description des interfaces à partir de la théorie statistique des champs. L'utilisant un champ externe $\phi^4$ dans l'Hamiltonien induit une séparation de phase en dessous de la température critique, ce qui donne une interface dont le profil peut-être calculé grâce aux équations dynamiques du modèle A dans le cas stationnaire. L'énergie libre associée à cette interface est directement reliée à la tension superficielle $\sigma$ par la relation de Cahn-Hilliard.
Depuis les équations de théorie statistique des champs, en considérant une interface d'épaisseur nulle, on retrouve les équations d'Edwards-Wilkinson pour le modèle A et le modèle B. La réduction du nombre de degrés de libertés par cette transformation d'un volume en surface permet leur étude grâce à des intégrales de chemin, qui seront détaillées plus tard.

Ensuite nous expliquons le modèle le plus simple sur réseaux, le modèle d'Ising, qui décrit un réseau orthogonal de spins avec une interaction uniquement entre les plus proches voisins et un champ externe. Ce modèle, où chaque site du réseau prend la valeur $\sigma_1 = \pm1$, permet également de modéliser un système de gaz sur réseau ou de liquides binaires en faisant un changement de variable sur $\sigma_i$. En appliquant la même approximation d'interface d'épaisseur nulle que précédement, on trouve cette fois-ci le modèle Solid-On-Solid. La fonction de partition SOS a l'avantage d'être diagonalisée grâce à la matrice de transfert, dont le formalisme est développé. Dans la limite thermodynamique, seul l'état fondamental (qui possède la plus grande valeur propre de la matrice de transfert) est nécessaire pour obtenir toutes les observables du système. 

Dans ce premier chapitre, une grande importance est également donnée quant à la différence entre les différents ensembles statistiques. La complexité de l'ensemble canonique provient de la contrainte sur la hauteur totale de l'interface qui ne peut être intégré au formalisme des matrices de transfert. Pour étudier ces différences il faut donc utiliser des méthodes numériques. Dans l'ensemble canonique, il est également possible d'introduire des états stationnaires hors-équilibre grâce à l'advection d'un champ de vitesse, que nous introduisons ici mais discuterons dans le dernier chapitre.\\

{\bf \large Méthodes numériques} \\

Dans ce chapitre nous développons le fonctionnement des simulations de Monte Carlo Métropolis, qui permettent d'explorer l'espace des configurations et ainsi obtenir les valeurs moyennes d'observables. Ces méthodes sont particulièrement adaptées aux systèmes sur réseau comme le modèle d'Ising ou Solid-On-Solid, et on utilise alors l'algorithme de Glauber ou de Kawasaki selon l'ensemble thermodynamique dans lequel on se place. 

Néanmoins, l'énergie libre n'est pas une valeur mesurable directement dans les simulations de Monte Carlo, et il faut alors utiliser des méthodes indirectes. La méthode de Vasilyev consiste à découpler progressivement une rangée du système afin d'obtenir la dérivée de l'énergie libre par rapport à la taille du système. Cette méthode ne fonctionnant pas dans le cas du modèle SOS à cause de l'absence de terme de volume dans son hamiltonien, on utilise alors la méthode Lopes-Jacquin-Holdsworth, qui a été développée dans le cas d'un champ magnétique uniforme. Puisque l'intensité du champ magnétique est la valeur conjugée de la magnétisation totale, on calcule alors l'intégrale de la magnétisation entre deux intensités afin d'obtenir la différence entre l'énergie libre des deux systèmes. Puisque cette méthode utilise la magnétisaiton, elle ne peut donc pas être utilisée lorsque le paramètre d'ordre est conservé, ce qui nous mène à une conclusion : il n'existe pas de méthode pour la mesure de l'énergie libre dans des simulations numériques de Kawasaki pour le modèle SOS.

Le chapitre se terme avec une petite liste d'astuces à connaître pour celui qui désirerait reproduire mes résultats numériques, principalement sur l'optimisation du code et la parallélisation.\\

{\bf \large Interfaces à l'équilibre et effets de taille finie} \\

Lorsque la longueur de corrélation est du même ordre de grandeur que la taille du système, la contrainte imposée sur les modes mous de fluctuations ajoute une partie singulière à l'énergie libre. Cette dépendance de l'énergie libre en fonction de la taille du système implique une force thermodynamique qui s'appelle force de Casimir dans le cas des fluctuations du champ électromagnétique entre deux plaques diélectriques parfaitement conductrices, ou effet Casimir critique dans les systèmes critiques. Après avoir exposé ces deux effets grâce à la mécanique quantique et le groupe de renormalisation, nous étudions cette force de confinement dans le cas des interfaces confinées. 

En utilisant le formalisme de Matsubara pour l'équation du propagateur de l'hamiltonien dans le cas d'une interface continue soumise à champ externe, nous explicitons la distribution de probabilité de l'interface, l'énergie libre, la fonction de corrélation à deux points et la longueur de corrélation en fonction de l'énergie de l'état fondamental et du premier état excité. En appliquant ce formalisme à une interface libre et confinée, on retrouve des résultats connus sur la force thermodynamique. Dans le cas où cette interface ne possède pas de tension superficielle (comme c'est le cas dans les systèmes critiques), la correction de taille finie à la tension superficielle proposée par Privman nous permet de retrouver quantitivement le même comportement que pour la force de Casimir critique. Nous utilisons égalemnet ce formalisme dans le cas où l'interface est confinée à cause d'une pression afin de trouver la hauteur de moyenne de l'interface et sa variance.

Puisque la méthode Lopes-Jacquin-Holdsworth de calcul de l'énergie libre dans les simulations de Monte Carlo n'est pas utilisable dans la dynamique de Kawasaki et que la méthode de Vasilyev ne l'est pas non plus pour les modèles SOS, nous généralisons la méthode LJH pour des champs externes non-uniformes, ce qui permet l'intégration sur une magnétisation généralisée qui n'est plus conservée dans une dynamique de Kawasaki. Nous montrons que cette généralisation est en accord avec la matrice de transfert SOS, et observons le même comportement de l'énergie libre pour les dynamiques de Glauber et de Kawasaki.

Ce troisième chapitre s'achève par la diagonalisation exacte de la matrice de transfert du modèle SOS en absence de potentiel externe, généralisant ainsi les résultats de Privman, et nous étudions les limites à faible et haute température ainsi que dans la limite thermodynamique.\\

{\bf \large Le modèle Particles-Over-Particles} \\

Le mnodèle SOS est un modèle d'interface provenant de l'approximation à basse température du modèle d'Ising. Nous développons dans ce quatrième chapitre un nouveau modèle : le modèle POP. Ce nouveau modèle prend en compte le terme d'entropie associé au modèle d'Ising et qui se retrouve aisément lorsque l'on fait des simulations numériques sur Ising, mais qui font défaut dans SOS. Ce terme d'entropie apparaît lorsque l'on considère non plus juste la hauteur de l'interface (comme dans SOS), mais également le nombre de particules en-dessous. En labelisant ainsi les particules, il devient possible de créer des modèles avec $M$ types différents de particules, chacune étant régie par un coefficient cinétique ou un coefficient de diffusion dans le cas où elles appartiennent à des ensembles thermodynamiques différents. Nous finissons le chapitre en montrant comment procéder aux simulations numériques dans ce modèle, que nous comparons à la matrice de transfert. \\

{\bf \large Interfaces stationnaires hors-équilibre} \\

Lorsqu'une interface est advectée par un champ de vitesse, sa largeur et sa longueur de corrélation sont modifiées. Le cinquième et dernier chapitre s'intéresse à un écoulement uniforme et constant. 

À cause de l'invariance gagliléenne de translation dans le référentiel de l'écoulement, l'équation dynamique du modèle B reste inchangée. Le cas d'un champ soumis au modèle B et couplé à un autre champ soumis au modèle A (afin de briser l'invariance galiléenne) a été l'objet d'un article publié dans {\bf Journal of Statistical Mechanics: Theory and Experiment} en mars 2020, et la première partie de ce chapitre est une reproduction de l'article original. À partir des équations couplées donnant le modèle C, on trouve une relation fermée pour la dynamique d'une interface soumis à un écoulement uniiforme et constant dans l'espace de Fourrier. Cette méthode donne est valable pour une théorie gaussienne de l'interface, et l'écoulement introduit de la nouvelle physique, principalement que la tension de surface effective du système ainsi que la longueur de corrélation des fluctuations de l'interface sont augmentées. Ces résultats sont en accord qualitative avec les expériences de cisaillement sur les colloïdes. Tandis que la théorie sur les grands longueurs d'onde que nous trouvons se comporte comme celles d'une onde capillaire, nous trouvons les ordres suivants qui sont générés par le spectre de la hauteur des fluctuations. De plus, nous montrons comment ce modèle permet de décrire l'interface entre deux phases de colloïdes ayant une activité différente en jouant sur la température de chaque phase, ce qui nous donne de nouveaux résultats que nous ne pourrions trouver dans le cas à l'équilibre.

Nous passons ensuite aux modèles sur réseau, et étudions la dépendance de la variance des hauteurs et de la tension superficielle de l'interface SOS en fonction de l'écoulement. Dans le cas POP à deux types de particules, nous développons un formalisme très similaire à celui de l'article publié qui montre comment l'augmentation de l'écoulement induit un découplage dans la fonction de corrélation entre les différents types de particules dans la limite où l'écoulement est infini.