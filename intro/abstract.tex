{\bf Effets d'écoulements et de confinement dans les modèles discrets et continus d'interfaces} \\

Cette thèse examine les propriétés de l'interface entre deux phases dans un système avec des phases séparées. Nous regardons comment les effets de taille finis modifient les propriétés statistiques de ces interfaces, en particulier comment la dépendance de l'énergie libre par rapport à la taille du système donne lieu à des interactions de Casimir critique à longue portée proche du point critique.
Souvent, les interfaces sont décrites par des modèles simplifiés ou coarse-grained dont les seuls degrés de liberté sont les hauteurs de l'interface. Nous rappelons comment les propriétés statiques et dynamiques de ces interfaces sont retrouvées à partir de modèles microscopiques de spins et de théorie statistique des champs.
Nous étudions ensuite les effets de taille finie pour les interfaces continues comme le modèle Edwards-Wilkinson ou discrètes comme le modèle Solid-On-Solid, et discutons leur pertinence dans le cadre de l'effet Casimir critique.

Dans la seconde partie de la thèse, nous examinons des modèles d'interfaces sous écoulement qui possède des états stationnaires hors-équilibres. Nous développons 
ces équations dans le cadre du modèle C d'une interface, qui a un état stationnaire hors-équilibre lorsque soumis à un écoulement uniforme. L'état stationnaire hors-équilibre résultant exhibe des propriétés retrouvées dans les expériences sur des colloïdes sous cisaillement, notamment la suppression des fluctuations de la hauteur de l'interface et une augmentation de la longueur de corrélation des fluctuations. 
Finalement, nous proposons un nouveau modèle pour des interfaces uni-dimensionnelles qui est une modification du modèle Solid-on-Solid qui contient un terme supplémentaire d'entropie, ce qui le rend plus approprié pour l'étude de la diffusion et du cisaillement.
\\ \\
{\bf Keywords : } Modèles d'interface, Théorie d'ordonnancement des phases, Modèle d'Ising, Modèle Solid-On-Solid, Force de Casimir, Écoulements stationnaires hors-équilibre



{\bf Confinement and driving effects on continuous and discrete model interfaces}\\
This thesis examines the properties  of the interface between two phases in  phase separated systems. We are interested 
in how finite size effects modify the statistical properties of these interfaces, in particular  the dependence of the free energy on the system size 
gives rise to long range critical Casimir forces close to the critical point. Often the interfaces in phase separated systems are
described by simplified or coarse grained models whose only degrees of freedom are the interface height. We review how the statics and dynamics of
these  interface models can be derived from microscopic spin models and statistical field theories. We then examine finite size effects for 
continuous interface models such as the Edwards Wilkinson model and discrete models such as the Solid-On-Solid model  and discuss their 
relevance to the critical Casimir effect. In the second part of the thesis we examine models of driven interfaces which have nonequilibrium steady states. 
We develop a model C type model of an interface which shows a nonequlibrium steady
state even with constant driving. The resulting nonequlibrium steady state shows properties seen in experiments on sheared colloidal systems,
notably the suppression of height fluctuations but   a increase in the  correlation length of the fluctuations. Finally we propose a new model for
one dimensional interfaces which is a modification of  the solid-on-solid model and which contains extra entropic terms which make it more
appropriate to study diffusive dynamics and driving.

\\ \\
{\bf Keywords : }  Interface models, phase ordering dynamics, Ising model, Solid-On-Solid model, Casimir force, driven nonequilibrium steady states

\cleardoublepage
{\bf \huge Résumé en français}
\cleardoublepage



{\bf \huge Abstract}