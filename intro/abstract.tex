{\bf Effets d'écoulements et de confinement dans les modèles discrets et continus d'interfaces} \\

Lorsqu'un système statistique est confiné ou soumis à un écoulement, on observe une modification de l'énergie libre et des propriétés sous-jacentes du système, dont une force de confinement connue dans les systèmes critiques sous le nom de force de Casimir critique. Dans cette thèse, nous nous intéressons aux systèmes à une interface dans la phase ordonnée. 

À partir de la théorie statistique des champs, nous développons les équations de la dynamique d'une interface continue dans les modèles A et B. En utilisant les intégrales de chemin sur des interfaces possédant un hamiltonien gaussien, nous trouvons la distribution de probabilité de l'interface et la force de confinement pour une interface libre ou soumise à une pression constante. Nous mettons également au point un système d'équations couplées propre au modèle C pour montrer qu'un écoulement uniforme parallèle à l'interface diminue la fluctuation des hauteurs et augmente la longueur de corrélation du système.

Pour les modèles discrets, nous introduisons le modèle Solid-On-Solid issu du modèle d'Ising à basse température, et utilisons le formalisme des matrices de transfert afin de calculer la force de confinement. Nous généralisons également une méthode de mesure de l'énergie libre dans les simulations numériques, et en profitons pour observer les différences d'énergie libre entre les ensembles thermodynamiques. Finalement, nous proposons un nouveau modèle discret, qui contrairement au modèle SOS, n'est pas un modèle discret d'interface mais est une meilleure approximation du modèle d'Ising à basse température.
\\ \\
{\bf Mots-clés :} théorie statistique des champs, modèle A/B, Ising, Solid-On-Solid, force de Casimir, écoulements \\

{\bf Confinement and driving effects on continuous and discrete model interfaces}\\

When a statistical system is confined or driven, the free energy is modified, leading to effects such as the critical Casimir force. In this thesis, we are interested in interface systems in the ordered phase.

From statistical field theory, we develop the dynamical equations of continuous interfaces both in model A and B. Using path integrals on interfaces having gaussian hamiltonians, we find the probability density distribution and the confinement force for a free or under constant pressure interface. We also propose a coupled system as in model C, and show that uniform driving along the interface suppreses height fluctuations and increases the correlation length of fluctuations. 

For discrete models, we introduce the Solid-On-Solid model which is a low-temperature approximation of the Ising model, and use the transfer matrix method to compute the confinement force. We also generalize a method to numerically compute the free energy and use it to describe the free energy difference between statistical ensembles. Finally, we propose a new model, which contrary to SOS, is not an interface model but is a better approximation of the Ising model at low temperature.
\\ \\
{\bf Keywords : }  statistical field theory, model A/B, Ising, Solid-On-Solid, Casimir force, driving

\newpage
{\bf \huge Résumé en français}

\newpage
{\bf \huge Abstract}